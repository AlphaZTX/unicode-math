%%^^A%% um-code-msg.dtx -- part of UNICODE-MATH <wspr.io/unicode-math>
%%^^A%% Definitions of error, warning, and log messages.

% \section{Error messages}
% \seclabel{codemsg}
%
%    \begin{macrocode}
%<*package>
%    \end{macrocode}
%
% Wrapper functions:
%    \begin{macrocode}
\cs_new:Npn \@@_error:n     { \msg_error:nn     {unicode-math} }
\cs_new:Npn \@@_error:nx    { \msg_error:nnx    {unicode-math} }
\cs_new:Npn \@@_warning:n   { \msg_warning:nn   {unicode-math} }
\cs_new:Npn \@@_warning:nnn { \msg_warning:nnxx {unicode-math} }
\cs_new:Npn \@@_log:n       { \msg_log:nn       {unicode-math} }
\cs_new:Npn \@@_log:nx      { \msg_log:nnx      {unicode-math} }
%    \end{macrocode}
%
%    \begin{macrocode}
\cs_generate_variant:Nn \msg_new:nnn  {nnx}
\cs_generate_variant:Nn \msg_new:nnnn {nnxx}
\cs_new:Nn \@@_msg_new:nnn  { \msg_new:nnx  {#1} {#2} { \tl_trim_spaces:n {#3} } }
\cs_new:Nn \@@_msg_new:nnnn { \msg_new:nnxx {#1} {#2} { \tl_trim_spaces:n {#3} } { \tl_trim_spaces:n {#4} } }
%    \end{macrocode}
%
%    \begin{macrocode}
\char_set_catcode_space:n {32}
%    \end{macrocode}
%
%    \begin{macrocode}
\@@_msg_new:nnn {unicode-math} {no-tfrac}
{
  Small fraction command \protect\tfrac\ not defined.\\
  Load amsmath or define it manually before loading unicode-math.
}
\@@_msg_new:nnn {unicode-math} {default-math-font}
{
  Defining the default maths font as '\l_@@_fontname_tl'.
}
\@@_msg_new:nnn {unicode-math} {setup-implicit}
{
  Setup alphabets: implicit mode.
}
\@@_msg_new:nnn {unicode-math} {setup-explicit}
{
  Setup alphabets: explicit mode.
}
\@@_msg_new:nnn {unicode-math} {alph-initialise}
{
  Initialising \@backslashchar math#1.
}
\@@_msg_new:nnn {unicode-math} {setup-alph}
{
  Setup alphabet: #1.
}
\@@_msg_new:nnn {unicode-math} {no-alphabet}
{
  I am trying to set up alphabet"#1" but there are no configuration settings for it.
  (See source file "unicode-math-alphabets.dtx" to debug.)
}
\@@_msg_new:nnn {unicode-math} {no-named-range}
 {
  I am trying to define new alphabet "#2" in range "#1", but range "#1" hasn't been defined yet.
 }
\@@_msg_new:nnn {unicode-math} {missing-alphabets}
 {
  Missingmathalphabetsinfont "\fontname\g_@@_curr_font_cmd_tl" \\ \\
  \seq_map_function:NN \l_@@_missing_alph_seq \@@_print_indent:n
 }
\cs_new:Nn \@@_print_indent:n { \space\space\space\space #1 \\ }
\@@_msg_new:nnn {unicode-math} {macro-expected}
{
  I've expected that #1 is a macro, but it isn't.
}
\@@_msg_new:nnn {unicode-math} {wrong-meaning}
{
  I've expected #1 to have the meaning #3, but it has the meaning #2.
}
\@@_msg_new:nnn {unicode-math} {patch-macro}
{
  I'm going to patch macro #1.
}
\@@_msg_new:nnn {unicode-math} {mathtools-overbracket} {
  Using \token_to_str:N \overbracket\ and
         \token_to_str:N \underbracket\ from
	 `mathtools' package.\\
  \\
  Use \token_to_str:N \Uoverbracket\ and
       \token_to_str:N \Uunderbracket\ for
       original `unicode-math' definition.
}
\@@_msg_new:nnn {unicode-math} {mathtools-colon} {
  I'm going to overwrite the following commands from
  the `mathtools' package: \\ \\
  \ \ \ \ \token_to_str:N \dblcolon,
  \token_to_str:N \coloneqq,
  \token_to_str:N \Coloneqq,
  \token_to_str:N \eqqcolon. \\ \\
  Note that since I won't overwrite the other colon-like
  commands, using them will lead to inconsistencies.
}
\@@_msg_new:nnn {unicode-math} {colonequals} {
  I'm going to overwrite the following commands from
  the `colonequals' package: \\ \\
  \ \ \ \ \token_to_str:N \ratio,
          \token_to_str:N \coloncolon,
          \token_to_str:N \minuscolon, \\
  \ \ \ \ \token_to_str:N \colonequals,
          \token_to_str:N \equalscolon,
          \token_to_str:N \coloncolonequals. \\ \\
  Note that since I won't overwrite the other colon-like
  commands, using them will lead to inconsistencies.
  Furthermore, changing \token_to_str:N \colonsep \c_space_tl
  or \token_to_str:N \doublecolonsep \c_space_tl won't have
  any effect on the re-defined commands.
}
\@@_msg_new:nnn {unicode-math} {bad-cs-in-range}
  {
    Command `#1` in math range is not recognised as a maths symbol.
    Check file "unicode-math-table.tex" for allowable commands.
  }
\@@_msg_new:nnn {unicode-math} {legacy-char-not-supported}
  {
    Command `#1` is a legacy maths symbol that is not supported by unicode-math.
  }
\@@_msg_new:nnn {unicode-math} {range-not-bf-sf}
  {
    Range alphabets cannot include alphabets referring to `bf`, `sf`, or `bfsf`
    since they relate to input commands not output glyphs.
    Use `bfit` or `bfup` (etc.) to specify which.
  }
%    \end{macrocode}
%
%    \begin{macrocode}
\char_set_catcode_ignore:n {32}
%    \end{macrocode}
%
%    \begin{macrocode}
%</package>
%    \end{macrocode}

\endinput

% /©
%
% ------------------------------------------------
% The UNICODE-MATH package  <wspr.io/unicode-math>
% ------------------------------------------------
% This package is free software and may be redistributed and/or modified under
% the conditions of the LaTeX Project Public License, version 1.3c or higher
% (your choice): <http://www.latex-project.org/lppl/>.
% ------------------------------------------------
% Copyright 2006-2017  Will Robertson, LPPL "maintainer"
% Copyright 2010-2017  Philipp Stephani
% Copyright 2011-2017  Joseph Wright
% Copyright 2012-2015  Khaled Hosny
% ------------------------------------------------
%
% ©/
