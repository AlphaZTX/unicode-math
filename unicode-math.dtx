% \iffalse
% !TEX TS-program = XeLaTeX
% ^^A %%%%%%%%%%%%%%%%%%%%%%%%%%%%%%%
% ^^A   SELF-EXTRACTION BEGINS HERE
% ^^A %%%%%%%%%%%%%%%%%%%%%%%%%%%%%%%
%<*internal>
\begingroup
\input l3docstrip.tex
\begingroup
  \catcode`\_ = 12 %
  \long\gdef\prepareActiveModule#1{%
    \if\relax#1\relax
       \let\replaceModuleInLine\empty
    \else
      \def\replaceModuleInLine{%
        \replaceAllIn\inLine{@@@@}{!!!!!FOURAT!!!!!}%
        \replaceAllIn\inLine{__@@}{__#1}%
        \replaceAllIn\inLine{_@@}{__#1}%
        \replaceAllIn\inLine{@@}{__#1}%
        \replaceAllIn\inLine{!!!!!FOURAT!!!!!}{@@}% i.e., use @@@@ when you need literal @@ in code
      }%
    \fi
  }
\endgroup
\keepsilent
\let\MetaPrefix\DoubleperCent
\declarepreamble\texpreamble
Copyright 2006-2015   Will Robertson <will.robertson@latex-project.org>
Copyright 2010-2013 Philipp Stephani <st_philipp@yahoo.de>
Copyright 2012-2015     Khaled Hosny <khaledhosny@eglug.org>

This package is free software and may be redistributed and/or modified under
the conditions of the LaTeX Project Public License, version 1.3c or higher
(your choice): <http://www.latex-project.org/lppl/>.

This work is "maintained" by Will Robertson.
\endpreamble
\nopostamble
\askforoverwritefalse
\let\MetaPrefix\DoubleperCent
\usepreamble\texpreamble
\ifx\UMDEBUG\undefined
  \def\UMDEBUG{}%
\else
  \def\UMDEBUG{,debug}%
\fi
\generate{\file{unicode-math.sty}{
  \from{unicode-math.dtx}{preamble\UMDEBUG}
  \from{unicode-math-msg.dtx}{msg\UMDEBUG}
  \from{unicode-math-usv.dtx}{usv\UMDEBUG}
  \from{unicode-math.dtx}{load\UMDEBUG}
  \from{unicode-math-alphabets.dtx}{alphabets\UMDEBUG}
}}
\generate{\file{unicode-math-xetex.sty}{
  \from{unicode-math.dtx}{package,XE\UMDEBUG}
  \from{unicode-math-compat.dtx}{compat,XE\UMDEBUG}
}}
\generate{\file{unicode-math-luatex.sty}{
  \from{unicode-math.dtx}{package,LU\UMDEBUG}
  \from{unicode-math-compat.dtx}{compat,LU\UMDEBUG}
}}
\def\tempa{plain}
\ifx\tempa\fmtname\endgroup\expandafter\bye\fi
\generate{\file{dtx-style.sty}{\from{\jobname.dtx}{dtx-style}}}
\endgroup
\ProvidesFile{unicode-math.dtx}
%</internal>
%<preamble&!XE&!LU>\ProvidesPackage{unicode-math}
%<preamble&XE>\ProvidesPackage{unicode-math-xetex}
%<preamble&LU>\ProvidesPackage{unicode-math-luatex}
%<*preamble>
  [2014/07/30 v0.8 Unicode maths in XeLaTeX and LuaLaTeX]
%</preamble>
%<*internal>
\def\DOCUMENTEND{F}
% !TEX TS-program = XeLaTeX

\providecommand\DOCUMENTEND{T}
\documentclass[a4paper]{ltxdoc}

\if \DOCUMENTEND T \OnlyDescription \fi

\makeatletter
\GetFileInfo{unicode-math.dtx}
\let\umfiledate\filedate
\let\umfileversion\fileversion

\CheckSum{0}
\EnableCrossrefs
\CodelineIndex

\errorcontextlines=999

\def\@dotsep{1000}
\setcounter{tocdepth}{2}
\setlength\columnseprule{0.4pt}
\renewcommand\tableofcontents{\relax
  \begin{multicols}{2}[\section*{\contentsname}]\relax
    \@starttoc{toc}\relax
  \end{multicols}}

\setcounter{IndexColumns}{2}
\renewenvironment{theglossary}
  {\small\list{}{}
     \item\relax
     \glossary@prologue\GlossaryParms
     \let\item\@idxitem \ignorespaces
     \def\pfill{\hspace*{\fill}}}
  {\endlist}

\usepackage[svgnames]{xcolor}
\usepackage{array,booktabs,calc,enumitem,fancyvrb,graphicx,ifthen,longtable,refstyle,subfig,topcapt,url,varioref,underscore}
\setcounter{LTchunksize}{100}
\usepackage[slash-delimiter=frac,nabla=literal]{unicode-math}
\usepackage{metalogo,hologo}

\fvset{fontsize=\small,xleftmargin=2em}
\usepackage[it]{titlesec}

\setmainfont{texgyrepagella}%
 [
  Extension = .otf ,
  UprightFont = *-regular ,
  ItalicFont = *-italic ,
  BoldFont = *-bold ,
  BoldItalicFont = *-bolditalic ,
 ]
\setsansfont{Iwona}%
 [
  Scale=MatchLowercase,
  Extension = .otf,
  UprightFont = *-Regular,
  ItalicFont  = *-Italic,
  BoldFont    = *-Bold,
  BoldItalicFont = *-BoldItalic,
 ]
\setmonofont{Inconsolatazi4-Regular.otf}%
 [
  Scale=MatchLowercase,
  BoldFont=Inconsolatazi4-Bold.otf
 ]

\setmathfont{texgyrepagella-math.otf}
\setmathfont[version=xits]{xits-math.otf}
\newfontface\umfont{xits-math.otf}

\usepackage{hypdoc}
\hypersetup{linktocpage}

% work around some issue turning | into "j" inside mathsf in the definition of \Module:
% (also prettify)
\def\Module#1{{\footnotesize\color{red}$\langle$\texttt{#1}$\rangle$}}

\usepackage{minitoc}

\linespread{1.1}
\frenchspacing

\definecolor{niceblue}{rgb}{0.2,0.4,0.8}

\def\theCodelineNo{\textcolor{niceblue}{\sffamily\tiny\arabic{CodelineNo}}}

\newcommand*\name[1]{{#1}}
\newcommand*\pkg[1]{\textsf{#1}}
\newcommand*\feat[1]{\texttt{#1}}
\newcommand*\opt[1]{\texttt{#1}}

\newcommand*\note[1]{\unskip\footnote{#1}}

\let\latin\textit
\def\eg{\latin{e.g.}}
\def\Eg{\latin{E.g.}}
\def\ie{\latin{i.e.}}
\def\etc{\@ifnextchar.{\latin{etc}}{\latin{etc.}\@}}

\def\STIX{\textsc{stix}}
\def\MacOSX{Mac~OS~X}
\def\ascii{\textsc{ascii}}
\def\OMEGA{Omega}

\newcounter{argument}

\makeatletter
\g@addto@macro\endmacro{\setcounter{argument}{0}}
\makeatother

\newcommand*\darg[1]{%
  \stepcounter{argument}%
  {\ttfamily\char`\#\theargument~:~}#1\par\noindent\ignorespaces
}
\newcommand*\doarg[1]{%
  \stepcounter{argument}%
  {\ttfamily\makebox[0pt][r]{[}\char`\#\theargument]:~}#1\par\noindent\ignorespaces
}

\newcommand\codeline[1]{\par{\centering#1\par\noindent}\ignorespaces}

\newcommand\unichar[1]{\textsc{u}+\texttt{\small#1}}

\setlength\parindent{2em}

\def \MakePrivateLetters {%
  \catcode`\@=11\relax
  \catcode`\_=11\relax
  \catcode`\:=11\relax
}

\def\partname{Part}

\makeatother

\begin{document}

\title{Experimental Unicode mathematical typesetting: The \pkg{unicode-math} package}
\author{Will Robertson, Philipp Stephani and Khaled Hosny\\
        \texttt{will.robertson@latex-project.org}}
\date{\umfiledate \qquad \umfileversion}

\maketitle

\begin{abstract}
\noindent
This document describes the \pkg{unicode-math} package, which is
intended as an implementation of Unicode
maths for \LaTeX\ using the \XeTeX\ and Lua\TeX\ typesetting engines.
With this package, changing maths fonts is as easy as changing
text fonts --- and there are more and more maths fonts appearing now.
Maths input can also be simplified with Unicode since literal glyphs may be
entered instead of control sequences in your document source.

The package provides support for both \XeTeX\ and Lua\TeX. The different
engines provide differing levels of support for Unicode maths.
Please let us know of any troubles.

Alongside this documentation file, you should be able to find a minimal
example demonstrating the use of the package,
`\texttt{unimath-example.ltx}'. It also comes with a separate document,
`\texttt{unimath-symbols.pdf}',
containing a complete listing of mathematical symbols defined by
\pkg{unicode-math}, including comparisons between different fonts.

Finally, while the STIX fonts may be used with this package, accessing
their alphabets in their `private user area' is not yet supported.
(Of these additional alphabets there is a separate caligraphic design
distinct to the script design already included.)
Better support for the STIX fonts is planned for an upcoming revision of the
package after any problems have been ironed out with the initial version.

\end{abstract}

\doparttoc\faketableofcontents

\newpage
\part{User documentation}
\parttoc

\clearpage
\section{Introduction}

This document describes the \pkg{unicode-math} package, which is an
\emph{experimental} implementation of a macro to Unicode glyph encoding for
mathematical characters.

Users who desire to specify maths alphabets only (Greek and Latin letters,
and Arabic numerals)
may wish to use Andrew Moschou's \pkg{mathspec} package instead.
(\XeTeX-only at time of writing.)

\section{Acknowledgements}

Many thanks to:
Microsoft for developing the mathematics extension to OpenType as part of
Microsoft Office~2007;
Jonathan Kew for implementing Unicode math support in \XeTeX;
Taco Hoekwater for implementing Unicode math support in \LuaTeX;
Barbara Beeton for her prodigious effort compiling the definitive list of Unicode math
glyphs and their \LaTeX\ names (inventing them where necessary), and also
for her thoughtful replies to my sometimes incessant questions;
Philipp Stephani for extending the package to support \LuaTeX.
Ross Moore and Chris Rowley have provided moral and technical support
from the very early days with great insight into the issues we face trying
to extend and use \TeX\ in the future.
Apostolos Syropoulos, Joel Salomon, Khaled Hosny, and Mariusz Wodzicki
have been fantastic beta testers.

\section{Getting started}

Load \pkg{unicode-math} as a regular \LaTeX\ package. It should be loaded
after any other maths or font-related package in case it needs to overwrite
their definitions. Here's an example:
\begin{Verbatim}
\usepackage{amsmath} % if desired
\usepackage{unicode-math}
\setmathfont{Asana-Math.otf}
\end{Verbatim}
Three OpenType maths fonts are included by default in \TeX\ Live 2011:
Latin Modern Math, Asana Math, and XITS Math.
These can be loaded directly with their filename
with both \XeLaTeX\ and \LuaLaTeX; resp.,
\begin{Verbatim}
\setmathfont{latinmodern-math.otf}
\setmathfont{Asana-Math.otf}
\setmathfont{xits-math.otf}
\end{Verbatim}
Other OpenType maths fonts may be loaded in the usual way; please see the
\pkg{fontspec} documentation for more information.

Once the package is loaded, traditional TFM-based fonts are not supported any more;
you can only switch to a different OpenType math font using the \cs{setmathfont} command.
If you do not load an OpenType maths font before |\begin{document}|, Latin Modern Math (see above) will be loaded automatically.

\subsection{New commands}
\textbf{New v0.8:}
\pkg{unicode-math} provides the following commands to select specific `alphabets' within the unicode maths font: (usage, e.g.: |$\symbfsf{g}$|${}\to\symbfsf{g}$)
\begin{quote}
\ExplSyntaxOn
\clist_map_inline:nn {
  normal, literal, up, bfup, bfit, sfup, sfit, bfsfup, bfsfit, bfsf,
  bb, bbit, scr, bfscr, cal, bfcal, frak, bffrak,
  sf, bf, tt, it,
  }{\cs{sym#1}~}
\ExplSyntaxOff
\end{quote}
Many of these are also defined with `familiar' synonyms:
\begin{quote}
\ExplSyntaxOn
\clist_map_inline:nn {
   normal, bb, bbit, scr, bfscr, cal, bfcal, frak, bffrak,
   bfup, bfit, sfup, sfit, bfsfup, bfsfit, bfsf,
   sf, bf, tt, it,
  }{\mbox{\cs{math#1}}~}
\ExplSyntaxOff
\end{quote}

\begin{table}[t]\centering
  \topcaption{\pkg{unicode-math} commands.}
  \tablabel{symvsmath}
  \begin{tabular}{lll}
    \toprule
    \pkg{unicode-math} command & Synonym \\
    \midrule
    |\symnormal|  & |\mathnormal| \\
    |\symliteral| &               \\
    |\symup|      &               \\
    |\symbfup|    & |\mathbfup|   \\
    |\symbfit|    & |\mathbfit|   \\
    |\symsfup|    & |\mathsfup|   \\
    |\symsfit|    & |\mathsfit|   \\
    |\symbfsfup|  & |\mathbfsfup| \\
    |\symbfsfit|  & |\mathbfsfit| \\
    |\symbfsf|    & |\mathbfsf|   \\
    |\symbb|      & |\mathbb|     \\
    |\symbbit|    & |\mathbbit|   \\
    |\symscr|     & |\mathscr|    \\
    |\symbfscr|   & |\mathbfscr|  \\
    |\symcal|     & |\mathcal|    \\
    |\symbfcal|   & |\mathbfcal|  \\
    |\symfrak|    & |\mathfrak|   \\
    |\symbffrak|  & |\mathbffrak| \\
    |\symsf|      & |\mathsf|     \\
    |\symbf|      & |\mathbf|     \\
    |\symtt|      & |\mathtt|     \\
    |\symit|      & |\mathit|     \\
    \bottomrule
  \end{tabular}
\end{table}

So what about \cs{mathup}, \cs{mathit}, \cs{mathbf}, \cs{mathsf}, and \cs{mathtt}?
(N.B.: \cs{mathrm} is defined as a synonym for \cs{mathup}, but the latter is prefered as it is a script-agnostic term.)
These commands have `overloaded' meanings in \LaTeX, and it's important to consider the subtle differences between, e.g., \cs{symbf} and \cs{mathbf}.
The former switches to single-letter mathematical symbols, whereas the second switches to a text font that behaves correctly in mathematics but should be used for multi-letter identifiers.
These four commands (and \cs{mathrm}) are defined in the traditional \LaTeX\ manner.
Further details are discussed in \secref{mathselect}.

Additional similar commands can be defined using
\begin{Verbatim}
\setmathfontface\mathfoo{...}
\end{Verbatim}

\subsection{Package options}
Package options may be set when the package as loaded or at any later
stage with the \cs{unimathsetup} command. Therefore, the following two
examples are equivalent:
\begin{Verbatim}
\usepackage[math-style=TeX]{unicode-math}
% OR
\usepackage{unicode-math}
\unimathsetup{math-style=TeX}
\end{Verbatim}
Note, however, that some package options affects how maths is initialised
and changing an option such as |math-style| will not take effect until a
new maths font is set up.

Package options may \emph{also} be used when declaring new maths fonts,
passed via options to the \cs{setmathfont} command.
Therefore, the following two examples are equivalent:
\begin{Verbatim}
\unimathsetup{math-style=TeX}
\setmathfont{Cambria Math}
% OR
\setmathfont{Cambria Math}[math-style=TeX]
\end{Verbatim}

A short list of package options is shown in \tabref{pkgopt}.
See following sections for more information.

\begin{table}\centering
  \topcaption{Package options.}
  \tablabel{pkgopt}
  \begin{tabular}{lll}
    \toprule
    Option & Description & See\dots \\
    \midrule
    |math-style| & Style of letters & \secref{math-style} \\
    |bold-style| & Style of bold letters & \secref{bold-style} \\
    |sans-style| & Style of sans serif letters & \secref{sans-style} \\
    |nabla|      & Style of the nabla symbol & \secref{nabla} \\
    |partial|    & Style of the partial symbol & \secref{partial} \\
    |colon| & Behaviour of \cs{colon} & \secref{colon} \\
    |slash-delimiter| & Glyph to use for `stretchy' slash & \secref{slash-delimiter} \\
    \bottomrule
  \end{tabular}
\end{table}


\section{Unicode maths font setup}

In the ideal case, a single Unicode font will contain all maths glyphs we
need. The file |unicode-math-table.tex| (based on Barbara Beeton's \STIX\ table)
provides the mapping between Unicode
maths glyphs and macro names (all 3298 — or however many — of them!). A
single command
\codeline{\cmd\setmathfont\marg{font name}\oarg{font features}}
implements this for every every symbol and alphabetic variant.
That means |x| to $x$, |\xi| to $\xi$, |\leq| to $\leq$, etc., |\symscr{H}|
to $\symscr{H}$ and so on, all for Unicode glyphs within a single font.

This package deals well with Unicode characters for maths
input. This includes using literal Greek letters in formulae,
resolving to upright or italic depending on preference.

Font features specific to \pkg{unicode-math} are shown in \tabref{mathfontfeatures}.
Package options (see \tabref{pkgopt}) may also be used.
Other \pkg{fontspec} features are also valid.

\begin{table}\centering
  \topcaption{Maths font options.}
  \tablabel{mathfontfeatures}
  \begin{tabular}{lll}
    \toprule
    Option & Description & See\dots \\
    \midrule
    |range| & Style of letters & \secref{range} \\
    |script-font| & Font to use for sub- and super-scripts & \secref{sscript} \\
    |script-features| & Font features for sub- and super-scripts & \secref{sscript} \\
    |sscript-font| & Font to use for nested sub- and super-scripts & \secref{sscript} \\
    |sscript-features| & Font features for nested sub- and super-scripts & \secref{sscript} \\
    \bottomrule
  \end{tabular}
\end{table}

\subsection{Using multiple fonts}
\seclabel{range}

There will probably be few cases where a single Unicode maths font suffices
(simply due to glyph coverage). The \STIX\ font comes to mind as a
possible exception. It will therefore be necessary to delegate specific
Unicode ranges of glyphs to separate fonts:
  \codeline{\cmd\setmathfont\marg{font name}|[range=|\meta{unicode range}|,|\meta{font features}|]|}
where \meta{unicode range} is a comma-separated list of Unicode slot numbers and ranges such as |{"27D0-"27EB,"27FF,"295B-"297F}|.
Note that \TeX's syntax for accessing the slot number of a character, such as |`\+|, will also work here.

You may also use the macro for accessing the glyph, such as \cs{int}, or whole collection of symbols with the same math type, such as \cs{mathopen}, or complete math styles such as \cs{symbb}.
(Only numerical slots, however, can be used in ranged declarations.)

\subsubsection{Control over alphabet ranges}

As discussed earlier, Unicode mathematics consists of a number of `alphabet styles' within a single font. In \pkg{unicode-math}, these ranges are indicated with the following (hopefully self-explanatory) labels:
\begin{quote}\ttfamily
\ExplSyntaxOn
\clist_use:Nn \g__um_named_ranges_clist {\,,\,~}
\ExplSyntaxOff
\end{quote}
Fonts can be selected for specified ranges only using the following syntax, in which case all other maths font setup remains untouched:
\begin{itemize}
\item |[range=bb]| to use the font for `|bb|' letters only.
\item |[range=bfsfit/{greek,Greek}]| for Greek lowercase and uppercase only (also with |latin|, |Latin|, |num| as possible options for Latin lower-/upper-case and numbers, resp.).
\item |[range=up->sfup]| to map to different output styles.
\end{itemize}

Note that `meta-styles' such as `|bf|' and `|sf|' are not included here since they are context dependent. Use |[range=bfup]| and |[range=bfit]| to effect changes to the particular ranges selected by `|bf|' (and similarly for `|sf|').

If a particular math style is not defined in the font, we fall back onto the lower-base plane (i.e., `upright') glyphs.
Therefore, to use an \ascii-encoded fractur font, for example, write
\begin{Verbatim}
  \setmathfont{SomeFracturFont}[range=frak]
\end{Verbatim}
and because the math plane fractur glyphs will be missing, \pkg{unicode-math} will know to use the \ascii\ ones instead.
If necessary this behaviour can be forced with |[range=frak->up]|, since the `|up|' range corresponds to \ascii\ letters.

%If you wanted to swap the maths symbols with sans serif forms, it would be possible to write |[range={up->sfup,it->sfit}]|.
%Note, however, that at present Unicode does not encode glyphs for sans serif Greek (\tabref{mathalphabets}).

Users of the impressive Minion Math fonts (commercial) may use remapping to access the bold glyphs using:
\begin{Verbatim}
  \setmathfont{MinionMath-Regular.otf}
  \setmathfont{MinionMath-Bold.otf}[range={bfup->up,bfit->it}]
\end{Verbatim}
To set up the complete range of optical sizes for these fonts, a font declaration such as the following may be used: (adjust may be desired according to the font size of the document)
\begin{Verbatim}
\setmathfont{Minion Math}[
 SizeFeatures = {
  {Size =      -6.01,  Font = MinionMath-Tiny},
  {Size =  6.01-8.41,  Font = MinionMath-Capt},
  {Size =  8.41-13.01, Font = MinionMath-Regular},
  {Size = 13.01-19.91, Font = MinionMath-Subh},
  {Size = 19.91-,      Font = MinionMath-Disp}
 }]

\setmathfont{Minion Math}[range = {bfup->up,bfit->it},
 SizeFeatures = {
  {Size =      -6.01,  Font = MinionMath-BoldTiny},
  {Size =  6.01-8.41,  Font = MinionMath-BoldCapt},
  {Size =  8.41-13.01, Font = MinionMath-Bold},
  {Size = 13.01-19.91, Font = MinionMath-BoldSubh},
  {Size = 19.91-,      Font = MinionMath-BoldDisp}
 }]
\end{Verbatim}
\textbf{v0.8:} Note that in previous versions of \pkg{unicode-math}, these features were labelled |[range=\mathbb]| and so on. This old syntax is still supported for backwards compatibility, but is now discouraged.


\subsection{Script and scriptscript fonts/features}
\seclabel{sscript}

Cambria Math uses OpenType font features to activate smaller optical sizes
for scriptsize and scriptscriptsize symbols (the $B$ and $C$, respectively,
in $A_{B_C}$).
Other typefaces (such as Minion Math) may use entirely separate font files.

The features |script-font| and |sscript-font| allow alternate fonts to be
selected for the script and scriptscript sizes, and |script-features| and
|sscript-features| to apply different OpenType features to them.

By default |script-features| is defined as |Style=MathScript| and |sscript-features| is |Style=MathScriptScript|.
These correspond to the two levels of OpenType's |ssty| feature tag.
If the |(s)script-features| options are specified manually, you must
additionally specify the |Style| options as above.


\subsection{Maths `versions'}

\LaTeX\ uses a concept known as `maths versions' to switch math fonts
mid-document.
This is useful because it is more efficient than loading a complete maths
font from scratch every time---especially with thousands of glyphs in the case of Unicode maths!
The canonical example for maths versions is to select a `bold' maths font
which might be suitable for section headings, say.
(Not everyone agrees with this typesetting choice, though; be careful.)

To select a new maths font in a particular version, use the syntax
  \codeline{\cmd\setmathfont\marg{font name}|[version=|\meta{version name}|,|\meta{font features}|]|}
and to switch between maths versions mid-document use the standard \LaTeX\ command
\cmd\mathversion\marg{version name}.


\subsection{Legacy maths `alphabet' commands}
\seclabel{mathselect}

\LaTeX\ traditionally uses \cs{DeclareMathAlphabet} and \cs{SetMathAlphabet} to define document commands such as \cs{mathit}, \cs{mathbf}, and so on.
While these commands can still be used, \pkg{unicode-math} defines a wrapper command to assist with the creation of new such maths alphabet commands.
This command is known as \cs{setmathface} in symmetry with \pkg{fontspec}'s \cs{newfontface} command; it takes syntax:
\begin{quote}
  \cmd\setmathfontface\meta{command}\marg{font name}|[|\meta{font features}|]|

  \makebox[0pt][l]{\cmd\setmathfontface\meta{command}\marg{font name}|[||version=|\meta{version name}|,|\meta{font features}|]|}
\end{quote}
For example, if you want to define a new legacy maths alphabet font \cs{mathittt}:
\begin{verbatim}
  \setmathfontface\mathittt{texgyrecursor-italic.otf}
  ...
  $\mathittt{foo} = \mathittt{a} + \mathittt{b}$
\end{verbatim}


\subsubsection{Default `text math' fonts}

The five `text math' fonts, discussed above, are: \cs{mathrm}, \cs{mathbf}, \cs{mathit}, \cs{mathsf}, and \cs{mathtt}.
These commands are also defined with their original definition under synonyms \cs{mathtextrm}, \cs{mathtextbf}, and so on.

When selecting document fonts using \pkg{fontspec} commands such as \cs{setmainfont}, \pkg{unicode-math} inserts some additional code into \pkg{fontspec} that keeps the current default fonts `in sync' with their corresponding \cs{mathrm} commands, etc.

For example, in standard \LaTeX, \cs{mathsf} doesn't change even if the main document font is changed using |\renewcommand\sfdefault{...}|. With \pkg{unicode-math} loaded, after writing |\setsansfont{Helvetica}|, \cs{mathsf} will now be set in Helvetica.

If the \cs{mathsf} font is set explicitly at any time in the preamble, this `auto-following' does not occur. The legacy math font switches can be defined either with commands defined by \pkg{fontspec} (|\setmathrm|, |\setmathsf|, etc.) or using the more general |\setmathfontface\mathsf| interface defined by \pkg{unicode-math}.


\subsubsection{Replacing `text math' fonts by symbols}

For certain types of documents that use legacy input syntax (say you're typesetting a new version of a book written in the 1990s), it would be preferable to use |\symbf| rather than |\mathbf| en masse.
For example, if bold maths is used only for vectors and matrices, a dedicated symbol font will produce better spacing and will better match the main math font.

Alternatively, you may have used an old version of \pkg{unicode-math} (pre-v0.8), when the \cs{symXYZ} commands were not defined and \cs{mathbf} behaved like \cs{symbf} does now.
A series of package options (\tabref{legacyfontswitch}) are provided to facilitate switching the definition of \cs{mathXYZ} for the five legacy text math font definitions.

\begin{table}
  \centering
  \topcaption{Maths text font configuration options. Note that \cs{mathup} and \cs{mathrm} are aliases of each other and cannot be configured separately.}
  \tablabel{legacyfontswitch}
  \begin{tabular}{lll}
    \toprule
    Defaults (from `text' font) & From `maths symbols' \\
    \midrule
    |mathrm=text| &   |mathrm=sym |  \\
    |mathup=text|\rlap{$^\ast$} &   |mathup=sym|{}\rlap{$^\ast$}  \\
    |mathit=text| &   |mathit=sym |  \\
    |mathsf=text| &   |mathsf=sym |  \\
    |mathbf=text| &   |mathbf=sym |  \\
    |mathtt=text| &   |mathtt=sym |  \\
    \bottomrule
  \end{tabular}
\end{table}

A `smart' macro is intended for a future version of \pkg{unicode-math} that can automatically distinguish between single- and multi-letter arguments to \cs{mathbf} and use either the maths symbol or the `text math' font as appropriate.


\subsubsection{Operator font}

\LaTeX\ defines an internal command \cs{operator@font} for typesetting elements such as |\sin| and |\cos|.
This font is selected from the legacy |operators| NFSS `MathAlphabet', which is no longer relevant in the context of \pkg{unicode-math}.
By default, the \cs{operator@font} command is defined to switch to the \cs{mathrm} font.
You may now change these using the command:
\begin{Verbatim}
\setoperatorfont\mathit
\end{Verbatim}
Or, to select a \pkg{unicode-math} range:
\begin{Verbatim}
\setoperatorfont\symscr
\end{Verbatim}
\setoperatorfont\symscr
For example, after the latter above, |$\sin x$| will produce `$\sin x$'.

\mathversion{normal}
\setoperatorfont\mathrm


\section{Maths input}

\XeTeX's Unicode support allows maths input through two methods. Like
classical \TeX, macros such as \cmd\alpha, \cmd\sum, \cmd\pm, \cmd\leq, and
so on, provide verbose access to the entire repertoire of characters defined
by Unicode. The literal characters themselves may be used instead, for more
readable input files.

\subsection{Math `style'}
\seclabel{math-style}

Classically, \TeX\ uses italic lowercase Greek letters and \emph{upright}
uppercase Greek letters for variables in mathematics. This is contrary to
the \textsc{iso} standards of using italic forms for both upper- and lowercase.
Furthermore, in various historical contexts, often associated with French typesetting, it was common to use upright uppercase \emph{Latin} letters as well as upright
upper- and lowercase Greek, but italic lowercase latin. Finally, it is not unknown to use upright letters
for all characters, as seen in the Euler fonts.

The \pkg{unicode-math} package accommodates these possibilities with the
option \opt{math-style} that takes one of four (case sensitive) arguments:
\opt{TeX}, \opt{ISO}, \opt{french}, or \opt{upright}.\footnote{Interface inspired by Walter Schmidt's \pkg{lucimatx} package.}
The \opt{math-style} options' effects are shown in brief in \tabref{math-style}.

The philosophy behind the interface to the mathematical symbols
lies in \LaTeX's attempt of separating content and formatting. Because input
source text may come from a variety of places, the upright and
`mathematical' italic Latin and Greek alphabets are \emph{unified} from the
point of view of having a specified meaning in the source text. That is, to
get a mathematical ‘$x$’, either the \ascii\ (`keyboard') letter |x| may
be typed, or the actual Unicode character may be used. Similarly for Greek
letters. The upright or italic forms are then chosen based on the
|math-style| package option.

If glyphs are desired that do not map as per the package option (for
example, an upright `g' is desired but typing |$g$| yields `$g$'),
\emph{markup} is required to specify this; to follow from the example:
|\symup{g}|.
Maths style commands such as \cmd\symup\ are detailed later.

\paragraph{`Literal' interface}
Some may not like this convention of normalising their input.
For them, an upright |x| is an upright `x' and that's that.
(This will be the case when obtaining source text from copy/pasting PDF or
Microsoft Word documents, for example.)
For these users, the |literal| option to |math-style| will effect this behaviour.
The \cs{symliteral}\marg{syms} command can also be used, regardless of package setting, to force the style to match the literal input characters.
This is a `mirror' to \cs{symnormal}\marg{syms} (also alias \cs{mathnormal}) which `resets' the character mapping in its argument to that originally set up through package options.

\begin{table}
  \centering
  \topcaption{Effects of the \opt{math-style} package option.}
  \tablabel{math-style}
  \begin{tabular}{@{}>{\ttfamily}lcc@{}}
    \toprule
      & \multicolumn{2}{c}{Example} \\
       \cmidrule(l){2-3}
      \rmfamily Package option & Latin & Greek \\
    \midrule
      math-style=ISO & $(a,z,B,X)$ & $\symit{(\alpha,\beta,\Gamma,\Xi)}$ \\
      math-style=TeX & $(a,z,B,X)$ & $(\symit\alpha,\symit\beta,\symup\Gamma,\symup\Xi)$ \\
      math-style=french & $(a,z,\symup B,\symup X)$ & $(\symup\alpha,\symup\beta,\symup\Gamma,\symup\Xi)$ \\
      math-style=upright & $(\symup a,\symup z,\symup B,\symup X)$ & $(\symup\alpha,\symup\beta,\symup\Gamma,\symup\Xi)$ \\
    \bottomrule
  \end{tabular}
\end{table}


\subsection{Bold style}
\seclabel{bold-style}

Similar as in the previous section, ISO standards differ somewhat to \TeX's
conventions (and classical typesetting) for `boldness' in mathematics. In
the past, it has been customary to use bold \emph{upright} letters to denote
things like vectors and matrices. For example, \( \symbfup{M} =
(\mitM_x,\mitM_y,\mitM_z) \). Presumably, this was due to the relatively
scarcity of bold italic fonts in the pre-digital typesetting era.
It has been suggested by some that \emph{italic} bold symbols should be used nowadays instead, but this practise is certainly not widespread.

Bold Greek letters have simply been bold variant glyphs of their regular
weight, as in \( \mbfitxi = (\mitxi_\mitr,\mitxi_\mitphi,\mitxi_\mittheta)
\).
Confusingly, the syntax in \LaTeX\ traditionally has been different for obtaining `normal' bold symbols in Latin and Greek: \cmd\mathbf\ in the former (`$\symbfup{M}$'), and \cmd\bm\ (or
\cmd\boldsymbol, deprecated) in the latter (`$\mbfitxi$').

In \pkg{unicode-math}, the \cmd\symbf\ command works directly with both
Greek and Latin maths characters and depending on package option
either switches to upright for Latin letters (|bold-style=TeX|) as well or
keeps them italic (|bold-style=ISO|).
To match the package options for non-bold characters, with option
|bold-style=upright| all bold characters are upright, and
|bold-style=literal| does not change the upright/italic shape of the letter.
The \opt{bold-style} options' effects are shown in brief in \tabref{bold-style}.

Upright and italic bold mathematical letters input as direct Unicode
characters are normalised with the same rules. For example, with
|bold-style=TeX|, a literal bold italic latin character will be typeset
upright.

Note that \opt{bold-style} is independent of \opt{math-style}, although if
the former is not specified then matching defaults are chosen based on the
latter.

\begin{table}
  \centering
  \topcaption{Effects of the \opt{bold-style} package option.}
  \tablabel{bold-style}
  \begin{tabular}{@{}>{\ttfamily}lcc@{}}
    \toprule
      & \multicolumn{2}{c}{Example} \\
       \cmidrule(l){2-3}
      \rmfamily Package option & Latin & Greek \\
    \midrule
      bold-style=ISO & $(\symbfit a, \symbfit z, \symbfit B, \symbfit X)$ & $(\symbfit\alpha, \symbfit\beta, \symbfit\Gamma, \symbfit\Xi)$ \\
      bold-style=TeX & $(\symbfup a,\symbfup z,\symbfup B,\symbfup X)$ & $(\symbfit\alpha, \symbfit\beta,\symbfup \Gamma,\symbfup \Xi)$ \\
      bold-style=upright & $(\symbfup a,\symbfup z,\symbfup B,\symbfup X)$ & $(\symbfup \alpha,\symbfup \beta,\symbfup \Gamma,\symbfup \Xi)$ \\
    \bottomrule
  \end{tabular}
\end{table}


\subsection{Sans serif style}
\seclabel{sans-style}

Unicode contains upright and italic, medium and bold mathematical style characters.
These may be explicitly selected with the \cs{mathsfup}, \cs{mathsfit}, \cs{mathbfsfup}, and \cs{mathbfsfit}
commands discussed in \secref{all-math-alphabets}.

How should the generic \cs{mathsf} behave? Unlike bold, sans serif is used much more sparingly
in mathematics. I've seen recommendations to typeset tensors in sans serif italic
or sans serif italic bold (e.g., examples in the \pkg{isomath} and \pkg{mattens} packages).
But \LaTeX's \cs{mathsf} is \textsl{upright} sans serif.

Therefore I reluctantly add the package options |[sans-style=upright]| and |[sans-style=italic]| to control the behaviour of \cs{mathsf}.
The |upright| style sets up the command to use upright sans serif, including Greek;
the |italic| style switches to using italic in both Latin and Greek.
In other words, this option simply changes the meaning of \cs{mathsf} to either \cs{mathsfup} or \cs{mathsfit}, respectively.
Please let me know if more granular control is necessary here.

There is also a |[sans-style=literal]| setting, set automatically with |[math-style=literal]|, which retains the uprightness of the input characters used when selecting the sans serif output.

\subsubsection{What about bold sans serif?}

While you might want your bold upright and your sans serif italic, I don't believe you'd also want
your bold sans serif upright (or all vice versa, if that's even conceivable). Therefore, bold sans
serif follows from the setting for sans serif; it is completely independent of the setting for bold.

In other words, \cs{mathbfsf} is either \cs{mathbfsfup} or \cs{mathbfsfit} based on |[sans-style=upright]| or |[sans-style=italic]|, respectively. And \texttt{[sans-style = literal]} causes \cs{mathbfsf} to retain the same italic or upright shape as the input, and turns it bold sans serif.

N.B.: there is no medium-weight sans serif Greek range in Unicode.
Therefore, |\symsf{\alpha}| does not make sense (it produces `$\symsf{\alpha}$'), while |\symbfsf{\alpha}| gives `$\symbfsfup{\alpha}$' or `$\symbfsfit{\alpha}$' according to the |sans-style|.

\subsection{All (the rest) of the mathematical styles}
\seclabel{all-math-alphabets}

Unicode contains separate codepoints for most if not all variations of style
shape one may wish to use in mathematical notation. The complete list is shown
in \tabref{mathalphabets}. Some of these have been covered in the previous sections.

The math font switching commands do not nest; therefore if you want
sans serif bold, you must write |\symbfsf{...}| rather than |\symbf{\symsf{...}}|.
This may change in the future.

\begin{table}
\caption{Mathematical styles defined in Unicode. Black dots indicate an style exists in the font specified; blue dots indicate shapes that should always be taken from the upright font even in the italic style. See main text for description of \cs{mathbbit}.}
\tablabel{mathalphabets}
\centering
\def\Y{\textbullet}
\def\M{\textcolor[rgb]{0.5,0.5,1}{\textbullet}}
\begin{tabular}{@{} lll l ccc @{}}
\toprule
\multicolumn{3}{c}{Font} & & \multicolumn{3}{c}{Alphabet} \\
\cmidrule(r){1-3}
\cmidrule(l){5-7}
Style & Shape & Series & Switch & Latin & Greek & Numerals \\
\midrule
Serif      & Upright & Normal & \cs{mathup}     & \Y & \Y & \Y  \\
           &         & Bold   & \cs{mathbfup}   & \Y & \Y & \Y  \\
           & Italic  & Normal & \cs{mathit}     & \Y & \Y & \M  \\
           &         & Bold   & \cs{mathbfit}   & \Y & \Y & \M  \\
Sans serif & Upright & Normal & \cs{mathsfup}   & \Y &    & \Y  \\
           & Italic  & Normal & \cs{mathsfit}   & \Y &    & \M  \\
           & Upright & Bold   & \cs{mathbfsfup} & \Y & \Y & \Y  \\
           & Italic  & Bold   & \cs{mathbfsfit} & \Y & \Y & \M  \\
Typewriter & Upright & Normal & \cs{mathtt}     & \Y &    & \Y  \\
Double-struck & Upright & Normal & \cs{mathbb}     & \Y &    & \Y  \\
              & Italic  & Normal & \cs{mathbbit}   & \Y &    &  \\
Script     & Upright & Normal & \cs{mathscr}    & \Y &    &     \\
           &         & Bold   & \cs{mathbfscr}  & \Y &    &     \\
Fraktur    & Upright & Normal & \cs{mathfrak}   & \Y &    &     \\
           &         & Bold   & \cs{mathbffrac} & \Y &    &     \\
\bottomrule
\end{tabular}
\end{table}

\subsubsection{Double-struck}

The double-struck style (also known as `blackboard bold') consists of
upright Latin letters $\{\symbb{a}$--$\symbb{z}$,$\symbb{A}$$\symbb{Z}\}$,
numerals $\symbb{0}$--$\symbb{9}$, summation symbol $\symbb\sum$, and four
Greek letters only: $\{\symbb{\gamma\pi\Gamma\Pi}\}$.

While |\symbb{\sum}| does produce a double-struck summation symbol,
its limits aren't properly aligned. Therefore,
either the literal character or the control sequence \cs{Bbbsum} are
recommended instead.

There are also five Latin \emph{italic} double-struck letters: $\symbbit{Ddeij}$.
These can be accessed (if not with their literal characters or control sequences)
with the \cs{mathbbit} style switch, but note that only those five letters
will give the expected output.

\subsubsection{Caligraphic vs.\ Script variants}

The Unicode maths encoding contains a style for `Script' letters,
and while by default \cs{mathcal} and \cs{mathscr}
are synonyms, there are some situations when a
separate `Caligraphic' style is needed as well.

If a font contains alternate glyphs for a separat caligraphic style,
they can be selected explicitly as shown below.
This feature is currently only supported by the XITS~Math font, where
the caligraphic letters are accessed with the same glyph slots as the
script letters but with the first stylistic set feature (|ss01|) applied.
\begin{verbatim}
  \setmathfont{xits-math.otf}[range={cal,bfcal},StylisticSet=1]
\end{verbatim}
An example is shown below.
\begin{quote}
\setmathfont{xits-math.otf}[range=scr]
\setmathfont{xits-math.otf}[range=cal,StylisticSet=1]
The Script style (\cs{mathscr}) in XITS Math is: $\symscr{ABCXYZ}$\par
The Caligraphic style (\cs{mathcal}) in XITS Math is: $\symcal{ABCXYZ}$
\end{quote}


\subsection{Miscellanea}

\subsubsection{Nabla}
\seclabel{nabla}

 The symbol $\nabla$ comes in the six forms shown in \tabref{nabla}.
 We want an individual option to specify whether we want upright or italic
 nabla by default (when either upright or italic nabla is used in the
 source). \TeX\ classically uses an upright nabla, and \textsc{iso}
 standards agree with this convention.
 The package options |nabla=upright| and
 |nabla=italic| switch between the two choices, and |nabla=literal| respects
 the shape of the input character. This is then inherited
 through \cmd\symbf; \cmd\symit\ and \cmd\symup\ can be used to force one
 way or the other.

|nabla=italic| is the default. |nabla=literal| is
activated automatically after |math-style=literal|.

\begin{table}
  \begin{minipage}[b]{0.49\textwidth}
  \centering
  \topcaption{The various forms of nabla.}
  \tablabel{nabla}
  \let \tmpshow\empty
  \begin{tabular}{@{}llc@{}}
      \toprule
    \multicolumn{2}{@{}l}{Description} & Glyph
     \\ \cmidrule(r){1-2}\cmidrule(l){3-3}
    Upright & Serif & $\symup\nabla$ \\
    & Bold serif & $\symbfup\nabla$ \\
    & Bold sans & $\symbfsfup\nabla$ \\
      \cmidrule(lr){1-2}\cmidrule(lr){3-3}
    Italic & Serif & $\symit\nabla$ \\
    & Bold serif & $\symbfit\nabla$ \\
    & Bold sans  & $\symbfsfit\nabla$ \\
      \bottomrule
  \end{tabular}
  \end{minipage}\hfill
  \begin{minipage}[b]{0.49\textwidth}
  \centering
  \topcaption{The partial differential.}
  \tablabel{partial}
  \begin{tabular}{@{}llc@{}}
      \toprule
    \multicolumn{2}{@{}l}{Description} & Glyph
     \\ \cmidrule(r){1-2}\cmidrule(l){3-3}
    Regular   & Upright & $\symup\partial$ \\
              & Italic  & $\symit\partial$ \\
    Bold      & Upright & $\symbfup\partial$ \\
              & Italic  & $\symbfit\partial$ \\
    Sans bold & Upright & $\symbfsfup\partial$ \\
              & Italic  & $\symbfsfit\partial$ \\
      \bottomrule
  \end{tabular}
  \end{minipage}
\end{table}


\subsubsection{Partial}
\seclabel{partial}

The same applies to the symbols \unichar{2202} partial differential and
\unichar{1D715} math italic partial differential.

At time of writing, both the Cambria Math and STIX fonts display these
two glyphs in the same italic style, but this is hopefully a bug that will
be corrected in the future~--- the `plain' partial differential should
really have an upright shape.

Use the |partial=upright| or |partial=italic| package options to specify
which one you would like, or |partial=literal| to have the same character
used in the output as was used for the input.
The default is (always, unless someone requests and
argues otherwise) |partial=italic|.\footnote{A good argument would revolve
around some international standards body recommending upright over italic.
I just don't have the time right now to look it up.} |partial=literal|
is activated following |math-style=literal|.

See \tabref{partial} for the variations on the partial differential symbol.


\subsubsection{Primes}

Primes ($x'$) may be input in several ways. You may use any combination
the \ascii\ straight quote (\texttt{\char`\'}) or the Unicode prime \unichar{2032}
($'$); when multiple primes occur next to each other, they chain
together to form double, triple, or quadruple primes if the font contains
pre-drawn glyphs. The individual prime glyphs are accessed, as usual,
with the \cs{prime} command, and the double-, triple-, and quadruple-prime
glyphs are available with \cs{dprime}, \cs{trprime}, and \cs{qprime},
respectively.

If the font does not contain the pre-drawn glyphs or more than four primes
are used, the single prime glyph is used multiple times with a negative
kern to get the spacing right. There is no user interface to adjust this
negative kern yet (because I haven't decided what it should look like);
if you need to, write something like this:
\begin{Verbatim}
\ExplSyntaxOn
\muskip_gset:Nn \g_@@_primekern_muskip { -\thinmuskip/2 }
\ExplySyntaxOff
\end{Verbatim}
Backwards or reverse primes behave in exactly the same way; use the \ascii\
back tick (\texttt{\char`\`}) or the Unicode reverse prime \unichar{2035}
({\umfont\char"2035}).
The command to access the backprime is \cs{backprime}, and
multiple backwards primes can accessed with \cs{backdprime},
\cs{backtrprime}, and \cs{backqprime}.

In all cases above, no error checking is performed if you attempt to
access a multi-prime glyph in a font that doesn't contain one. For this
reason, it may be safer to write |x''''| instead of |x\qprime|
in general.

If you ever need to enter the straight quote |'| or the backtick |`| in
maths mode, these glyphs can be accessed with \cs{mathstraightquote} and
\cs{mathbacktick}.

\subsubsection{Unicode subscripts and superscripts}

You may, if you wish, use Unicode subscripts and superscripts in your
source document. For basic expressions, the use of these characters
can make the input more readable.
Adjacent sub- or super-scripts will be concatenated into a single
expression.

The range of subscripts and superscripts supported by this package
are shown in \figref{superscripts,subscripts}. Please request more if
you think it is appropriate.

\begin{figure}\centering
\fbox{\fontspec{Charis SIL}\Large
A
^^^^2070 ^^^^00b9 ^^^^00b2 ^^^^00b3 ^^^^2074 ^^^^2075 ^^^^2076 ^^^^2077
^^^^2078 ^^^^2079 ^^^^207a ^^^^207b ^^^^207c ^^^^207d ^^^^207e ^^^^2071
^^^^207f ^^^^207f ^^^^02b0 ^^^^02b2 ^^^^02b3 ^^^^02b7 ^^^^02b8
Z}
\caption{
  The Unicode superscripts supported as input characters.
  These are the literal glyphs from Charis SIL,
  not the output seen when used for maths input.
  The `A' and `Z' are to provide context for the size and
  location of the superscript glyphs.
}
\figlabel{superscripts}
\end{figure}

\begin{figure}\centering
\fbox{\fontspec{Charis SIL}\Large
A
^^^^2080 ^^^^2081 ^^^^2082 ^^^^2083 ^^^^2084 ^^^^2085 ^^^^2086 ^^^^2087
^^^^2088 ^^^^2089 ^^^^208a ^^^^208b ^^^^208c ^^^^208d ^^^^208e ^^^^2090
^^^^2091 ^^^^1d62 ^^^^2092 ^^^^1d63 ^^^^1d64 ^^^^1d65 ^^^^2093 ^^^^1d66
^^^^1d67 ^^^^1d68 ^^^^1d69 ^^^^1d6a
Z}
\caption{
  The Unicode subscripts supported as input characters.
  See note from \figref{superscripts}.
}
\figlabel{subscripts}
\end{figure}

\subsubsection{Colon}
\seclabel{colon}

The colon is one of the few confusing characters of Unicode maths.
In \TeX, \texttt{:} is defined as a colon with relation spacing: `$a:b$'.
While \cs{colon} is defined as a colon with punctuation spacing: `$a\colon b$'.

In Unicode, \unichar{003A} {colon} is defined as a punctuation symbol,
while \unichar{2236} {ratio} is the colon-like symbol used in mathematics to denote
ratios and other things.

This breaks the usual straightforward mapping from control sequence to Unicode input character
to (the same) Unicode glyph.

To preserve input compatibility, we remap the \ascii\ input character `\texttt{:}' to \unichar{2236}.
Typing a literal \unichar{2236} char will result in the same output.
If \pkg{amsmath} is loaded, then the definition of \cs{colon} is inherited from there
(it looks like a punctuation colon with additional space around it).
Otherwise, \cs{colon} is made to output a colon with \cs{mathpunct} spacing.

The package option |colon=literal| forces \ascii\ input `|:|' to be printed as \cs{mathcolon} instead.


\subsubsection{Slashes and backslashes}
\seclabel{slash-delimiter}

There are several slash-like symbols defined in Unicode. The complete list is shown in \tabref{slashes}.

\begin{table}\centering
\caption{Slashes and backslashes.}
\tablabel{slashes}
\begin{tabular}{@{}cl@{}cl@{}}
\toprule
Slot & Name & Glyph & Command  \\
\midrule
\unichar{002F} & \textsc{solidus}                 & \umfont \char"002F & \cs{slash} \\
\unichar{2044} & \textsc{fraction slash}          & \umfont \char"2044 & \cs{fracslash} \\
\unichar{2215} & \textsc{division slash}          & \umfont \char"2215 & \cs{divslash} \\
\unichar{29F8} & \textsc{big solidus}             & \umfont \char"29F8 & \cs{xsol} \\
\midrule
\unichar{005C} & \textsc{reverse solidus}         & \umfont \char"005C & \cs{backslash} \\
\unichar{2216} & \textsc{set minus}               & \umfont \char"2216 & \cs{smallsetminus} \\
\unichar{29F5} & \textsc{reverse solidus operator}& \umfont \char"29F5 & \cs{setminus} \\
\unichar{29F9} & \textsc{big reverse solidus}     & \umfont \char"29F9 & \cs{xbsol} \\
\bottomrule
\end{tabular}
\end{table}

In regular \LaTeX\ we can write \cs{left}\cs{slash}\dots\cs{right}\cs{backslash}
and so on and obtain extensible delimiter-like symbols. Not all of the Unicode slashes
are suitable for this (and do not have the font support to do it).

\paragraph{Slash}

Of \unichar{2044} {fraction slash}, TR25 says that it is:
\begin{quote}
\dots used to build up simple fractions in running text\dots
however parsers of mathematical texts should be prepared to handle fraction slash
when it is received from other sources.
\end{quote}

\unichar{2215} {division slash} should be used when division is represented
without a built-up fraction; $\pi\approx22/7$, for example.

\unichar{29F8} {big solidus} is a `big operator' (like $\sum$).

\paragraph{Backslash}

The \unichar{005C} {reverse solidus} character \cs{backslash} is used for denoting
double cosets: $A\backslash B$. (So I'm led to believe.)
It may be used as a `stretchy' delimiter if supported by the font.

MathML uses \unichar{2216} {set minus} like this: $A\smallsetminus B$.\footnote{\S4.4.5.11 \url{http://www.w3.org/TR/MathML3/}}
The \LaTeX\ command name \cs{smallsetminus} is used for backwards compatibility.

Presumably, \unichar{29F5} {reverse solidus operator} is intended to
be used in a similar way, but it could also (perhaps?) be used to
represent `inverse division': $\pi\approx7\mathbin{\backslash}22$.^^A
\footnote{This is valid syntax in the Octave and Matlab programming languages,
in which it means matrix inverse pre-multiplication. I.e., $A\mathbin{\backslash} B\equiv A^{-1}B$.}
The \LaTeX\ name for this character is \cs{setminus}.

Finally, \unichar{29F9} {big reverse solidus} is a `big operator' (like $\sum$).

\paragraph{How to use all of these things}

Unfortunately, font support for the above characters/glyphs is rather inconsistent.
In Cambria Math, the only slash that grows (say when writing
\[
\left.\left[\begin{array}{cc} a & b \\ c & d\end{array}\right]\middle\slash
      \left[\begin{array}{cc} 1 & 1 \\ 1 & 0\end{array}\right] \right.\quad )
\]
is the \textsc{fraction slash}, which we just established above is
sort of only supposed to be used in text.

Of the above characters, the following are allowed to be used after
\cs{left}, \cs{middle}, and \cs{right}:
\begin{itemize}
\item \cs{fracslash};
\item \cs{slash}; and,
\item \cs{backslash} (the only reverse slash).
\end{itemize}

However, we assume that there is only \emph{one} stretchy slash
in the font; this is assumed by default to be \unichar{002F} {solidus}.
Writing \cs{left/} or \cs{left}\cs{slash} or \cs{left}\cs{fracslash}
will all result in the same stretchy delimiter being used.

The delimiter used can be changed with the |slash-delimiter| package option.
Allowed values are |ascii|, |frac|, and |div|, corresponding to the respective
Unicode slots.

For example: as mentioned above, Cambria Math's stretchy slash is
\unichar{2044} {fraction slash}. When using Cambria Math, then
\pkg{unicode-math} should be loaded with the |slash-delimiter=frac| option.
(This should be a font option rather than a package option, but
it will change soon.)


\subsubsection{Growing and non-growing accents}
\seclabel{growing-accents}

There are a few accents for which \TeX\ has both non-growing and growing
versions.  Among these are \cs{hat} and \cs{tilde}; the corresponding growing
versions are called \cs{widehat} and \cs{widetilde}, respectively.

Older versions of \XeTeX\ and \LuaTeX\ did not support this distinction,
however, and \emph{all} accents there were growing automatically. (I.e.,
\cs{hat} and \cs{widehat} are equivalent.) As of \LuaTeX\ v0.65 and \XeTeX\
v0.9998, these wide/non-wide commands will again behave in their expected
manner.


\subsubsection{Pre-drawn fraction characters}

Pre-drawn fractions \unichar{00BC}--\unichar{00BE}, \unichar{2150}--\unichar{215E}
are not suitable for use in mathematics output. However, they can be useful
as input characters to abbreviate common fractions.
\begin{center}
\fontspec{DejaVuSerif.ttf} ^^A available in TeX Live 2012 if not earlier
¼ ½ ¾  ↉ ⅐ ⅑ ⅒ ⅓ ⅔ ⅕ ⅖ ⅗ ⅘ ⅙ ⅚ ⅛ ⅜ ⅝ ⅞
\end{center}
For example, instead of writing `|\tfrac12 x|', you may consider it more readable to have
`|½x|' in the source instead.

If the \cs{tfrac} command exists (i.e., if \pkg{amsmath} is loaded or
you have specially defined \cs{tfrac} for this purpose), it will be used
to typeset the fractions. If not, regular \cs{frac} will be used. The command
to use (\cs{tfrac} or \cs{frac}) can be forced either way with the package
option |active-frac=small| or |active-frac=normalsize|, respectively.

\subsubsection{Circles}

Unicode defines a large number of different types of circles for a variety
of mathematical purposes. There are thirteen alone just considering the
all white and all black ones, shown in \tabref{circles}.

\LaTeX\ defines considerably fewer: \cs{circ} and \cs{bigcirc} for white;
\cs{bullet} for black. This package maps those commands to \cs{vysmwhtcircle},
\cs{mdlgwhtcircle}, and \cs{smblkcircle}, respectively.

\begin{table}\centering
\def\showchar#1#2#3{ \textsc{u}+{\small\ttfamily #1} & \texttt{\string#3} & \umfont \char"#1 \\}
\begin{tabular}{@{}llc@{}}
\toprule
Slot & Command & Glyph \\
\midrule
\showchar{00B7}{centerdot}{\cdotp}
\showchar{22C5}{small middle dot}{\cdot}
\showchar{2219}{bullet operator}{\vysmblkcircle}
\showchar{2022}{round bullet, filled}{\smblkcircle}
\showchar{2981}{z notation spot}{\mdsmblkcircle}
\showchar{26AB}{medium black circle}{\mdblkcircle}
\showchar{25CF}{circle, filled}{\mdlgblkcircle}
\showchar{2B24}{black large circle}{\lgblkcircle}
\bottomrule
\end{tabular}
\def\showchar#1#2#3{ \umfont \char"#1 & \texttt{\string#3} & \textsc{u}+{\small\ttfamily #1} \\}
\begin{tabular}{@{}cll@{}}
\toprule
Glyph & Command & Slot \\
\midrule
\\
\\
\showchar{2218}{composite function (small circle)}{\vysmwhtcircle}
\showchar{25E6}{white bullet}{\smwhtcircle}
\showchar{26AC}{medium small white circle}{\mdsmwhtcircle}
\showchar{26AA}{medium white circle}{\mdwhtcircle}
\showchar{25CB}{large circle}{\mdlgwhtcircle}
\showchar{25EF}{large circle}{\lgwhtcircle}
\bottomrule
\end{tabular}
\caption{Filled and hollow Unicode circles.}
\tablabel{circles}
\end{table}

\subsubsection{Triangles}

While there aren't as many different sizes of triangle as there are circle,
there's some important distinctions to make between a few similar characters. See \tabref{uptriangles} for the full summary.

These triangles all have different intended meanings. Note for backwards
compatibility with \TeX, \unichar{25B3} has \emph{two} different mappings
in \pkg{unicode-math}. \cs{bigtriangleup} is intended as a binary operator
whereas \cs{triangle} is intended to be used as a letter-like symbol.

But you're better off if you're using the latter form to indicate an
increment to use the glyph intended for this purpose, \unichar{2206}: $\increment x$.

Finally, given that $\triangle$ and $\increment$ are provided for you
already, it is better off to only use upright Greek Delta $\Delta$ if you're
actually using it as a symbolic entity such as a variable on its own.

\begin{table}\centering
\begin{tabular}{@{}llcl@{}}
\toprule
Slot & Command & Glyph & Class \\
\midrule
\unichar{25B5} & \cs{vartriangle}      & \umfont \char"25B5 & binary \\
\unichar{25B3} & \cs{bigtriangleup}    & \umfont \char"25B3 & binary \\
\unichar{25B3} & \cs{triangle}         & \umfont \char"25B3 & ordinary \\
\unichar{2206} & \cs{increment}        & \umfont \char"2206 & ordinary \\
\unichar{0394} & \cs{mathup}\cs{Delta} & \umfont \char"0394 & ordinary \\
\bottomrule
\end{tabular}
\caption{Different upwards pointing triangles.}
\tablabel{uptriangles}
\end{table}

\iffalse
\subsubsection{Normalising some input characters}

I believe
all variant forms should be used as legal input that is normalised to
a consistent output glyph, because we want to be fault-tolerant in the input.
Here are the duplicates:
\begin{quote}\obeylines
\unichar {251} {latin small letter alpha}
\unichar {25B} {latin small letter epsilon}
\unichar {263} {latin small letter gamma}
\unichar {269} {latin small letter iota}
\unichar {278} {latin small letter phi}
\unichar {28A} {latin small letter upsilon}
\unichar {190} {latin capital letter epsilon}
\unichar {194} {latin capital letter gamma}
\unichar {196} {latin capital letter iota}
\unichar {1B1} {latin capital letter upsilon}
\end{quote}

(Not yet implemented.)
\fi

\section{Advanced}

\subsection{Warning messages}

This package can produce a number of informational messages to try and inform the user when something might be going wrong due to package conflicts or something else.
As an experimental feature, these can be turn off on an individual basis with the package option |warnings-off| which takes a comma-separated list of warnings to suppress.
A warning will give you its name when printed on the console output; e.g.,
\begin{Verbatim}
  * unicode-math warning: "mathtools-colon"
  *
  * ... <warning message> ...
\end{Verbatim}
This warning could be suppressed by loading the package as follows:
\begin{Verbatim}
  \usepackage[warnings-off={mathtools-colon}]{unicode-math}
\end{Verbatim}

\subsection{Programmer's interface}

(Tentative and under construction.)
If you are writing some code that needs to know the current
maths style (\cs{mathbf}, \cs{mathit}, etc.), you can query the
variable \cs{l_@@_mathstyle_tl}. It will contain the maths style
without the leading `math' string; for example,
|\symbf { \show \l_@@_mathstyle_tl }|
will produce `bf'.

\StopEventually{\end{document}}

\clearpage
\appendix

\section{\STIX\ table data extraction}\label{part:awk}

The source for the \TeX\ names for the very large number of mathematical
glyphs are provided via Barbara Beeton's table file for the \STIX\ project
(|ams.org/STIX|). A version is located at
|http://www.ams.org/STIX/bnb/stix-tbl.asc|
but check |http://www.ams.org/STIX/| for more up-to-date info.

This table is converted into a form suitable for reading by \TeX.
A single file is produced containing all (more than 3298) symbols.
Future optimisations might include generating various (possibly overlapping) subsets
so not all definitions must be read just to redefine a small range of symbols.
Performance for now seems to be acceptable without such measures.

This file is currently developed outside this DTX file. It will be
incorporated when the final version is ready. (I know this is not how
things are supposed to work!)


\section{Documenting maths support in the NFSS}

In the following, \meta{NFSS decl.} stands for something like |{T1}{lmr}{m}{n}|.

\begin{description}
\item[Maths symbol fonts] Fonts for symbols: $\propto$, $\leq$, $\rightarrow$

\cmd\DeclareSymbolFont\marg{name}\meta{NFSS decl.}\\
Declares a named maths font such as |operators| from which symbols are defined with \cmd\DeclareMathSymbol.

\item[Maths alphabet fonts] Fonts for {\font\1=cmmi10 at 10pt\1 ABC}\,–\,{\font\1=cmmi10 at 10pt\1 xyz}, {\font\1=eufm10 at 10pt\1 ABC}\,–\,{\font\1=cmsy10 at 10pt\1 XYZ}, etc.

\cmd\DeclareMathAlphabet\marg{cmd}\meta{NFSS decl.}

For commands such as \cmd\mathbf, accessed
through maths mode that are unaffected by the current text font, and which are used for
alphabetic symbols in the \ascii\ range.

\cmd\DeclareSymbolFontAlphabet\marg{cmd}\marg{name}

Alternative (and optimisation) for \cmd\DeclareMathAlphabet\ if a single font is being used
for both alphabetic characters (as above) and symbols.

\item[Maths `versions'] Different maths weights can be defined with the following, switched
in text with the \cmd\mathversion\marg{maths version} command.

\cmd\SetSymbolFont\marg{name}\marg{maths version}\meta{NFSS decl.}\\
\cmd\SetMathAlphabet\marg{cmd}\marg{maths version}\meta{NFSS decl.}

\item[Maths symbols] Symbol definitions in maths for both characters (=) and macros (\cmd\eqdef):
\cmd\DeclareMathSymbol\marg{symbol}\marg{type}\marg{named font}\marg{slot}
This is the macro that actually defines which font each symbol comes from and how they behave.
\end{description}
Delimiters and radicals use wrappers around \TeX's \cmd\delimiter/\cmd\radical\ primitives,
which are re-designed in \XeTeX. The syntax used in \LaTeX's NFSS is therefore not so relevant here.
\begin{description}
\item[Delimiters] A special class of maths symbol which enlarge themselves in certain contexts.

\cmd\DeclareMathDelimiter\marg{symbol}\marg{type}\marg{sym.\ font}\marg{slot}\marg{sym.\ font}\marg{slot}

\item[Radicals] Similar to delimiters (\cmd\DeclareMathRadical\ takes the same syntax) but
behave `weirdly'.
\end{description}
In those cases, glyph slots in \emph{two} symbol fonts are required; one for the small (`regular') case,
the other for situations when the glyph is larger. This is not the case in \XeTeX.

Accents are not included yet.

\paragraph{Summary}

For symbols, something like:
\begin{Verbatim}
\def\DeclareMathSymbol#1#2#3#4{
  \global\mathchardef#1"\mathchar@type#2
    \expandafter\hexnumber@\csname sym#2\endcsname
    {\hexnumber@{\count\z@}\hexnumber@{\count\tw@}}}
\end{Verbatim}
For characters, something like:
\begin{Verbatim}
\def\DeclareMathSymbol#1#2#3#4{
  \global\mathcode`#1"\mathchar@type#2
    \expandafter\hexnumber@\csname sym#2\endcsname
    {\hexnumber@{\count\z@}\hexnumber@{\count\tw@}}}
\end{Verbatim}

\section{Legacy \TeX\ font dimensions}

\centerline{%
\begin{tabular}[t]{@{}lp{4cm}@{}}
\toprule
\multicolumn{2}{@{}c@{}}{Text fonts} \\
\midrule
$\phi_1$ & slant per pt                \\
$\phi_2$ & interword space             \\
$\phi_3$ & interword stretch           \\
$\phi_4$ & interword shrink            \\
$\phi_5$ & x-height                    \\
$\phi_6$ & quad width                  \\
$\phi_7$ & extra space                 \\
$\phi_8$ & cap height (\XeTeX\ only)   \\
\bottomrule
\end{tabular}
\quad
\begin{tabular}[t]{@{}lp{4cm}@{}}
\toprule
\multicolumn{2}{@{}c@{}}{Maths font, \cs{fam}2} \\
\midrule
$\sigma_5$    & x height                    \\
$\sigma_6$    & quad                        \\
$\sigma_8$    & num1                        \\
$\sigma_9$    & num2                        \\
$\sigma_{10}$ & num3                        \\
$\sigma_{11}$ & denom1                      \\
$\sigma_{12}$ & denom2                      \\
$\sigma_{13}$ & sup1                        \\
$\sigma_{14}$ & sup2                        \\
$\sigma_{15}$ & sup3                        \\
$\sigma_{16}$ & sub1                        \\
$\sigma_{17}$ & sub2                        \\
$\sigma_{18}$ & sup drop                    \\
$\sigma_{19}$ & sub drop                    \\
$\sigma_{20}$ & delim1                      \\
$\sigma_{21}$ & delim2                      \\
$\sigma_{22}$ & axis height                 \\
\bottomrule
\end{tabular}
\quad
\begin{tabular}[t]{@{}lp{4cm}@{}}
\toprule
\multicolumn{2}{@{}c@{}}{Maths font, \cs{fam}3} \\
\midrule
$\xi_8$    & default rule thickness      \\
$\xi_9$    & big op spacing1             \\
$\xi_{10}$ & big op spacing2             \\
$\xi_{11}$ & big op spacing3             \\
$\xi_{12}$ & big op spacing4             \\
$\xi_{13}$ & big op spacing5             \\
\bottomrule
\end{tabular}
}


\section{\Hologo{XeTeX} math font dimensions}

These are the extended \cmd\fontdimen s available for suitable fonts
in \XeTeX. Note that Lua\TeX\ takes an alternative route, and this package
will eventually provide a wrapper interface to the two (I hope).

\newcounter{mfdimen}
\setcounter{mfdimen}{9}
\newcommand\mathfontdimen[2]{^^A
  \stepcounter{mfdimen}^^A
  \themfdimen & {\scshape\small #1} & #2\vspace{0.5ex} \tabularnewline}

\begin{longtable}{
  @{}c>{\raggedright\parfillskip=0pt}p{4cm}>{\raggedright}p{7cm}@{}}
\toprule \cmd\fontdimen & Dimension name & Description\tabularnewline\midrule \endhead
\bottomrule\endfoot
\mathfontdimen{Script\-Percent\-Scale\-Down}
{Percentage of scaling down for script level 1. Suggested value: 80\%.}
\mathfontdimen{Script\-Script\-Percent\-Scale\-Down}
{Percentage of scaling down for script level 2 (Script\-Script). Suggested value: 60\%.}
\mathfontdimen{Delimited\-Sub\-Formula\-Min\-Height}
{Minimum height required for a delimited expression to be treated as a subformula. Suggested value: normal line height\,×\,1.5.}
\mathfontdimen{Display\-Operator\-Min\-Height}
{Minimum height of n-ary operators (such as integral and summation) for formulas in display mode.}
\mathfontdimen{Math\-Leading}
{White space to be left between math formulas to ensure proper line spacing. For example, for applications that treat line gap as a part of line ascender, formulas with ink  going above (os2.sTypoAscender + os2.sTypoLineGap – MathLeading) or with ink going below os2.sTypoDescender will result in increasing line height.}
\mathfontdimen{Axis\-Height}
{Axis height of the font. }
\mathfontdimen{Accent\-Base\-Height}
{Maximum (ink) height of accent base that does not require raising the accents. Suggested: x-height of the font (os2.sxHeight) plus any possible overshots. }
\mathfontdimen{Flattened\-Accent\-Base\-Height}
{Maximum (ink) height of accent base that does not require flattening the accents. Suggested: cap height of the font (os2.sCapHeight).}
\mathfontdimen{Subscript\-Shift\-Down}
{The standard shift down applied to subscript elements. Positive for moving in the downward direction. Suggested: os2.ySubscriptYOffset.}
\mathfontdimen{Subscript\-Top\-Max}
{Maximum allowed height of the (ink) top of subscripts that does not require moving subscripts further down. Suggested: /5 x-height.}
\mathfontdimen{Subscript\-Baseline\-Drop\-Min}
{Minimum allowed drop of the baseline of subscripts relative to the (ink) bottom of the base. Checked for bases that are treated as a box or extended shape. Positive for subscript baseline dropped below the base bottom.}
\mathfontdimen{Superscript\-Shift\-Up}
{Standard shift up applied to superscript elements. Suggested: os2.ySuperscriptYOffset.}
\mathfontdimen{Superscript\-Shift\-Up\-Cramped}
{Standard shift of superscripts relative to the base, in cramped style.}
\mathfontdimen{Superscript\-Bottom\-Min}
{Minimum allowed height of the (ink) bottom of superscripts that does not require moving subscripts further up. Suggested: ¼ x-height.}
\mathfontdimen{Superscript\-Baseline\-Drop\-Max}
{Maximum allowed drop of the baseline of superscripts relative to the (ink) top of the base. Checked for bases that are treated as a box or extended shape. Positive for superscript baseline below the base top.}
\mathfontdimen{Sub\-Superscript\-Gap\-Min}
{Minimum gap between the superscript and subscript ink. Suggested: 4×default rule thickness.}
\mathfontdimen{Superscript\-Bottom\-Max\-With\-Subscript}
{The maximum level to which the (ink) bottom of superscript can be pushed to increase the gap between superscript and subscript, before subscript starts being moved down.
Suggested: /5 x-height.}
\mathfontdimen{Space\-After\-Script}
{Extra white space to be added after each subscript and superscript. Suggested: 0.5pt for a 12 pt font.}
\mathfontdimen{Upper\-Limit\-Gap\-Min}
{Minimum gap between the (ink) bottom of the upper limit, and the (ink) top of the base operator. }
\mathfontdimen{Upper\-Limit\-Baseline\-Rise\-Min}
{Minimum distance between baseline of upper limit and (ink) top of the base operator.}
\mathfontdimen{Lower\-Limit\-Gap\-Min}
{Minimum gap between (ink) top of the lower limit, and (ink) bottom of the base operator.}
\mathfontdimen{Lower\-Limit\-Baseline\-Drop\-Min}
{Minimum distance between baseline of the lower limit and (ink) bottom of the base operator.}
\mathfontdimen{Stack\-Top\-Shift\-Up}
{Standard shift up applied to the top element of a stack.}
\mathfontdimen{Stack\-Top\-Display\-Style\-Shift\-Up}
{Standard shift up applied to the top element of a stack in display style.}
\mathfontdimen{Stack\-Bottom\-Shift\-Down}
{Standard shift down applied to the bottom element of a stack. Positive for moving in the downward direction.}
\mathfontdimen{Stack\-Bottom\-Display\-Style\-Shift\-Down}
{Standard shift down applied to the bottom element of a stack in display style. Positive for moving in the downward direction.}
\mathfontdimen{Stack\-Gap\-Min}
{Minimum gap between (ink) bottom of the top element of a stack, and the (ink) top of the bottom element. Suggested: 3×default rule thickness.}
\mathfontdimen{Stack\-Display\-Style\-Gap\-Min}
{Minimum gap between (ink) bottom of the top element of a stack, and the (ink) top of the bottom element in display style. Suggested: 7×default rule thickness.}
\mathfontdimen{Stretch\-Stack\-Top\-Shift\-Up}
{Standard shift up applied to the top element of the stretch stack.}
\mathfontdimen{Stretch\-Stack\-Bottom\-Shift\-Down}
{Standard shift down applied to the bottom element of the stretch stack. Positive for moving in the downward direction.}
\mathfontdimen{Stretch\-Stack\-Gap\-Above\-Min}
{Minimum gap between the ink of the stretched element, and the (ink) bottom of the element above. Suggested: Upper\-Limit\-Gap\-Min}
\mathfontdimen{Stretch\-Stack\-Gap\-Below\-Min}
{Minimum gap between the ink of the stretched element, and the (ink) top of the element below. Suggested: Lower\-Limit\-Gap\-Min.}
\mathfontdimen{Fraction\-Numerator\-Shift\-Up}
{Standard shift up applied to the numerator. }
\mathfontdimen{Fraction\-Numerator\-Display\-Style\-Shift\-Up}
{Standard shift up applied to the numerator in display style. Suggested: Stack\-Top\-Display\-Style\-Shift\-Up.}
\mathfontdimen{Fraction\-Denominator\-Shift\-Down}
{Standard shift down applied to the denominator. Positive for moving in the downward direction.}
\mathfontdimen{Fraction\-Denominator\-Display\-Style\-Shift\-Down}
{Standard shift down applied to the denominator in display style. Positive for moving in the downward direction. Suggested: Stack\-Bottom\-Display\-Style\-Shift\-Down.}
\mathfontdimen{Fraction\-Numerator\-Gap\-Min}
{Minimum tolerated gap between the (ink) bottom of the numerator and the ink of the fraction bar. Suggested: default rule thickness}
\mathfontdimen{Fraction\-Num\-Display\-Style\-Gap\-Min}
{Minimum tolerated gap between the (ink) bottom of the numerator and the ink of the fraction bar in display style. Suggested: 3×default rule thickness.}
\mathfontdimen{Fraction\-Rule\-Thickness}
{Thickness of the fraction bar. Suggested: default rule thickness.}
\mathfontdimen{Fraction\-Denominator\-Gap\-Min}
{Minimum tolerated gap between the (ink) top of the denominator and the ink of the fraction bar. Suggested: default rule thickness}
\mathfontdimen{Fraction\-Denom\-Display\-Style\-Gap\-Min}
{Minimum tolerated gap between the (ink) top of the denominator and the ink of the fraction bar in display style. Suggested: 3×default rule thickness.}
\mathfontdimen{Skewed\-Fraction\-Horizontal\-Gap}
{Horizontal distance between the top and bottom elements of a skewed fraction.}
\mathfontdimen{Skewed\-Fraction\-Vertical\-Gap}
{Vertical distance between the ink of the top and bottom elements of a skewed fraction.}
\mathfontdimen{Overbar\-Vertical\-Gap}
{Distance between the overbar and the (ink) top of he base. Suggested: 3×default rule thickness.}
\mathfontdimen{Overbar\-Rule\-Thickness}
{Thickness of overbar. Suggested: default rule thickness.}
\mathfontdimen{Overbar\-Extra\-Ascender}
{Extra white space reserved above the overbar. Suggested: default rule thickness.}
\mathfontdimen{Underbar\-Vertical\-Gap}
{Distance between underbar and (ink) bottom of the base. Suggested: 3×default rule thickness.}
\mathfontdimen{Underbar\-Rule\-Thickness}
{Thickness of underbar. Suggested: default rule thickness.}
\mathfontdimen{Underbar\-Extra\-Descender}
{Extra white space reserved below the underbar. Always positive. Suggested: default rule thickness.}
\mathfontdimen{Radical\-Vertical\-Gap}
{Space between the (ink) top of the expression and the bar over it. Suggested: 1¼ default rule thickness.}
\mathfontdimen{Radical\-Display\-Style\-Vertical\-Gap}
{Space between the (ink) top of the expression and the bar over it. Suggested: default rule thickness + ¼ x-height. }
\mathfontdimen{Radical\-Rule\-Thickness}
{Thickness of the radical rule. This is the thickness of the rule in designed or constructed radical signs. Suggested: default rule thickness.}
\mathfontdimen{Radical\-Extra\-Ascender}
{Extra white space reserved above the radical. Suggested: Radical\-Rule\-Thickness.}
\mathfontdimen{Radical\-Kern\-Before\-Degree}
{Extra horizontal kern before the degree of a radical, if such is present. Suggested: 5/18 of em.}
\mathfontdimen{Radical\-Kern\-After\-Degree}
{Negative kern after the degree of a radical, if such is present. Suggested: −10/18 of em.}
\mathfontdimen{Radical\-Degree\-Bottom\-Raise\-Percent}
{Height of the bottom of the radical degree, if such is present, in proportion to the ascender of the radical sign. Suggested: 60\%.}
\end{longtable}

\if \DOCUMENTEND T \end{document} \fi




\DocInput{unicode-math.dtx}
\DocInput{unicode-math-msg.dtx}
\DocInput{unicode-math-usv.dtx}
\DocInput{unicode-math-alphabets.dtx}
\DocInput{unicode-math-compat.dtx}
\end{document}
%</internal>
% \fi
%
% \clearpage
% \part{Package implementation}
% \parttoc
%
% The prefix for \pkg{unicode-math} is \texttt{um}:
%    \begin{macrocode}
%<@@=um>
%    \end{macrocode}
%
% \section{Header code}
%
% We (later on) bifurcate the package based on the engine being used.
% These separate package files are indicated with the Docstrip flags \textsf{LU} and \textsf{XE}, respectively.
% Shared code executed before loading the engine-specific code is indicated with the flag \textsf{preamble}.
%    \begin{macrocode}
%<*load>
\luatex_if_engine:T { \RequirePackage{unicode-math-luatex} }
\xetex_if_engine:T  { \RequirePackage{unicode-math-xetex}  }
%</load>
%    \end{macrocode}
% The shared part of the code starts here before the split above.
%    \begin{macrocode}
%<*preamble&!XE&!LU>
%    \end{macrocode}
%
% Bail early if using pdf\TeX.
%    \begin{macrocode}
\usepackage{ifxetex,ifluatex}
\ifxetex
  \ifdim\number\XeTeXversion\XeTeXrevision in<0.9998in%
    \PackageError{unicode-math}{%
      Cannot run with this version of XeTeX!\MessageBreak
      You need XeTeX 0.9998 or newer.%
    }\@ehd
  \fi
\else\ifluatex
  \ifnum\luatexversion<64%
    \PackageError{unicode-math}{%
      Cannot run with this version of LuaTeX!\MessageBreak
      You need LuaTeX 0.64 or newer.%
    }\@ehd
  \fi
\else
  \PackageError{unicode-math}{%
    Cannot be run with pdfLaTeX!\MessageBreak
    Use XeLaTeX or LuaLaTeX instead.%
  }\@ehd
\fi\fi
%    \end{macrocode}
%
% \paragraph{Packages}
%    \begin{macrocode}
\RequirePackage{expl3}[2015/03/01]
\RequirePackage{xparse}
\RequirePackage{l3keys2e}
\RequirePackage{fontspec}[2015/03/14]
\RequirePackage{catchfile}
\RequirePackage{fix-cm} % avoid some warnings
\RequirePackage{filehook}
%    \end{macrocode}
%
%    \begin{macrocode}
\ExplSyntaxOn
%    \end{macrocode}
%
% For \pkg{fontspec}:
%    \begin{macrocode}
\cs_generate_variant:Nn \fontspec_set_family:Nnn {Nx}
\cs_generate_variant:Nn \fontspec_set_fontface:NNnn {NNx}
%    \end{macrocode}
%
% \paragraph{Conditionals}
%
%    \begin{macrocode}
\bool_new:N \l_@@_ot_math_bool
\bool_new:N \l_@@_init_bool
\bool_new:N \l_@@_implicit_alph_bool
\bool_new:N \g_@@_mainfont_already_set_bool
%    \end{macrocode}
% For \opt{math-style}:
%    \begin{macrocode}
\bool_new:N \g_@@_literal_bool
\bool_new:N \g_@@_upLatin_bool
\bool_new:N \g_@@_uplatin_bool
\bool_new:N \g_@@_upGreek_bool
\bool_new:N \g_@@_upgreek_bool
%    \end{macrocode}
% For \opt{bold-style}:
%    \begin{macrocode}
\bool_new:N \g_@@_bfliteral_bool
\bool_new:N \g_@@_bfupLatin_bool
\bool_new:N \g_@@_bfuplatin_bool
\bool_new:N \g_@@_bfupGreek_bool
\bool_new:N \g_@@_bfupgreek_bool
%    \end{macrocode}
% For \opt{sans-style}:
%    \begin{macrocode}
\bool_new:N \g_@@_upsans_bool
\bool_new:N \g_@@_sfliteral_bool
%    \end{macrocode}
% For assorted package options:
%    \begin{macrocode}
\bool_new:N \g_@@_upNabla_bool
\bool_new:N \g_@@_uppartial_bool
\bool_new:N \g_@@_literal_Nabla_bool
\bool_new:N \g_@@_literal_partial_bool
\bool_new:N \g_@@_texgreek_bool
\bool_set_true:N \g_@@_texgreek_bool
\bool_new:N \l_@@_smallfrac_bool
\bool_new:N \g_@@_literal_colon_bool
\bool_new:N \g_@@_mathrm_text_bool
\bool_new:N \g_@@_mathit_text_bool
\bool_new:N \g_@@_mathbf_text_bool
\bool_new:N \g_@@_mathsf_text_bool
\bool_new:N \g_@@_mathtt_text_bool
%    \end{macrocode}
%
% \paragraph{Variables}
%    \begin{macrocode}
\int_new:N \g_@@_fam_int
%    \end{macrocode}
%
% For displaying in warning messages, etc.:
%    \begin{macrocode}
\tl_const:Nn \c_@@_math_alphabet_name_latin_tl {Latin,~lowercase}
\tl_const:Nn \c_@@_math_alphabet_name_Latin_tl {Latin,~uppercase}
\tl_const:Nn \c_@@_math_alphabet_name_greek_tl {Greek,~lowercase}
\tl_const:Nn \c_@@_math_alphabet_name_Greek_tl {Greek,~uppercase}
\tl_const:Nn \c_@@_math_alphabet_name_num_tl   {Numerals}
\tl_const:Nn \c_@@_math_alphabet_name_misc_tl  {Misc.}
%    \end{macrocode}
%
%    \begin{macrocode}
\tl_new:N \l_@@_mathstyle_tl
%    \end{macrocode}
%
% Used to store the font switch for the \cs{operator@font}.
%    \begin{macrocode}
\tl_new:N \g_@@_operator_mathfont_tl
%    \end{macrocode}
%
% Variables:
%    \begin{macrocode}
\seq_new:N \l_@@_missing_alph_seq
\seq_new:N \l_@@_mathalph_seq
\seq_new:N \l_@@_char_range_seq
\seq_new:N \l_@@_mclass_range_seq
\seq_new:N \l_@@_cmd_range_seq
%    \end{macrocode}
%
% \begin{macro}{\g_@@_mathclasses_seq}
% Every math class.
%    \begin{macrocode}
\seq_new:N \g_@@_mathclasses_seq
\seq_set_from_clist:Nn \g_@@_mathclasses_seq
  {
    \mathord,\mathalpha,\mathbin,\mathrel,\mathpunct,
     \mathop,
    \mathopen,\mathclose,
    \mathfence,\mathover,\mathunder,
     \mathaccent,\mathbotaccent,\mathaccentwide,\mathbotaccentwide
  }
%    \end{macrocode}
% \end{macro}
%

% \begin{macro}{\g_@@_default_mathalph_seq}
% This sequence stores the alphabets in each math style.
%    \begin{macrocode}
\seq_new:N \g_@@_default_mathalph_seq
%    \end{macrocode}
% \end{macro}
%
% \begin{macro}{\g_@@_mathstyles_seq}
% This is every `named range' and every `math style' known to \pkg{unicode-math}.
% A named range is such as "bfit" and "sfit", which are also math styles (with \cs{symbfit} and \cs{symsfit}).
% `Mathstyles' are a superset of named ranges and also include commands such as \cs{symbf} and \cs{symsf}.
%
% N.B. for parsing purposes `named ranges' are defined as strings!
%    \begin{macrocode}
\seq_new:N \g_@@_named_ranges_seq
\seq_new:N \g_@@_mathstyles_seq
%    \end{macrocode}
% \end{macro}
%
%    \begin{macrocode}
\muskip_new:N \g_@@_primekern_muskip
\muskip_gset:Nn \g_@@_primekern_muskip { -\thinmuskip/2 }% arbitrary
\int_new:N \l_@@_primecount_int
\prop_new:N \g_@@_supers_prop
\prop_new:N \g_@@_subs_prop
\tl_new:N \l_not_token_name_tl
%    \end{macrocode}
%
% \subsection{Extras}
%
% What might end up being provided by the kernel.
%
% \begin{macro}{\@@_glyph_if_exist:nTF}
%: TODO: Generalise for arbitrary fonts! \cs{l_@@_font} is not always the one used for a specific glyph!!
%    \begin{macrocode}
\prg_new_conditional:Nnn \@@_glyph_if_exist:n {p,TF,T,F}
 {
  \etex_iffontchar:D \l_@@_font #1 \scan_stop:
    \prg_return_true:
  \else:
    \prg_return_false:
  \fi:
 }
%    \end{macrocode}
% \end{macro}
%
% \begin{macro}{\@@_set_mathcode:nnnn}
% \begin{macro}{\@@_set_mathcode:nnn}
% \begin{macro}{\@@_set_mathchar:NNnn}
% \begin{macro}{\@@_set_mathchar:cNnn}
% \begin{macro}{\@@_set_delcode:nnn}
% \begin{macro}{\@@_radical:nn}
% \begin{macro}{\@@_delimiter:Nnn}
% \begin{macro}{\@@_accent:nnn}
% \begin{macro}{\@@_accent_keyword:}
% These are all wrappers for the primitive commands that take numerical
% input only.
%    \begin{macrocode}
\cs_set:Npn \@@_set_mathcode:nnnn #1#2#3#4 {
  \Umathcode \int_eval:n {#1} =
    \mathchar@type#2 \csname sym#3\endcsname \int_eval:n {#4} \scan_stop:
}
\cs_set:Npn \@@_set_mathcode:nnn #1#2#3 {
  \Umathcode \int_eval:n {#1} =
    \mathchar@type#2 \csname sym#3\endcsname \int_eval:n {#1} \scan_stop:
}
\cs_set:Npn \@@_set_mathchar:NNnn #1#2#3#4 {
  \Umathchardef #1 =
    \mathchar@type#2 \csname sym#3\endcsname \int_eval:n {#4} \scan_stop:
}
\cs_new:Nn \@@_set_delcode:nnn {
  \Udelcode#2 = \csname sym#1\endcsname #3 \scan_stop:
}
\cs_new:Nn \@@_radical:nn {
  \Uradical \csname sym#1\endcsname #2 \scan_stop:
}
\cs_new:Nn \@@_delimiter:Nnn {
  \Udelimiter \mathchar@type#1 \csname sym#2\endcsname #3 \scan_stop:
}
\cs_new:Nn \@@_accent:nnn {
  \Umathaccent #1~ \mathchar@type\mathaccent \use:c { sym #2 } #3 \scan_stop:
}
%    \end{macrocode}
%
%    \begin{macrocode}
\cs_generate_variant:Nn \@@_set_mathchar:NNnn {c}
%    \end{macrocode}
% \end{macro}
% \end{macro}
% \end{macro}
% \end{macro}
% \end{macro}
% \end{macro}
% \end{macro}
% \end{macro}
% \end{macro}
%
%
% \begin{macro}{\@@_char_gmake_mathactive:N}
% \begin{macro}{\@@_char_gmake_mathactive:n}
%    \begin{macrocode}
\cs_new:Nn \@@_char_gmake_mathactive:N
 {
  \global\mathcode `#1 = "8000 \scan_stop:
 }
\cs_new:Nn \@@_char_gmake_mathactive:n
 {
  \global\mathcode #1 = "8000 \scan_stop:
 }
%    \end{macrocode}
% \end{macro}
% \end{macro}
%
% \subsection{Alphabet Unicode positions}
%
% Before we begin, let's define the positions of the various Unicode
% alphabets so that our code is a little more readable.\footnote{`\textsc{u.s.v.}' stands
% for `Unicode scalar value'.}
%
% Rather than `readable', in the end, this makes the code more extensible.
%    \begin{macrocode}
\cs_new:Nn \usv_set:nnn 
 { \tl_set:cn { g_@@_#1_#2_usv } {#3} }
\cs_new:Nn \@@_to_usv:nn
 { \use:c { g_@@_#1_#2_usv } }
\prg_new_conditional:Nnn \@@_usv_if_exist:nn {T,F,TF}
 {
  \cs_if_exist:cTF { g_@@_#1_#2_usv }
   \prg_return_true: \prg_return_false:
 }
%    \end{macrocode}
%
% \subsection{Package options}
%
% \begin{macro}{\unimathsetup}
% This macro can be used in lieu of or later to override
% options declared when the package is loaded.
%    \begin{macrocode}
\DeclareDocumentCommand \unimathsetup {m}
 { \keys_set:nn {unicode-math} {#1} }
%    \end{macrocode}
% \end{macro}
%
% \begin{macro}{\@@_keys_choices:nn}
% To simplify the creation of option keys, let's iterate in pairs rather than worry about equals signs and commas.
%    \begin{macrocode}
\cs_new:Nn \@@_keys_choices:nn
 {
  \cs_set:Npn \@@_keys_choices_fn:nn { \@@_keys_choices_aux:nnn {#1} }
  \use:x
   {
    \exp_not:N \keys_define:nn {unicode-math}
     {
      #1 .choice: ,
      \@@_tl_map_dbl:nN {#2} \@@_keys_choices_fn:nn
     }
   }
 }
\cs_new:Nn \@@_keys_choices_aux:nnn { #1 / #2 .code:n = { \exp_not:n {#3} } , }

\cs_new:Nn \@@_tl_map_dbl:nN
  {
    \__@@_tl_map_dbl:Nnn #2 #1 \q_recursion_tail {}{} \q_recursion_stop
  }
\cs_new:Nn \__@@_tl_map_dbl:Nnn
  {
    \quark_if_recursion_tail_stop:n {#2}
    \quark_if_recursion_tail_stop:n {#3}
    #1 {#2} {#3}
    \__@@_tl_map_dbl:Nnn #1
 }
%    \end{macrocode}
% \end{macro}
%
% \paragraph{Compatibility}
%    \begin{macrocode}
\@@_keys_choices:nn {mathup}
 {
  {sym}  { \bool_set_false:N \g_@@_mathrm_text_bool }
  {text} { \bool_set_true:N  \g_@@_mathrm_text_bool }
 }
\@@_keys_choices:nn {mathrm}
 {
  {sym}  { \bool_set_false:N \g_@@_mathrm_text_bool }
  {text} { \bool_set_true:N  \g_@@_mathrm_text_bool }
 }
\@@_keys_choices:nn {mathit}
 {
  {sym}  { \bool_set_false:N \g_@@_mathit_text_bool }
  {text} { \bool_set_true:N  \g_@@_mathit_text_bool }
 }
\@@_keys_choices:nn {mathbf}
 {
  {sym}  { \bool_set_false:N \g_@@_mathbf_text_bool }
  {text} { \bool_set_true:N  \g_@@_mathbf_text_bool }
 }
\@@_keys_choices:nn {mathsf}
 {
  {sym}  { \bool_set_false:N \g_@@_mathsf_text_bool }
  {text} { \bool_set_true:N  \g_@@_mathsf_text_bool }
 }
\@@_keys_choices:nn {mathtt}
 {
  {sym}  { \bool_set_false:N \g_@@_mathtt_text_bool }
  {text} { \bool_set_true:N  \g_@@_mathtt_text_bool }
 }
%    \end{macrocode}
%
% \paragraph{math-style}
%    \begin{macrocode}
\@@_keys_choices:nn {normal-style}
 {
       {ISO} {
              \bool_set_false:N \g_@@_literal_bool
              \bool_set_false:N \g_@@_upGreek_bool
              \bool_set_false:N \g_@@_upgreek_bool
              \bool_set_false:N \g_@@_upLatin_bool
              \bool_set_false:N \g_@@_uplatin_bool
             }
       {TeX} {
              \bool_set_false:N \g_@@_literal_bool
              \bool_set_true:N  \g_@@_upGreek_bool
              \bool_set_false:N \g_@@_upgreek_bool
              \bool_set_false:N \g_@@_upLatin_bool
              \bool_set_false:N \g_@@_uplatin_bool
             }
    {french} {
              \bool_set_false:N \g_@@_literal_bool
              \bool_set_true:N  \g_@@_upGreek_bool
              \bool_set_true:N  \g_@@_upgreek_bool
              \bool_set_true:N  \g_@@_upLatin_bool
              \bool_set_false:N \g_@@_uplatin_bool
             }
   {upright} {
              \bool_set_false:N \g_@@_literal_bool
              \bool_set_true:N  \g_@@_upGreek_bool
              \bool_set_true:N  \g_@@_upgreek_bool
              \bool_set_true:N  \g_@@_upLatin_bool
              \bool_set_true:N  \g_@@_uplatin_bool
             }
   {literal} {
              \bool_set_true:N  \g_@@_literal_bool
             }
 }
%    \end{macrocode}
%
%    \begin{macrocode}
\@@_keys_choices:nn {math-style}
 {
      {ISO} {
             \unimathsetup { nabla=upright, partial=italic,
              normal-style=ISO, bold-style=ISO, sans-style=italic }
            }
      {TeX} {
             \unimathsetup { nabla=upright, partial=italic,
               normal-style=TeX, bold-style=TeX, sans-style=upright }
            }
   {french} {
             \unimathsetup { nabla=upright, partial=upright,
               normal-style=french, bold-style=upright, sans-style=upright }
            }
  {upright} {
             \unimathsetup { nabla=upright, partial=upright,
               normal-style=upright, bold-style=upright, sans-style=upright }
            }
  {literal} {
             \unimathsetup { colon=literal, nabla=literal, partial=literal,
               normal-style=literal, bold-style=literal, sans-style=literal }
            }
 }
%    \end{macrocode}
%
% \paragraph{bold-style}
%    \begin{macrocode}
\@@_keys_choices:nn {bold-style}
 {
      {ISO} {
             \bool_set_false:N \g_@@_bfliteral_bool
             \bool_set_false:N \g_@@_bfupGreek_bool
             \bool_set_false:N \g_@@_bfupgreek_bool
             \bool_set_false:N \g_@@_bfupLatin_bool
             \bool_set_false:N \g_@@_bfuplatin_bool
            }
      {TeX} {
             \bool_set_false:N \g_@@_bfliteral_bool
             \bool_set_true:N  \g_@@_bfupGreek_bool
             \bool_set_false:N \g_@@_bfupgreek_bool
             \bool_set_true:N  \g_@@_bfupLatin_bool
             \bool_set_true:N  \g_@@_bfuplatin_bool
            }
  {upright} {
             \bool_set_false:N \g_@@_bfliteral_bool
             \bool_set_true:N  \g_@@_bfupGreek_bool
             \bool_set_true:N  \g_@@_bfupgreek_bool
             \bool_set_true:N  \g_@@_bfupLatin_bool
             \bool_set_true:N  \g_@@_bfuplatin_bool
            }
  {literal} {
             \bool_set_true:N  \g_@@_bfliteral_bool
            }
 }
%    \end{macrocode}
%
% \paragraph{sans-style}
%    \begin{macrocode}
\@@_keys_choices:nn {sans-style}
 {
  {italic}  { \bool_set_false:N \g_@@_upsans_bool    }
  {upright} { \bool_set_true:N  \g_@@_upsans_bool    }
  {literal} { \bool_set_true:N  \g_@@_sfliteral_bool }
 }
%    \end{macrocode}
%
%
% \paragraph{Nabla and partial}
%    \begin{macrocode}
\@@_keys_choices:nn {nabla}
 {
  {upright} {
              \bool_set_false:N \g_@@_literal_Nabla_bool
              \bool_set_true:N  \g_@@_upNabla_bool
            }
  {italic}  {
              \bool_set_false:N \g_@@_literal_Nabla_bool
              \bool_set_false:N \g_@@_upNabla_bool
            }
  {literal} { \bool_set_true:N  \g_@@_literal_Nabla_bool }
 }
%    \end{macrocode}
%
%    \begin{macrocode}
\@@_keys_choices:nn {partial}
 {
  {upright} {
              \bool_set_false:N \g_@@_literal_partial_bool
              \bool_set_true:N  \g_@@_uppartial_bool
            }
  {italic}  {
              \bool_set_false:N \g_@@_literal_partial_bool
              \bool_set_false:N \g_@@_uppartial_bool
            }
  {literal} { \bool_set_true:N  \g_@@_literal_partial_bool }
 }
%    \end{macrocode}
%
% \paragraph{Epsilon and phi shapes}
%    \begin{macrocode}
\@@_keys_choices:nn {vargreek-shape}
 {
  {unicode} { \bool_set_false:N \g_@@_texgreek_bool }
  {TeX}     { \bool_set_true:N  \g_@@_texgreek_bool }
 }
%    \end{macrocode}
%
% \paragraph{Colon style}
%    \begin{macrocode}
\@@_keys_choices:nn {colon}
 {
  {literal} { \bool_set_true:N  \g_@@_literal_colon_bool }
  {TeX}     { \bool_set_false:N \g_@@_literal_colon_bool }
 }
%    \end{macrocode}
%
% \paragraph{Slash delimiter style}
%    \begin{macrocode}
\@@_keys_choices:nn {slash-delimiter}
 {
  {ascii} { \tl_set:Nn \g_@@_slash_delimiter_usv {"002F} }
  {frac}  { \tl_set:Nn \g_@@_slash_delimiter_usv {"2044} }
  {div}   { \tl_set:Nn \g_@@_slash_delimiter_usv {"2215} }
 }
%    \end{macrocode}
%
%
% \paragraph{Active fraction style}
%    \begin{macrocode}
\@@_keys_choices:nn {active-frac}
 {
   {small}
   {
    \cs_if_exist:NTF \tfrac
     { \bool_set_true:N \l_@@_smallfrac_bool }
     {
      \@@_warning:n {no-tfrac}
      \bool_set_false:N \l_@@_smallfrac_bool
     }
    \use:c {@@_setup_active_frac:}
   }
   
   {normalsize}
   {
    \bool_set_false:N \l_@@_smallfrac_bool
    \use:c {@@_setup_active_frac:}
   }
 }
%    \end{macrocode}
%
% \paragraph{Debug/tracing}
%
%
%    \begin{macrocode}
\keys_define:nn {unicode-math}
  {
    warnings-off .code:n =
      {
        \clist_map_inline:nn {#1}
          { \msg_redirect_name:nnn { unicode-math } { ##1 } { none } }
      }
  }
%    \end{macrocode}
%
%    \begin{macrocode}
\@@_keys_choices:nn {trace}
 {
  {on}    {} % default
  {debug} { \msg_redirect_module:nnn { unicode-math } { log } { warning } }
  {off}   { \msg_redirect_module:nnn { unicode-math } { log } { none } }
 }
%    \end{macrocode}
%
%    \begin{macrocode}
\unimathsetup {math-style=TeX}
\unimathsetup {slash-delimiter=ascii}
\unimathsetup {trace=off}
\unimathsetup {mathrm=text,mathit=text,mathbf=text,mathsf=text,mathtt=text}
\cs_if_exist:NT \tfrac { \unimathsetup {active-frac=small} }
\ProcessKeysOptions {unicode-math}
%    \end{macrocode}
%
% \subsection{Programmers' interface}
%
% \begin{macro}{\unimath_get_mathstyle:}
% This command expands to the currently math style.
%    \begin{macrocode}
\cs_new:Nn \unimath_get_mathstyle:
 {
  \tl_use:N \l_@@_mathstyle_tl
 }
%    \end{macrocode}
% \end{macro}
%
% End of preamble code.
%    \begin{macrocode}
%</preamble&!XE&!LU>
%    \end{macrocode}
%
% (Error messages and warning definitions go here from the |msg| chunk
%  defined in \secref[vref]{codemsg}.)
%
% \section{Bifurcation}
%
% And here the split begins. Most of the code is still shared, but
% code for \LuaTeX\ uses the `\textsf{LU}' flag and code for \XeTeX\ uses `\textsf{XE}'.
%
%    \begin{macrocode}
%<*package&(XE|LU)>
\ExplSyntaxOn
%    \end{macrocode}
%
% \subsection{Engine differences}
%
% \XeTeX\ before version 0.9999 did not support |\U| prefix for extended math
% primitives, and while \LuaTeX\ had it from the start, prior 0.75.0 the
% \LaTeX\ format did not provide them without the |\luatex| prefix.
% We assume that users of \pkg{unicode-math} are using up-to-date engines however.
%
%    \begin{macrocode}
%<*LU>
\RequirePackage{luaotfload}   [2014/05/18]
\RequirePackage{lualatex-math}[2011/08/07]
%</LU>
%    \end{macrocode}
%
%
% \subsection{Overcoming \texorpdfstring{\cmd\@onlypreamble}{\textbackslash @onlypreamble}}
%
% The requirement of only setting up the maths fonts in the preamble is now removed. The following list might be overly ambitious.
%    \begin{macrocode}
\tl_map_inline:nn
 {
  \new@mathgroup\cdp@list\cdp@elt\DeclareMathSizes
  \@DeclareMathSizes\newmathalphabet\newmathalphabet@@\newmathalphabet@@@
  \DeclareMathVersion\define@mathalphabet\define@mathgroup\addtoversion
  \version@list\version@elt\alpha@list\alpha@elt
  \restore@mathversion\init@restore@version\dorestore@version\process@table
  \new@mathversion\DeclareSymbolFont\group@list\group@elt
  \new@symbolfont\SetSymbolFont\SetSymbolFont@\get@cdp
  \DeclareMathAlphabet\new@mathalphabet\SetMathAlphabet\SetMathAlphabet@
  \DeclareMathAccent\set@mathaccent\DeclareMathSymbol\set@mathchar
  \set@mathsymbol\DeclareMathDelimiter\@xxDeclareMathDelimiter
  \@DeclareMathDelimiter\@xDeclareMathDelimiter\set@mathdelimiter
  \set@@mathdelimiter\DeclareMathRadical\mathchar@type
  \DeclareSymbolFontAlphabet\DeclareSymbolFontAlphabet@
 }
 {
  \tl_remove_once:Nn \@preamblecmds {\do#1}
 }
%    \end{macrocode}
%
% \section{Fundamentals}
%
% \subsection{Setting math chars, math codes, etc.}
%
% \begin{macro}{\@@_set_mathsymbol:nNNn}
% \darg{A \LaTeX\ symbol font, e.g., \texttt{operators}}
% \darg{Symbol macro, \eg, \cmd\alpha}
% \darg{Type, \eg, \cmd\mathalpha}
% \darg{Slot, \eg, \texttt{"221E}}
% There are a bunch of tests to perform to process the various characters.
% The following assignments should all be fairly straightforward.
%
% The catcode setting is to work around (strange?) behaviour in LuaTeX in which catcode 11 characters don't have italic correction for maths.
% We don't adjust ascii chars, however, because certain punctuation should not have their catcodes changed.
%    \begin{macrocode}
\cs_set:Nn \@@_set_mathsymbol:nNNn
 {
  \bool_if:nT
   {
    \int_compare_p:nNn {#4} > {127} &&
    \int_compare_p:nNn { \char_value_catcode:n {#4} } = {11}
   }
   { \char_set_catcode_other:n {#4} }

  \tl_case:Nn #3
   {
    \mathord   { \@@_set_mathcode:nnn {#4} {#3} {#1} }
    \mathalpha { \@@_set_mathcode:nnn {#4} {#3} {#1} }
    \mathbin   { \@@_set_mathcode:nnn {#4} {#3} {#1} }
    \mathrel   { \@@_set_mathcode:nnn {#4} {#3} {#1} }
    \mathpunct { \@@_set_mathcode:nnn {#4} {#3} {#1} }
    \mathop    { \@@_set_big_operator:nnn {#1} {#2} {#4} }
    \mathopen  { \@@_set_math_open:nnn    {#1} {#2} {#4} }
    \mathclose { \@@_set_math_close:nnn   {#1} {#2} {#4} }
    \mathfence { \@@_set_math_fence:nnnn  {#1} {#2} {#3} {#4} }
    \mathaccent
     { \@@_set_math_accent:Nnnn #2 {fixed} {#1} {#4} }
    \mathbotaccent
     { \@@_set_math_accent:Nnnn #2 {bottom~ fixed} {#1} {#4} }
    \mathaccentwide
     { \@@_set_math_accent:Nnnn #2 {} {#1} {#4} }
    \mathbotaccentwide
     { \@@_set_math_accent:Nnnn #2 {bottom} {#1} {#4} }
    \mathover
     { \@@_set_math_overunder:Nnnn #2 {} {#1} {#4} }
    \mathunder
     { \@@_set_math_overunder:Nnnn #2 {bottom} {#1} {#4} }
   }
 }
%    \end{macrocode}
% \end{macro}
%
%    \begin{macrocode}
\edef\mathfence{\string\mathfence}
\edef\mathover{\string\mathover}
\edef\mathunder{\string\mathunder}
\edef\mathbotaccent{\string\mathbotaccent}
\edef\mathaccentwide{\string\mathaccentwide}
\edef\mathbotaccentwide{\string\mathbotaccentwide}
%    \end{macrocode}
%
%
% \begin{macro}{\@@_set_big_operator:nnn}
% \darg{Symbol font name}
% \darg{Macro to assign}
% \darg{Glyph slot}
% In the examples following, say we're defining for the symbol \cmd\sum\ ($\sum$).
% In order for literal Unicode characters to be used in the source and still
% have the correct limits behaviour, big operators are made math-active.
% This involves three steps:
% \begin{itemize}
% \item
% The active math char is defined to expand to the macro \cs{sum_sym}.
% (Later, the control sequence \cs{sum} will be assigned the math char.)
% \item
% Declare the plain old mathchardef for the control sequence \cmd\sumop.
% (This follows the convention of \LaTeX/\pkg{amsmath}.)
% \item
% Define \cs{sum_sym} as \cmd\sumop, followed by \cmd\nolimits\ if necessary.
% \end{itemize}
% Whether the \cmd\nolimits\ suffix is inserted is controlled by the
% token list \cs{l_@@_nolimits_tl}, which contains a list of such characters.
% This list is checked dynamically to allow it to be updated mid-document.
%
% Examples of expansion, by default, for two big operators:
% \begin{quote}
% (~\cs{sum} $\to$~) $\sum$ $\to$ \cs{sum_sym} $\to$ \cs{sumop}\cs{nolimits}\par
% (~\cs{int} $\to$~) $\int$ $\to$ \cs{int_sym} $\to$ \cs{intop}
% \end{quote}
%    \begin{macrocode}
\cs_new:Nn \@@_set_big_operator:nnn
 {
  \group_begin:
    \char_set_catcode_active:n {#3}
    \@@_char_gmake_mathactive:n {#3}
    \@@_active_char_set:wc #3 \q_nil { \cs_to_str:N #2 _sym }
  \group_end:
  
  \@@_set_mathchar:cNnn {\cs_to_str:N #2 op} \mathop {#1} {#3}
  
  \cs_gset:cpx { \cs_to_str:N #2 _sym }
   {
    \exp_not:c { \cs_to_str:N #2 op   }
    \exp_not:n { \tl_if_in:NnT \l_@@_nolimits_tl {#2} \nolimits }
   }
 }
%    \end{macrocode}
% \end{macro}
%
% \begin{macro}{\@@_set_math_open:nnn}
% \darg{Symbol font name}
% \darg{Macro to assign}
% \darg{Glyph slot}
%    \begin{macrocode}
\cs_new:Nn \@@_set_math_open:nnn
 {
  \tl_if_in:NnTF \l_@@_radicals_tl {#2}
   {
     \cs_gset_protected_nopar:cpx {\cs_to_str:N #2 sign}
       { \@@_radical:nn {#1} {#3} }
     \tl_set:cn {l_@@_radical_\cs_to_str:N #2_tl} {\use:c{sym #1}~ #3}
   }
   {
     \@@_set_delcode:nnn {#1} {#3} {#3}
     \@@_set_mathcode:nnn {#3} \mathopen {#1}
     \cs_gset_protected_nopar:Npx #2
       { \@@_delimiter:Nnn \mathopen {#1} {#3} }
   }
 }
%    \end{macrocode}
% \end{macro}
%
% \begin{macro}{\@@_set_math_close:nnn}
% \darg{Symbol font name}
% \darg{Macro to assign}
% \darg{Glyph slot}
%    \begin{macrocode}
\cs_new:Nn \@@_set_math_close:nnn
 {
  \@@_set_delcode:nnn {#1} {#3} {#3}
  \@@_set_mathcode:nnn {#3} \mathclose {#1}
  \cs_gset_protected_nopar:Npx #2
    { \@@_delimiter:Nnn \mathclose {#1} {#3} }
 }
%    \end{macrocode}
% \end{macro}
%
% \begin{macro}{\@@_set_math_fence:nnnn}
% \darg{Symbol font name}
% \darg{Macro to assign}
% \darg{Type, \eg, \cmd\mathalpha}
% \darg{Glyph slot}
%    \begin{macrocode}
\cs_new:Nn \@@_set_math_fence:nnnn
 {
  \@@_set_mathcode:nnn {#4} {#3} {#1}
  \@@_set_delcode:nnn  {#1} {#4} {#4}
  \cs_gset_protected_nopar:cpx {l \cs_to_str:N #2}
    { \@@_delimiter:Nnn \mathopen  {#1} {#4} }
  \cs_gset_protected_nopar:cpx {r \cs_to_str:N #2}
    { \@@_delimiter:Nnn \mathclose {#1} {#4} }
 }
%    \end{macrocode}
% \end{macro}
%
% \begin{macro}{\@@_set_math_accent:Nnnn}
% \darg{Accend command}
% \darg{Accent type (string)}
% \darg{Symbol font name}
% \darg{Glyph slot}
%    \begin{macrocode}
\cs_new:Nn \@@_set_math_accent:Nnnn
 {
  \cs_gset_protected_nopar:Npx #1
   { \@@_accent:nnn {#2} {#3} {#4} }
 }
%    \end{macrocode}
% \end{macro}
%
% \begin{macro}{\@@_set_math_overunder:Nnnn}
% \darg{Accend command}
% \darg{Accent type (string)}
% \darg{Symbol font name}
% \darg{Glyph slot}
%    \begin{macrocode}
\cs_new:Nn \@@_set_math_overunder:Nnnn
 {
  \cs_gset_protected_nopar:Npx #1 ##1
   {
    \mathop
     { \@@_accent:nnn {#2} {#3} {#4} {##1} }
    \limits
   }
 }
%    \end{macrocode}
% \end{macro}
%
% \subsection{\cs{setmathalphabet}}
%
% \begin{macro}{\setmathalphabet}
%    \begin{macrocode}
\keys_define:nn {@@_mathface}
 {
  version .code:n =
   { \tl_set:Nn \l_@@_mversion_tl {#1} }
 }

\DeclareDocumentCommand \setmathfontface { m O{} m O{} }
 {
  \tl_clear:N \l_@@_mversion_tl
 
  \keys_set_known:nnN {@@_mathface} {#2,#4} \l_@@_keyval_clist
  \exp_args:Nnx \fontspec_set_family:Nxn \l_@@_tmpa_tl
   { ItalicFont={}, BoldFont={}, \exp_not:V \l_@@_keyval_clist } {#3}
  
  \tl_if_empty:NT \l_@@_mversion_tl
   {
    \tl_set:Nn \l_@@_mversion_tl {normal}
    \DeclareMathAlphabet #1 {\g_fontspec_encoding_tl} {\l_@@_tmpa_tl} {\mddefault} {\updefault}
   }
  \SetMathAlphabet #1 {\l_@@_mversion_tl} {\g_fontspec_encoding_tl} {\l_@@_tmpa_tl} {\mddefault} {\updefault}
  
  % integrate with fontspec's \setmathrm etc:
  \tl_case:Nn #1
   {
    \mathrm { \cs_set_eq:NN \g__fontspec_mathrm_tl \l_@@_tmpa_tl }
    \mathsf { \cs_set_eq:NN \g__fontspec_mathsf_tl \l_@@_tmpa_tl }
    \mathtt { \cs_set_eq:NN \g__fontspec_mathtt_tl \l_@@_tmpa_tl }
   }
 }

\@onlypreamble \setmathfontface
%    \end{macrocode}
% Note that \LaTeX's SetMathAlphabet simply doesn't work to "reset" a maths alphabet font after \verb"\begin{document}", so unlike most of the other maths commands around we still restrict this one to the preamble.
% \end{macro}
%
% \begin{macro}{\setoperatorfont}
% TODO: add check?
%    \begin{macrocode}
\DeclareDocumentCommand \setoperatorfont {m}
 { \tl_set:Nn \g_@@_operator_mathfont_tl {#1} }
\setoperatorfont{\mathrm}
%    \end{macrocode}
% \end{macro}
%
% \subsection{Hooks into \pkg{fontspec}}
%
% Historically, \cs{mathrm} and so on were completely overwritten by \pkg{unicode-math}, and \pkg{fontspec}'s methods for setting these fonts in the classical manner were bypassed.
%
% While we could now re-activate the way that \pkg{fontspec} does the following, because we can now change maths fonts whenever it's better to define new commands in \pkg{unicode-math} to define the \cs{mathXYZ} fonts.
%
% \subsubsection{Text font}
%    \begin{macrocode}
\cs_generate_variant:Nn \tl_if_eq:nnT {o} 
\cs_set:Nn \__fontspec_setmainfont:nn
 {
  \fontspec_set_family:Nnn \rmdefault {#1}{#2}
  \tl_if_eq:onT {\g__fontspec_mathrm_tl} {\rmdefault}
   {
%<XE>  \fontspec_set_family:Nnn \g__fontspec_mathrm_tl {#1} {#2}
%<LU>  \fontspec_set_family:Nnn \g__fontspec_mathrm_tl {Renderer=Basic,#1} {#2}
    \SetMathAlphabet\mathrm{normal}\g_fontspec_encoding_tl\g__fontspec_mathrm_tl\mddefault\updefault
    \SetMathAlphabet\mathit{normal}\g_fontspec_encoding_tl\g__fontspec_mathrm_tl\mddefault\itdefault
    \SetMathAlphabet\mathbf{normal}\g_fontspec_encoding_tl\g__fontspec_mathrm_tl\bfdefault\updefault
   }
  \normalfont
  \ignorespaces
 }

\cs_set:Nn \__fontspec_setsansfont:nn
 {
  \fontspec_set_family:Nnn \sfdefault {#1}{#2}
  \tl_if_eq:onT {\g__fontspec_mathsf_tl} {\sfdefault}
   {
%<XE>  \fontspec_set_family:Nnn \g__fontspec_mathsf_tl {#1} {#2}
%<LU>  \fontspec_set_family:Nnn \g__fontspec_mathsf_tl {Renderer=Basic,#1} {#2}
    \SetMathAlphabet\mathsf{normal}\g_fontspec_encoding_tl\g__fontspec_mathsf_tl\mddefault\updefault
    \SetMathAlphabet\mathsf{bold}  \g_fontspec_encoding_tl\g__fontspec_mathsf_tl\bfdefault\updefault
   }
  \normalfont
  \ignorespaces
 }

\cs_set:Nn \__fontspec_setmonofont:nn
 {
  \fontspec_set_family:Nnn \ttdefault {#1}{#2}
  \tl_if_eq:onT {\g__fontspec_mathtt_tl} {\ttdefault}
   {
%<XE>  \fontspec_set_family:Nnn \g__fontspec_mathtt_tl {#1} {#2}
%<LU>  \fontspec_set_family:Nnn \g__fontspec_mathtt_tl {Renderer=Basic,#1} {#2}
    \SetMathAlphabet\mathtt{normal}\g_fontspec_encoding_tl\g__fontspec_mathtt_tl\mddefault\updefault
    \SetMathAlphabet\mathtt{bold}  \g_fontspec_encoding_tl\g__fontspec_mathtt_tl\bfdefault\updefault
   }
  \normalfont
  \ignorespaces
 }
%    \end{macrocode}
%
% \subsubsection{Maths font}
% If the maths fonts are set explicitly, then the text commands above will not execute their branches to set the maths font alphabets.
%    \begin{macrocode}
\cs_set:Nn \__fontspec_setmathrm:nn
 {
%<XE>  \fontspec_set_family:Nnn \g__fontspec_mathrm_tl {#1} {#2}
%<LU>  \fontspec_set_family:Nnn \g__fontspec_mathrm_tl {Renderer=Basic,#1} {#2}
  \SetMathAlphabet\mathrm{normal}\g_fontspec_encoding_tl\g__fontspec_mathrm_tl\mddefault\updefault
  \SetMathAlphabet\mathit{normal}\g_fontspec_encoding_tl\g__fontspec_mathrm_tl\mddefault\itdefault
  \SetMathAlphabet\mathbf{normal}\g_fontspec_encoding_tl\g__fontspec_mathrm_tl\bfdefault\updefault
 }
\cs_set:Nn \__fontspec_setboldmathrm:nn
 {
%<XE>  \fontspec_set_family:Nnn \g__fontspec_bfmathrm_tl {#1} {#2}
%<LU>  \fontspec_set_family:Nnn \g__fontspec_bfmathrm_tl {Renderer=Basic,#1} {#2}
  \SetMathAlphabet\mathrm{bold}\g_fontspec_encoding_tl\g__fontspec_bfmathrm_tl\mddefault\updefault
  \SetMathAlphabet\mathbf{bold}\g_fontspec_encoding_tl\g__fontspec_bfmathrm_tl\bfdefault\updefault
  \SetMathAlphabet\mathit{bold}\g_fontspec_encoding_tl\g__fontspec_bfmathrm_tl\mddefault\itdefault
 }
\cs_set:Nn \__fontspec_setmathsf:nn
 {
%<XE>  \fontspec_set_family:Nnn \g__fontspec_mathsf_tl {#1} {#2}
%<LU>  \fontspec_set_family:Nnn \g__fontspec_mathsf_tl {Renderer=Basic,#1} {#2}
  \SetMathAlphabet\mathsf{normal}\g_fontspec_encoding_tl\g__fontspec_mathsf_tl\mddefault\updefault
  \SetMathAlphabet\mathsf{bold}  \g_fontspec_encoding_tl\g__fontspec_mathsf_tl\bfdefault\updefault
 }
\cs_set:Nn \__fontspec_setmathtt:nn
 {
%<XE>  \fontspec_set_family:Nnn \g__fontspec_mathtt_tl {#1} {#2}
%<LU>  \fontspec_set_family:Nnn \g__fontspec_mathtt_tl {Renderer=Basic,#1} {#2}
  \SetMathAlphabet\mathtt{normal}\g_fontspec_encoding_tl\g__fontspec_mathtt_tl\mddefault\updefault
  \SetMathAlphabet\mathtt{bold}  \g_fontspec_encoding_tl\g__fontspec_mathtt_tl\bfdefault\updefault
 }
%    \end{macrocode}
%
%
% \subsection{The main \cs{setmathfont} macro}
%
% Using a |range| including large character sets such as \cmd\mathrel,
% \cmd\mathalpha, \etc, is \emph{very slow}!
% I hope to improve the performance somehow.
%
% \begin{macro}{\setmathfont}
% \doarg{font features (first optional argument retained for backwards compatibility)}
% \darg{font name}
% \doarg{font features}
%    \begin{macrocode}
\DeclareDocumentCommand \setmathfont { O{} m O{} }
 {
  \tl_set:Nn \l_@@_fontname_tl {#2}
  \@@_init:
%    \end{macrocode}
% Grab the current size information:
% (is this robust enough? Maybe it should be preceded by \cmd\normalsize).
% The macro \cmd\S@\meta{size}
% contains the definitions of the sizes used for maths letters, subscripts and subsubscripts in
% \cmd\tf@size, \cmd\sf@size, and \cmd\ssf@size, respectively.
%    \begin{macrocode}
  \cs_if_exist:cF { S@ \f@size } { \calculate@math@sizes }
  \csname S@\f@size\endcsname
%    \end{macrocode}
% Parse options and tell people what's going on:
%    \begin{macrocode}
  \keys_set_known:nnN {unicode-math} {#1,#3} \l_@@_unknown_keys_clist
  \bool_if:NT \l_@@_init_bool { \@@_log:n {default-math-font} }
%    \end{macrocode}
% Use \pkg{fontspec} to select a font to use.
% After loading the font, we detect what sizes it recommends for scriptsize and scriptscriptsize, so after setting those values appropriately, we reload the font to take these into account.
%    \begin{macrocode}

%<debug>  \csname TIC\endcsname
  \@@_fontspec_select_font:
%<debug>  \csname TOC\endcsname
  \bool_if:nT { \l_@@_ot_math_bool && !\g_@@_mainfont_already_set_bool }
   {
    \@@_declare_math_sizes:
    \@@_fontspec_select_font: 
   }
%    \end{macrocode}
% Now define |\@@_symfont_tl| as the \LaTeX\ math font to access everything:
%    \begin{macrocode}
  \cs_if_exist:cF { sym \@@_symfont_tl }
    {
      \DeclareSymbolFont{\@@_symfont_tl}
        {\encodingdefault}{\l_@@_family_tl}{\mddefault}{\updefault}
    }
  \SetSymbolFont{\@@_symfont_tl}{\l_@@_mversion_tl}
    {\encodingdefault}{\l_@@_family_tl}{\mddefault}{\updefault}
%    \end{macrocode}
% Set the bold math version.
%    \begin{macrocode}
  \tl_set:Nn \l_@@_tmpa_tl {normal}
  \tl_if_eq:NNT \l_@@_mversion_tl \l_@@_tmpa_tl
    {
     \SetSymbolFont{\@@_symfont_tl}{bold}
      {\encodingdefault}{\l_@@_family_tl}{\bfdefault}{\updefault}
    }
%    \end{macrocode}
% Declare the math sizes (i.e., scaling of superscripts) for the specific
% values for this font,
% and set defaults for math fams two and three for legacy compatibility:
%    \begin{macrocode}
  \bool_if:nT { \l_@@_ot_math_bool && !\g_@@_mainfont_already_set_bool }
   {
    \bool_set_true:N \g_@@_mainfont_already_set_bool
    \@@_setup_legacy_fam_two:
    \@@_setup_legacy_fam_three:
   }
%    \end{macrocode}
% And now we input every single maths char.
%    \begin{macrocode}
%<debug>  \csname TIC\endcsname
  \@@_input_math_symbol_table:
%<debug>  \csname TOC\endcsname
%    \end{macrocode}
% Finally,
% \begin{itemize}
% \item Remap symbols that don't take their natural mathcode
% \item Activate any symbols that need to be math-active
% \item Enable wide/narrow accents
% \item Assign delimiter codes for symbols that need to grow
% \item Setup the maths alphabets (\cs{mathbf} etc.)
% \end{itemize}
%    \begin{macrocode}
  \@@_remap_symbols:
  \@@_setup_mathactives:
  \@@_setup_delcodes:
%<debug>  \csname TIC\endcsname
  \@@_setup_alphabets:
%<debug>  \csname TOC\endcsname
  \@@_setup_negations:
%    \end{macrocode}
% Prevent spaces, and that's it:
%    \begin{macrocode}
  \ignorespaces
 }
%    \end{macrocode}
% \end{macro}
%
% Backward compatibility alias.
%    \begin{macrocode}
\cs_set_eq:NN \resetmathfont \setmathfont
%    \end{macrocode}
%
% \begin{macro}{\@@_init:}
%    \begin{macrocode}
\cs_new:Nn \@@_init:
 {
%    \end{macrocode}
% \begin{itemize}
% \item Initially assume we're using a proper OpenType font with unicode maths.
%    \begin{macrocode}
  \bool_set_true:N  \l_@@_ot_math_bool
%    \end{macrocode}
% \item Erase any conception \LaTeX\ has of previously defined math symbol fonts;
% this allows \cmd\DeclareSymbolFont\ at any point in the document.
%    \begin{macrocode}
  \cs_set_eq:NN \glb@currsize \scan_stop:
%    \end{macrocode}
% \item To start with, assume we're defining the font for every math symbol character.
%    \begin{macrocode}
  \bool_set_true:N \l_@@_init_bool
  \seq_clear:N \l_@@_char_range_seq
  \clist_clear:N \l_@@_char_nrange_clist
  \seq_clear:N \l_@@_mathalph_seq
  \seq_clear:N \l_@@_missing_alph_seq
%    \end{macrocode}
% \item By default use the `normal' math version.
%    \begin{macrocode}
  \tl_set:Nn \l_@@_mversion_tl {normal}
%    \end{macrocode}
% \item Other range initialisations.
%    \begin{macrocode}
  \tl_set:Nn \@@_symfont_tl {operators}
  \cs_set_eq:NN \_@@_sym:nnn \@@_process_symbol_noparse:nnn
  \cs_set_eq:NN \@@_set_mathalphabet_char:nnn \@@_mathmap_noparse:nnn
  \cs_set_eq:NN \@@_remap_symbol:nnn \@@_remap_symbol_noparse:nnn
  \cs_set_eq:NN \@@_maybe_init_alphabet:n \@@_init_alphabet:n
  \cs_set_eq:NN \@@_map_char_single:nn \@@_map_char_noparse:nn
  \cs_set_eq:NN \@@_assign_delcode:nn \@@_assign_delcode_noparse:nn
  \cs_set_eq:NN \@@_make_mathactive:nNN \@@_make_mathactive_noparse:nNN
%    \end{macrocode}
% \item Define default font features for the script and scriptscript font.
%    \begin{macrocode}
  \tl_set:Nn \l_@@_script_features_tl  {Style=MathScript}
  \tl_set:Nn \l_@@_sscript_features_tl {Style=MathScriptScript}
  \tl_set_eq:NN \l_@@_script_font_tl   \l_@@_fontname_tl
  \tl_set_eq:NN \l_@@_sscript_font_tl  \l_@@_fontname_tl
%    \end{macrocode}
% \end{itemize}
%    \begin{macrocode}
 }
%    \end{macrocode}
% \end{macro}
%
%
% \begin{macro}{\@@_declare_math_sizes:}
% Set the math sizes according to the recommended font parameters:
%    \begin{macrocode}
\cs_new:Nn \@@_declare_math_sizes:
  {
    \dim_compare:nF { \fontdimen 10 \l_@@_font == 0pt }
      {
        \DeclareMathSizes { \f@size } { \f@size }
          { \@@_fontdimen_to_scale:nn {10} {\l_@@_font} }
          { \@@_fontdimen_to_scale:nn {11} {\l_@@_font} }
      }
  }
%    \end{macrocode}
% \end{macro}
%
%
%
% \begin{macro}{\@@_setup_legacy_fam_two:}
% \TeX\ won't load the same font twice at the same scale, so we need to magnify this one by an imperceptable amount.
%    \begin{macrocode}
\cs_new:Nn \@@_setup_legacy_fam_two:
  {
    \fontspec_set_family:Nxn \l_@@_family_tl
      {
      \l_@@_font_keyval_tl,
      Scale=1.00001,
      FontAdjustment =
       {
        \fontdimen8\font= \@@_get_fontparam:nn {43} {FractionNumeratorDisplayStyleShiftUp}\relax
        \fontdimen9\font= \@@_get_fontparam:nn {42} {FractionNumeratorShiftUp}\relax
        \fontdimen10\font=\@@_get_fontparam:nn {32} {StackTopShiftUp}\relax
        \fontdimen11\font=\@@_get_fontparam:nn {45} {FractionDenominatorDisplayStyleShiftDown}\relax
        \fontdimen12\font=\@@_get_fontparam:nn {44} {FractionDenominatorShiftDown}\relax
        \fontdimen13\font=\@@_get_fontparam:nn {21} {SuperscriptShiftUp}\relax
        \fontdimen14\font=\@@_get_fontparam:nn {21} {SuperscriptShiftUp}\relax
        \fontdimen15\font=\@@_get_fontparam:nn {22} {SuperscriptShiftUpCramped}\relax
        \fontdimen16\font=\@@_get_fontparam:nn {18} {SubscriptShiftDown}\relax
        \fontdimen17\font=\@@_get_fontparam:nn {18} {SubscriptShiftDownWithSuperscript}\relax
        \fontdimen18\font=\@@_get_fontparam:nn {24} {SuperscriptBaselineDropMax}\relax
        \fontdimen19\font=\@@_get_fontparam:nn {20} {SubscriptBaselineDropMin}\relax
        \fontdimen20\font=0pt\relax % delim1 = FractionDelimiterDisplaySize
        \fontdimen21\font=0pt\relax % delim2 = FractionDelimiterSize
        \fontdimen22\font=\@@_get_fontparam:nn {15} {AxisHeight}\relax
       }
      } {\l_@@_fontname_tl}
    \SetSymbolFont{symbols}{\l_@@_mversion_tl}
      {\encodingdefault}{\l_@@_family_tl}{\mddefault}{\updefault}

    \tl_set:Nn \l_@@_tmpa_tl {normal}
    \tl_if_eq:NNT \l_@@_mversion_tl \l_@@_tmpa_tl
      {
      \SetSymbolFont{symbols}{bold}
        {\encodingdefault}{\l_@@_family_tl}{\bfdefault}{\updefault}
      }
  }
%    \end{macrocode}
% \end{macro}
%
% \begin{macro}{\@@_setup_legacy_fam_three:}
% Similarly, this font is shrunk by an imperceptable amount for \TeX\ to load it again.
%    \begin{macrocode}
\cs_new:Nn \@@_setup_legacy_fam_three:
  {
    \fontspec_set_family:Nxn \l_@@_family_tl
      {
      \l_@@_font_keyval_tl,
      Scale=0.99999,
      FontAdjustment={
        \fontdimen8\font= \@@_get_fontparam:nn {48} {FractionRuleThickness}\relax
        \fontdimen9\font= \@@_get_fontparam:nn {28} {UpperLimitGapMin}\relax
        \fontdimen10\font=\@@_get_fontparam:nn {30} {LowerLimitGapMin}\relax
        \fontdimen11\font=\@@_get_fontparam:nn {29} {UpperLimitBaselineRiseMin}\relax
        \fontdimen12\font=\@@_get_fontparam:nn {31} {LowerLimitBaselineDropMin}\relax
        \fontdimen13\font=0pt\relax
      }
    } {\l_@@_fontname_tl}
    \SetSymbolFont{largesymbols}{\l_@@_mversion_tl}
      {\encodingdefault}{\l_@@_family_tl}{\mddefault}{\updefault}

    \tl_set:Nn \l_@@_tmpa_tl {normal}
    \tl_if_eq:NNT \l_@@_mversion_tl \l_@@_tmpa_tl
      {
      \SetSymbolFont{largesymbols}{bold}
        {\encodingdefault}{\l_@@_family_tl}{\bfdefault}{\updefault}
      }
  }
%    \end{macrocode}
% \end{macro}
%
%
%    \begin{macrocode}
\cs_new:Nn \@@_get_fontparam:nn
%<XE>  { \the\fontdimen#1\l_@@_font\relax }
%<LU>  { \directlua{fontspec.mathfontdimen("l_@@_font","#2")} }
%    \end{macrocode}
%
%
%
% \begin{macro}{\@@_fontspec_select_font:}
% Select the font with \cs{fontspec} and define \cs{l_@@_font} from it.
%    \begin{macrocode}
\cs_new:Nn \@@_fontspec_select_font:
 {
  \tl_set:Nx \l_@@_font_keyval_tl {
%<LU>     Renderer = Basic,
    BoldItalicFont = {}, ItalicFont = {},
    Script = Math,
    SizeFeatures =
     {
      {
       Size = \tf@size-
      } ,
      {
       Size = \sf@size-\tf@size ,
       Font = \l_@@_script_font_tl ,
       \l_@@_script_features_tl
      } ,
      {
       Size = -\sf@size ,
       Font = \l_@@_sscript_font_tl ,
       \l_@@_sscript_features_tl
      }
     } ,
    \l_@@_unknown_keys_clist
  }
  \fontspec_set_fontface:NNxn \l_@@_font \l_@@_family_tl
    {\l_@@_font_keyval_tl} {\l_@@_fontname_tl}
%    \end{macrocode}
% Check whether we're using a real maths font:
%    \begin{macrocode}
  \group_begin:
    \fontfamily{\l_@@_family_tl}\selectfont
    \fontspec_if_script:nF {math} {\bool_gset_false:N \l_@@_ot_math_bool}
  \group_end:
 }
%    \end{macrocode}
% \end{macro}
%
%
% \subsubsection{Functions for setting up symbols with mathcodes}
% \seclabel{mathsymbol}
%
% \begin{macro}{\@@_process_symbol_noparse:nnn}
% \begin{macro}{\@@_process_symbol_parse:nnn}
% If the \feat{range} font feature has been used, then only
% a subset of the Unicode glyphs are to be defined.
% See \secref{rangeproc} for the code that enables this.
%    \begin{macrocode}
\cs_set:Nn \@@_process_symbol_noparse:nnn
 {
  \@@_set_mathsymbol:nNNn {\@@_symfont_tl} #2 #3 {#1}
 }
%    \end{macrocode}
%    \begin{macrocode}
\cs_set:Nn \@@_process_symbol_parse:nnn
 {
  \@@_if_char_spec:nNNT {#1} {#2} {#3}
   {
    \@@_process_symbol_noparse:nnn {#1} {#2} {#3}
   }
 }
%    \end{macrocode}
% \end{macro}
% \end{macro}
%
%
% \begin{macro}{\@@_remap_symbols:}
% \begin{macro}{\@@_remap_symbol_noparse:nnn}
% \begin{macro}{\@@_remap_symbol_parse:nnn}
% This function is used to define the mathcodes for those chars which should
% be mapped to a different glyph than themselves.
%    \begin{macrocode}
\cs_new:Npn \@@_remap_symbols:
 {
  \@@_remap_symbol:nnn{`\-}{\mathbin}{"02212}% hyphen to minus
  \@@_remap_symbol:nnn{`\*}{\mathbin}{"02217}% text asterisk to "centred asterisk"
  \bool_if:NF \g_@@_literal_colon_bool
   {
    \@@_remap_symbol:nnn{`\:}{\mathrel}{"02236}% colon to ratio (i.e., punct to rel)
   }
 }
%    \end{macrocode}
% \end{macro}
% Where |\@@_remap_symbol:nnn| is defined to be one of these two, depending
% on the range setup:
%    \begin{macrocode}
\cs_new:Nn \@@_remap_symbol_parse:nnn
 {
  \@@_if_char_spec:nNNT {#3} {\@nil} {#2}
   { \@@_remap_symbol_noparse:nnn {#1} {#2} {#3} }
 }
\cs_new:Nn \@@_remap_symbol_noparse:nnn
 {
  \clist_map_inline:nn {#1}
   { \@@_set_mathcode:nnnn {##1} {#2} {\@@_symfont_tl} {#3} }
 }
%    \end{macrocode}
% \end{macro}
% \end{macro}
%
%
% \subsubsection{Active math characters}
%
% There are more math active chars later in the subscript/superscript section.
% But they don't need to be able to be typeset directly.
%
% \begin{macro}{\@@_setup_mathactives:}
%    \begin{macrocode}
\cs_new:Npn \@@_setup_mathactives:
 {
  \@@_make_mathactive:nNN {"2032} \@@_prime_single_mchar \mathord
  \@@_make_mathactive:nNN {"2033} \@@_prime_double_mchar \mathord
  \@@_make_mathactive:nNN {"2034} \@@_prime_triple_mchar \mathord
  \@@_make_mathactive:nNN {"2057} \@@_prime_quad_mchar   \mathord
  \@@_make_mathactive:nNN {"2035} \@@_backprime_single_mchar \mathord
  \@@_make_mathactive:nNN {"2036} \@@_backprime_double_mchar \mathord
  \@@_make_mathactive:nNN {"2037} \@@_backprime_triple_mchar \mathord
  \@@_make_mathactive:nNN {`\'} \mathstraightquote \mathord
  \@@_make_mathactive:nNN {`\`} \mathbacktick      \mathord
 }
%    \end{macrocode}
% \end{macro}
%
% \begin{macro}{\@@_make_mathactive:nNN}
% Makes |#1| a mathactive char, and gives cs |#2| the meaning of mathchar |#1|
% with class |#3|.
% You are responsible for giving active |#1| a particular meaning!
%    \begin{macrocode}
\cs_new:Nn \@@_make_mathactive_parse:nNN
  {
    \@@_if_char_spec:nNNT {#1} #2 #3
      { \@@_make_mathactive_noparse:nNN {#1} #2 #3 }
  }
\cs_new:Nn \@@_make_mathactive_noparse:nNN
  {
    \@@_set_mathchar:NNnn #2 #3 {\@@_symfont_tl} {#1}
    \@@_char_gmake_mathactive:n {#1}
  }
%    \end{macrocode}
% \end{macro}
%
% \subsubsection{Delimiter codes}
%
%
% \begin{macro}{\@@_assign_delcode:nn}
%    \begin{macrocode}
\cs_new:Nn \@@_assign_delcode_noparse:nn
 {
  \@@_set_delcode:nnn \@@_symfont_tl {#1} {#2}
 }
\cs_new:Nn \@@_assign_delcode_parse:nn
 {
  \@@_if_char_spec:nNNT {#2} {\@nil} {\@nil}
   {
    \@@_assign_delcode_noparse:nn {#1} {#2}
   }
 }
%    \end{macrocode}
% \end{macro}
%
%
% \begin{macro}{\@@_assign_delcode:n}
% Shorthand.
%    \begin{macrocode}
\cs_new:Nn \@@_assign_delcode:n { \@@_assign_delcode:nn {#1} {#1} }
%    \end{macrocode}
% \end{macro}
%
%
%
% \begin{macro}{\@@_setup_delcodes:}
% Some symbols that aren't mathopen/mathclose still need to have delimiter codes assigned.
% The list of vertical arrows may be incomplete.
% On the other hand, many fonts won't support them all being stretchy.
% And some of them are probably not meant to stretch, either. But adding them here doesn't hurt.
%    \begin{macrocode}
\cs_new:Npn \@@_setup_delcodes:
 {
  % ensure \left. and \right. work:
  \@@_set_delcode:nnn \@@_symfont_tl {`\.} {\c_zero}
  % this is forcefully done to fix a bug -- indicates a larger problem!
  
  \@@_assign_delcode:nn {`\/}   {\g_@@_slash_delimiter_usv}
  \@@_assign_delcode:nn {"2044} {\g_@@_slash_delimiter_usv} % fracslash
  \@@_assign_delcode:nn {"2215} {\g_@@_slash_delimiter_usv} % divslash
  \@@_assign_delcode:n {"005C} % backslash
  \@@_assign_delcode:nn {`\<} {"27E8} % angle brackets with ascii notation
  \@@_assign_delcode:nn {`\>} {"27E9} % angle brackets with ascii notation
  \@@_assign_delcode:n {"2191} % up arrow
  \@@_assign_delcode:n {"2193} % down arrow
  \@@_assign_delcode:n {"2195} % updown arrow
  \@@_assign_delcode:n {"219F} % up arrow twohead
  \@@_assign_delcode:n {"21A1} % down arrow twohead
  \@@_assign_delcode:n {"21A5} % up arrow from bar
  \@@_assign_delcode:n {"21A7} % down arrow from bar
  \@@_assign_delcode:n {"21A8} % updown arrow from bar
  \@@_assign_delcode:n {"21BE} % up harpoon right
  \@@_assign_delcode:n {"21BF} % up harpoon left
  \@@_assign_delcode:n {"21C2} % down harpoon right
  \@@_assign_delcode:n {"21C3} % down harpoon left
  \@@_assign_delcode:n {"21C5} % arrows up down
  \@@_assign_delcode:n {"21F5} % arrows down up
  \@@_assign_delcode:n {"21C8} % arrows up up
  \@@_assign_delcode:n {"21CA} % arrows down down
  \@@_assign_delcode:n {"21D1} % double up arrow
  \@@_assign_delcode:n {"21D3} % double down arrow
  \@@_assign_delcode:n {"21D5} % double updown arrow
  \@@_assign_delcode:n {"21DE} % up arrow double stroke
  \@@_assign_delcode:n {"21DF} % down arrow double stroke
  \@@_assign_delcode:n {"21E1} % up arrow dashed
  \@@_assign_delcode:n {"21E3} % down arrow dashed
  \@@_assign_delcode:n {"21E7} % up white arrow
  \@@_assign_delcode:n {"21E9} % down white arrow
  \@@_assign_delcode:n {"21EA} % up white arrow from bar
  \@@_assign_delcode:n {"21F3} % updown white arrow
 }
%    \end{macrocode}
% \end{macro}
%
%
%
%
% \subsection{(Big) operators}
%
% Turns out that \XeTeX\ is clever enough to deal with big operators for us
% automatically with \cmd\Umathchardef. Amazing!
%
% However, the limits aren't set automatically; that is, we want to define,
% a la Plain \TeX\ \etc, |\def\int{\intop\nolimits}|, so there needs to be a
% transformation from \cmd\int\ to \cmd\intop\ during the expansion of
% \cmd\_@@_sym:nnn\ in the appropriate contexts.
%
% \begin{macro}{\l_@@_nolimits_tl}
% This macro is a sequence containing those maths operators that require a
% \cmd\nolimits\ suffix.
% This list is used when processing |unicode-math-table.tex| to define such
% commands automatically (see the macro \cs{@@_set_mathsymbol:nNNn}).
% I've chosen essentially just the operators that look like integrals;
% hopefully a better mathematician can help me out here.
% I've a feeling that it's more useful \emph{not} to include the multiple
% integrals such as $\iiiint$, but that might be a matter of preference.
%    \begin{macrocode}
\tl_new:N \l_@@_nolimits_tl
\tl_set:Nn \l_@@_nolimits_tl
 {
  \int\iint\iiint\iiiint\oint\oiint\oiiint
  \intclockwise\varointclockwise\ointctrclockwise\sumint
  \intbar\intBar\fint\cirfnint\awint\rppolint
  \scpolint\npolint\pointint\sqint\intlarhk\intx
  \intcap\intcup\upint\lowint
 }
%    \end{macrocode}
% \end{macro}
%
% \begin{macro}{\addnolimits}
% This macro appends material to the macro containing the list of operators
% that don't take limits.
%    \begin{macrocode}
\DeclareDocumentCommand \addnolimits {m}
 {
  \tl_put_right:Nn \l_@@_nolimits_tl {#1}
 }
%    \end{macrocode}
% \end{macro}
%
% \begin{macro}{\removenolimits}
% Can this macro be given a better name?
% It removes an item from the nolimits list.
%    \begin{macrocode}
\DeclareDocumentCommand \removenolimits {m}
 {
  \tl_remove_all:Nn \l_@@_nolimits_tl {#1}
 }
%    \end{macrocode}
% \end{macro}
%
% \subsection{Radicals}
%
% The radical for square root is organised in \cs{@@_set_mathsymbol:nNNn}.
% I think it's the only radical ever.
% (Actually, there is also \cs{cuberoot} and \cs{fourthroot}, but they don't
%  seem to behave as proper radicals.)
%
% Also, what about right-to-left square roots?
%
% \begin{macro}{\l_@@_radicals_tl}
% We organise radicals in the same way as nolimits-operators.
%    \begin{macrocode}
\tl_new:N \l_@@_radicals_tl
\tl_set:Nn \l_@@_radicals_tl {\sqrt \longdivision}
%    \end{macrocode}
% \end{macro}
%
% \subsection{Maths accents}
%
% Maths accents should just work \emph{if they are available in the font}.
%
% \subsection{Common interface for font parameters}
%
% \XeTeX\ and \LuaTeX\ have different interfaces for math font parameters.
% We use \LuaTeX’s interface because it’s much better, but rename the primitives to be more \LaTeX3-like.
% There are getter and setter commands for each font parameter.
% The names of the parameters is derived from the \LuaTeX\ names, with underscores inserted between words.
% For every parameter \cs{Umath\meta{\LuaTeX\ name}}, we define an expandable getter command \cs{@@_\meta{\LaTeX3 name}:N} and a protected setter command \cs{@@_set_\meta{\LaTeX3 name}:Nn}.
% The getter command takes one of the style primitives (\cs{displaystyle} etc.)\ and expands to the font parameter, which is a \meta{dimension}.
% The setter command takes a style primitive and a dimension expression, which is parsed with \cs{dim_eval:n}.
%
% Often, the mapping between font dimensions and font parameters is bijective, but there are cases which require special attention:
% \begin{itemize}
% \item Some parameters map to different dimensions in display and non-display styles.
% \item Likewise, one parameter maps to different dimensions in non-cramped and cramped styles.
% \item There are a few parameters for which \XeTeX\ doesn’t seem to provide \cs{fontdimen}s; in this case the getter and setter commands are left undefined.
% \end{itemize}
%
% \paragraph{Cramped style tokens}
% \LuaTeX\ has \cs{crampeddisplaystyle} etc.,\ but they are loaded as \cs{luatexcrampeddisplaystyle} etc.\ by the \pkg{luatextra} package.
% \XeTeX, however, doesn’t have these primitives, and their syntax cannot really be emulated.
% Nevertheless, we define these commands as quarks, so they can be used as arguments to the font parameter commands (but nowhere else).
% Making these commands available is necessary because we need to make a distinction between cramped and non-cramped styles for one font parameter.
%
% \begin{macro}{\@@_new_cramped_style:N}
% \darg{command}
% Define \meta{command} as a new cramped style switch.
% For \LuaTeX, simply rename the correspronding primitive.
% For \XeTeX, define \meta{command} as a new quark.
%    \begin{macrocode}
\cs_new_protected_nopar:Nn \@@_new_cramped_style:N
%<XE>  { \quark_new:N #1 }
%<LU>  { \cs_new_eq:Nc #1 { luatex \cs_to_str:N #1 } }
%    \end{macrocode}
% \end{macro}
%
% \begin{macro}{\crampeddisplaystyle}
% \begin{macro}{\crampedtextstyle}
% \begin{macro}{\crampedscriptstyle}
% \begin{macro}{\crampedscriptscriptstyle}
% The cramped style commands.
%    \begin{macrocode}
\@@_new_cramped_style:N \crampeddisplaystyle
\@@_new_cramped_style:N \crampedtextstyle
\@@_new_cramped_style:N \crampedscriptstyle
\@@_new_cramped_style:N \crampedscriptscriptstyle
%    \end{macrocode}
% \end{macro}
% \end{macro}
% \end{macro}
% \end{macro}
%
% \paragraph{Font dimension mapping}
% Font parameters may differ between the styles.
% \LuaTeX\ accounts for this by having the parameter primitives take a style token argument.
% To replicate this behavior in \XeTeX, we have to map style tokens to specific combinations of font dimension numbers and math fonts (\cs{textfont} etc.).
%
% \begin{macro}{\@@_font_dimen:Nnnnn}
% \darg{style token}
% \darg{font dimen for display style}
% \darg{font dimen for cramped display style}
% \darg{font dimen for non-display styles}
% \darg{font dimen for cramped non-display styles}
% Map math style to \XeTeX\ math font dimension.
% \meta{style token} must be one of the style switches (\cs{displaystyle}, \cs{crampeddisplaystyle}, \dots).
% The other parameters are integer constants referring to font dimension numbers.
% The macro expands to a dimension which contains the appropriate font dimension.
%    \begin{macrocode}
%<*XE>
  \cs_new_nopar:Npn \@@_font_dimen:Nnnnn #1 #2 #3 #4 #5 {
    \fontdimen
    \cs_if_eq:NNTF #1 \displaystyle {
      #2 \textfont
    } {
      \cs_if_eq:NNTF #1 \crampeddisplaystyle {
        #3 \textfont
      } {
        \cs_if_eq:NNTF #1 \textstyle {
          #4 \textfont
        } {
          \cs_if_eq:NNTF #1 \crampedtextstyle {
            #5 \textfont
          } {
            \cs_if_eq:NNTF #1 \scriptstyle {
              #4 \scriptfont
            } {
              \cs_if_eq:NNTF #1 \crampedscriptstyle {
                #5 \scriptfont
              } {
                \cs_if_eq:NNTF #1 \scriptscriptstyle {
                  #4 \scriptscriptfont
                } {
%    \end{macrocode}
% Should we check here if the style is invalid?
%    \begin{macrocode}
                  #5 \scriptscriptfont
                }
              }
            }
          }
        }
      }
    }
%    \end{macrocode}
% Which family to use?
%    \begin{macrocode}
    \c_two
  }
%</XE>
%    \end{macrocode}
% \end{macro}
%
% \paragraph{Font parameters}
% This paragraph contains macros for defining the font parameter interface, as well as the definition for all font parameters known to \LuaTeX.
%
% \begin{macro}{\@@_font_param:nnnnn}
% \darg{name}
% \darg{font dimension for non-cramped display style}
% \darg{font dimension for cramped display style}
% \darg{font dimension for non-cramped non-display styles}
% \darg{font dimension for cramped non-display styles}
% This macro defines getter and setter functions for the font parameter \meta{name}.
% The \LuaTeX\ font parameter name is produced by removing all underscores and prefixing the result with |luatexUmath|.
% The \XeTeX\ font dimension numbers must be integer constants.
%    \begin{macrocode}
\cs_new_protected_nopar:Nn \@@_font_param:nnnnn
%<*XE>
{
  \@@_font_param_aux:ccnnnn { @@_ #1 :N } { @@_set_ #1 :Nn }
    { #2 } { #3 } { #4 } { #5 }
}
%</XE>
%<*LU>
{
  \tl_set:Nn \l_@@_tmpa_tl { #1 }
  \tl_remove_all:Nn \l_@@_tmpa_tl { _ }
  \@@_font_param_aux:ccc { @@_ #1 :N } { @@_set_ #1 :Nn }
    { luatexUmath \l_@@_tmpa_tl }
}
%</LU>
%    \end{macrocode}
% \end{macro}
%
% \begin{macro}{\@@_font_param:nnn}
% \darg{name}
% \darg{font dimension for display style}
% \darg{font dimension for non-display styles}
% This macro defines getter and setter functions for the font parameter \meta{name}.
% The \LuaTeX\ font parameter name is produced by removing all underscores and prefixing the result with |luatexUmath|.
% The \XeTeX\ font dimension numbers must be integer constants.
%    \begin{macrocode}
\cs_new_protected_nopar:Nn \@@_font_param:nnn
 {
  \@@_font_param:nnnnn { #1 } { #2 } { #2 } { #3 } { #3 }
 }
%    \end{macrocode}
% \end{macro}
%
% \begin{macro}{\@@_font_param:nn}
% \darg{name}
% \darg{font dimension}
% This macro defines getter and setter functions for the font parameter \meta{name}.
% The \LuaTeX\ font parameter name is produced by removing all underscores and prefixing the result with |luatexUmath|.
% The \XeTeX\ font dimension number must be an integer constant.
%    \begin{macrocode}
\cs_new_protected_nopar:Nn \@@_font_param:nn
 {
  \@@_font_param:nnnnn { #1 } { #2 } { #2 } { #2 } { #2 }
 }
%    \end{macrocode}
% \end{macro}
%
% \begin{macro}{\@@_font_param:n}
% \darg{name}
% This macro defines getter and setter functions for the font parameter \meta{name}, which is considered unavailable in \XeTeX\@.
% The \LuaTeX\ font parameter name is produced by removing all underscores and prefixing the result with |luatexUmath|.
%    \begin{macrocode}
\cs_new_protected_nopar:Nn \@@_font_param:n
%<XE>  { }
%<LU>  { \@@_font_param:nnnnn { #1 } { 0 } { 0 } { 0 } { 0 } }
%    \end{macrocode}
% \end{macro}
%
% \begin{macro}{\@@_font_param_aux:NNnnnn}
% \begin{macro}{\@@_font_param_aux:NNN}
% Auxiliary macros for generating font parameter accessor macros.
%    \begin{macrocode}
%<*XE>
\cs_new_protected_nopar:Nn \@@_font_param_aux:NNnnnn
  {
    \cs_new_nopar:Npn #1 ##1
     {
      \@@_font_dimen:Nnnnn ##1 { #3 } { #4 } { #5 } { #6 }
     }
    \cs_new_protected_nopar:Npn #2 ##1 ##2
     {
      #1 ##1 \dim_eval:n { ##2 }
     }
  }
\cs_generate_variant:Nn \@@_font_param_aux:NNnnnn { cc }
%</XE>
%<*LU>
\cs_new_protected_nopar:Nn \@@_font_param_aux:NNN
  {
    \cs_new_nopar:Npn #1 ##1
     {
      #3 ##1
     }
    \cs_new_protected_nopar:Npn #2 ##1 ##2
     {
      #3 ##1 \dim_eval:n { ##2 }
     }
  }
\cs_generate_variant:Nn \@@_font_param_aux:NNN { ccc }
%</LU>
%    \end{macrocode}
% \end{macro}
% \end{macro}
%
% Now all font parameters that are listed in the \LuaTeX\ reference follow.
%    \begin{macrocode}
\@@_font_param:nn { axis } { 15 }
\@@_font_param:nn { operator_size } { 13 }
\@@_font_param:n { fraction_del_size }
\@@_font_param:nnn { fraction_denom_down } { 45 } { 44 }
\@@_font_param:nnn { fraction_denom_vgap } { 50 } { 49 }
\@@_font_param:nnn { fraction_num_up } { 43 } { 42 }
\@@_font_param:nnn { fraction_num_vgap } { 47 } { 46 }
\@@_font_param:nn { fraction_rule } { 48 }
\@@_font_param:nn { limit_above_bgap } { 29 }
\@@_font_param:n { limit_above_kern }
\@@_font_param:nn { limit_above_vgap } { 28 }
\@@_font_param:nn { limit_below_bgap } { 31 }
\@@_font_param:n { limit_below_kern }
\@@_font_param:nn { limit_below_vgap } { 30 }
\@@_font_param:nn { over_delimiter_vgap } { 41 }
\@@_font_param:nn { over_delimiter_bgap } { 38 }
\@@_font_param:nn { under_delimiter_vgap } { 40 }
\@@_font_param:nn { under_delimiter_bgap } { 39 }
\@@_font_param:nn { overbar_kern } { 55 }
\@@_font_param:nn { overbar_rule } { 54 }
\@@_font_param:nn { overbar_vgap } { 53 }
\@@_font_param:n { quad }
\@@_font_param:nn { radical_kern } { 62 }
\@@_font_param:nn { radical_rule } { 61 }
\@@_font_param:nnn { radical_vgap } { 60 } { 59 }
\@@_font_param:nn { radical_degree_before } { 63 }
\@@_font_param:nn { radical_degree_after } { 64 }
\@@_font_param:nn { radical_degree_raise } { 65 }
\@@_font_param:nn { space_after_script } { 27 }
\@@_font_param:nnn { stack_denom_down } { 35 } { 34 }
\@@_font_param:nnn { stack_num_up } { 33 } { 32 }
\@@_font_param:nnn { stack_vgap } { 37 } { 36 }
\@@_font_param:nn { sub_shift_down } { 18 }
\@@_font_param:nn { sub_shift_drop } { 20 }
\@@_font_param:n { subsup_shift_down }
\@@_font_param:nn { sub_top_max } { 19 }
\@@_font_param:nn { subsup_vgap } { 25 }
\@@_font_param:nn { sup_bottom_min } { 23 }
\@@_font_param:nn { sup_shift_drop } { 24 }
\@@_font_param:nnnnn { sup_shift_up } { 21 } { 22 } { 21 } { 22 }
\@@_font_param:nn { supsub_bottom_max } { 26 }
\@@_font_param:nn { underbar_kern } { 58 }
\@@_font_param:nn { underbar_rule } { 57 }
\@@_font_param:nn { underbar_vgap } { 56 }
\@@_font_param:n { connector_overlap_min }
%    \end{macrocode}
%
% \section{Font features}
%
% \subsection{Math version}
%    \begin{macrocode}
\keys_define:nn {unicode-math}
  {
    version .code:n =
      {
        \tl_set:Nn \l_@@_mversion_tl {#1}
        \DeclareMathVersion {\l_@@_mversion_tl}
      }
  }
%    \end{macrocode}
%
% \subsection{Script and scriptscript font options}
%    \begin{macrocode}
\keys_define:nn {unicode-math}
 {
  script-features  .tl_set:N =  \l_@@_script_features_tl ,
  sscript-features .tl_set:N = \l_@@_sscript_features_tl ,
       script-font .tl_set:N =      \l_@@_script_font_tl ,
      sscript-font .tl_set:N =     \l_@@_sscript_font_tl ,
 }
%    \end{macrocode}
%
% \subsection{Range processing}
% \seclabel{rangeproc}
%
%    \begin{macrocode}
\keys_define:nn {unicode-math}
 {
  range .code:n =
   {
    \bool_set_false:N \l_@@_init_bool
%    \end{macrocode}
% Set processing functions if we're not defining the full Unicode math repetoire.
% Math symbols are defined with \cmd\_@@_sym:nnn; see \secref{mathsymbol}
% for the individual definitions
%    \begin{macrocode}
    \int_incr:N \g_@@_fam_int
    \tl_set:Nx \@@_symfont_tl {@@_fam\int_use:N\g_@@_fam_int}
    \cs_set_eq:NN \_@@_sym:nnn \@@_process_symbol_parse:nnn
    \cs_set_eq:NN \@@_set_mathalphabet_char:Nnn \@@_mathmap_parse:Nnn
    \cs_set_eq:NN \@@_remap_symbol:nnn \@@_remap_symbol_parse:nnn
    \cs_set_eq:NN \@@_maybe_init_alphabet:n \use_none:n
    \cs_set_eq:NN \@@_map_char_single:nn \@@_map_char_parse:nn
    \cs_set_eq:NN \@@_assign_delcode:nn \@@_assign_delcode_parse:nn
    \cs_set_eq:NN \@@_make_mathactive:nNN \@@_make_mathactive_parse:nNN
%    \end{macrocode}
% Proceed by filling up the various `range' seqs according to the user options.
%    \begin{macrocode}
    \seq_clear:N \l_@@_char_range_seq
    \seq_clear:N \l_@@_mclass_range_seq
    \seq_clear:N \l_@@_cmd_range_seq
    \seq_clear:N \l_@@_mathalph_seq

    \clist_map_inline:nn {#1}
     {
      \@@_if_mathalph_decl:nTF {##1}
       {
        \seq_put_right:Nx \l_@@_mathalph_seq
         {
          { \exp_not:V \l_@@_tmpa_tl }
          { \exp_not:V \l_@@_tmpb_tl }
          { \exp_not:V \l_@@_tmpc_tl }
         }
       }
       {
%    \end{macrocode}
% Four cases:
% math class matching the known list;
% single item that is a control sequence---command name;
% single item that isn't---edge case, must be 0--9;
% none of the above---char range.
%    \begin{macrocode}
        \seq_if_in:NnTF \g_@@_mathclasses_seq {##1}
          { \seq_put_right:Nn \l_@@_mclass_range_seq {##1} }
          {
            \bool_if:nTF { \tl_if_single_p:n {##1} && \token_if_cs_p:N ##1 }
              { \seq_put_right:Nn \l_@@_cmd_range_seq {##1} }
              { \seq_put_right:Nn \l_@@_char_range_seq {##1} }
          }
       }
     }
   }
 }
%    \end{macrocode}
%
%
% \begin{macro}{\@@_if_mathalph_decl:nTF}
% Possible forms of input:\\
% |\mathscr|\\
% |\mathscr->\mathup|\\
% |\mathscr/{Latin}|\\
% |\mathscr/{Latin}->\mathup|\\
% Outputs:\\
% |tmpa|: math style (\eg, |\mathscr|)\\
% |tmpb|: alphabets (\eg, |Latin|)\\
% |tmpc|: remap style (\eg, |\mathup|). Defaults to |tmpa|.
%
% The remap style can also be |\mathcal->stixcal|, which I marginally prefer
% in the general case.
%    \begin{macrocode}
\prg_new_conditional:Nnn \@@_if_mathalph_decl:n {TF}
 {
  \tl_set:Nn  \l_@@_tmpa_tl {#1}  
  \tl_clear:N \l_@@_tmpb_tl
  \tl_clear:N \l_@@_tmpc_tl
  
  \tl_if_in:NnT \l_@@_tmpa_tl {->}
   { \exp_after:wN \@@_split_arrow:w \l_@@_tmpa_tl \q_nil }
   
  \tl_if_in:NnT \l_@@_tmpa_tl {/}
   { \exp_after:wN \@@_split_slash:w \l_@@_tmpa_tl \q_nil }
  
  \tl_set:Nx \l_@@_tmpa_tl { \tl_to_str:N \l_@@_tmpa_tl }  
  \exp_args:NNx \tl_remove_all:Nn \l_@@_tmpa_tl { \token_to_str:N \math }
  \exp_args:NNx \tl_remove_all:Nn \l_@@_tmpa_tl { \token_to_str:N \sym }
  \tl_trim_spaces:N \l_@@_tmpa_tl

  \tl_if_empty:NT \l_@@_tmpc_tl
   { \tl_set_eq:NN \l_@@_tmpc_tl \l_@@_tmpa_tl }
  
  \seq_if_in:NVTF \g_@@_named_ranges_seq \l_@@_tmpa_tl
   { \prg_return_true: } { \prg_return_false: }
 }
%    \end{macrocode}
%    \begin{macrocode}
\cs_set:Npn \@@_split_arrow:w #1->#2 \q_nil
 {
  \tl_set:Nx \l_@@_tmpa_tl { \tl_trim_spaces:n {#1} }
  \tl_set:Nx \l_@@_tmpc_tl { \tl_trim_spaces:n {#2} }
 }
%    \end{macrocode}
%    \begin{macrocode}
\cs_set:Npn \@@_split_slash:w #1/#2 \q_nil
 {
  \tl_set:Nx \l_@@_tmpa_tl { \tl_trim_spaces:n {#1} }
  \tl_set:Nx \l_@@_tmpb_tl { \tl_trim_spaces:n {#2} }
 }
%    \end{macrocode}
% \end{macro}
%
% Pretty basic comma separated range processing.
% Donald Arseneau's \pkg{selectp} package has a cleverer technique.
%
% \begin{macro}{\@@_if_char_spec:nNNT}
% \darg{Unicode character slot}
% \darg{control sequence (character macro)}
% \darg{control sequence (math class)}
% \darg{code to execute}
% This macro expands to |#4|
% if any of its arguments are contained in \cmd\l_@@_char_range_seq.
% This list can contain either character ranges (for checking with |#1|) or control sequences.
% These latter can either be the command name of a specific character, \emph{or} the math
% type of one (\eg, \cmd\mathbin).
%
% Character ranges are passed to \cs{@@_if_char_spec:nNNT}, which accepts input in the form shown in \tabref{ranges}.
%
% \begin{table}[htbp]
% \centering
% \topcaption{Ranges accepted by \cs{@@_if_char_spec:nNNT}.}
% \label{tab:ranges}
% \begin{tabular}{>{\ttfamily}cc}
% \textrm{Input} & Range \\
% \hline
% x & $r=x$ \\
% x- & $r\geq x$ \\
% -y & $r\leq y$ \\
% x-y & $x \leq r \leq y$ \\
% \end{tabular}
% \end{table}
%
% We have three tests, performed sequentially in order of execution time.
% Any test finding a match jumps directly to the end.
%    \begin{macrocode}
\cs_new:Nn \@@_if_char_spec:nNNT
  {
    % math class:
    \seq_if_in:NnT \l_@@_mclass_range_seq {#3}
      { \use_none_delimit_by_q_nil:w }

    % command name:
    \seq_if_in:NnT \l_@@_cmd_range_seq {#2}
      { \use_none_delimit_by_q_nil:w }

    % character slot:
    \seq_map_inline:Nn \l_@@_char_range_seq
      {
        \@@_int_if_slot_in_range:nnT {#1} {##1}
          { \seq_map_break:n { \use_none_delimit_by_q_nil:w } }
      }

    % the following expands to nil if no match was found:
    \use_none:nnn
    \q_nil
    \use:n
      {
        \clist_put_right:Nx \l_@@_char_nrange_clist { \int_eval:n {#1} }
        #4
      }
  }
%    \end{macrocode}
% \end{macro}
%
% \begin{macro}{\@@_int_if_slot_in_range:nnT}
% A `numrange' is like |-2,5-8,12,17-| (can be unsorted).
%
% Four cases, four argument types:
% \begin{Verbatim}
% input    #2     #3      #4
% "1  "   [ 1] - [qn] - [   ] qs
% "1- "   [ 1] - [  ] - [qn-] qs
% " -3"   [  ] - [ 3] - [qn-] qs
% "1-3"   [ 1] - [ 3] - [qn-] qs
% \end{Verbatim}
%
%    \begin{macrocode}
\cs_new:Nn \@@_int_if_slot_in_range:nnT
  { \@@_numrange_parse:nwT {#1} #2 - \q_nil - \q_stop {#3} }
%    \end{macrocode}
%
%    \begin{macrocode}
\cs_set:Npn \@@_numrange_parse:nwT #1 #2 - #3 - #4 \q_stop #5
  {
    \tl_if_empty:nTF {#4} { \int_compare:nT {#1=#2} {#5} }
      {
    \tl_if_empty:nTF {#3} { \int_compare:nT {#1>=#2} {#5} }
      {
    \tl_if_empty:nTF {#2} { \int_compare:nT {#1<=#3} {#5} }
      {
    \int_compare:nT {#1>=#2} { \int_compare:nT {#1<=#3} {#5} }
      } } }
  }
%    \end{macrocode}
% \end{macro}
%
%
% \subsection{Resolving Greek symbol name control sequences}
%
% \begin{macro}{\@@_resolve_greek:}
% This macro defines \cmd\Alpha\dots\cmd\omega\ as their corresponding
% Unicode (mathematical italic) character. Remember that the mapping
% to upright or italic happens with the mathcode definitions, whereas these macros
% just stand for the literal Unicode characters.
%    \begin{macrocode}
\AtBeginDocument{\@@_resolve_greek:}
\cs_new:Npn \@@_resolve_greek:
 {
  \clist_map_inline:nn
   {
    Alpha,Beta,Gamma,Delta,Epsilon,Zeta,Eta,Theta,Iota,Kappa,Lambda,
    alpha,beta,gamma,delta,        zeta,eta,theta,iota,kappa,lambda,
    Mu,Nu,Xi,Omicron,Pi,Rho,Sigma,Tau,Upsilon,Phi,Chi,Psi,Omega,
    mu,nu,xi,omicron,pi,rho,sigma,tau,upsilon,    chi,psi,omega,
    varTheta,
    varsigma,vartheta,varkappa,varrho,varpi
   }
   {
    \tl_set:cx {##1} { \exp_not:c { mit ##1 } }
    \tl_set:cx {up ##1} { \exp_not:N \symup \exp_not:c { ##1 } }
    \tl_set:cx {it ##1} { \exp_not:N \symit \exp_not:c { ##1 } }
   }
  \tl_set:Nn \epsilon
   { \bool_if:NTF \g_@@_texgreek_bool \mitvarepsilon \mitepsilon }
  \tl_set:Nn \phi
   { \bool_if:NTF \g_@@_texgreek_bool \mitvarphi \mitphi }
  \tl_set:Nn \varepsilon
   { \bool_if:NTF \g_@@_texgreek_bool \mitepsilon \mitvarepsilon }
  \tl_set:Nn \varphi
   { \bool_if:NTF \g_@@_texgreek_bool \mitphi \mitvarphi }
 }
%    \end{macrocode}
% \end{macro}
%
%
%
%
%
%
%
% \section{Maths alphabets}
% \label{part:mathmap}
%
% Defining commands like \cmd\mathrm\ is not as simple with Unicode fonts.
% In traditional \TeX{} maths font setups, you simply switch between different `families' (\cmd\fam), which is analogous to changing from one font to another---a symbol such as `a' will be upright in one font, bold in another, and so on.
%
% In pkg{unicode-math}, a different mechanism is used to switch between styles. For every letter (start with ascii a-zA-Z and numbers to keep things simple for now), they are assigned a `mathcode' with \cmd\Umathcode\ that maps from input letter to output font glyph slot. This is done with the equivalent of
% \begin{Verbatim}
% \Umathcode`\a = 7 1 "1D44E\relax
% \Umathcode`\b = 7 1 "1D44F\relax
% \Umathcode`\c = 7 1 "1D450\relax
% ...
% \end{Verbatim}
% When switching from regular letters to, say, \cmd\mathrm, we now need to execute a new mapping:
% \begin{Verbatim}
% \Umathcode`\a = 7 1 `\a\relax
% \Umathcode`\b = 7 1 `\b\relax
% \Umathcode`\c = 7 1 `\c\relax
% ...
% \end{Verbatim}
% This is fairly straightforward to perform when we're defining our own commands such as \cmd\symbf\ and so on. However, this means that `classical' \TeX\ font setups will break, because with the original mapping still in place, the engine will be attempting to insert unicode maths glyphs from a standard font.
%
% \subsection{Hooks into \LaTeXe}
%
% To overcome this, we patch \cs{use@mathgroup}.
% (An alternative is to patch \cs{extract@alph@from@version}, which constructs the \cs{mathXYZ} commands, but this method fails if the command has been defined using \cs{DeclareSymbolFontAlphabet}.)
% As far as I can tell, this is only used inside of commands such as \cs{mathXYZ}, so this shouldn't have any major side-effects.
%
%    \begin{macrocode}
\cs_set:Npn \use@mathgroup #1 #2
 {
  \mode_if_math:T % <- not sure if this is really necessary since we've just checked for mmode and raised an error if not!
   {
    \math@bgroup
      \cs_if_eq:cNF {M@\f@encoding} #1 {#1}
      \@@_switchto_literal:
      \mathgroup #2 \relax
    \math@egroup
   }
 }
%    \end{macrocode}
%
%
%
% \subsection{Setting styles}
%
% Algorithm for setting alphabet fonts.
% By default, when |range| is empty, we are in \emph{implicit} mode.
% If |range| contains the name of the math alphabet, we are in \emph{explicit}
% mode and do things slightly differently.
%
% Implicit mode:
% \begin{itemize}
% \item Try and set all of the alphabet shapes.
% \item Check for the first glyph of each alphabet to detect if the font supports each
%       alphabet shape.
% \item For alphabets that do exist, overwrite whatever's already there.
% \item For alphabets that are not supported, \emph{do nothing}.
%       (This includes leaving the old alphabet definition in place.)
% \end{itemize}
%
% Explicit mode:
% \begin{itemize}
% \item Only set the alphabets specified.
% \item Check for the first glyph of the alphabet to detect if the font contains
%       the alphabet shape in the Unicode math plane.
% \item For Unicode math alphabets, overwrite whatever's already there.
% \item Otherwise, use the \ascii\ glyph slots instead.
% \end{itemize}
%
%
%
% \subsection{Defining the math style macros}
%
% We call the different shapes that a math alphabet can be a `math style'.
% Note that different alphabets can exist within the same math style. E.g.,
% we call `bold' the math style |bf| and within it there are upper and lower
% case Greek and Roman alphabets and Arabic numerals.
%
% \begin{macro}{\@@_prepare_mathstyle:n}
% \darg{math style name (e.g., \texttt{it} or \texttt{bb})}
% Define the high level math alphabet macros (\cs{mathit}, etc.) in terms of
% unicode-math definitions. Use \cs{bgroup}/\cs{egroup} so s'scripts scan the
% whole thing.
%
% The flag \cs{l_@@_mathstyle_tl} is for other applications to query the
% current math style.
%    \begin{macrocode}
\cs_new:Nn \@@_prepare_mathstyle:n
 {
  \seq_put_right:Nn \g_@@_mathstyles_seq {#1}
  \@@_init_alphabet:n {#1}
  \cs_set:cpn {_@@_sym_#1_aux:n}
   { \use:c {@@_switchto_#1:} \math@egroup }
  \cs_set_protected:cpx {sym#1}
   {
    \exp_not:n
     {
      \math@bgroup
      \mode_if_math:F
        {
          \egroup\expandafter
          \non@alpherr\expandafter{\csname sym#1\endcsname\space}
        }
      \tl_set:Nn \l_@@_mathstyle_tl {#1}
     }
    \exp_not:c {_@@_sym_#1_aux:n}
   }
 }
%    \end{macrocode}
% \end{macro}
%
%
% \begin{macro}{\@@_init_alphabet:n}
% \darg{math alphabet name (e.g., \texttt{it} or \texttt{bb})}
% This macro initialises the macros used to set up a math alphabet.
% First used when the math alphabet macro is first defined, but then used
% later when redefining a particular maths alphabet.
%    \begin{macrocode}
\cs_set:Nn \@@_init_alphabet:n
 {
  \@@_log:nx {alph-initialise} {#1}
  \cs_set_eq:cN {@@_switchto_#1:} \prg_do_nothing:
 }
%    \end{macrocode}
% \end{macro}
%
% \subsection{Definition of alphabets and styles}
%
% First of all, we break up unicode into `named ranges', such as |up|, |bb|, |sfup|, and so on, which refer to specific blocks of unicode that contain various symbols (usually alphabetical symbols).
%
%    \begin{macrocode}
\cs_new:Nn \@@_new_named_range:n
 {
  \prop_new:c {g_@@_named_range_#1_prop}
 }
\clist_set:Nn \g_@@_named_ranges_clist
 {
  up, it, tt, bfup, bfit, bb , bbit, scr, bfscr, cal, bfcal,
  frak, bffrak, sfup, sfit, bfsfup, bfsfit, bfsf 
 }
\clist_map_inline:Nn \g_@@_named_ranges_clist
 { \@@_new_named_range:n {#1} }
%    \end{macrocode}
%
% Each of these styles usually contains one or more `alphabets', which are currently |latin|, |Latin|, |greek|, |Greek|, |num|, and |misc|, although there's an implicit potential for more.
% |misc| is not included in the official list to avoid checking code.
%    \begin{macrocode}
\clist_new:N  \g_@@_alphabets_seq
\clist_set:Nn \g_@@_alphabets_seq { latin, Latin, greek, Greek, num }
%    \end{macrocode}
%
% Each alphabet style needs to be configured.
% This happens in the |unicode-math-alphabets.dtx| file.
%    \begin{macrocode}
\cs_new:Nn \@@_new_alphabet_config:nnn
 {
  \prop_if_exist:cF {g_@@_named_range_#1_prop}
   { \@@_warning:nnn {no-named-range} {#1} {#2} }

  \prop_gput:cnn {g_@@_named_range_#1_prop} { alpha_tl }
    {
     \prop_item:cn {g_@@_named_range_#1_prop} { alpha_tl }
     {#2}
    }
  % Q: do I need to bother removing duplicates?

  \cs_new:cn { @@_config_#1_#2:n } {#3}
 }
%    \end{macrocode}
%    \begin{macrocode}
\cs_new:Nn \@@_alphabet_config:nnn
 {
  \use:c {@@_config_#1_#2:n} {#3}
 }
%    \end{macrocode}
%    \begin{macrocode}
\prg_new_conditional:Nnn \@@_if_alphabet_exists:nn {T,TF}
 {
  \cs_if_exist:cTF {@@_config_#1_#2:n}
   \prg_return_true: \prg_return_false:
 }
%    \end{macrocode}
%
% The linking between named ranges and symbol style commands happens here.
% It's currently not using all of the machinery we're in the process of setting up above.
% Baby steps.
%    \begin{macrocode}
\cs_new:Nn \@@_default_mathalph:nnn
 {
  \seq_put_right:Nx \g_@@_named_ranges_seq { \tl_to_str:n {#1} }
  \seq_put_right:Nn \g_@@_default_mathalph_seq {{#1}{#2}{#3}}
  \prop_gput:cnn { g_@@_named_range_#1_prop } { default-alpha } {#2}
 }
\@@_default_mathalph:nnn {up    } {latin,Latin,greek,Greek,num,misc} {up    }
\@@_default_mathalph:nnn {it    } {latin,Latin,greek,Greek,misc}     {it    }
\@@_default_mathalph:nnn {bb    } {latin,Latin,num,misc}             {bb    }
\@@_default_mathalph:nnn {bbit  } {misc}                             {bbit  }
\@@_default_mathalph:nnn {scr   } {latin,Latin}                      {scr   }
\@@_default_mathalph:nnn {cal   } {Latin}                            {scr   }
\@@_default_mathalph:nnn {bfcal } {Latin}                            {bfscr }
\@@_default_mathalph:nnn {frak  } {latin,Latin}                      {frak  }
\@@_default_mathalph:nnn {tt    } {latin,Latin,num}                  {tt    }
\@@_default_mathalph:nnn {sfup  } {latin,Latin,num}                  {sfup  }
\@@_default_mathalph:nnn {sfit  } {latin,Latin}                      {sfit  }
\@@_default_mathalph:nnn {bfup  } {latin,Latin,greek,Greek,num,misc} {bfup  }
\@@_default_mathalph:nnn {bfit  } {latin,Latin,greek,Greek,misc}     {bfit  }
\@@_default_mathalph:nnn {bfscr } {latin,Latin}                      {bfscr }
\@@_default_mathalph:nnn {bffrak} {latin,Latin}                      {bffrak}
\@@_default_mathalph:nnn {bfsfup} {latin,Latin,greek,Greek,num,misc} {bfsfup}
\@@_default_mathalph:nnn {bfsfit} {latin,Latin,greek,Greek,misc}     {bfsfit}
%    \end{macrocode}
%
% \subsubsection{Define symbol style commands}
% Finally, all of the `symbol styles' commands are set up, which are the commands to access each of the named alphabet styles. There is not a one-to-one mapping between symbol style commands and named style ranges!
%    \begin{macrocode}
\clist_map_inline:nn
 {
  up, it, bfup, bfit, sfup, sfit, bfsfup, bfsfit, bfsf,
  tt, bb, bbit, scr, bfscr, cal, bfcal, frak, bffrak,
  normal, literal, sf, bf,
 }
 { \@@_prepare_mathstyle:n {#1} }
%    \end{macrocode}
%
%
% \subsubsection{New names for legacy textmath alphabet selection}
% In case a package option overwrites, say, \cs{mathbf} with \cs{symbf}.
%    \begin{macrocode}
\clist_map_inline:nn
 { rm, it, bf, sf, tt }
 { \cs_set_eq:cc { mathtext #1 } { math #1 } }
%    \end{macrocode}
% Perhaps these should actually be defined using a hypothetical unicode-math interface to creating new such styles. To come.
%
%
% \subsubsection{Replacing legacy pure-maths alphabets}
% The following are alphabets which do not have a math/text ambiguity.
%    \begin{macrocode}
\clist_map_inline:nn
 { 
   normal, bb , bbit, scr, bfscr, cal, bfcal, frak, bffrak, tt,
   bfup, bfit, sfup, sfit, bfsfup, bfsfit, bfsf
 }
 {
  \cs_set:cpx { math #1 } { \exp_not:c { sym #1 } }
 }
%    \end{macrocode}
%
%
% \subsubsection{New commands for ambiguous alphabets}
%    \begin{macrocode}
\AtBeginDocument{
\clist_map_inline:nn
 { rm, it, bf, sf, tt }
 {
  \cs_set_protected:cpx { math #1 }
   {
    \exp_not:n { \bool_if:NTF  } \exp_not:c { g_@@_ math #1 _text_bool}
     { \exp_not:c { mathtext #1 } }
     { \exp_not:c { sym #1 } }
   }
 }}
%    \end{macrocode}
%
% \paragraph{Alias \cs{mathrm} as legacy name for \cs{mathup}}
%    \begin{macrocode}
\cs_set_protected:Npn \mathup { \mathrm }
\cs_set_protected:Npn \symrm  { \symup  }
%    \end{macrocode}
%
%
% \subsubsection{Fixing up \cs{operator@font}}
%
%In LaTeX maths, the command |\operator@font| is defined that switches to the |operator| mathgroup. The classic example is the |\sin| in |$\sin{x}$|; essentially we're using |\mathrm| to typeset the upright symbols, but the syntax is |{\operator@font sin}|.
%
%It turns out that hooking into |\operator@font| is hard because all other maths font selection in 2e uses |\mathrm{...}| style.
%
%Then reading source2e a little more I stumbled upon: (in the definition of |\select@group|)
%\begin{quote}
% We surround |\select@group| with braces so that functions using it can be used directly after |_| or |^|. However, if we use oldstyle syntax where the math alphabet doesn’t have arguments (ie if |\math@bgroup| is not |\bgroup|) we need to get rid of the extra group.
%\end{quote}
%So there's a trick we can use.
%Because it's late and I'm tired, I went for the first thing that jumped out at me:
%\begin{Verbatim}
%    \documentclass{article}
%    \DeclareMathAlphabet\mathfoo{OT1}{lmdh}{m}{n}
%    \begin{document}
%    \makeatletter
%    ${\operator@font Mod}\, x$
%    
%    \def\operator@font{%
%      \let \math@bgroup \relax
%      \def \math@egroup {\let \math@bgroup \@@math@bgroup
%                         \let \math@egroup \@@math@egroup}%
%      \mathfoo}
%    ${\operator@font Mod}\, x$
%    \end{document}
%\end{Verbatim}
% We define a new math alphabet |\mathfoo| to select the Latin Modern Dunhill font, and then locally redefine |\math@bgroup| to allow |\mathfoo| to be used without an argument temporarily.
%
% Now that I've written this whole thing out, another solution pops to mind:
%\begin{Verbatim}
%    \documentclass{article}
%    \DeclareSymbolFont{foo}{OT1}{lmdh}{m}{n}
%    \DeclareSymbolFontAlphabet\mathfoo{foo}
%    \begin{document}
%    \makeatletter
%    ${\operator@font Mod}\, x$
%    
%    \def\operator@font{\mathgroup\symfoo}
%    ${\operator@font Mod}\, x$
%    \end{document}
%\end{Verbatim}
%I guess that's the better approach!!
%
% Or perhaps I should just use |\@fontswitch| to do the first solution with a nicer wrapper. I really should read things more carefully:
% \begin{macro}{\operator@font}
%    \begin{macrocode}
\cs_set:Npn \operator@font
 {
  \@@_switchto_literal:
  \@fontswitch {} { \g_@@_operator_mathfont_tl }
 }
%    \end{macrocode}
% \end{macro}
%
%
% \subsection{Defining the math alphabets per style}
%
% \begin{macro}{\@@_setup_alphabets:}
% This function is called within \cs{setmathfont} to configure the
% mapping between characters inside math styles.
%    \begin{macrocode}
\cs_new:Npn \@@_setup_alphabets:
 {
%    \end{macrocode}
% If |range=| has been used to configure styles, those choices will be in
% |\l_@@_mathalph_seq|. If not, set up the styles implicitly:
%    \begin{macrocode}
  \seq_if_empty:NTF \l_@@_mathalph_seq
   {
    \@@_log:n {setup-implicit}
    \seq_set_eq:NN \l_@@_mathalph_seq \g_@@_default_mathalph_seq
    \bool_set_true:N \l_@@_implicit_alph_bool
    \@@_maybe_init_alphabet:n  {sf}
    \@@_maybe_init_alphabet:n  {bf}
    \@@_maybe_init_alphabet:n  {bfsf}
   }
%    \end{macrocode}
% If |range=| has been used then we're in explicit mode:
%    \begin{macrocode}
   {
    \@@_log:n {setup-explicit}
    \bool_set_false:N \l_@@_implicit_alph_bool
    \cs_set_eq:NN \@@_set_mathalphabet_char:nnn \@@_mathmap_noparse:nnn
    \cs_set_eq:NN \@@_map_char_single:nn \@@_map_char_noparse:nn
   }

  % Now perform the mapping:
  \seq_map_inline:Nn \l_@@_mathalph_seq
   {
    \tl_set:No    \l_@@_style_tl       { \use_i:nnn   ##1 }
    \clist_set:No \l_@@_alphabet_clist { \use_ii:nnn  ##1 }
    \tl_set:No    \l_@@_remap_style_tl { \use_iii:nnn ##1 }

    % If no set of alphabets is defined:
    \clist_if_empty:NT \l_@@_alphabet_clist
     {
      \cs_set_eq:NN \@@_maybe_init_alphabet:n \@@_init_alphabet:n
      \prop_get:cnN { g_@@_named_range_ \l_@@_style_tl _prop }
       { default-alpha } \l_@@_alphabet_clist
     }
    
    \@@_setup_math_alphabet:
   }
  \seq_if_empty:NF \l_@@_missing_alph_seq { \@@_log:n { missing-alphabets } }
 }
%    \end{macrocode}
% \end{macro}
%
% \begin{macro}{\@@_setup_math_alphabet:}
%    \begin{macrocode}
\cs_new:Nn \@@_setup_math_alphabet:
 {
%    \end{macrocode}
% First check that at least one of the alphabets for the font shape is defined
% (this process is fast) \dots
%    \begin{macrocode}
  \clist_map_inline:Nn \l_@@_alphabet_clist
   {
    \tl_set:Nn \l_@@_alphabet_tl {##1}
    \@@_if_alphabet_exists:nnTF \l_@@_style_tl \l_@@_alphabet_tl
     {
      \str_if_eq_x:nnTF {\l_@@_alphabet_tl} {misc}
       {
        \@@_maybe_init_alphabet:n \l_@@_style_tl
        \clist_map_break:
       }
       {
        \@@_glyph_if_exist:nT { \@@_to_usv:nn {\l_@@_style_tl} {\l_@@_alphabet_tl} }
         {
          \@@_maybe_init_alphabet:n \l_@@_style_tl
          \clist_map_break:
         }
       }
     }
     { \msg_warning:nnx {unicode-math} {no-alphabet} { \l_@@_style_tl / \l_@@_alphabet_tl } }
   }
%    \end{macrocode}
% \dots and then loop through them defining the individual ranges:
% (currently this process is slow)
%    \begin{macrocode}
%<debug>  \csname TIC\endcsname
  \clist_map_inline:Nn \l_@@_alphabet_clist
   {
    \tl_set:Nx \l_@@_alphabet_tl { \tl_trim_spaces:n {##1} }
    \cs_if_exist:cT {@@_config_ \l_@@_style_tl _ \l_@@_alphabet_tl :n}
     {
      \exp_args:No \tl_if_eq:nnTF \l_@@_alphabet_tl {misc}
       {
        \@@_log:nx {setup-alph} {sym \l_@@_style_tl~(\l_@@_alphabet_tl)}
        \@@_alphabet_config:nnn {\l_@@_style_tl} {\l_@@_alphabet_tl} {\l_@@_remap_style_tl}
       }
       {
        \@@_glyph_if_exist:nTF { \@@_to_usv:nn {\l_@@_remap_style_tl} {\l_@@_alphabet_tl} }
         {
          \@@_log:nx {setup-alph} {sym \l_@@_style_tl~(\l_@@_alphabet_tl)}
          \@@_alphabet_config:nnn {\l_@@_style_tl} {\l_@@_alphabet_tl} {\l_@@_remap_style_tl}
         }
         {
          \bool_if:NTF \l_@@_implicit_alph_bool
           {
            \seq_put_right:Nx \l_@@_missing_alph_seq
             {
              \@backslashchar sym \l_@@_style_tl \space
              (\tl_use:c{c_@@_math_alphabet_name_ \l_@@_alphabet_tl _tl})
             }
           }
           {
            \@@_alphabet_config:nnn {\l_@@_style_tl} {\l_@@_alphabet_tl} {up}
           }
         }
       }
     }
   }
%<debug>  \csname TOC\endcsname
 }
%    \end{macrocode}
% \end{macro}
%
%
% \subsection{Mapping `naked' math characters}
%
% Before we show the definitions of the alphabet mappings using the functions
% |\@@_alphabet_config:nnn \l_@@_style_tl {##1} {...}|, we first want to define some functions
% to be used inside them to actually perform the character mapping.
%
% \subsubsection{Functions}
%
% \begin{macro}{\@@_map_char_single:nn}
% Wrapper for |\@@_map_char_noparse:nn| or |\@@_map_char_parse:nn|
% depending on the context.
%
% \begin{macro}{\@@_map_char_noparse:nn}
% \begin{macro}{\@@_map_char_parse:nn}
%    \begin{macrocode}
\cs_new:Nn \@@_map_char_noparse:nn
 { \@@_set_mathcode:nnnn {#1}{\mathalpha}{\@@_symfont_tl}{#2} }
%    \end{macrocode}
%
%    \begin{macrocode}
\cs_new:Nn \@@_map_char_parse:nn
 {
  \@@_if_char_spec:nNNT {#1} {\@nil} {\mathalpha}
   { \@@_map_char_noparse:nn {#1}{#2} }
 }
%    \end{macrocode}
% \end{macro}
% \end{macro}
% \end{macro}
%
% \begin{macro}{\@@_map_char_single:nnn}
% \darg{char name (`dotlessi')}
% \darg{from alphabet(s)}
% \darg{to alphabet}
% Logical interface to \cs{@@_map_char_single:nn}.
%    \begin{macrocode}
\cs_new:Nn \@@_map_char_single:nnn
 {
  \@@_map_char_single:nn { \@@_to_usv:nn {#1}{#3} }
                         { \@@_to_usv:nn {#2}{#3} }
 }
%    \end{macrocode}
% \end{macro}
%
%
% \begin{macro}{\@@_map_chars_range:nnnn}
% \darg{Number of chars (26)}
% \darg{From style, one or more (it)}
% \darg{To style (up)}
% \darg{Alphabet name (Latin)}
% First the function with numbers:
%    \begin{macrocode}
\cs_set:Nn \@@_map_chars_range:nnn
 {
  \int_step_inline:nnnn {0}{1}{#1-1}
   { \@@_map_char_single:nn {#2+##1}{#3+##1} }
 }
%    \end{macrocode}
% And the wrapper with names:
%    \begin{macrocode}
\cs_new:Nn \@@_map_chars_range:nnnn
 {
  \@@_map_chars_range:nnn {#1} { \@@_to_usv:nn {#2}{#4} }
                               { \@@_to_usv:nn {#3}{#4} }
 }
%    \end{macrocode}
% \end{macro}
%
% \subsubsection{Functions for `normal' alphabet symbols}
%
% \begin{macro}{\@@_set_normal_char:nnn}
%    \begin{macrocode}
\cs_set:Nn \@@_set_normal_char:nnn
 {
  \@@_usv_if_exist:nnT {#3} {#1}
  {
    \clist_map_inline:nn {#2}
     {
      \@@_set_mathalphabet_pos:nnnn {normal} {#1} {##1} {#3}
      \@@_map_char_single:nnn {##1} {#3} {#1}
     }
  }
 }
%    \end{macrocode}
% \end{macro}
%
%    \begin{macrocode}
\cs_new:Nn \@@_set_normal_Latin:nn
 {
  \clist_map_inline:nn {#1}
   {
    \@@_set_mathalphabet_Latin:nnn {normal} {##1} {#2}
    \@@_map_chars_range:nnnn {26} {##1} {#2} {Latin}
   }
 }
%    \end{macrocode}
%
%    \begin{macrocode}
\cs_new:Nn \@@_set_normal_latin:nn
 {
  \clist_map_inline:nn {#1}
   {
    \@@_set_mathalphabet_latin:nnn {normal} {##1} {#2}
    \@@_map_chars_range:nnnn {26} {##1} {#2} {latin}
   }
 }
%    \end{macrocode}
%
%    \begin{macrocode}
\cs_new:Nn \@@_set_normal_greek:nn
 {
  \clist_map_inline:nn {#1}
   {
    \@@_set_mathalphabet_greek:nnn {normal} {##1} {#2}
    \@@_map_chars_range:nnnn {25} {##1} {#2} {greek}
    \@@_map_char_single:nnn {##1} {#2} {varepsilon}
    \@@_map_char_single:nnn {##1} {#2} {vartheta}
    \@@_map_char_single:nnn {##1} {#2} {varkappa}
    \@@_map_char_single:nnn {##1} {#2} {varphi}
    \@@_map_char_single:nnn {##1} {#2} {varrho}
    \@@_map_char_single:nnn {##1} {#2} {varpi}
    \@@_set_mathalphabet_pos:nnnn {normal} {varepsilon} {##1} {#2}
    \@@_set_mathalphabet_pos:nnnn {normal} {vartheta} {##1} {#2}
    \@@_set_mathalphabet_pos:nnnn {normal} {varkappa} {##1} {#2}
    \@@_set_mathalphabet_pos:nnnn {normal} {varphi} {##1} {#2}
    \@@_set_mathalphabet_pos:nnnn {normal} {varrho} {##1} {#2}
    \@@_set_mathalphabet_pos:nnnn {normal} {varpi} {##1} {#2}
   }
 }
%    \end{macrocode}
%
%    \begin{macrocode}
\cs_new:Nn \@@_set_normal_Greek:nn
 {
  \clist_map_inline:nn {#1}
   {
    \@@_set_mathalphabet_Greek:nnn {normal} {##1} {#2}
    \@@_map_chars_range:nnnn {25} {##1} {#2} {Greek}
    \@@_map_char_single:nnn {##1} {#2} {varTheta}
    \@@_set_mathalphabet_pos:nnnn {normal} {varTheta} {##1} {#2}
   }
 }
%    \end{macrocode}
%
%    \begin{macrocode}
\cs_new:Nn \@@_set_normal_numbers:nn
 {
  \@@_set_mathalphabet_numbers:nnn {normal} {#1} {#2}
  \@@_map_chars_range:nnnn {10} {#1} {#2} {num}
 }
%    \end{macrocode}
%
%
% \subsection{Mapping chars inside a math style}
%
% \subsubsection{Functions for setting up the maths alphabets}
%
% \begin{macro}{\@@_set_mathalphabet_char:Nnn}
% This is a wrapper for either |\@@_mathmap_noparse:nnn| or
% |\@@_mathmap_parse:Nnn|, depending on the context.
% \end{macro}
%
% \begin{macro}{\@@_mathmap_noparse:nnn}
% \darg{Maths alphabet, \eg, `bb'}
% \darg{Input slot(s), \eg, the slot for `A' (comma separated)}
% \darg{Output slot, \eg, the slot for `$\mathbb{A}$'}
% Adds \cs{@@_set_mathcode:nnnn} declarations to the specified maths alphabet's definition.
%    \begin{macrocode}
\cs_new:Nn \@@_mathmap_noparse:nnn
 {
  \clist_map_inline:nn {#2}
   {
    \tl_put_right:cx {@@_switchto_#1:}
     {
      \@@_set_mathcode:nnnn {##1} {\mathalpha} {\@@_symfont_tl} {#3}
     }
   }
 }
%    \end{macrocode}
% \end{macro}
%
% \begin{macro}{\@@_mathmap_parse:nnn}
% \darg{Maths alphabet, \eg, `bb'}
% \darg{Input slot(s), \eg, the slot for `A' (comma separated)}
% \darg{Output slot, \eg, the slot for `$\mathbb{A}$'}
% When \cmd\@@_if_char_spec:nNNT\ is executed, it populates the \cmd\l_@@_char_nrange_clist\
% macro with slot numbers corresponding to the specified range. This range is used to
% conditionally add \cs{@@_set_mathcode:nnnn} declaractions to the maths alphabet definition.
%    \begin{macrocode}
\cs_new:Nn \@@_mathmap_parse:nnn
 {
  \clist_if_in:NnT \l_@@_char_nrange_clist {#3}
   {
    \@@_mathmap_noparse:nnn {#1}{#2}{#3}
   }
 }
%    \end{macrocode}
% \end{macro}
%
% \begin{macro}{\@@_set_mathalphabet_char:nnnn}
% \darg{math style command}
% \darg{input math alphabet name}
% \darg{output math alphabet name}
% \darg{char name to map}
%    \begin{macrocode}
\cs_new:Nn \@@_set_mathalphabet_char:nnnn
 {
  \@@_set_mathalphabet_char:nnn {#1} { \@@_to_usv:nn {#2} {#4} }
                                     { \@@_to_usv:nn {#3} {#4} }
 }
%    \end{macrocode}
% \end{macro}
%
% \begin{macro}{\@@_set_mathalph_range:nnnn}
% \darg{Number of iterations}
% \darg{Maths alphabet}
% \darg{Starting input char (single)}
% \darg{Starting output char}
% Loops through character ranges setting \cmd\mathcode.
% First the version that uses numbers:
%    \begin{macrocode}
\cs_new:Nn \@@_set_mathalph_range:nnnn
 {
  \int_step_inline:nnnn {0} {1} {#1-1}
    { \@@_set_mathalphabet_char:nnn {#2} { ##1 + #3 } { ##1 + #4 } }
 }
%    \end{macrocode}
% Then the wrapper version that uses names:
%    \begin{macrocode}
\cs_new:Nn \@@_set_mathalph_range:nnnnn
 {
  \@@_set_mathalph_range:nnnn {#1} {#2} { \@@_to_usv:nn {#3} {#5} }
                                        { \@@_to_usv:nn {#4} {#5} }
 }
%    \end{macrocode}
% \end{macro}
%
% \subsubsection{Individual mapping functions for different alphabets}
%
%    \begin{macrocode}
\cs_new:Nn \@@_set_mathalphabet_pos:nnnn
 {
  \@@_usv_if_exist:nnT {#4} {#2}
   {
    \clist_map_inline:nn {#3}
      { \@@_set_mathalphabet_char:nnnn {#1} {##1} {#4} {#2} }
   }
 }
%    \end{macrocode}
%
%    \begin{macrocode}
\cs_new:Nn \@@_set_mathalphabet_numbers:nnn
 {
  \clist_map_inline:nn {#2}
    { \@@_set_mathalph_range:nnnnn {10} {#1} {##1} {#3} {num} }
 }
%    \end{macrocode}
%
%    \begin{macrocode}
\cs_new:Nn \@@_set_mathalphabet_Latin:nnn
 {
  \clist_map_inline:nn {#2}
    { \@@_set_mathalph_range:nnnnn {26} {#1} {##1} {#3} {Latin} }
 }
%    \end{macrocode}
%
%    \begin{macrocode}
\cs_new:Nn \@@_set_mathalphabet_latin:nnn
 {
  \clist_map_inline:nn {#2}
   {
    \@@_set_mathalph_range:nnnnn {26} {#1} {##1} {#3} {latin}
    \@@_set_mathalphabet_char:nnnn    {#1} {##1} {#3} {h}
   }
 }
%    \end{macrocode}
%
%    \begin{macrocode}
\cs_new:Nn \@@_set_mathalphabet_Greek:nnn
 {
  \clist_map_inline:nn {#2}
   {
    \@@_set_mathalph_range:nnnnn {25} {#1} {##1} {#3} {Greek}
    \@@_set_mathalphabet_char:nnnn    {#1} {##1} {#3} {varTheta}
   }
 }
%    \end{macrocode}
%
%    \begin{macrocode}
\cs_new:Nn \@@_set_mathalphabet_greek:nnn
 {
  \clist_map_inline:nn {#2}
   {
    \@@_set_mathalph_range:nnnnn {25} {#1} {##1} {#3} {greek}
    \@@_set_mathalphabet_char:nnnn    {#1} {##1} {#3} {varepsilon}
    \@@_set_mathalphabet_char:nnnn    {#1} {##1} {#3} {vartheta}
    \@@_set_mathalphabet_char:nnnn    {#1} {##1} {#3} {varkappa}
    \@@_set_mathalphabet_char:nnnn    {#1} {##1} {#3} {varphi}
    \@@_set_mathalphabet_char:nnnn    {#1} {##1} {#3} {varrho}
    \@@_set_mathalphabet_char:nnnn    {#1} {##1} {#3} {varpi}
   }
 }
%    \end{macrocode}
%
%
%
% \section{A token list to contain the data of the math table}
%
% Instead of \cmd\input-ing the unicode math table every time we
% want to re-read its data, we save it within a macro. This has two
% advantages: 1.~it should be slightly faster, at the expense of memory;
% 2.~we don't need to worry about catcodes later, since they're frozen
% at this point.
%
% In time, the case statement inside |set_mathsymbol| will be moved in here
% to avoid re-running it every time.
%    \begin{macrocode}
\cs_new:Npn \@@_symbol_setup:
 {
  \cs_set:Npn \UnicodeMathSymbol ##1##2##3##4
   {
    \exp_not:n { \_@@_sym:nnn {##1} {##2} {##3} }
   }
 }
%    \end{macrocode}
%
%    \begin{macrocode}
\CatchFileEdef \g_@@_mathtable_tl {unicode-math-table.tex} {\@@_symbol_setup:}
%    \end{macrocode}
%
%
% \begin{macro}{\@@_input_math_symbol_table:}
% This function simply expands to the token list containing all the data.
%    \begin{macrocode}
\cs_new:Nn \@@_input_math_symbol_table: {\g_@@_mathtable_tl}
%    \end{macrocode}
% \end{macro}
%
%
% \section{Definitions of the active math characters}
%
% Here we define every Unicode math codepoint an equivalent macro name.
% The two are equivalent, in a |\let\xyz=^^^^1234| kind of way.
%
% \begin{macro}{\@@_cs_set_eq_active_char:Nw}
% \begin{macro}{\@@_active_char_set:wc}
% We need to do some trickery to transform the |\_@@_sym:nnn| argument
% |"ABCDEF| into the \XeTeX\ `caret input' form |^^^^^abcdef|. It is \emph{very important}
% that the argument has five characters. Otherwise we need to change the number of |^| chars.
%
% To do this, turn |^| into a regular `other' character and define the macro
% to perform the lowercasing and |\let|. \cmd\scantokens\ changes the carets
% back into their original meaning after the group has ended and |^|'s catcode returns to normal.
%    \begin{macrocode}
\group_begin:
  \char_set_catcode_other:N \^
  \cs_gset:Npn \@@_cs_set_eq_active_char:Nw #1 = "#2 \q_nil
   {
    \tex_lowercase:D
     {
      \tl_rescan:nn
       {
        \ExplSyntaxOn
        \char_set_catcode_other:N \{
        \char_set_catcode_other:N \}
        \char_set_catcode_other:N \&
        \char_set_catcode_other:N \%
        \char_set_catcode_other:N \$
       }
       {
        \cs_gset_eq:NN #1 ^^^^^#2
       }
     }
   }
%    \end{macrocode}
% Making |^| the right catcode isn't strictly necessary right now but it helps
% to future proof us with, e.g., breqn.
% Because we're inside a |\tl_rescan:nn|, use plain old \TeX\ syntax to avoid
% any catcode problems.
%    \begin{macrocode}
  \cs_new:Npn \@@_active_char_set:wc "#1 \q_nil #2
   {
    \tex_lowercase:D
     {
      \tl_rescan:nn { \ExplSyntaxOn }
        { \cs_gset_protected_nopar:Npx ^^^^^#1 { \exp_not:c {#2} } }
     }
   }
\group_end:
%    \end{macrocode}
% \end{macro}
% \end{macro}
%
% Now give \cmd\_@@_sym:nnn\ a definition in terms of \cmd\@@_cs_set_eq_active_char:Nw\
% and we're good to go.
%
% Ensure catcodes are appropriate;
% make sure |#| is an `other' so that we don't get confused with \cs{mathoctothorpe}.
%    \begin{macrocode}
\AtBeginDocument{\@@_define_math_chars:}
\cs_new:Nn \@@_define_math_chars:
 {
  \group_begin:
    \char_set_catcode_math_superscript:N \^
    \cs_set:Npn \_@@_sym:nnn ##1##2##3
     {
      \tl_if_in:nnT 
       { \mathord \mathalpha \mathbin \mathrel \mathpunct \mathop \mathfence }
       {##3}
      {
        \@@_cs_set_eq_active_char:Nw ##2 = ##1 \q_nil \ignorespaces
      }
     }
    \char_set_catcode_other:N \#
    \@@_input_math_symbol_table:
  \group_end:
 }
%    \end{macrocode}
% Fix \cs{backslash}, which is defined as the escape char character
% above:
%    \begin{macrocode}
\group_begin:
  \lccode`\*=`\\
  \char_set_catcode_escape:N \|
  \char_set_catcode_other:N \\
  |lowercase
   {
    |AtBeginDocument
     {
      |let|backslash=*
     }
   }
|group_end:
%    \end{macrocode}
%
% \section{Fall-back font}
%
% Want to load Latin Modern Math if nothing else.
% Reset the `font already loaded' boolean so that a new font being set will do the right thing.
% TODO: need a better way to do this for the general case.
%    \begin{macrocode}
\AtBeginDocument { \@@_load_lm_if_necessary: }
\cs_new:Nn \@@_load_lm_if_necessary:
  {
    \cs_if_exist:NF \l_@@_fontname_tl
      {
        % TODO: update this when lmmath-bold.otf is released
        \setmathfont{latinmodern-math.otf}[BoldFont={latinmodern-math.otf}]
        \bool_set_false:N \g_@@_mainfont_already_set_bool
      }
  }
%    \end{macrocode}
%
% \section{Epilogue}
%
% Lots of little things to tidy up.
%
% \subsection{Primes}
%
% We need a new `prime' algorithm. Unicode math has four pre-drawn prime glyphs.
% \begin{quote}\obeylines
% \unichar{2032} {prime} (\cs{prime}): $x\prime$
% \unichar{2033} {double prime} (\cs{dprime}): $x\dprime$
% \unichar{2034} {triple prime} (\cs{trprime}): $x\trprime$
% \unichar{2057} {quadruple prime} (\cs{qprime}): $x\qprime$
% \end{quote}
% As you can see, they're all drawn at the correct height without being superscripted.
% However, in a correctly behaving OpenType font,
% we also see different behaviour after the \texttt{ssty} feature is applied:
% \begin{quote}
% \font\1="Cambria Math:script=math,+ssty=0"\1
% \char"1D465\char"2032\quad
% \char"1D465\char"2033\quad
% \char"1D465\char"2034\quad
% \char"1D465\char"2057
% \end{quote}
% The glyphs are now `full size' so that when placed inside a superscript,
% their shape will match the originally sized ones. Many thanks to Ross Mills
% of Tiro Typeworks for originally pointing out this behaviour.
%
% In regular \LaTeX, primes can be entered with the straight quote character
% |'|, and multiple straight quotes chain together to produce multiple
% primes. Better results can be achieved in \pkg{unicode-math} by chaining
% multiple single primes into a pre-drawn multi-prime glyph; consider
% $x\prime{}\prime{}\prime$ vs.\ $x\trprime$.
%
% For Unicode maths, we wish to conserve this behaviour and augment it with
% the possibility of adding any combination of Unicode prime or any of the
% $n$-prime characters. E.g., the user might copy-paste a double prime from
% another source and then later type another single prime after it; the output
% should be the triple prime.
%
% Our algorithm is:
% \begin{itemize}[nolistsep]
% \item Prime encountered; pcount=1.
% \item Scan ahead; if prime: pcount:=pcount+1; repeat.
% \item If not prime, stop scanning.
% \item If pcount=1, \cs{prime}, end.
% \item If pcount=2, check \cs{dprime}; if it exists, use it, end; if not, goto last step.
% \item Ditto pcount=3 \& \cs{trprime}.
% \item Ditto pcount=4 \& \cs{qprime}.
% \item If pcount>4 or the glyph doesn't exist, insert pcount \cs{prime}s with \cs{primekern} between each.
% \end{itemize}
%
% This is a wrapper to insert a superscript; if there is a subsequent
% trailing superscript, then it is included within the insertion.
%    \begin{macrocode}
\cs_new:Nn \@@_arg_i_before_egroup:n {#1\egroup}
\cs_new:Nn \@@_superscript:n
 {
  ^\bgroup #1
  \peek_meaning_remove:NTF ^ \@@_arg_i_before_egroup:n \egroup
 }
%    \end{macrocode}
%
%    \begin{macrocode}
\cs_new:Nn \@@_nprimes:Nn
 {
  \@@_superscript:n
   {
    #1
    \prg_replicate:nn {#2-1} { \mskip \g_@@_primekern_muskip #1 }
   }
 }

\cs_new:Nn \@@_nprimes_select:nn
 {
  \int_case:nnF {#2}
   {
    {1} { \@@_superscript:n {#1} }
    {2} {
      \@@_glyph_if_exist:nTF {"2033}
        { \@@_superscript:n {\@@_prime_double_mchar} }
        { \@@_nprimes:Nn #1 {#2} }
    }
    {3} {
      \@@_glyph_if_exist:nTF {"2034}
        { \@@_superscript:n {\@@_prime_triple_mchar} }
        { \@@_nprimes:Nn #1 {#2} }
    }
    {4} {
      \@@_glyph_if_exist:nTF {"2057}
        { \@@_superscript:n {\@@_prime_quad_mchar} }
        { \@@_nprimes:Nn #1 {#2} }
    }
   }
   {
    \@@_nprimes:Nn #1 {#2}
   }
 }
\cs_new:Nn \@@_nbackprimes_select:nn
 {
  \int_case:nnF {#2}
   {
    {1} { \@@_superscript:n {#1} }
    {2} {
      \@@_glyph_if_exist:nTF {"2036}
        { \@@_superscript:n {\@@_backprime_double_mchar} }
        { \@@_nprimes:Nn #1 {#2} }
    }
    {3} {
      \@@_glyph_if_exist:nTF {"2037}
        { \@@_superscript:n {\@@_backprime_triple_mchar} }
        { \@@_nprimes:Nn #1 {#2} }
    }
   }
   {
    \@@_nprimes:Nn #1 {#2}
   }
 }
%    \end{macrocode}
%
% Scanning is annoying because I'm too lazy to do it for the general case.
%
%    \begin{macrocode}
\cs_new:Npn \@@_scan_prime:
 {
  \cs_set_eq:NN \@@_superscript:n \use:n
  \int_zero:N \l_@@_primecount_int
  \@@_scanprime_collect:N \@@_prime_single_mchar
 }
\cs_new:Npn \@@_scan_dprime:
 {
  \cs_set_eq:NN \@@_superscript:n \use:n
  \int_set:Nn \l_@@_primecount_int {1}
  \@@_scanprime_collect:N \@@_prime_single_mchar
 }
\cs_new:Npn \@@_scan_trprime:
 {
  \cs_set_eq:NN \@@_superscript:n \use:n
  \int_set:Nn \l_@@_primecount_int {2}
  \@@_scanprime_collect:N \@@_prime_single_mchar
 }
\cs_new:Npn \@@_scan_qprime:
 {
  \cs_set_eq:NN \@@_superscript:n \use:n
  \int_set:Nn \l_@@_primecount_int {3}
  \@@_scanprime_collect:N \@@_prime_single_mchar
 }
\cs_new:Npn \@@_scan_sup_prime:
 {
  \int_zero:N \l_@@_primecount_int
  \@@_scanprime_collect:N \@@_prime_single_mchar
 }
\cs_new:Npn \@@_scan_sup_dprime:
 {
  \int_set:Nn \l_@@_primecount_int {1}
  \@@_scanprime_collect:N \@@_prime_single_mchar
 }
\cs_new:Npn \@@_scan_sup_trprime:
 {
  \int_set:Nn \l_@@_primecount_int {2}
  \@@_scanprime_collect:N \@@_prime_single_mchar
 }
\cs_new:Npn \@@_scan_sup_qprime:
 {
  \int_set:Nn \l_@@_primecount_int {3}
  \@@_scanprime_collect:N \@@_prime_single_mchar
 }
\cs_new:Nn \@@_scanprime_collect:N
 {
  \int_incr:N \l_@@_primecount_int
  \peek_meaning_remove:NTF ' 
   { \@@_scanprime_collect:N #1 }
   {
    \peek_meaning_remove:NTF \@@_scan_prime:
     { \@@_scanprime_collect:N #1 }
     {
      \peek_meaning_remove:NTF ^^^^2032
       { \@@_scanprime_collect:N #1 }
       {
        \peek_meaning_remove:NTF \@@_scan_dprime:
         {
          \int_incr:N \l_@@_primecount_int
          \@@_scanprime_collect:N #1
         }
         {
          \peek_meaning_remove:NTF ^^^^2033
           {
            \int_incr:N \l_@@_primecount_int
            \@@_scanprime_collect:N #1
           }
           {
            \peek_meaning_remove:NTF \@@_scan_trprime:
             {
              \int_add:Nn \l_@@_primecount_int {2}
              \@@_scanprime_collect:N #1
             }
             {
              \peek_meaning_remove:NTF ^^^^2034
               {
                \int_add:Nn \l_@@_primecount_int {2}
                \@@_scanprime_collect:N #1
               }
               {
                \peek_meaning_remove:NTF \@@_scan_qprime:
                 {
                  \int_add:Nn \l_@@_primecount_int {3}
                  \@@_scanprime_collect:N #1
                 }
                 {
                  \peek_meaning_remove:NTF ^^^^2057
                   {
                    \int_add:Nn \l_@@_primecount_int {3}
                    \@@_scanprime_collect:N #1
                   }
                   {
                    \@@_nprimes_select:nn {#1} {\l_@@_primecount_int}
                   }
                 }
               }
             }
           }
         }
       }
     }
   }
 }
\cs_new:Npn \@@_scan_backprime:
 {
  \cs_set_eq:NN \@@_superscript:n \use:n
  \int_zero:N \l_@@_primecount_int
  \@@_scanbackprime_collect:N \@@_backprime_single_mchar
 }
\cs_new:Npn \@@_scan_backdprime:
 {
  \cs_set_eq:NN \@@_superscript:n \use:n
  \int_set:Nn \l_@@_primecount_int {1}
  \@@_scanbackprime_collect:N \@@_backprime_single_mchar
 }
\cs_new:Npn \@@_scan_backtrprime:
 {
  \cs_set_eq:NN \@@_superscript:n \use:n
  \int_set:Nn \l_@@_primecount_int {2}
  \@@_scanbackprime_collect:N \@@_backprime_single_mchar
 }
\cs_new:Npn \@@_scan_sup_backprime:
 {
  \int_zero:N \l_@@_primecount_int
  \@@_scanbackprime_collect:N \@@_backprime_single_mchar
 }
\cs_new:Npn \@@_scan_sup_backdprime:
 {
  \int_set:Nn \l_@@_primecount_int {1}
  \@@_scanbackprime_collect:N \@@_backprime_single_mchar
 }
\cs_new:Npn \@@_scan_sup_backtrprime:
 {
  \int_set:Nn \l_@@_primecount_int {2}
  \@@_scanbackprime_collect:N \@@_backprime_single_mchar
 }
\cs_new:Nn \@@_scanbackprime_collect:N
 {
  \int_incr:N \l_@@_primecount_int
  \peek_meaning_remove:NTF `
   {
    \@@_scanbackprime_collect:N #1
   }
   {
    \peek_meaning_remove:NTF \@@_scan_backprime:
     {
      \@@_scanbackprime_collect:N #1
     }
     {
      \peek_meaning_remove:NTF ^^^^2035
       {
        \@@_scanbackprime_collect:N #1
       }
       {
        \peek_meaning_remove:NTF \@@_scan_backdprime:
         {
          \int_incr:N \l_@@_primecount_int
          \@@_scanbackprime_collect:N #1
         }
         {
          \peek_meaning_remove:NTF ^^^^2036
           {
            \int_incr:N \l_@@_primecount_int
            \@@_scanbackprime_collect:N #1
           }
           {
            \peek_meaning_remove:NTF \@@_scan_backtrprime:
             {
              \int_add:Nn \l_@@_primecount_int {2}
              \@@_scanbackprime_collect:N #1
             }
             {
              \peek_meaning_remove:NTF ^^^^2037
               {
                \int_add:Nn \l_@@_primecount_int {2}
                \@@_scanbackprime_collect:N #1
               }
               {
                \@@_nbackprimes_select:nn {#1} {\l_@@_primecount_int}
               }
             }
           }
         }
       }
     }
   }
 }
%    \end{macrocode}
%
%    \begin{macrocode}
\AtBeginDocument{\@@_define_prime_commands: \@@_define_prime_chars:}
\cs_new:Nn \@@_define_prime_commands:
 {
  \cs_set_eq:NN \prime       \@@_prime_single_mchar
  \cs_set_eq:NN \dprime      \@@_prime_double_mchar
  \cs_set_eq:NN \trprime     \@@_prime_triple_mchar
  \cs_set_eq:NN \qprime      \@@_prime_quad_mchar
  \cs_set_eq:NN \backprime   \@@_backprime_single_mchar
  \cs_set_eq:NN \backdprime  \@@_backprime_double_mchar
  \cs_set_eq:NN \backtrprime \@@_backprime_triple_mchar
 }
\group_begin:
  \char_set_catcode_active:N \'
  \char_set_catcode_active:N \`
  \char_set_catcode_active:n {"2032}
  \char_set_catcode_active:n {"2033}
  \char_set_catcode_active:n {"2034}
  \char_set_catcode_active:n {"2057}
  \char_set_catcode_active:n {"2035}
  \char_set_catcode_active:n {"2036}
  \char_set_catcode_active:n {"2037}
  \cs_gset:Nn \@@_define_prime_chars:
   {
    \cs_set_eq:NN '        \@@_scan_sup_prime:
    \cs_set_eq:NN ^^^^2032 \@@_scan_sup_prime:
    \cs_set_eq:NN ^^^^2033 \@@_scan_sup_dprime:
    \cs_set_eq:NN ^^^^2034 \@@_scan_sup_trprime:
    \cs_set_eq:NN ^^^^2057 \@@_scan_sup_qprime:
    \cs_set_eq:NN `        \@@_scan_sup_backprime:
    \cs_set_eq:NN ^^^^2035 \@@_scan_sup_backprime:
    \cs_set_eq:NN ^^^^2036 \@@_scan_sup_backdprime:
    \cs_set_eq:NN ^^^^2037 \@@_scan_sup_backtrprime:
   }
\group_end:
%    \end{macrocode}
%
% \subsection{Unicode radicals}
%
%    \begin{macrocode}
\AtBeginDocument{\@@_redefine_radical:}
\cs_new:Nn \@@_redefine_radical:
%<*XE>
 {
  \@ifpackageloaded { amsmath } { }
   {
%    \end{macrocode}
% \begin{macro}{\r@@t}
% \darg{A mathstyle (for \cmd\mathpalette)}
% \darg{Leading superscript for the sqrt sign}
% A re-implementation of \LaTeX's hard-coded n-root sign using the appropriate \cmd\fontdimen s.
%    \begin{macrocode}
    \cs_set_nopar:Npn \r@@@@t ##1 ##2
     {
      \hbox_set:Nn \l_tmpa_box
       {
        \c_math_toggle_token
        \m@th
        ##1
        \sqrtsign { ##2 }
        \c_math_toggle_token
       }
      \@@_mathstyle_scale:Nnn ##1 { \kern }
       { \fontdimen 63 \l_@@_font }
      \box_move_up:nn
       {
        (\box_ht:N \l_tmpa_box - \box_dp:N \l_tmpa_box)
        * \number \fontdimen 65 \l_@@_font / 100
       }
       { \box_use:N \rootbox }
      \@@_mathstyle_scale:Nnn ##1 { \kern }
       { \fontdimen 64 \l_@@_font }
      \box_use_clear:N \l_tmpa_box
     }
%    \end{macrocode}
% \end{macro}
%    \begin{macrocode}
   }
 }
%</XE>
%<*LU>
 {
  \@ifpackageloaded { amsmath } { }
   {
%    \end{macrocode}
% \begin{macro}{\root}
% Redefine this macro for \LuaTeX, which provides us a nice primitive to use.
%    \begin{macrocode}
    \cs_set:Npn \root ##1 \of ##2
     {
      \luatexUroot \l_@@_radical_sqrt_tl { ##1 } { ##2 }
     }
%    \end{macrocode}
% \end{macro}
%    \begin{macrocode}
   }
 }
%</LU>
%    \end{macrocode}
%
%
% \begin{macro}{\@@_fontdimen_to_percent:nn}
% \begin{macro}{\@@_fontdimen_to_scale:nn}
% \darg{Font dimen number}
% \darg{Font `variable'}
% \cmd\fontdimen s |10|, |11|, and |65| aren't actually dimensions, they're percentage values given in units of |sp|.
% \cs{@@_fontdimen_to_percent:nn} takes a font dimension number and outputs the decimal value of the associated parameter.
% \cs{@@_fontdimen_to_scale:nn} returns a dimension correspond to the current
% font size relative proportion based on that percentage.
%    \begin{macrocode}
\cs_new:Nn \@@_fontdimen_to_percent:nn
 {
  \fp_eval:n { \dim_to_decimal:n { \fontdimen #1 #2 } * 65536 / 100 }
 }
\cs_new:Nn \@@_fontdimen_to_scale:nn
 {
  \fp_eval:n {\@@_fontdimen_to_percent:nn {#1} {#2} * \f@size } pt
 }
%    \end{macrocode}
% \end{macro}
% \end{macro}
%
% \begin{macro}{\@@_mathstyle_scale:Nnn}
% \darg{A math style (\cs{scriptstyle}, say)}
% \darg{Macro that takes a non-delimited length argument (like \cmd\kern)}
% \darg{Length control sequence to be scaled according to the math style}
% This macro is used to scale the lengths reported by \cmd\fontdimen\ according to the scale factor for script- and scriptscript-size objects.
%    \begin{macrocode}
\cs_new:Nn \@@_mathstyle_scale:Nnn
 {
  \ifx#1\scriptstyle
    #2 \@@_fontdimen_to_percent:nn {10} \l_@@_font #3
  \else
    \ifx#1\scriptscriptstyle
      #2 \@@_fontdimen_to_percent:nn {11} \l_@@_font #3
    \else
      #2 #3
    \fi
  \fi
 }
%    \end{macrocode}
% \end{macro}
%
% \subsection{Unicode sub- and super-scripts}
%
% The idea here is to enter a scanning state after a superscript or subscript
% is encountered.
% If subsequent superscripts or subscripts (resp.) are found,
% they are lumped together.
% Each sub/super has a corresponding regular size
% glyph which is used by \XeTeX\ to typeset the results; this means that the
% actual subscript/superscript glyphs are never seen in the output
% document~--- they are only used as input characters.
%
% Open question: should the superscript-like `modifiers' (\unichar{1D2C}
% {modifier capital letter a} and on) be included here?
%    \begin{macrocode}
\group_begin:
%    \end{macrocode}
% \paragraph{Superscripts}
% Populate a property list with superscript characters; their meaning as their
% key, for reasons that will become apparent soon, and their replacement as
% each key's value.
% Then make the superscript active and bind it to the scanning function.
%
% \cs{scantokens} makes this process much simpler since we can activate the
% char and assign its meaning in one step.
%    \begin{macrocode}
\cs_new:Nn \@@_setup_active_superscript:nn
 {
  \prop_gput:Non \g_@@_supers_prop   {\meaning #1} {#2}
  \char_set_catcode_active:N #1
  \@@_char_gmake_mathactive:N #1
  \scantokens
   {
    \cs_gset:Npn #1
     {
      \tl_set:Nn \l_@@_ss_chain_tl {#2}
      \cs_set_eq:NN \@@_sub_or_super:n \sp
      \tl_set:Nn \l_@@_tmpa_tl {supers}
      \@@_scan_sscript:
     }
   }
 }
%    \end{macrocode}
% Bam:
%    \begin{macrocode}
\@@_setup_active_superscript:nn {^^^^2070} {0}
\@@_setup_active_superscript:nn {^^^^00b9} {1}
\@@_setup_active_superscript:nn {^^^^00b2} {2}
\@@_setup_active_superscript:nn {^^^^00b3} {3}
\@@_setup_active_superscript:nn {^^^^2074} {4}
\@@_setup_active_superscript:nn {^^^^2075} {5}
\@@_setup_active_superscript:nn {^^^^2076} {6}
\@@_setup_active_superscript:nn {^^^^2077} {7}
\@@_setup_active_superscript:nn {^^^^2078} {8}
\@@_setup_active_superscript:nn {^^^^2079} {9}
\@@_setup_active_superscript:nn {^^^^207a} {+}
\@@_setup_active_superscript:nn {^^^^207b} {-}
\@@_setup_active_superscript:nn {^^^^207c} {=}
\@@_setup_active_superscript:nn {^^^^207d} {(}
\@@_setup_active_superscript:nn {^^^^207e} {)}
\@@_setup_active_superscript:nn {^^^^2071} {i}
\@@_setup_active_superscript:nn {^^^^207f} {n}
\@@_setup_active_superscript:nn {^^^^02b0} {h}
\@@_setup_active_superscript:nn {^^^^02b2} {j}
\@@_setup_active_superscript:nn {^^^^02b3} {r}
\@@_setup_active_superscript:nn {^^^^02b7} {w}
\@@_setup_active_superscript:nn {^^^^02b8} {y}
%    \end{macrocode}
% \paragraph{Subscripts} Ditto above.
%    \begin{macrocode}
\cs_new:Nn \@@_setup_active_subscript:nn
 {
  \prop_gput:Non \g_@@_subs_prop   {\meaning #1} {#2}
  \char_set_catcode_active:N #1
  \@@_char_gmake_mathactive:N #1
  \scantokens
   {
    \cs_gset:Npn #1
     {
      \tl_set:Nn \l_@@_ss_chain_tl {#2}
      \cs_set_eq:NN \@@_sub_or_super:n \sb
      \tl_set:Nn \l_@@_tmpa_tl {subs}
      \@@_scan_sscript:
     }
   }
 }
%    \end{macrocode}
% A few more subscripts than superscripts:
%    \begin{macrocode}
\@@_setup_active_subscript:nn {^^^^2080} {0}
\@@_setup_active_subscript:nn {^^^^2081} {1}
\@@_setup_active_subscript:nn {^^^^2082} {2}
\@@_setup_active_subscript:nn {^^^^2083} {3}
\@@_setup_active_subscript:nn {^^^^2084} {4}
\@@_setup_active_subscript:nn {^^^^2085} {5}
\@@_setup_active_subscript:nn {^^^^2086} {6}
\@@_setup_active_subscript:nn {^^^^2087} {7}
\@@_setup_active_subscript:nn {^^^^2088} {8}
\@@_setup_active_subscript:nn {^^^^2089} {9}
\@@_setup_active_subscript:nn {^^^^208a} {+}
\@@_setup_active_subscript:nn {^^^^208b} {-}
\@@_setup_active_subscript:nn {^^^^208c} {=}
\@@_setup_active_subscript:nn {^^^^208d} {(}
\@@_setup_active_subscript:nn {^^^^208e} {)}
\@@_setup_active_subscript:nn {^^^^2090} {a}
\@@_setup_active_subscript:nn {^^^^2091} {e}
\@@_setup_active_subscript:nn {^^^^1d62} {i}
\@@_setup_active_subscript:nn {^^^^2092} {o}
\@@_setup_active_subscript:nn {^^^^1d63} {r}
\@@_setup_active_subscript:nn {^^^^1d64} {u}
\@@_setup_active_subscript:nn {^^^^1d65} {v}
\@@_setup_active_subscript:nn {^^^^2093} {x}
\@@_setup_active_subscript:nn {^^^^1d66} {\beta}
\@@_setup_active_subscript:nn {^^^^1d67} {\gamma}
\@@_setup_active_subscript:nn {^^^^1d68} {\rho}
\@@_setup_active_subscript:nn {^^^^1d69} {\phi}
\@@_setup_active_subscript:nn {^^^^1d6a} {\chi}
%    \end{macrocode}
%
%    \begin{macrocode}
\group_end:
%    \end{macrocode}
% The scanning command, evident in its purpose:
%    \begin{macrocode}
\cs_new:Npn \@@_scan_sscript:
 {
  \@@_scan_sscript:TF
   {
    \@@_scan_sscript:
   }
   {
    \@@_sub_or_super:n {\l_@@_ss_chain_tl}
   }
 }
%    \end{macrocode}
% The main theme here is stolen from the source to the various \cs{peek_} functions.
% Consider this function as simply boilerplate:
% TODO: move all this to expl3, and don't use internal expl3 macros.
%    \begin{macrocode}
\cs_new:Npn \@@_scan_sscript:TF #1#2
 {
  \tl_set:Nx \__peek_true_aux:w { \exp_not:n{ #1 } }
  \tl_set_eq:NN \__peek_true:w \__peek_true_remove:w
  \tl_set:Nx \__peek_false:w { \exp_not:n { \group_align_safe_end: #2 } }
  \group_align_safe_begin:
    \peek_after:Nw \@@_peek_execute_branches_ss:
 }
%    \end{macrocode}
% We do not skip spaces when scanning ahead, and we explicitly wish to
% bail out on encountering a space or a brace.
%    \begin{macrocode}
\cs_new:Npn \@@_peek_execute_branches_ss:
 {
  \bool_if:nTF
   {
    \token_if_eq_catcode_p:NN \l_peek_token \c_group_begin_token ||
    \token_if_eq_catcode_p:NN \l_peek_token \c_group_end_token ||
    \token_if_eq_meaning_p:NN \l_peek_token \c_space_token
   }
   { \__peek_false:w  }
   { \@@_peek_execute_branches_ss_aux: }
 }
%    \end{macrocode}
% This is the actual comparison code.
% Because the peeking has already tokenised the next token,
% it's too late to extract its charcode directly. Instead,
% we look at its meaning, which remains a `character' even
% though it is itself math-active. If the character is ever
% made fully active, this will break our assumptions!
%
% If the char's meaning exists as a property list key, we
% build up a chain of sub-/superscripts and iterate. (If not, exit and
% typeset what we've already collected.)
%    \begin{macrocode}
\cs_new:Npn \@@_peek_execute_branches_ss_aux:
 {
  \prop_if_in:coTF
    {g_@@_\l_@@_tmpa_tl _prop} {\meaning\l_peek_token}
    {
      \prop_get:coN
        {g_@@_\l_@@_tmpa_tl _prop} {\meaning\l_peek_token} \l_@@_tmpb_tl
      \tl_put_right:NV \l_@@_ss_chain_tl \l_@@_tmpb_tl
      \__peek_true:w
    }
    { \__peek_false:w }
 }
%    \end{macrocode}
%
% \subsubsection{Active fractions}
% Active fractions can be setup independently of any maths font definition;
% all it requires is a mapping from the Unicode input chars to the relevant
% \LaTeX\ fraction declaration.
%
%    \begin{macrocode}
\cs_new:Npn \@@_define_active_frac:Nw #1 #2/#3
 {
  \char_set_catcode_active:N #1
  \@@_char_gmake_mathactive:N #1
  \tl_rescan:nn
   {
    \catcode`\_=11\relax
    \catcode`\:=11\relax
   }
   {
    \cs_gset:Npx #1
     {
      \bool_if:NTF \l_@@_smallfrac_bool {\exp_not:N\tfrac} {\exp_not:N\frac}
          {#2} {#3}
     }
   }
 }
%    \end{macrocode}
% These are redefined for each math font selection in case the |active-frac|
% feature changes.
%    \begin{macrocode}
\cs_new:Npn \@@_setup_active_frac:
 {
  \group_begin:
  \@@_define_active_frac:Nw  ^^^^2189  0/3
  \@@_define_active_frac:Nw  ^^^^2152  1/{10}
  \@@_define_active_frac:Nw  ^^^^2151  1/9
  \@@_define_active_frac:Nw  ^^^^215b  1/8
  \@@_define_active_frac:Nw  ^^^^2150  1/7
  \@@_define_active_frac:Nw  ^^^^2159  1/6
  \@@_define_active_frac:Nw  ^^^^2155  1/5
  \@@_define_active_frac:Nw  ^^^^00bc  1/4
  \@@_define_active_frac:Nw  ^^^^2153  1/3
  \@@_define_active_frac:Nw  ^^^^215c  3/8
  \@@_define_active_frac:Nw  ^^^^2156  2/5
  \@@_define_active_frac:Nw  ^^^^00bd  1/2
  \@@_define_active_frac:Nw  ^^^^2157  3/5
  \@@_define_active_frac:Nw  ^^^^215d  5/8
  \@@_define_active_frac:Nw  ^^^^2154  2/3
  \@@_define_active_frac:Nw  ^^^^00be  3/4
  \@@_define_active_frac:Nw  ^^^^2158  4/5
  \@@_define_active_frac:Nw  ^^^^215a  5/6
  \@@_define_active_frac:Nw  ^^^^215e  7/8
  \group_end:
 }
\@@_setup_active_frac:
%    \end{macrocode}
%
% \subsection{Synonyms and all the rest}
%
% These are symbols with multiple names. Eventually to be taken care of
% automatically by the maths characters database.
%    \begin{macrocode}
\def\to{\rightarrow}
\def\le{\leq}
\def\ge{\geq}
\def\neq{\ne}
\def\triangle{\mathord{\bigtriangleup}}
\def\bigcirc{\mdlgwhtcircle}
\def\circ{\vysmwhtcircle}
\def\bullet{\smblkcircle}
\def\mathyen{\yen}
\def\mathsterling{\sterling}
\def\diamond{\smwhtdiamond}
\def\emptyset{\varnothing}
\def\hbar{\hslash}
\def\land{\wedge}
\def\lor{\vee}
\def\owns{\ni}
\def\gets{\leftarrow}
\def\mathring{\ocirc}
\def\lnot{\neg}
\def\longdivision{\longdivisionsign}
%    \end{macrocode}
% These are somewhat odd: (and their usual Unicode uprightness does not match their amssymb glyphs)
%    \begin{macrocode}
\def\backepsilon{\upbackepsilon}
\def\eth{\matheth}
%    \end{macrocode}
% These are names that are `frozen' in HTML but have dumb names:
%    \begin{macrocode}
\def\dbkarow {\dbkarrow} 
\def\drbkarow{\drbkarrow}
\def\hksearow{\hksearrow}
\def\hkswarow{\hkswarrow}
%    \end{macrocode}
%
% Due to the magic of OpenType math, big operators are automatically
% enlarged when necessary. Since there isn't a separate unicode glyph for
% `small integral', I'm not sure if there is a better way to do this:
%    \begin{macrocode}
\def\smallint{\mathop{\textstyle\int}\limits}
%    \end{macrocode}
%
% \begin{macro}{\underbar}
%    \begin{macrocode}
\cs_set_eq:NN \latexe_underbar:n \underbar
\renewcommand\underbar
 {
  \mode_if_math:TF \mathunderbar \latexe_underbar:n
 }
%    \end{macrocode}
% \end{macro}
%
% \begin{macro}{\colon}
% Define \cs{colon} as a mathpunct `|:|'.
% This is wrong: it should be \unichar{003A} {colon} instead!
% We hope no-one will notice.
%    \begin{macrocode}
\@ifpackageloaded{amsmath}
 {
  % define their own colon, perhaps I should just steal it. (It does look much better.)
 }
 {
  \cs_set_protected:Npn \colon
   {
    \bool_if:NTF \g_@@_literal_colon_bool {:} { \mathpunct{:} }
   }
 }
%    \end{macrocode}
% \end{macro}
%
% \begin{macro}{\digamma}
% \begin{macro}{\Digamma}
% I might end up just changing these in the table.
%    \begin{macrocode}
\def\digamma{\updigamma}
\def\Digamma{\upDigamma}
%    \end{macrocode}
% \end{macro}
% \end{macro}
%
% \paragraph{Symbols}
%    \begin{macrocode}
\cs_set:Npn \| {\Vert}
%    \end{macrocode}
% \cs{mathinner} items:
%    \begin{macrocode}
\cs_set:Npn \mathellipsis {\mathinner{\unicodeellipsis}}
\cs_set:Npn \cdots {\mathinner{\unicodecdots}}
%    \end{macrocode}
%
%    \begin{macrocode}
\cs_set_eq:NN \@@_text_slash: \slash
\cs_set_protected:Npn \slash
 {
  \mode_if_math:TF {\mathslash} {\@@_text_slash:}
 }
%    \end{macrocode}
%
% \paragraph{\cs{not}}
% The situation of \cs{not} symbol is currently messy, in Unicode it is defined
% as a combining mark so naturally it should be treated as a math accent,
% however neither Lua\TeX\ nor \XeTeX\ correctly place it as it needs special
% treatment compared to other accents, furthermore a math accent changes the
% spacing of its nucleus, so \cs{not=} will be spaced as an ordinary not
% relational symbol, which is undesired.
%
% Here modify \cs{not} to a macro that tries to use predefined negated symbols,
% which would give better results in most cases, until there is more robust
% solution in the engines.
%
% This code is based on an answer to a TeX -- Stack Exchange question by Enrico
% Gregorio\footnote{\url{http://tex.stackexchange.com/a/47260/729}}.
%
%    \begin{macrocode}
\cs_new:Npn \@@_newnot:N #1
 {
   \tl_set:Nx \l_not_token_name_tl { \token_to_str:N #1 }
   \exp_args:Nx \tl_if_empty:nF { \tl_tail:V \l_not_token_name_tl }
    {
     \tl_set:Nx \l_not_token_name_tl { \tl_tail:V \l_not_token_name_tl }
    }
   \cs_if_exist:cTF { n \l_not_token_name_tl }
    {
     \use:c { n \l_not_token_name_tl }
    }
    {
     \cs_if_exist:cTF { not \l_not_token_name_tl }
      {
       \use:c { not \l_not_token_name_tl }
      }
      {
       \@@_oldnot: #1
      }
    }
 }
%    \end{macrocode}
%    \begin{macrocode}
\cs_set_eq:NN \@@_oldnot: \not
\AtBeginDocument{\cs_set_eq:NN \not \@@_newnot:N}
%    \end{macrocode}
%    \begin{macrocode}
\cs_new_protected_nopar:Nn \@@_setup_negations:
 {
  \cs_gset:cpn { not= }    { \neq }
  \cs_gset:cpn { not< }    { \nless }
  \cs_gset:cpn { not> }    { \ngtr }
  \cs_gset:Npn  \ngets     { \nleftarrow }
  \cs_gset:Npn  \nsimeq    { \nsime }
  \cs_gset:Npn  \nequal    { \ne }
  \cs_gset:Npn  \nle       { \nleq }
  \cs_gset:Npn  \nge       { \ngeq }
  \cs_gset:Npn  \ngreater  { \ngtr }
  \cs_gset:Npn  \nforksnot { \forks }
 }
%    \end{macrocode}
%
%    \begin{macrocode}
%</package&(XE|LU)>
%    \end{macrocode}
%
\endinput


