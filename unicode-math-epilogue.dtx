
% \section{Epilogue}
%
%    \begin{macrocode}
%<*package&(XE|LU)>
%    \end{macrocode}
%
% Lots of little things to tidy up.
%
% \subsection{Primes}
%
% We need a new `prime' algorithm. Unicode math has four pre-drawn prime glyphs.
% \begin{quote}\obeylines
% \unichar{2032} {prime} (\cs{prime}): $x\prime$
% \unichar{2033} {double prime} (\cs{dprime}): $x\dprime$
% \unichar{2034} {triple prime} (\cs{trprime}): $x\trprime$
% \unichar{2057} {quadruple prime} (\cs{qprime}): $x\qprime$
% \end{quote}
% As you can see, they're all drawn at the correct height without being superscripted.
% However, in a correctly behaving OpenType font,
% we also see different behaviour after the \texttt{ssty} feature is applied:
% \begin{quote}
% \font\1="Cambria Math:script=math,+ssty=0"\1
% \char"1D465\char"2032\quad
% \char"1D465\char"2033\quad
% \char"1D465\char"2034\quad
% \char"1D465\char"2057
% \end{quote}
% The glyphs are now `full size' so that when placed inside a superscript,
% their shape will match the originally sized ones. Many thanks to Ross Mills
% of Tiro Typeworks for originally pointing out this behaviour.
%
% In regular \LaTeX, primes can be entered with the straight quote character
% |'|, and multiple straight quotes chain together to produce multiple
% primes. Better results can be achieved in \pkg{unicode-math} by chaining
% multiple single primes into a pre-drawn multi-prime glyph; consider
% $x\prime{}\prime{}\prime$ vs.\ $x\trprime$.
%
% For Unicode maths, we wish to conserve this behaviour and augment it with
% the possibility of adding any combination of Unicode prime or any of the
% $n$-prime characters. E.g., the user might copy-paste a double prime from
% another source and then later type another single prime after it; the output
% should be the triple prime.
%
% Our algorithm is:
% \begin{itemize}[nolistsep]
% \item Prime encountered; pcount=1.
% \item Scan ahead; if prime: pcount:=pcount+1; repeat.
% \item If not prime, stop scanning.
% \item If pcount=1, \cs{prime}, end.
% \item If pcount=2, check \cs{dprime}; if it exists, use it, end; if not, goto last step.
% \item Ditto pcount=3 \& \cs{trprime}.
% \item Ditto pcount=4 \& \cs{qprime}.
% \item If pcount>4 or the glyph doesn't exist, insert pcount \cs{prime}s with \cs{primekern} between each.
% \end{itemize}
%
% This is a wrapper to insert a superscript; if there is a subsequent
% trailing superscript, then it is included within the insertion.
%    \begin{macrocode}
\cs_new:Nn \@@_arg_i_before_egroup:n {#1\egroup}
\cs_new:Nn \@@_superscript:n
 {
  ^\bgroup #1
  \peek_meaning_remove:NTF ^ \@@_arg_i_before_egroup:n \egroup
 }
%    \end{macrocode}
%
%    \begin{macrocode}
\cs_new:Nn \@@_nprimes:Nn
 {
  \@@_superscript:n
   {
    #1
    \prg_replicate:nn {#2-1} { \mskip \g_@@_primekern_muskip #1 }
   }
 }

\cs_new:Nn \@@_nprimes_select:nn
 {
  \int_case:nnF {#2}
   {
    {1} { \@@_superscript:n {#1} }
    {2} {
      \@@_glyph_if_exist:nTF {"2033}
        { \@@_superscript:n {\@@_prime_double_mchar} }
        { \@@_nprimes:Nn #1 {#2} }
    }
    {3} {
      \@@_glyph_if_exist:nTF {"2034}
        { \@@_superscript:n {\@@_prime_triple_mchar} }
        { \@@_nprimes:Nn #1 {#2} }
    }
    {4} {
      \@@_glyph_if_exist:nTF {"2057}
        { \@@_superscript:n {\@@_prime_quad_mchar} }
        { \@@_nprimes:Nn #1 {#2} }
    }
   }
   {
    \@@_nprimes:Nn #1 {#2}
   }
 }
\cs_new:Nn \@@_nbackprimes_select:nn
 {
  \int_case:nnF {#2}
   {
    {1} { \@@_superscript:n {#1} }
    {2} {
      \@@_glyph_if_exist:nTF {"2036}
        { \@@_superscript:n {\@@_backprime_double_mchar} }
        { \@@_nprimes:Nn #1 {#2} }
    }
    {3} {
      \@@_glyph_if_exist:nTF {"2037}
        { \@@_superscript:n {\@@_backprime_triple_mchar} }
        { \@@_nprimes:Nn #1 {#2} }
    }
   }
   {
    \@@_nprimes:Nn #1 {#2}
   }
 }
%    \end{macrocode}
%
% Scanning is annoying because I'm too lazy to do it for the general case.
%
%    \begin{macrocode}
\cs_new:Npn \@@_scan_prime:
 {
  \cs_set_eq:NN \@@_superscript:n \use:n
  \int_zero:N \l_@@_primecount_int
  \@@_scanprime_collect:N \@@_prime_single_mchar
 }
\cs_new:Npn \@@_scan_dprime:
 {
  \cs_set_eq:NN \@@_superscript:n \use:n
  \int_set:Nn \l_@@_primecount_int {1}
  \@@_scanprime_collect:N \@@_prime_single_mchar
 }
\cs_new:Npn \@@_scan_trprime:
 {
  \cs_set_eq:NN \@@_superscript:n \use:n
  \int_set:Nn \l_@@_primecount_int {2}
  \@@_scanprime_collect:N \@@_prime_single_mchar
 }
\cs_new:Npn \@@_scan_qprime:
 {
  \cs_set_eq:NN \@@_superscript:n \use:n
  \int_set:Nn \l_@@_primecount_int {3}
  \@@_scanprime_collect:N \@@_prime_single_mchar
 }
\cs_new:Npn \@@_scan_sup_prime:
 {
  \int_zero:N \l_@@_primecount_int
  \@@_scanprime_collect:N \@@_prime_single_mchar
 }
\cs_new:Npn \@@_scan_sup_dprime:
 {
  \int_set:Nn \l_@@_primecount_int {1}
  \@@_scanprime_collect:N \@@_prime_single_mchar
 }
\cs_new:Npn \@@_scan_sup_trprime:
 {
  \int_set:Nn \l_@@_primecount_int {2}
  \@@_scanprime_collect:N \@@_prime_single_mchar
 }
\cs_new:Npn \@@_scan_sup_qprime:
 {
  \int_set:Nn \l_@@_primecount_int {3}
  \@@_scanprime_collect:N \@@_prime_single_mchar
 }
\cs_new:Nn \@@_scanprime_collect:N
 {
  \int_incr:N \l_@@_primecount_int
  \peek_meaning_remove:NTF '
   { \@@_scanprime_collect:N #1 }
   {
    \peek_meaning_remove:NTF \@@_scan_prime:
     { \@@_scanprime_collect:N #1 }
     {
      \peek_meaning_remove:NTF ^^^^2032
       { \@@_scanprime_collect:N #1 }
       {
        \peek_meaning_remove:NTF \@@_scan_dprime:
         {
          \int_incr:N \l_@@_primecount_int
          \@@_scanprime_collect:N #1
         }
         {
          \peek_meaning_remove:NTF ^^^^2033
           {
            \int_incr:N \l_@@_primecount_int
            \@@_scanprime_collect:N #1
           }
           {
            \peek_meaning_remove:NTF \@@_scan_trprime:
             {
              \int_add:Nn \l_@@_primecount_int {2}
              \@@_scanprime_collect:N #1
             }
             {
              \peek_meaning_remove:NTF ^^^^2034
               {
                \int_add:Nn \l_@@_primecount_int {2}
                \@@_scanprime_collect:N #1
               }
               {
                \peek_meaning_remove:NTF \@@_scan_qprime:
                 {
                  \int_add:Nn \l_@@_primecount_int {3}
                  \@@_scanprime_collect:N #1
                 }
                 {
                  \peek_meaning_remove:NTF ^^^^2057
                   {
                    \int_add:Nn \l_@@_primecount_int {3}
                    \@@_scanprime_collect:N #1
                   }
                   {
                    \@@_nprimes_select:nn {#1} {\l_@@_primecount_int}
                   }
                 }
               }
             }
           }
         }
       }
     }
   }
 }
\cs_new:Npn \@@_scan_backprime:
 {
  \cs_set_eq:NN \@@_superscript:n \use:n
  \int_zero:N \l_@@_primecount_int
  \@@_scanbackprime_collect:N \@@_backprime_single_mchar
 }
\cs_new:Npn \@@_scan_backdprime:
 {
  \cs_set_eq:NN \@@_superscript:n \use:n
  \int_set:Nn \l_@@_primecount_int {1}
  \@@_scanbackprime_collect:N \@@_backprime_single_mchar
 }
\cs_new:Npn \@@_scan_backtrprime:
 {
  \cs_set_eq:NN \@@_superscript:n \use:n
  \int_set:Nn \l_@@_primecount_int {2}
  \@@_scanbackprime_collect:N \@@_backprime_single_mchar
 }
\cs_new:Npn \@@_scan_sup_backprime:
 {
  \int_zero:N \l_@@_primecount_int
  \@@_scanbackprime_collect:N \@@_backprime_single_mchar
 }
\cs_new:Npn \@@_scan_sup_backdprime:
 {
  \int_set:Nn \l_@@_primecount_int {1}
  \@@_scanbackprime_collect:N \@@_backprime_single_mchar
 }
\cs_new:Npn \@@_scan_sup_backtrprime:
 {
  \int_set:Nn \l_@@_primecount_int {2}
  \@@_scanbackprime_collect:N \@@_backprime_single_mchar
 }
\cs_new:Nn \@@_scanbackprime_collect:N
 {
  \int_incr:N \l_@@_primecount_int
  \peek_meaning_remove:NTF `
   {
    \@@_scanbackprime_collect:N #1
   }
   {
    \peek_meaning_remove:NTF \@@_scan_backprime:
     {
      \@@_scanbackprime_collect:N #1
     }
     {
      \peek_meaning_remove:NTF ^^^^2035
       {
        \@@_scanbackprime_collect:N #1
       }
       {
        \peek_meaning_remove:NTF \@@_scan_backdprime:
         {
          \int_incr:N \l_@@_primecount_int
          \@@_scanbackprime_collect:N #1
         }
         {
          \peek_meaning_remove:NTF ^^^^2036
           {
            \int_incr:N \l_@@_primecount_int
            \@@_scanbackprime_collect:N #1
           }
           {
            \peek_meaning_remove:NTF \@@_scan_backtrprime:
             {
              \int_add:Nn \l_@@_primecount_int {2}
              \@@_scanbackprime_collect:N #1
             }
             {
              \peek_meaning_remove:NTF ^^^^2037
               {
                \int_add:Nn \l_@@_primecount_int {2}
                \@@_scanbackprime_collect:N #1
               }
               {
                \@@_nbackprimes_select:nn {#1} {\l_@@_primecount_int}
               }
             }
           }
         }
       }
     }
   }
 }
%    \end{macrocode}
%
%    \begin{macrocode}
\AtBeginDocument{\@@_define_prime_commands: \@@_define_prime_chars:}
\cs_new:Nn \@@_define_prime_commands:
 {
  \cs_set_eq:NN \prime       \@@_prime_single_mchar
  \cs_set_eq:NN \dprime      \@@_prime_double_mchar
  \cs_set_eq:NN \trprime     \@@_prime_triple_mchar
  \cs_set_eq:NN \qprime      \@@_prime_quad_mchar
  \cs_set_eq:NN \backprime   \@@_backprime_single_mchar
  \cs_set_eq:NN \backdprime  \@@_backprime_double_mchar
  \cs_set_eq:NN \backtrprime \@@_backprime_triple_mchar
 }
\group_begin:
  \char_set_catcode_active:N \'
  \char_set_catcode_active:N \`
  \char_set_catcode_active:n {"2032}
  \char_set_catcode_active:n {"2033}
  \char_set_catcode_active:n {"2034}
  \char_set_catcode_active:n {"2057}
  \char_set_catcode_active:n {"2035}
  \char_set_catcode_active:n {"2036}
  \char_set_catcode_active:n {"2037}
  \cs_gset:Nn \@@_define_prime_chars:
   {
    \cs_set_eq:NN '        \@@_scan_sup_prime:
    \cs_set_eq:NN ^^^^2032 \@@_scan_sup_prime:
    \cs_set_eq:NN ^^^^2033 \@@_scan_sup_dprime:
    \cs_set_eq:NN ^^^^2034 \@@_scan_sup_trprime:
    \cs_set_eq:NN ^^^^2057 \@@_scan_sup_qprime:
    \cs_set_eq:NN `        \@@_scan_sup_backprime:
    \cs_set_eq:NN ^^^^2035 \@@_scan_sup_backprime:
    \cs_set_eq:NN ^^^^2036 \@@_scan_sup_backdprime:
    \cs_set_eq:NN ^^^^2037 \@@_scan_sup_backtrprime:
   }
\group_end:
%    \end{macrocode}
%
% \subsection{Unicode radicals}
%
% Make sure \cs{Uroot} is defined in the case where the \LaTeX{}
% kernel doesn't make it available with its native name.
%    \begin{macrocode}
%<*LU>
\cs_if_exist:NF \Uroot
  { \cs_new_eq:NN \Uroot \luatexUroot }
%</LU>
%    \end{macrocode}
%
%    \begin{macrocode}
\AtBeginDocument{\@@_redefine_radical:}
\cs_new:Nn \@@_redefine_radical:
%<*XE>
 {
  \@ifpackageloaded { amsmath } { }
   {
%    \end{macrocode}
% \begin{macro}{\r@@t}
% \darg{A mathstyle (for \cmd\mathpalette)}
% \darg{Leading superscript for the sqrt sign}
% A re-implementation of \LaTeX's hard-coded n-root sign using the appropriate \cmd\fontdimen s.
%    \begin{macrocode}
    \cs_set_nopar:Npn \r@@@@t ##1 ##2
     {
      \hbox_set:Nn \l_tmpa_box
       {
        \c_math_toggle_token
        \m@th
        ##1
        \sqrtsign { ##2 }
        \c_math_toggle_token
       }
      \@@_mathstyle_scale:Nnn ##1 { \kern }
       { \fontdimen 63 \l_@@_font }
      \box_move_up:nn
       {
        (\box_ht:N \l_tmpa_box - \box_dp:N \l_tmpa_box)
        * \number \fontdimen 65 \l_@@_font / 100
       }
       { \box_use:N \rootbox }
      \@@_mathstyle_scale:Nnn ##1 { \kern }
       { \fontdimen 64 \l_@@_font }
      \box_use_clear:N \l_tmpa_box
     }
%    \end{macrocode}
% \end{macro}
%    \begin{macrocode}
   }
 }
%</XE>
%<*LU>
 {
  \@ifpackageloaded { amsmath } { }
   {
%    \end{macrocode}
% \begin{macro}{\root}
% Redefine this macro for \LuaTeX, which provides us a nice primitive to use.
%    \begin{macrocode}
    \cs_set:Npn \root ##1 \of ##2
     {
       \Uroot \l_@@_radical_sqrt_tl { ##1 } { ##2 }
     }
%    \end{macrocode}
% \end{macro}
%    \begin{macrocode}
   }
 }
%</LU>
%    \end{macrocode}
%
%
% \begin{macro}{\@@_fontdimen_to_percent:nn}
% \begin{macro}{\@@_fontdimen_to_scale:nn}
% \darg{Font dimen number}
% \darg{Font `variable'}
% \cmd\fontdimen s |10|, |11|, and |65| aren't actually dimensions, they're percentage values given in units of |sp|.
% \cs{@@_fontdimen_to_percent:nn} takes a font dimension number and outputs the decimal value of the associated parameter.
% \cs{@@_fontdimen_to_scale:nn} returns a dimension correspond to the current
% font size relative proportion based on that percentage.
%    \begin{macrocode}
\cs_new:Nn \@@_fontdimen_to_percent:nn
 {
  \fp_eval:n { \dim_to_decimal:n { \fontdimen #1 #2 } * 65536 / 100 }
 }
\cs_new:Nn \@@_fontdimen_to_scale:nn
 {
  \fp_eval:n {\@@_fontdimen_to_percent:nn {#1} {#2} * \f@size } pt
 }
%    \end{macrocode}
% \end{macro}
% \end{macro}
%
% \begin{macro}{\@@_mathstyle_scale:Nnn}
% \darg{A math style (\cs{scriptstyle}, say)}
% \darg{Macro that takes a non-delimited length argument (like \cmd\kern)}
% \darg{Length control sequence to be scaled according to the math style}
% This macro is used to scale the lengths reported by \cmd\fontdimen\ according to the scale factor for script- and scriptscript-size objects.
%    \begin{macrocode}
\cs_new:Nn \@@_mathstyle_scale:Nnn
 {
  \ifx#1\scriptstyle
    #2 \@@_fontdimen_to_percent:nn {10} \l_@@_font #3
  \else
    \ifx#1\scriptscriptstyle
      #2 \@@_fontdimen_to_percent:nn {11} \l_@@_font #3
    \else
      #2 #3
    \fi
  \fi
 }
%    \end{macrocode}
% \end{macro}
%
% \subsection{Unicode sub- and super-scripts}
%
% The idea here is to enter a scanning state after a superscript or subscript
% is encountered.
% If subsequent superscripts or subscripts (resp.) are found,
% they are lumped together.
% Each sub/super has a corresponding regular size
% glyph which is used by \XeTeX\ to typeset the results; this means that the
% actual subscript/superscript glyphs are never seen in the output
% document~--- they are only used as input characters.
%
% Open question: should the superscript-like `modifiers' (\unichar{1D2C}
% {modifier capital letter a} and on) be included here?
%    \begin{macrocode}
\group_begin:
%    \end{macrocode}
% \paragraph{Superscripts}
% Populate a property list with superscript characters; themselves as their
% key, and their replacement as each key's value.
% Then make the superscript active and bind it to the scanning function.
%
% \cs{scantokens} makes this process much simpler since we can activate the
% char and assign its meaning in one step.
%    \begin{macrocode}
\cs_new:Nn \@@_setup_active_superscript:nn
 {
  \prop_gput:Nnn \g_@@_supers_prop   {#1} {#2}
  \char_set_catcode_active:N #1
  \@@_char_gmake_mathactive:N #1
  \scantokens
   {
    \cs_gset:Npn #1
     {
      \tl_set:Nn \l_@@_ss_chain_tl {#2}
      \cs_set_eq:NN \@@_sub_or_super:n \sp
      \tl_set:Nn \l_@@_tmpa_tl {supers}
      \@@_scan_sscript:
     }
   }
 }
%    \end{macrocode}
% Bam:
%    \begin{macrocode}
\@@_setup_active_superscript:nn {^^^^2070} {0}
\@@_setup_active_superscript:nn {^^^^00b9} {1}
\@@_setup_active_superscript:nn {^^^^00b2} {2}
\@@_setup_active_superscript:nn {^^^^00b3} {3}
\@@_setup_active_superscript:nn {^^^^2074} {4}
\@@_setup_active_superscript:nn {^^^^2075} {5}
\@@_setup_active_superscript:nn {^^^^2076} {6}
\@@_setup_active_superscript:nn {^^^^2077} {7}
\@@_setup_active_superscript:nn {^^^^2078} {8}
\@@_setup_active_superscript:nn {^^^^2079} {9}
\@@_setup_active_superscript:nn {^^^^207a} {+}
\@@_setup_active_superscript:nn {^^^^207b} {-}
\@@_setup_active_superscript:nn {^^^^207c} {=}
\@@_setup_active_superscript:nn {^^^^207d} {(}
\@@_setup_active_superscript:nn {^^^^207e} {)}
\@@_setup_active_superscript:nn {^^^^2071} {i}
\@@_setup_active_superscript:nn {^^^^207f} {n}
\@@_setup_active_superscript:nn {^^^^02b0} {h}
\@@_setup_active_superscript:nn {^^^^02b2} {j}
\@@_setup_active_superscript:nn {^^^^02b3} {r}
\@@_setup_active_superscript:nn {^^^^02b7} {w}
\@@_setup_active_superscript:nn {^^^^02b8} {y}
%    \end{macrocode}
% \paragraph{Subscripts} Ditto above.
%    \begin{macrocode}
\cs_new:Nn \@@_setup_active_subscript:nn
 {
  \prop_gput:Nnn \g_@@_subs_prop   {#1} {#2}
  \char_set_catcode_active:N #1
  \@@_char_gmake_mathactive:N #1
  \scantokens
   {
    \cs_gset:Npn #1
     {
      \tl_set:Nn \l_@@_ss_chain_tl {#2}
      \cs_set_eq:NN \@@_sub_or_super:n \sb
      \tl_set:Nn \l_@@_tmpa_tl {subs}
      \@@_scan_sscript:
     }
   }
 }
%    \end{macrocode}
% A few more subscripts than superscripts:
%    \begin{macrocode}
\@@_setup_active_subscript:nn {^^^^2080} {0}
\@@_setup_active_subscript:nn {^^^^2081} {1}
\@@_setup_active_subscript:nn {^^^^2082} {2}
\@@_setup_active_subscript:nn {^^^^2083} {3}
\@@_setup_active_subscript:nn {^^^^2084} {4}
\@@_setup_active_subscript:nn {^^^^2085} {5}
\@@_setup_active_subscript:nn {^^^^2086} {6}
\@@_setup_active_subscript:nn {^^^^2087} {7}
\@@_setup_active_subscript:nn {^^^^2088} {8}
\@@_setup_active_subscript:nn {^^^^2089} {9}
\@@_setup_active_subscript:nn {^^^^208a} {+}
\@@_setup_active_subscript:nn {^^^^208b} {-}
\@@_setup_active_subscript:nn {^^^^208c} {=}
\@@_setup_active_subscript:nn {^^^^208d} {(}
\@@_setup_active_subscript:nn {^^^^208e} {)}
\@@_setup_active_subscript:nn {^^^^2090} {a}
\@@_setup_active_subscript:nn {^^^^2091} {e}
\@@_setup_active_subscript:nn {^^^^2095} {h}
\@@_setup_active_subscript:nn {^^^^1d62} {i}
\@@_setup_active_subscript:nn {^^^^2c7c} {j}
\@@_setup_active_subscript:nn {^^^^2096} {k}
\@@_setup_active_subscript:nn {^^^^2097} {l}
\@@_setup_active_subscript:nn {^^^^2098} {m}
\@@_setup_active_subscript:nn {^^^^2099} {n}
\@@_setup_active_subscript:nn {^^^^2092} {o}
\@@_setup_active_subscript:nn {^^^^209a} {p}
\@@_setup_active_subscript:nn {^^^^1d63} {r}
\@@_setup_active_subscript:nn {^^^^209b} {s}
\@@_setup_active_subscript:nn {^^^^209c} {t}
\@@_setup_active_subscript:nn {^^^^1d64} {u}
\@@_setup_active_subscript:nn {^^^^1d65} {v}
\@@_setup_active_subscript:nn {^^^^2093} {x}
\@@_setup_active_subscript:nn {^^^^1d66} {\beta}
\@@_setup_active_subscript:nn {^^^^1d67} {\gamma}
\@@_setup_active_subscript:nn {^^^^1d68} {\rho}
\@@_setup_active_subscript:nn {^^^^1d69} {\phi}
\@@_setup_active_subscript:nn {^^^^1d6a} {\chi}
%    \end{macrocode}
%
%    \begin{macrocode}
\group_end:
%    \end{macrocode}
% The scanning command, which collects a chain of subscripts or a chain
% of superscripts and then typesets what it has collected.
%    \begin{macrocode}
\cs_new:Npn \@@_scan_sscript:
 {
  \@@_scan_sscript:TF
   {
    \@@_scan_sscript:
   }
   {
    \@@_sub_or_super:n {\l_@@_ss_chain_tl}
   }
 }
%    \end{macrocode}
% We do not skip spaces when scanning ahead, and we explicitly wish to
% bail out on encountering a space or a brace.  These cases are filtered
% using \cs{peek_N_type:TF}.  Otherwise the token can be taken as an
% \texttt{N}-type argument.  Then we search for it in the appropriate
% property list (\cs{l_@@_tmpa_tl} is |subs| or |supers|).
% If found, add the value to the current chain of sub/superscripts.
% Remember to put the character back in the input otherwise.
% The \cs{group_align_safe_begin:} and \cs{group_align_safe_end:} are
% needed in case |#3| is |&|.
%    \begin{macrocode}
\cs_new:Npn \@@_scan_sscript:TF #1#2
 {
  \peek_N_type:TF
   {
    \group_align_safe_begin:
    \@@_scan_sscript_aux:nnN {#1} {#2}
   }
   {#2}
 }
\cs_new:Npn \@@_scan_sscript_aux:nnN #1#2#3
 {
  \prop_get:cnNTF {g_@@_\l_@@_tmpa_tl _prop} {#3} \l_@@_tmpb_tl
   {
    \tl_put_right:NV \l_@@_ss_chain_tl \l_@@_tmpb_tl
    \group_align_safe_end:
    #1
   }
   { \group_align_safe_end: #2 #3 }
 }
%    \end{macrocode}
%
% \subsubsection{Active fractions}
% Active fractions can be setup independently of any maths font definition;
% all it requires is a mapping from the Unicode input chars to the relevant
% \LaTeX\ fraction declaration.
%
%    \begin{macrocode}
\cs_new:Npn \@@_define_active_frac:Nw #1 #2/#3
 {
  \char_set_catcode_active:N #1
  \@@_char_gmake_mathactive:N #1
  \tl_rescan:nn
   {
    \catcode`\_=11\relax
    \catcode`\:=11\relax
   }
   {
    \cs_gset:Npx #1
     {
      \bool_if:NTF \l_@@_smallfrac_bool {\exp_not:N\tfrac} {\exp_not:N\frac}
          {#2} {#3}
     }
   }
 }
%    \end{macrocode}
% These are redefined for each math font selection in case the |active-frac|
% feature changes.
%    \begin{macrocode}
\cs_new:Npn \@@_setup_active_frac:
 {
  \group_begin:
  \@@_define_active_frac:Nw  ^^^^2189  0/3
  \@@_define_active_frac:Nw  ^^^^2152  1/{10}
  \@@_define_active_frac:Nw  ^^^^2151  1/9
  \@@_define_active_frac:Nw  ^^^^215b  1/8
  \@@_define_active_frac:Nw  ^^^^2150  1/7
  \@@_define_active_frac:Nw  ^^^^2159  1/6
  \@@_define_active_frac:Nw  ^^^^2155  1/5
  \@@_define_active_frac:Nw  ^^^^00bc  1/4
  \@@_define_active_frac:Nw  ^^^^2153  1/3
  \@@_define_active_frac:Nw  ^^^^215c  3/8
  \@@_define_active_frac:Nw  ^^^^2156  2/5
  \@@_define_active_frac:Nw  ^^^^00bd  1/2
  \@@_define_active_frac:Nw  ^^^^2157  3/5
  \@@_define_active_frac:Nw  ^^^^215d  5/8
  \@@_define_active_frac:Nw  ^^^^2154  2/3
  \@@_define_active_frac:Nw  ^^^^00be  3/4
  \@@_define_active_frac:Nw  ^^^^2158  4/5
  \@@_define_active_frac:Nw  ^^^^215a  5/6
  \@@_define_active_frac:Nw  ^^^^215e  7/8
  \group_end:
 }
\@@_setup_active_frac:
%    \end{macrocode}
%
% \subsection{Synonyms and all the rest}
%
% These are symbols with multiple names. Eventually to be taken care of
% automatically by the maths characters database.
%    \begin{macrocode}
\protected\def\to{\rightarrow}
\protected\def\le{\leq}
\protected\def\ge{\geq}
\protected\def\neq{\ne}
\protected\def\triangle{\mathord{\bigtriangleup}}
\protected\def\bigcirc{\mdlgwhtcircle}
\protected\def\circ{\vysmwhtcircle}
\protected\def\bullet{\smblkcircle}
\protected\def\mathyen{\yen}
\protected\def\mathsterling{\sterling}
\protected\def\diamond{\smwhtdiamond}
\protected\def\emptyset{\varnothing}
\protected\def\hbar{\hslash}
\protected\def\land{\wedge}
\protected\def\lor{\vee}
\protected\def\owns{\ni}
\protected\def\gets{\leftarrow}
\protected\def\mathring{\ocirc}
\protected\def\lnot{\neg}
\protected\def\longdivision{\longdivisionsign}
%    \end{macrocode}
% These are somewhat odd: (and their usual Unicode uprightness does not match their amssymb glyphs)
%    \begin{macrocode}
\protected\def\backepsilon{\upbackepsilon}
\protected\def\eth{\matheth}
%    \end{macrocode}
% These are names that are `frozen' in HTML but have dumb names:
%    \begin{macrocode}
\protected\def\dbkarow {\dbkarrow}
\protected\def\drbkarow{\drbkarrow}
\protected\def\hksearow{\hksearrow}
\protected\def\hkswarow{\hkswarrow}
%    \end{macrocode}
%
% Due to the magic of OpenType math, big operators are automatically
% enlarged when necessary. Since there isn't a separate unicode glyph for
% `small integral', I'm not sure if there is a better way to do this:
%    \begin{macrocode}
\protected\def\smallint{\mathop{\textstyle\int}\limits}
%    \end{macrocode}
%
% \begin{macro}{\underbar}
%    \begin{macrocode}
\cs_set_eq:NN \latexe_underbar:n \underbar
\renewcommand\underbar
 {
  \mode_if_math:TF \mathunderbar \latexe_underbar:n
 }
%    \end{macrocode}
% \end{macro}
%
% \begin{macro}{\colon}
% Define \cs{colon} as a mathpunct `|:|'.
% This is wrong: it should be \unichar{003A} {colon} instead!
% We hope no-one will notice.
%    \begin{macrocode}
\@ifpackageloaded{amsmath}
 {
  % define their own colon, perhaps I should just steal it. (It does look much better.)
 }
 {
  \cs_set_protected:Npn \colon
   {
    \bool_if:NTF \g_@@_literal_colon_bool {:} { \mathpunct{:} }
   }
 }
%    \end{macrocode}
% \end{macro}
%
% \begin{macro}{\digamma}
% \begin{macro}{\Digamma}
% I might end up just changing these in the table.
%    \begin{macrocode}
\protected\def\digamma{\updigamma}
\protected\def\Digamma{\upDigamma}
%    \end{macrocode}
% \end{macro}
% \end{macro}
%
% \paragraph{Symbols}
%    \begin{macrocode}
\cs_set_protected:Npn \| {\Vert}
%    \end{macrocode}
% \cs{mathinner} items:
%    \begin{macrocode}
\cs_set_protected:Npn \mathellipsis {\mathinner{\unicodeellipsis}}
\cs_set_protected:Npn \cdots {\mathinner{\unicodecdots}}
%    \end{macrocode}
%
%    \begin{macrocode}
\cs_set_eq:NN \@@_text_slash: \slash
\cs_set_protected:Npn \slash
 {
  \mode_if_math:TF {\mathslash} {\@@_text_slash:}
 }
%    \end{macrocode}
%
% \paragraph{\cs{not}}
% The situation of \cs{not} symbol is currently messy, in Unicode it is defined
% as a combining mark so naturally it should be treated as a math accent,
% however neither Lua\TeX\ nor \XeTeX\ correctly place it as it needs special
% treatment compared to other accents, furthermore a math accent changes the
% spacing of its nucleus, so \cs{not=} will be spaced as an ordinary not
% relational symbol, which is undesired.
%
% Here modify \cs{not} to a macro that tries to use predefined negated symbols,
% which would give better results in most cases, until there is more robust
% solution in the engines.
%
% This code is based on an answer to a TeX -- Stack Exchange question by Enrico
% Gregorio\footnote{\url{http://tex.stackexchange.com/a/47260/729}}.
%
%    \begin{macrocode}
\cs_new:Npn \@@_newnot:N #1
 {
   \tl_set:Nx \l_not_token_name_tl { \token_to_str:N #1 }
   \exp_args:Nx \tl_if_empty:nF { \tl_tail:V \l_not_token_name_tl }
    {
     \tl_set:Nx \l_not_token_name_tl { \tl_tail:V \l_not_token_name_tl }
    }
   \cs_if_exist:cTF { n \l_not_token_name_tl }
    {
     \use:c { n \l_not_token_name_tl }
    }
    {
     \cs_if_exist:cTF { not \l_not_token_name_tl }
      {
       \use:c { not \l_not_token_name_tl }
      }
      {
       \@@_oldnot: #1
      }
    }
 }
%    \end{macrocode}
%    \begin{macrocode}
\cs_set_eq:NN \@@_oldnot: \not
\AtBeginDocument{\cs_set_eq:NN \not \@@_newnot:N}
%    \end{macrocode}
%    \begin{macrocode}
\cs_new_protected_nopar:Nn \@@_setup_negations:
 {
  \cs_gset:cpn { not= }    { \neq }
  \cs_gset:cpn { not< }    { \nless }
  \cs_gset:cpn { not> }    { \ngtr }
  \cs_gset:Npn  \ngets     { \nleftarrow }
  \cs_gset:Npn  \nsimeq    { \nsime }
  \cs_gset:Npn  \nequal    { \ne }
  \cs_gset:Npn  \nle       { \nleq }
  \cs_gset:Npn  \nge       { \ngeq }
  \cs_gset:Npn  \ngreater  { \ngtr }
  \cs_gset:Npn  \nforksnot { \forks }
 }
%    \end{macrocode}
%
%
%
% \section{Fall-back font}
%
% Want to load Latin Modern Math if nothing else.
% Reset the `font already loaded' boolean so that a new font being set will do the right thing.
% TODO: need a better way to do this for the general case.
%    \begin{macrocode}
\AtBeginDocument { \@@_load_lm_if_necessary: }
\cs_new:Nn \@@_load_lm_if_necessary:
  {
    \cs_if_exist:NF \l_@@_fontname_tl
      {
        % TODO: update this when lmmath-bold.otf is released
        \setmathfont{latinmodern-math.otf}[BoldFont={latinmodern-math.otf}]
        \bool_set_false:N \g_@@_mainfont_already_set_bool
      }
  }
%    \end{macrocode}
%
%    \begin{macrocode}
%</package&(XE|LU)>
%    \end{macrocode}
%
\endinput


