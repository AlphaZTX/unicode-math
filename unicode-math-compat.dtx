% \subsection{Compatibility}
%
%    \begin{macrocode}
%<*compat>
%    \end{macrocode}
%
% \begin{macro}{\@@_check_and_fix:NNnnnn}
% \darg{command}
% \darg{factory command}
% \darg{parameter text}
% \darg{expected replacement text}
% \darg{new replacement text for \LuaTeX}
% \darg{new replacement text for \XeTeX}
% Tries to patch \meta{command}.
% If \meta{command} is undefined, do nothing.
% Otherwise it must be a macro with the given \meta{parameter text} and \meta{expected replacement text}, created by the given \meta{factory command} or equivalent.
% In this case it will be overwritten using the \meta{parameter text} and the \meta{new replacement text for \LuaTeX} or the \meta{new replacement text for \XeTeX}, depending on the engine.
% Otherwise issue a warning and don’t overwrite.
%    \begin{macrocode}
\cs_new_protected_nopar:Nn \@@_check_and_fix:NNnnnn
 {
  \cs_if_exist:NT #1
   {
    \token_if_macro:NTF #1
     {
      \group_begin:
      #2 \@@_tmpa:w #3 { #4 }
      \cs_if_eq:NNTF #1 \@@_tmpa:w
       {
        \msg_info:nnx { unicode-math } { patch-macro }
          { \token_to_str:N #1 }
        \group_end:
        #2 #1 #3
%<XE>          { #6 }
%<LU>          { #5 }
       }
       {
        \msg_warning:nnxxx { unicode-math } { wrong-meaning }
          { \token_to_str:N #1 } { \token_to_meaning:N #1 }
          { \token_to_meaning:N \@@_tmpa:w }
        \group_end:
       }
     }
     {
      \msg_warning:nnx { unicode-math } { macro-expected }
        { \token_to_str:N #1 }
     }
   }
 }
%    \end{macrocode}
% \end{macro}
%
% \begin{macro}{\@@_check_and_fix:NNnnn}
% \darg{command}
% \darg{factory command}
% \darg{parameter text}
% \darg{expected replacement text}
% \darg{new replacement text}
% Tries to patch \meta{command}.
% If \meta{command} is undefined, do nothing.
% Otherwise it must be a macro with the given \meta{parameter text} and \meta{expected replacement text}, created by the given \meta{factory command} or equivalent.
% In this case it will be overwritten using the \meta{parameter text} and the \meta{new replacement text}.
% Otherwise issue a warning and don’t overwrite.
%    \begin{macrocode}
\cs_new_protected_nopar:Nn \@@_check_and_fix:NNnnn
 {
  \@@_check_and_fix:NNnnnn #1 #2 { #3 } { #4 } { #5 } { #5 }
 }
%    \end{macrocode}
% \end{macro}
%
% \begin{macro}{\@@_check_and_fix_luatex:NNnnn}
% \begin{macro}{\@@_check_and_fix_luatex:cNnnn}
% \darg{command}
% \darg{factory command}
% \darg{parameter text}
% \darg{expected replacement text}
% \darg{new replacement text}
% Tries to patch \meta{command}.
% If \XeTeX\ is the current engine or \meta{command} is undefined, do nothing.
% Otherwise it must be a macro with the given \meta{parameter text} and \meta{expected replacement text}, created by the given \meta{factory command} or equivalent.
% In this case it will be overwritten using the \meta{parameter text} and the \meta{new replacement text}.
% Otherwise issue a warning and don’t overwrite.
%    \begin{macrocode}
\cs_new_protected_nopar:Nn \@@_check_and_fix_luatex:NNnnn
 {
%<LU>    \@@_check_and_fix:NNnnn #1 #2 { #3 } { #4 } { #5 }
 }
\cs_generate_variant:Nn \@@_check_and_fix_luatex:NNnnn { c }
%    \end{macrocode}
% \end{macro}
% \end{macro}
%
% \paragraph{\pkg{url}}
% Simply need to get \pkg{url} in a state such that
% when it switches to math mode and enters \ascii\ characters, the maths
% setup (i.e., \pkg{unicode-math}) doesn't remap the symbols into Plane 1.
% Which is, of course, what \cs{mathup} is doing.
%
% This is the same as writing, e.g., |\def\UrlFont{\ttfamily\@@_switchto_up:}|
% but activates automatically so old documents that might change the \cs{url}
% font still work correctly.
%    \begin{macrocode}
\AtEndOfPackageFile * {url}
 {
  \tl_put_left:Nn \Url@FormatString { \@@_switchto_up: }
  \tl_put_right:Nn \UrlSpecials
   {
    \do\`{\mathchar`\`}
    \do\'{\mathchar`\'}
    \do\${\mathchar`\$}
    \do\&{\mathchar`\&}
   }
 }
%    \end{macrocode}
%
% \paragraph{\pkg{amsmath}}
% Since the mathcode of |`\-| is greater than eight bits, this piece of |\AtBeginDocument| code from \pkg{amsmath} dies if we try and set the maths font in the preamble:
%    \begin{macrocode}
\AtEndOfPackageFile * {amsmath}
 {
%<*XE>
    \tl_remove_once:Nn \@begindocumenthook
     {
      \mathchardef\std@minus\mathcode`\-\relax
      \mathchardef\std@equal\mathcode`\=\relax
     }
    \def\std@minus{\Umathcharnum\Umathcodenum`\-\relax}
    \def\std@equal{\Umathcharnum\Umathcodenum`\=\relax}
%</XE>
%    \end{macrocode}
%
%    \begin{macrocode}
  \cs_set:Npn \@cdots {\mathinner{\cdots}}
  \cs_set_eq:NN \dotsb@ \cdots
%    \end{macrocode}
% This isn't as clever as the \pkg{amsmath} definition but I think it works:
%    \begin{macrocode}
%<*XE>
    \def \resetMathstrut@
     {%
      \setbox\z@\hbox{$($}%)
      \ht\Mathstrutbox@\ht\z@ \dp\Mathstrutbox@\dp\z@
     }
%    \end{macrocode}
% The |subarray| environment uses inappropriate font dimensions.
%    \begin{macrocode}
    \@@_check_and_fix:NNnnn \subarray \cs_set:Npn { #1 }
     {
      \vcenter
      \bgroup
      \Let@
      \restore@math@cr
      \default@tag
      \baselineskip \fontdimen 10~ \scriptfont \tw@
      \advance \baselineskip \fontdimen 12~ \scriptfont \tw@
      \lineskip \thr@@@@ \fontdimen 8~ \scriptfont \thr@@@@
      \lineskiplimit \lineskip
      \ialign
      \bgroup
      \ifx c #1 \hfil \fi
      $ \m@th \scriptstyle ## $
      \hfil
      \crcr
     }
     {
      \vcenter
      \c_group_begin_token
      \Let@
      \restore@math@cr
      \default@tag
      \skip_set:Nn \baselineskip
       {
%    \end{macrocode}
% Here we use stack top shift + stack bottom shift, which sounds reasonable.
%    \begin{macrocode}
        \@@_stack_num_up:N \scriptstyle
        + \@@_stack_denom_down:N \scriptstyle
       }
%    \end{macrocode}
% Here we use the minimum stack gap.
%    \begin{macrocode}
      \lineskip \@@_stack_vgap:N \scriptstyle
      \lineskiplimit \lineskip
      \ialign
      \c_group_begin_token
      \token_if_eq_meaning:NNT c #1 { \hfil }
      \c_math_toggle_token
      \m@th
      \scriptstyle
      \c_parameter_token \c_parameter_token
      \c_math_toggle_token
      \hfil
      \crcr
     }
%</XE>
%    \end{macrocode}
% The roots need a complete rework.
%    \begin{macrocode}
  \@@_check_and_fix_luatex:NNnnn \plainroot@ \cs_set_nopar:Npn { #1 \of #2 }
   {
    \setbox \rootbox \hbox
     {
      $ \m@th \scriptscriptstyle { #1 } $
     }
    \mathchoice
      { \r@@@@t \displaystyle      { #2 } }
      { \r@@@@t \textstyle         { #2 } }~
      { \r@@@@t \scriptstyle       { #2 } }
      { \r@@@@t \scriptscriptstyle { #2 } }
    \egroup
   }
   {
    \bool_if:nTF
     {
      \int_compare_p:nNn { \uproot@ } = { \c_zero }
      && \int_compare_p:nNn { \leftroot@ } = { \c_zero }
     }
     {
        \Uroot \l_@@_radical_sqrt_tl { #1 } { #2 }
     }
     {
      \hbox_set:Nn \rootbox
       {
        \c_math_toggle_token
        \m@th
        \scriptscriptstyle { #1 }
        \c_math_toggle_token
       }
      \mathchoice
        { \r@@@@t \displaystyle      { #2 } }
        { \r@@@@t \textstyle         { #2 } }
        { \r@@@@t \scriptstyle       { #2 } }
        { \r@@@@t \scriptscriptstyle { #2 } }
     }
    \c_group_end_token
   }
  \@@_check_and_fix:NNnnnn \r@@@@t \cs_set_nopar:Npn { #1 #2 }
   {
    \setboxz@h { $ \m@th #1 \sqrtsign { #2 } $ }
    \dimen@ \ht\z@
    \advance \dimen@ -\dp\z@
    \setbox\@ne \hbox { $ \m@th #1 \mskip \uproot@ mu $ }
    \advance \dimen@ by 1.667 \wd\@ne
    \mkern -\leftroot@ mu
    \mkern 5mu
    \raise .6\dimen@ \copy\rootbox
    \mkern -10mu
    \mkern \leftroot@ mu
    \boxz@
   }
   {
    \hbox_set:Nn \l_tmpa_box
     {
      \c_math_toggle_token
      \m@th
      #1
      \mskip \uproot@ mu
      \c_math_toggle_token
     }
      \Uroot \l_@@_radical_sqrt_tl
     {
      \box_move_up:nn { \box_wd:N \l_tmpa_box }
       {
        \hbox:n
         {
          \c_math_toggle_token
          \m@th
          \mkern -\leftroot@ mu
          \box_use:N \rootbox
          \mkern \leftroot@ mu
          \c_math_toggle_token
         }
       }
     }
     { #2 }
   }
   {
    \hbox_set:Nn \l_tmpa_box
     {
      \c_math_toggle_token
      \m@th
      #1
      \sqrtsign { #2 }
      \c_math_toggle_token
     }
    \hbox_set:Nn \l_tmpb_box
     {
      \c_math_toggle_token
      \m@th
      #1
      \mskip \uproot@ mu
      \c_math_toggle_token
     }
    \mkern -\leftroot@ mu
    \@@_mathstyle_scale:Nnn #1 { \kern }
     {
      \fontdimen 63 \l_@@_font
     }
    \box_move_up:nn
     {
      \box_wd:N \l_tmpb_box
      + (\box_ht:N \l_tmpa_box - \box_dp:N \l_tmpa_box)
      * \number \fontdimen 65 \l_@@_font / 100
     } 
     {
      \box_use:N \rootbox
     }
    \@@_mathstyle_scale:Nnn #1 { \kern }
     {
      \fontdimen 64 \l_@@_font
     }
    \mkern \leftroot@ mu
    \box_use_clear:N \l_tmpa_box
   }
 }
%    \end{macrocode}
%
% \paragraph{\pkg{amsopn}}
% This code is to improve the output of analphabetic symbols in text of operator names (\cs{sin}, \cs{cos}, etc.). Just comment out the offending lines for now:
%    \begin{macrocode}
%<*XE>
\AtEndOfPackageFile * {amsopn}
 {
  \cs_set:Npn \newmcodes@
   {
    \mathcode`\'39\scan_stop:
    \mathcode`\*42\scan_stop:
    \mathcode`\."613A\scan_stop:
%%  \ifnum\mathcode`\-=45 \else
%%    \mathchardef\std@minus\mathcode`\-\relax
%%  \fi
    \mathcode`\-45\scan_stop:
    \mathcode`\/47\scan_stop:
    \mathcode`\:"603A\scan_stop:
   }
 }
%</XE>
%    \end{macrocode}
%
% \paragraph{\pkg{mathtools}}
% \pkg{mathtools}’s |\cramped| command and others that make use of its internal version use an incorrect font dimension.
%
%    \begin{macrocode}
%<*XE>
\AtEndOfPackageFile * { mathtools }
 {
    \@@_check_and_fix:NNnnn
        \MT_cramped_internal:Nn \cs_set_nopar:Npn { #1 #2 }
     {
      \sbox \z@
       {
        $
        \m@th
        #1
        \nulldelimiterspace = \z@
        \radical \z@ { #2 }
        $
       }
      \ifx #1 \displaystyle
        \dimen@ = \fontdimen 8 \textfont 3
        \advance \dimen@ .25 \fontdimen 5 \textfont 2
      \else
        \dimen@ = 1.25 \fontdimen 8
        \ifx #1 \textstyle
          \textfont
        \else
          \ifx #1 \scriptstyle
            \scriptfont
          \else
            \scriptscriptfont
          \fi
        \fi
        3
      \fi
      \advance \dimen@ -\ht\z@
      \ht\z@ = -\dimen@
      \box\z@
     }
%    \end{macrocode}
% The \XeTeX\ version is pretty similar to the legacy version, only using the correct font dimensions.
% Note we used `\verb|\XeTeXradical|' with the family 255 to be almost sure
% that the radical rule width is not set. Former use of `\newfam` had an
% upsetting effect on legacy math alphabets.
%    \begin{macrocode}
     {
      \hbox_set:Nn \l_tmpa_box
       {
        \color@setgroup
        \c_math_toggle_token
        \m@th
        #1
        \dim_zero:N \nulldelimiterspace
        \XeTeXradical \c_two_hundred_fifty_five \c_zero { #2 }
        \c_math_toggle_token
        \color@endgroup
       }
      \box_set_ht:Nn \l_tmpa_box
       {
        \box_ht:N \l_tmpa_box
%    \end{macrocode}
% Here we use the radical vertical gap.
%    \begin{macrocode}
        - \@@_radical_vgap:N #1
       }
      \box_use_clear:N \l_tmpa_box
     }
 }
%</XE>
%    \end{macrocode}
%
% \begin{macro}{\overbracket}
% \begin{macro}{\underbracket}
%   \pkg{mathtools}’s |\overbracket| and |\underbracket| take optional
%   arguments and are defined in terms of rules, so we keep them, and rename
%   ours to |\Uoverbracket| and |\Uunderbracket|.
%    \begin{macrocode}
\AtEndOfPackageFile * { mathtools }
 {
    \cs_set_eq:NN \MToverbracket  \overbracket
    \cs_set_eq:NN \MTunderbracket \underbracket

    \AtBeginDocument
     {
      \msg_warning:nn { unicode-math } { mathtools-overbracket }

	\def\downbracketfill#1#2
	 {%
%    \end{macrocode}
%   Original definition used the height of |\braceld| which is not available
%   with Unicode fonts, so we are hard coding the $5/18$ex suggested by
%   \pkg{mathtools}’s documentation.
%    \begin{macrocode}
            \edef\l_MT_bracketheight_fdim{.27ex}%
            \downbracketend{#1}{#2}
            \leaders \vrule \@height #1 \@depth \z@ \hfill
            \downbracketend{#1}{#2}%
     }
	\def\upbracketfill#1#2
	 {%
            \edef\l_MT_bracketheight_fdim{.27ex}%
            \upbracketend{#1}{#2}
            \leaders \vrule \@height \z@ \@depth #1 \hfill
            \upbracketend{#1}{#2}%
     }
	\let\Uoverbracket =\overbracket
	\let\Uunderbracket=\underbracket
        \let\overbracket  =\MToverbracket
        \let\underbracket =\MTunderbracket
     }% end of AtBeginDocument
%    \end{macrocode}
% \end{macro}
% \end{macro}
%
% \begin{macro}{\dblcolon}
% \begin{macro}{\coloneqq}
% \begin{macro}{\Coloneqq}
% \begin{macro}{\eqqcolon}
%   \pkg{mathtools} defines several commands as combinations of colons and
%   other characters, but with meanings incompatible to \pkg{unicode-math}.
%   Thus we issue a warning.  Because \pkg{mathtools} uses
%   \cmd{\providecommand} \cmd{\AtBeginDocument}, we can just define the
%   offending commands here.
%    \begin{macrocode}
  \msg_warning:nn { unicode-math } { mathtools-colon }
  \NewDocumentCommand \dblcolon { } { \Colon }
  \NewDocumentCommand \coloneqq { } { \coloneq }
  \NewDocumentCommand \Coloneqq { } { \Coloneq }
  \NewDocumentCommand \eqqcolon { } { \eqcolon }
 }
%    \end{macrocode}
% \end{macro}
% \end{macro}
% \end{macro}
% \end{macro}
%
% \paragraph{\pkg{colonequals}}
%
% \begin{macro}{\ratio}
% \begin{macro}{\coloncolon}
% \begin{macro}{\minuscolon}
% \begin{macro}{\colonequals}
% \begin{macro}{\equalscolon}
% \begin{macro}{\coloncolonequals}
%   Similarly to \pkg{mathtools}, the \pkg{colonequals} defines several colon
%   combinations.  Fortunately there are no name clashes, so we can just
%   overwrite their definitions.
%    \begin{macrocode}
\AtEndOfPackageFile * { colonequals }
 {
  \msg_warning:nn { unicode-math } { colonequals }
  \RenewDocumentCommand \ratio { } { \mathratio }
  \RenewDocumentCommand \coloncolon { } { \Colon }
  \RenewDocumentCommand \minuscolon { } { \dashcolon }
  \RenewDocumentCommand \colonequals { } { \coloneq }
  \RenewDocumentCommand \equalscolon { } { \eqcolon }
  \RenewDocumentCommand \coloncolonequals { } { \Coloneq }
 }
%    \end{macrocode}
% \end{macro}
% \end{macro}
% \end{macro}
% \end{macro}
% \end{macro}
% \end{macro}
%
%    \begin{macrocode}
%</compat>
%    \end{macrocode}