%%^^A%% um-code-opening.dtx -- part of UNICODE-MATH <wspr.io/unicode-math>
%%^^A%% Assorted initialisation tasks, including some low-level function definitions.

% \section{Start of the package code}
%
% The prefix for \pkg{unicode-math} is \texttt{um}:
%    \begin{macrocode}
%<@@=um>
%    \end{macrocode}
%
%    \begin{macrocode}
%<*package>
%    \end{macrocode}
%
%    \begin{macrocode}
%<*LU>
\RequirePackage{lualatex-math}
%</LU>
%    \end{macrocode}
%
%    \begin{macrocode}
\ExplSyntaxOn
%    \end{macrocode}
%
% \subsection{\pkg{expl3} variants}
%
% Variants needed from \pkg{expl3}:
%    \begin{macrocode}
\cs_set_protected_nopar:Npn \exp_last_unbraced:NNx { \::N \::x_unbraced \::: }
%    \end{macrocode}
%
% For \pkg{fontspec}:
%    \begin{macrocode}
\cs_generate_variant:Nn \fontspec_set_family:Nnn {Nx}
\cs_generate_variant:Nn \prop_get:NnNTF {cx}
\cs_generate_variant:Nn \tl_if_eq:nnF {o}
%    \end{macrocode}
%
%
% \subsection{Low level commands}
%
%    \begin{macrocode}
\cs_set_eq:NN \@@_group_begin: \group_begin:
\cs_set_eq:NN \@@_group_end:   \group_end:
\cs_set_eq:NN \@@_group_begin_frozen: \@@_group_begin:
\cs_set_eq:NN \@@_group_end_frozen:   \@@_group_end:
%    \end{macrocode}
%
%
% \subsection{Primitive font commands}
%
% What might end up being provided by the kernel.
%
% \begin{macro}{\@@_glyph_if_exist:NnTF}
%    \begin{macrocode}
\prg_new_conditional:Nnn \@@_glyph_if_exist:Nn {p,TF,T,F}
  {
    \etex_iffontchar:D #1 #2 \scan_stop:
      \prg_return_true:
    \else:
      \prg_return_false:
    \fi:
  }
%    \end{macrocode}
% \end{macro}
%
% \begin{macro}{\@@_fontface_gset_eq:NN}
%    \begin{macrocode}
\cs_set_protected:Nn \@@_fontface_gset_eq:NN
  {
    \tex_global:D \tex_let:D #1 #2
  }
\cs_generate_variant:Nn \@@_fontface_gset_eq:NN {cN}
%    \end{macrocode}
% \end{macro}
%
% \subsubsection{Mathcode and friends}
%
% \begin{macro}{\@@_set_mathcode:nnnn}
% \begin{macro}{\@@_set_mathcode:nnn}
% These are all wrappers for the primitive commands that take numerical
% input only.
%    \begin{macrocode}
\cs_set:Npn \@@_set_mathcode:nnnn #1#2#3#4
  {
    \Umathcode \int_eval:n {#1} =
      \mathchar@type#2 \csname sym#3\endcsname \int_eval:n {#4} \scan_stop:
  }
%    \end{macrocode}
%
%    \begin{macrocode}
\cs_set:Npn \@@_set_mathcode:nnn #1#2#3
  {
    \Umathcode \int_eval:n {#1} =
      \mathchar@type#2 \csname sym#3\endcsname \int_eval:n {#1} \scan_stop:
  }
%    \end{macrocode}
% \end{macro}
% \end{macro}
%
% \begin{macro}{\@@_set_mathchar:NNnn}
% \begin{macro}{\@@_set_mathchar:cNnn}
%    \begin{macrocode}
\cs_set:Npn \@@_set_mathchar:NNnn #1#2#3#4
  {
    \Umathchardef #1 =
      \mathchar@type#2 \csname sym#3\endcsname \int_eval:n {#4} \scan_stop:
  }
%    \end{macrocode}
% \end{macro}
% \end{macro}
%
%    \begin{macrocode}
\cs_generate_variant:Nn \@@_set_mathchar:NNnn {c}
%    \end{macrocode}
%
% \begin{macro}{\@@_set_delcode:nnn}
%    \begin{macrocode}
\cs_new:Nn \@@_set_delcode:nnn
  {
    \Udelcode#2 = \csname sym#1\endcsname #3 \scan_stop:
  }
%    \end{macrocode}
% \end{macro}
%
% \begin{macro}{\@@_radical:nn}
%    \begin{macrocode}
\cs_new:Nn \@@_radical:nn
  {
    \Uradical \csname sym#1\endcsname #2 \scan_stop:
  }
%    \end{macrocode}
% \end{macro}
%
% \begin{macro}{\@@_delimiter:Nnn}
%    \begin{macrocode}
\cs_new:Nn \@@_delimiter:Nnn
  {
    \Udelimiter \mathchar@type#1 \csname sym#2\endcsname #3 \scan_stop:
  }
%    \end{macrocode}
% \end{macro}
%
% \begin{macro}{\@@_accent:nnn}
%    \begin{macrocode}
\cs_new:Nn \@@_accent:nnn
  {
    \Umathaccent #1~ \mathchar@type\mathaccent \use:c { sym #2 } #3 \scan_stop:
  }
%    \end{macrocode}
% \end{macro}
%
% \begin{macro}{\@@_char_gmake_mathactive:N}
% \begin{macro}{\@@_char_gmake_mathactive:n}
%    \begin{macrocode}
\cs_new:Nn \@@_char_gmake_mathactive:N
  {
    \tex_global:D \tex_mathcode:D `#1 = "8000 \scan_stop:
  }
%    \end{macrocode}
%    \begin{macrocode}
\cs_new:Nn \@@_char_gmake_mathactive:n
  {
    \tex_global:D \tex_mathcode:D \int_eval:n {#1} = "8000 \scan_stop:
  }
%    \end{macrocode}
% \end{macro}
% \end{macro}
%
% \begin{macro}{\@@_mathactive_remap:nn}
% Makes |#1| math-active and defines its meaning to be |#2|.
% This is a global operation.
%    \begin{macrocode}
\cs_new:Nn \@@_mathactive_remap:nn
  {
    \group_begin:
      \cs_set_protected:Npn \@@_tmp: {#2}
      \@@_char_gmake_mathactive:n {#1}
      \char_gset_active_eq:nN {#1} \@@_tmp:
    \group_end:
  }
%    \end{macrocode}
%
% \subsubsection{Font parameters}
%
% \begin{macro}{\@@_copy_fontdimen:nnN}
%    \begin{macrocode}
\cs_new:Nn \@@_copy_fontdimen:nnN
  {
    \fontdimen #1 \font = \the \fontdimen #2 #3 \relax
  }
%    \end{macrocode}
% \end{macro}
%
% \begin{macro}{\@@_zero_fontdimen:n}
%    \begin{macrocode}
\cs_new:Nn \@@_zero_fontdimen:n
  {
    \fontdimen #1 \font = 0pt\relax
  }
%    \end{macrocode}
% \end{macro}
%
% \begin{macro}{\@@_fontdimen_from_param:Nnn}
% This function extracts the math font dimen \verb|#3| from the font \verb|#1|
% and sets fontdimen \verb|#2| of the same font to that value.
%
% Use \XeTeX's fontdimen approach because it's tidy. We don't need bells and whistles here.
%    \begin{macrocode}
%<*LU>
\cs_new_protected:Nn \@@_fontdimen_from_param:nn
  {
    \fontdimen #1 \font = \directlua{fontspec.mathfontdimen(font.current(),"#2")}\relax
  }
%</LU>
%    \end{macrocode}
% \end{macro}
%
%
% \subsection{Alphabet Unicode positions (USVs)}
%
% Before we begin, let's define the positions of the various Unicode
% alphabets so that our code is a little more readable.\footnote{`\textsc{u.s.v.}' stands
% for `Unicode scalar value'.}
%
% \begin{macro}{\usv_set:nnn,\@@_to_usv:nn}
% Rather than `readable', in the end, this makes the code more extensible.
%    \begin{macrocode}
\cs_new:Nn \usv_set:nnn  { \tl_const:cn { c_@@_#1_#2_usv } {#3} }
\cs_new:Nn \@@_to_usv:nn {       \use:c { c_@@_#1_#2_usv } }
%    \end{macrocode}
% \end{macro}
%
% \begin{macro}{\@@_usv_if_exist:nnTF}
%    \begin{macrocode}
\prg_new_conditional:Nnn \@@_usv_if_exist:nn {T,F,TF}
  {
    \cs_if_exist:cTF { c_@@_#1_#2_usv }
      \prg_return_true: \prg_return_false:
  }
%    \end{macrocode}
% \end{macro}
%
%
% \subsection{Overcoming \texorpdfstring{\cmd\@onlypreamble}{\textbackslash @onlypreamble}}
%
% The requirement of only setting up the maths fonts in the preamble is lifted.
% (Perhaps unwisely.)
%    \begin{macrocode}
\tl_map_inline:nn
  {
    \new@mathgroup\cdp@list\cdp@elt\DeclareMathSizes
    \@DeclareMathSizes\newmathalphabet\newmathalphabet@@\newmathalphabet@@@
    \DeclareMathVersion\define@mathalphabet\define@mathgroup\addtoversion
    \version@list\version@elt\alpha@list\alpha@elt
    \restore@mathversion\init@restore@version\dorestore@version\process@table
    \new@mathversion\DeclareSymbolFont\group@list\group@elt
    \new@symbolfont\SetSymbolFont\SetSymbolFont@\get@cdp
    \DeclareMathAlphabet\new@mathalphabet\SetMathAlphabet\SetMathAlphabet@
    \DeclareMathAccent\set@mathaccent\DeclareMathSymbol\set@mathchar
    \set@mathsymbol\DeclareMathDelimiter\@xxDeclareMathDelimiter
    \@DeclareMathDelimiter\@xDeclareMathDelimiter\set@mathdelimiter
    \set@@mathdelimiter\DeclareMathRadical\mathchar@type
    \DeclareSymbolFontAlphabet\DeclareSymbolFontAlphabet@
  }
  {
    \tl_remove_once:Nn \@preamblecmds {\do#1}
  }
%    \end{macrocode}
%
%
% \subsection{Wrappers for message functions}
%
% Messages themselves are defined in \secref{codemsg}.
%
%    \begin{macrocode}
\cs_new:Npn \@@_error:n     { \msg_error:nn     {unicode-math} }
\cs_new:Npn \@@_error:nx    { \msg_error:nnx    {unicode-math} }
\cs_new:Npn \@@_warning:n   { \msg_warning:nn   {unicode-math} }
\cs_new:Npn \@@_warning:nnn { \msg_warning:nnxx {unicode-math} }
\cs_new:Npn \@@_log:n       { \msg_log:nn       {unicode-math} }
\cs_new:Npn \@@_log:nx      { \msg_log:nnx      {unicode-math} }
%    \end{macrocode}
%
%    \begin{macrocode}
\cs_generate_variant:Nn \msg_new:nnn  {nnx}
\cs_generate_variant:Nn \msg_new:nnnn {nnxx}
\cs_new:Nn \@@_msg_new:nn  { \msg_new:nnx  {unicode-math} {#1} { \tl_trim_spaces:n {#2} } }
%    \end{macrocode}
%
%
%    \begin{macrocode}
%</package>
%    \end{macrocode}

\endinput


% \subsection{Sow/reap}
%
% This might make it into the kernel; not sure.
% Not actually used yet in the package!
%
% \begin{macro}{\__@@_sow_item:Nn}
% Helper variables for constructing \verb|c|-type arguments.
%    \begin{macrocode}
\cs_new:Nn \__@@_sow_item:Nn  { l___sow___ \token_to_str:N #1 ___ #2 ___tl }
%    \end{macrocode}
% \end{macro}
%
% \begin{macro}{\__@@_sow_int:N}
% Helper variables for constructing \verb|c|-type arguments.
%    \begin{macrocode}
\cs_new:Nn \__@@_sow_int:N  { l___sow___ \token_to_str:N #1 ___int }
%    \end{macrocode}
% \end{macro}
%
% \begin{macro}{\@@_sow_prepare:N}
%    \begin{macrocode}
\cs_new:Nn \@@_sow_prepare:N
  {
    \tl_clear:N #1
    \int_zero_new:c { \__@@_sow_int:N #1 }
  }
%    \end{macrocode}
% \end{macro}
%
% \begin{macro}{\@@_sow:Nn}
%    \begin{macrocode}
\cs_new:Nn \@@_sow:Nn
  {
    \int_incr:c { \__@@_sow_int:N #1 }
    \tl_clear_new:c
      {
        \__@@_sow_item:Nn #1
          { \int_use:c { \__@@_sow_int:N #1 } }
      }
    \tl_set:cn
      {
        \__@@_sow_item:Nn #1
          { \int_use:c { \__@@_sow_int:N #1 } }
      }
      { #2 }
  }
%    \end{macrocode}
% \end{macro}
%
% \begin{macro}{\@@_sow_reap:N}
%    \begin{macrocode}
\cs_new:Nn \@@_sow_reap:N
  {
    \cs_set:Npn \__@@_sow_tmp:n ##1
      {
        \exp_not:v { \__@@_sow_item:Nn #1 {##1} }
      }
    \tl_set:Nx #1
      {
        \int_step_function:nnnN {1} {1}
          { \int_use:c { \__@@_sow_int:N #1 } }
          \__@@_sow_tmp:n
      }

  }
%    \end{macrocode}
% \end{macro}


% /©
%
% ------------------------------------------------
% The UNICODE-MATH package  <wspr.io/unicode-math>
% ------------------------------------------------
% This package is free software and may be redistributed and/or modified under
% the conditions of the LaTeX Project Public License, version 1.3c or higher
% (your choice): <http://www.latex-project.org/lppl/>.
% ------------------------------------------------
% Copyright 2006-2018  Will Robertson, LPPL "maintainer"
% Copyright 2010-2017  Philipp Stephani
% Copyright 2011-2017  Joseph Wright
% Copyright 2012-2015  Khaled Hosny
% ------------------------------------------------
%
% ©/
