%%^^J%%  um-code-main.dtx  -  part of UNICODE-MATH <github.com/wspr/unicode-math>

% \section{The main \cs{setmathfont} macro}
%
%    \begin{macrocode}
%<*package>
%    \end{macrocode}
%
% Using a |range| including large character sets such as \cmd\mathrel,
% \cmd\mathalpha, \etc, is \emph{very slow}!
% I hope to improve the performance somehow.
%
% \begin{macro}{\@@_setmathfont:nn}
%    \begin{macrocode}
\cs_set:Nn \@@_setmathfont:nn
  {
    \tl_set:Nn \l_@@_fontname_tl {#2}
    \@@_init:
%    \end{macrocode}
% Grab the current size information:
% (is this robust enough? Maybe it should be preceded by \cmd\normalsize).
% The macro \cmd\S@\meta{size}
% contains the definitions of the sizes used for maths letters, subscripts and subsubscripts in
% \cmd\tf@size, \cmd\sf@size, and \cmd\ssf@size, respectively.
%    \begin{macrocode}
    \cs_if_exist:cF { S@ \f@size } { \calculate@math@sizes }
    \csname S@\f@size\endcsname
%    \end{macrocode}
% Parse options and tell people what's going on:
%    \begin{macrocode}
    \keys_set_known:nnN {unicode-math} {#1} \l_@@_unknown_keys_clist
    \bool_if:NT \l_@@_init_bool { \@@_log:n {default-math-font} }
%    \end{macrocode}
% Use \pkg{fontspec} to select a font to use.
% After loading the font, we detect what sizes it recommends for scriptsize and scriptscriptsize, so after setting those values appropriately, we reload the font to take these into account.
%    \begin{macrocode}
%<debug>  \csname TIC\endcsname
    \@@_fontspec_select_font:
%<debug>  \csname TOC\endcsname
    \bool_if:nT { \l_@@_ot_math_bool && !\g_@@_mainfont_already_set_bool }
      {
        \@@_declare_math_sizes:
        \@@_fontspec_select_font:
      }
%    \end{macrocode}
% Now define |\@@_symfont_tl| as the \LaTeX\ math font to access everything:
%    \begin{macrocode}
    \cs_if_exist:cF { sym \@@_symfont_tl }
      {
        \DeclareSymbolFont{\@@_symfont_tl}
          {\encodingdefault}{\l_@@_family_tl}{\mddefault}{\updefault}
      }
    \SetSymbolFont{\@@_symfont_tl}{\l_@@_mversion_tl}
      {\encodingdefault}{\l_@@_family_tl}{\mddefault}{\updefault}
%    \end{macrocode}
% Set the bold math version.
%    \begin{macrocode}
    \str_if_eq_x:nnT {\l_@@_mversion_tl} {normal}
      {
        \SetSymbolFont{\@@_symfont_tl}{bold}
          {\encodingdefault}{\l_@@_family_tl}{\bfdefault}{\updefault}
      }
%    \end{macrocode}
% Declare the math sizes (i.e., scaling of superscripts) for the specific
% values for this font,
% and set defaults for math fams two and three for legacy compatibility:
%    \begin{macrocode}
    \bool_if:nT { \l_@@_ot_math_bool && !\g_@@_mainfont_already_set_bool }
      {
        \bool_set_true:N \g_@@_mainfont_already_set_bool
        \@@_setup_legacy_fam_two:
        \@@_setup_legacy_fam_three:
      }
%    \end{macrocode}
% And now we input every single maths char.
%    \begin{macrocode}
%<debug>  \csname TIC\endcsname
    \@@_input_math_symbol_table:
%<debug>  \csname TOC\endcsname
%    \end{macrocode}
% Finally,
% \begin{itemize}
% \item Remap symbols that don't take their natural mathcode
% \item Activate any symbols that need to be math-active
% \item Enable wide/narrow accents
% \item Assign delimiter codes for symbols that need to grow
% \item Setup the maths alphabets (\cs{mathbf} etc.).
%       This is an extensive part of the code; see Section~\ref{sec:mathmap}.
% \item Setup negations, which are handled on an ad hoc basis; see Section~\ref{sec:negations}.
% \end{itemize}
%    \begin{macrocode}
    \@@_remap_symbols:
    \@@_setup_mathactives:
    \@@_setup_delcodes:
%<debug>  \csname TIC\endcsname
    \@@_setup_alphabets:
%<debug>  \csname TOC\endcsname
    \@@_setup_negations:
  }
%    \end{macrocode}
% \end{macro}
%
% \paragraph{Fall-back font}
%
% Want to load Latin Modern Math if nothing else.
% This needs to happen early so that all of the font-loading machinery executes before
% the other `AtBeginDocument' code.
%    \begin{macrocode}
\AtBeginDocument { \@@_load_lm_if_necessary: }
\cs_new:Nn \@@_load_lm_if_necessary:
  {
    \cs_if_exist:NF \l_@@_fontname_tl
      {
        % TODO: update this when lmmath-bold.otf is released
        \setmathfont{latinmodern-math.otf}[BoldFont={latinmodern-math.otf}]
        \bool_set_false:N \g_@@_mainfont_already_set_bool
      }
  }
%    \end{macrocode}
% Note that here we reset the `font already loaded' boolean so that a new font being set
% will do the right thing in terms of setting up defaults.
%
% TODO: need a better way to do this for the general case. (Maybe a `reset' command option?)
%
% \begin{macro}{\@@_init:}
%    \begin{macrocode}
\cs_new:Nn \@@_init:
  {
%    \end{macrocode}
% \begin{itemize}
% \item Initially assume we're using a proper OpenType font with unicode maths.
%    \begin{macrocode}
    \bool_set_true:N  \l_@@_ot_math_bool
%    \end{macrocode}
% \item Erase any conception \LaTeX\ has of previously defined math symbol fonts;
% this allows \cmd\DeclareSymbolFont\ at any point in the document.
%    \begin{macrocode}
    \cs_set_eq:NN \glb@currsize \scan_stop:
%    \end{macrocode}
% \item To start with, assume we're defining the font for every math symbol character.
%    \begin{macrocode}
    \bool_set_true:N \l_@@_init_bool
    \seq_clear:N \l_@@_char_range_seq
    \clist_clear:N \l_@@_char_nrange_clist
    \seq_clear:N \l_@@_mathalph_seq
    \seq_clear:N \l_@@_missing_alph_seq
%    \end{macrocode}
% \item By default use the `normal' math version.
%    \begin{macrocode}
    \tl_set:Nn \l_@@_mversion_tl {normal}
%    \end{macrocode}
% \item Other range initialisations.
%    \begin{macrocode}
    \tl_set:Nn \@@_symfont_tl {operators}
    \cs_set_eq:NN \_@@_sym:nnn \@@_process_symbol_noparse:nnn
    \cs_set_eq:NN \@@_set_mathalphabet_char:nnn \@@_mathmap_noparse:nnn
    \cs_set_eq:NN \@@_remap_symbol:nnn \@@_remap_symbol_noparse:nnn
    \cs_set_eq:NN \@@_maybe_init_alphabet:n \@@_init_alphabet:n
    \cs_set_eq:NN \@@_map_char_single:nn \@@_map_char_noparse:nn
    \cs_set_eq:NN \@@_assign_delcode:nn \@@_assign_delcode_noparse:nn
    \cs_set_eq:NN \@@_make_mathactive:nNN \@@_make_mathactive_noparse:nNN
%    \end{macrocode}
% \item Define default font features for the script and scriptscript font.
%    \begin{macrocode}
    \tl_set:Nn \l_@@_script_features_tl  {Style=MathScript}
    \tl_set:Nn \l_@@_sscript_features_tl {Style=MathScriptScript}
    \tl_set_eq:NN \l_@@_script_font_tl   \l_@@_fontname_tl
    \tl_set_eq:NN \l_@@_sscript_font_tl  \l_@@_fontname_tl
%    \end{macrocode}
% \end{itemize}
%    \begin{macrocode}
  }
%    \end{macrocode}
% \end{macro}
%
% \begin{macro}{\@@_declare_math_sizes:}
% Set the math sizes according to the recommended font parameters.
% TODO: this shouldn't need to be per-engine; check out why the wrappers aren't used.
%    \begin{macrocode}
\cs_new:Nn \@@_declare_math_sizes:
  {
%<*LU>
    \fp_compare:nF { \@@_script_style_size:n {ScriptPercentScaleDown} == 0 }
      {
        \DeclareMathSizes { \f@size } { \f@size }
          { \@@_script_style_size:n {ScriptPercentScaleDown} }
          { \@@_script_style_size:n {ScriptScriptPercentScaleDown} }
      }
%</LU>
%<*XE>
    \dim_compare:nF { \fontdimen 10 \l_@@_font == 0pt }
      {
        \DeclareMathSizes { \f@size } { \f@size }
          { \@@_fontdimen_to_scale:nn {10} {\l_@@_font} }
          { \@@_fontdimen_to_scale:nn {11} {\l_@@_font} }
      }
%</XE>
  }
%    \end{macrocode}
% \end{macro}
%
% \begin{macro}{\@@_script_style_size:n}
% Determine script- and scriptscriptstyle sizes using luaotfload:
%   \begin{macrocode}
%<*LU>
\cs_new:Nn \@@_script_style_size:n
  {
    \fp_eval:n {\directlua{tex.sprint(luaotfload.aux.get_math_dimension("l_@@_font","#1"))} * \f@size / 100 }
  }
%</LU>
%   \end{macrocode}
% \end{macro}
%
% \begin{macro}{\@@_setup_legacy_fam_two:}
% \TeX\ won't load the same font twice at the same scale, so we need to magnify this one by an imperceptable amount.
%    \begin{macrocode}
\cs_new:Nn \@@_setup_legacy_fam_two:
  {
    \fontspec_set_family:Nxn \l_@@_family_tl
      {
        \l_@@_font_keyval_tl,
        Scale=1.00001,
        FontAdjustment =
          {
            \@@_copy_fontparam:nnn { 8} {43} {FractionNumeratorDisplayStyleShiftUp}\relax
            \@@_copy_fontparam:nnn { 9} {42} {FractionNumeratorShiftUp}\relax
            \@@_copy_fontparam:nnn {10} {32} {StackTopShiftUp}\relax
            \@@_copy_fontparam:nnn {11} {45} {FractionDenominatorDisplayStyleShiftDown}\relax
            \@@_copy_fontparam:nnn {12} {44} {FractionDenominatorShiftDown}\relax
            \@@_copy_fontparam:nnn {13} {21} {SuperscriptShiftUp}\relax
            \@@_copy_fontparam:nnn {14} {21} {SuperscriptShiftUp}\relax
            \@@_copy_fontparam:nnn {15} {22} {SuperscriptShiftUpCramped}\relax
            \@@_copy_fontparam:nnn {16} {18} {SubscriptShiftDown}\relax
            \@@_copy_fontparam:nnn {17} {18} {SubscriptShiftDownWithSuperscript}\relax
            \@@_copy_fontparam:nnn {18} {24} {SuperscriptBaselineDropMax}\relax
            \@@_copy_fontparam:nnn {19} {20} {SubscriptBaselineDropMin}\relax
            \@@_zero_fontparam:n   {20} % delim1 = FractionDelimiterDisplaySize
            \@@_zero_fontparam:n   {21} % delim2 = FractionDelimiterSize
            \@@_copy_fontparam:nnn {22} {15} {AxisHeight}\relax
         }
      } {\l_@@_fontname_tl}

    \SetSymbolFont{symbols}{\l_@@_mversion_tl}
      {\encodingdefault}{\l_@@_family_tl}{\mddefault}{\updefault}

    \str_if_eq_x:nnT {\l_@@_mversion_tl} {normal}
      {
        \SetSymbolFont{symbols}{bold}
          {\encodingdefault}{\l_@@_family_tl}{\bfdefault}{\updefault}
      }
  }
%    \end{macrocode}
% \end{macro}
%
% \begin{macro}{\@@_setup_legacy_fam_three:}
% Similarly, this font is shrunk by an imperceptable amount for \TeX\ to load it again.
%    \begin{macrocode}
\cs_new:Nn \@@_setup_legacy_fam_three:
  {
    \fontspec_set_family:Nxn \l_@@_family_tl
      {
        \l_@@_font_keyval_tl,
        Scale=0.99999,
        FontAdjustment = {
          \@@_copy_fontparam:nnn { 8} {48} {FractionRuleThickness}\relax
          \@@_copy_fontparam:nnn { 9} {28} {UpperLimitGapMin}\relax
          \@@_copy_fontparam:nnn {10} {30} {LowerLimitGapMin}\relax
          \@@_copy_fontparam:nnn {11} {29} {UpperLimitBaselineRiseMin}\relax
          \@@_copy_fontparam:nnn {12} {31} {LowerLimitBaselineDropMin}\relax
          \@@_zero_fontparam:n   {13}
       }
      } {\l_@@_fontname_tl}

    \SetSymbolFont{largesymbols}{\l_@@_mversion_tl}
      {\encodingdefault}{\l_@@_family_tl}{\mddefault}{\updefault}

    \str_if_eq_x:nnT {\l_@@_mversion_tl} {normal}
      {
        \SetSymbolFont{largesymbols}{bold}
          {\encodingdefault}{\l_@@_family_tl}{\bfdefault}{\updefault}
      }
  }
%    \end{macrocode}
% \end{macro}
%
%
% \begin{macro}{\@@_fontspec_select_font:}
% Select the font with \cs{fontspec} and define \cs{l_@@_font} from it.
%    \begin{macrocode}
\cs_new:Nn \@@_fontspec_select_font:
  {
    \tl_set:Nx \l_@@_font_keyval_tl {
%<LU>     Renderer = Basic,
      BoldItalicFont = {}, ItalicFont = {},
      Script = Math,
      SizeFeatures =
        {
          {
            Size = \tf@size-
          } ,
          {
            Size = \sf@size-\tf@size ,
            Font = \l_@@_script_font_tl ,
            \l_@@_script_features_tl
          } ,
          {
            Size = -\sf@size ,
            Font = \l_@@_sscript_font_tl ,
            \l_@@_sscript_features_tl
          }
        } ,
      \l_@@_unknown_keys_clist
    }

  \fontspec_set_fontface:NNxn \l_@@_font \l_@@_family_tl
    {\l_@@_font_keyval_tl} {\l_@@_fontname_tl}
%    \end{macrocode}
% Check whether we're using a real maths font:
%    \begin{macrocode}
    \group_begin:
      \fontfamily{\l_@@_family_tl}\selectfont
      \fontspec_if_script:nF {math} {\bool_gset_false:N \l_@@_ot_math_bool}
    \group_end:
  }
%    \end{macrocode}
% \end{macro}
%
% \subsection{Functions for setting up symbols with mathcodes}
% \seclabel{mathsymbol}
%
% \begin{macro}{\@@_process_symbol_noparse:nnn}
% \begin{macro}{\@@_process_symbol_parse:nnn}
% If the \feat{range} font feature has been used, then only
% a subset of the Unicode glyphs are to be defined.
% See \secref{rangeproc} for the code that enables this.
%    \begin{macrocode}
\cs_set:Nn \@@_process_symbol_noparse:nnn
  {
    \@@_set_mathsymbol:nNNn {\@@_symfont_tl} #2 #3 {#1}
  }
%    \end{macrocode}
%    \begin{macrocode}
\cs_set:Nn \@@_process_symbol_parse:nnn
  {
    \@@_if_char_spec:nNNT {#1} {#2} {#3}
      {
        \@@_process_symbol_noparse:nnn {#1} {#2} {#3}
      }
  }
%    \end{macrocode}
% \end{macro}
% \end{macro}
%
% \begin{macro}{\@@_remap_symbols:}
% \begin{macro}{\@@_remap_symbol_noparse:nnn}
% \begin{macro}{\@@_remap_symbol_parse:nnn}
% This function is used to define the mathcodes for those chars which should
% be mapped to a different glyph than themselves.
%    \begin{macrocode}
\cs_new:Npn \@@_remap_symbols:
  {
    \@@_remap_symbol:nnn{`\-}{\mathbin}{"02212}% hyphen to minus
    \@@_remap_symbol:nnn{`\*}{\mathbin}{"02217}% text asterisk to "centred asterisk"
    \bool_if:NF \g_@@_literal_colon_bool
      {
        \@@_remap_symbol:nnn{`\:}{\mathrel}{"02236}% colon to ratio (i.e., punct to rel)
      }
  }
%    \end{macrocode}
% \end{macro}
% Where |\@@_remap_symbol:nnn| is defined to be one of these two, depending
% on the range setup:
%    \begin{macrocode}
\cs_new:Nn \@@_remap_symbol_parse:nnn
  {
    \@@_if_char_spec:nNNT {#3} {\@nil} {#2}
      { \@@_remap_symbol_noparse:nnn {#1} {#2} {#3} }
  }
\cs_new:Nn \@@_remap_symbol_noparse:nnn
  {
    \clist_map_inline:nn {#1}
      { \@@_set_mathcode:nnnn {##1} {#2} {\@@_symfont_tl} {#3} }
  }
%    \end{macrocode}
% \end{macro}
% \end{macro}
%
% \subsection{Active math characters}
%
% There are more math active chars later in the subscript/superscript section.
% But they don't need to be able to be typeset directly.
%
% \begin{macro}{\@@_setup_mathactives:}
%    \begin{macrocode}
\cs_new:Npn \@@_setup_mathactives:
  {
    \@@_make_mathactive:nNN {"2032} \@@_prime_single_mchar \mathord
    \@@_make_mathactive:nNN {"2033} \@@_prime_double_mchar \mathord
    \@@_make_mathactive:nNN {"2034} \@@_prime_triple_mchar \mathord
    \@@_make_mathactive:nNN {"2057} \@@_prime_quad_mchar   \mathord
    \@@_make_mathactive:nNN {"2035} \@@_backprime_single_mchar \mathord
    \@@_make_mathactive:nNN {"2036} \@@_backprime_double_mchar \mathord
    \@@_make_mathactive:nNN {"2037} \@@_backprime_triple_mchar \mathord
    \@@_make_mathactive:nNN {`\'} \mathstraightquote \mathord
    \@@_make_mathactive:nNN {`\`} \mathbacktick      \mathord
  }
%    \end{macrocode}
% \end{macro}
%
% \begin{macro}{\@@_make_mathactive:nNN}
% Makes |#1| a mathactive char, and gives cs |#2| the meaning of mathchar |#1|
% with class |#3|.
% You are responsible for giving active |#1| a particular meaning!
%    \begin{macrocode}
\cs_new:Nn \@@_make_mathactive_parse:nNN
  {
    \@@_if_char_spec:nNNT {#1} #2 #3
      { \@@_make_mathactive_noparse:nNN {#1} #2 #3 }
  }
%    \end{macrocode}
%
%    \begin{macrocode}
\cs_new:Nn \@@_make_mathactive_noparse:nNN
  {
    \@@_set_mathchar:NNnn #2 #3 {\@@_symfont_tl} {#1}
    \@@_char_gmake_mathactive:n {#1}
  }
%    \end{macrocode}
% \end{macro}
%
% \subsection{Delimiter codes}
%
% \begin{macro}{\@@_assign_delcode:nn}
%    \begin{macrocode}
\cs_new:Nn \@@_assign_delcode_noparse:nn
  {
    \@@_set_delcode:nnn \@@_symfont_tl {#1} {#2}
  }
%    \end{macrocode}
%
%    \begin{macrocode}
\cs_new:Nn \@@_assign_delcode_parse:nn
  {
    \@@_if_char_spec:nNNT {#2} {\@nil} {\@nil}
      {
        \@@_assign_delcode_noparse:nn {#1} {#2}
      }
  }
%    \end{macrocode}
% \end{macro}
%
% \begin{macro}{\@@_assign_delcode:n}
% Shorthand.
%    \begin{macrocode}
\cs_new:Nn \@@_assign_delcode:n { \@@_assign_delcode:nn {#1} {#1} }
%    \end{macrocode}
% \end{macro}
%
% \begin{macro}{\@@_setup_delcodes:}
% Some symbols that aren't mathopen/mathclose still need to have delimiter codes assigned.
% The list of vertical arrows may be incomplete.
% On the other hand, many fonts won't support them all being stretchy.
% And some of them are probably not meant to stretch, either. But adding them here doesn't hurt.
%    \begin{macrocode}
\cs_new:Npn \@@_setup_delcodes:
  {
    % ensure \left. and \right. work:
    \@@_set_delcode:nnn \@@_symfont_tl {`\.} {\c_zero}
    % this is forcefully done to fix a bug -- indicates a larger problem!

    \@@_assign_delcode:nn {`\/}   {\g_@@_slash_delimiter_usv}
    \@@_assign_delcode:nn {"2044} {\g_@@_slash_delimiter_usv} % fracslash
    \@@_assign_delcode:nn {"2215} {\g_@@_slash_delimiter_usv} % divslash
    \@@_assign_delcode:n {"005C} % backslash
    \@@_assign_delcode:nn {`\<} {"27E8} % angle brackets with ascii notation
    \@@_assign_delcode:nn {`\>} {"27E9} % angle brackets with ascii notation
    \@@_assign_delcode:n {"2191} % up arrow
    \@@_assign_delcode:n {"2193} % down arrow
    \@@_assign_delcode:n {"2195} % updown arrow
    \@@_assign_delcode:n {"219F} % up arrow twohead
    \@@_assign_delcode:n {"21A1} % down arrow twohead
    \@@_assign_delcode:n {"21A5} % up arrow from bar
    \@@_assign_delcode:n {"21A7} % down arrow from bar
    \@@_assign_delcode:n {"21A8} % updown arrow from bar
    \@@_assign_delcode:n {"21BE} % up harpoon right
    \@@_assign_delcode:n {"21BF} % up harpoon left
    \@@_assign_delcode:n {"21C2} % down harpoon right
    \@@_assign_delcode:n {"21C3} % down harpoon left
    \@@_assign_delcode:n {"21C5} % arrows up down
    \@@_assign_delcode:n {"21F5} % arrows down up
    \@@_assign_delcode:n {"21C8} % arrows up up
    \@@_assign_delcode:n {"21CA} % arrows down down
    \@@_assign_delcode:n {"21D1} % double up arrow
    \@@_assign_delcode:n {"21D3} % double down arrow
    \@@_assign_delcode:n {"21D5} % double updown arrow
    \@@_assign_delcode:n {"21DE} % up arrow double stroke
    \@@_assign_delcode:n {"21DF} % down arrow double stroke
    \@@_assign_delcode:n {"21E1} % up arrow dashed
    \@@_assign_delcode:n {"21E3} % down arrow dashed
    \@@_assign_delcode:n {"21E7} % up white arrow
    \@@_assign_delcode:n {"21E9} % down white arrow
    \@@_assign_delcode:n {"21EA} % up white arrow from bar
    \@@_assign_delcode:n {"21F3} % updown white arrow
  }
%    \end{macrocode}
% \end{macro}
%
% \subsection{(Big) operators}
%
% The engine does what is necessary to deal with big operators for us
% automatically with \cmd\Umathchardef.
% However, the limits aren't set automatically; that is, we want to define,
% a la Plain \TeX\ \etc, |\def\int{\intop\nolimits}|, so there needs to be a
% transformation from \cmd\int\ to \cmd\intop\ during the expansion of
% \cmd\_@@_sym:nnn\ in the appropriate contexts.
%
% \begin{macro}{\l_@@_nolimits_tl}
% This macro is a sequence containing those maths operators that require a
% \cmd\nolimits\ suffix.
% This list is used when processing |unicode-math-table.tex| to define such
% commands automatically (see the macro \cs{@@_set_mathsymbol:nNNn}).
% I've chosen essentially just the operators that look like integrals;
% hopefully a better mathematician can help me out here.
% I've a feeling that it's more useful \emph{not} to include the multiple
% integrals such as $\iiiint$, but that might be a matter of preference.
%    \begin{macrocode}
\tl_set:Nn \l_@@_nolimits_tl
  {
    \int\iint\iiint\iiiint\oint\oiint\oiiint
    \intclockwise\varointclockwise\ointctrclockwise\sumint
    \intbar\intBar\fint\cirfnint\awint\rppolint
    \scpolint\npolint\pointint\sqint\intlarhk\intx
    \intcap\intcup\upint\lowint
  }
%    \end{macrocode}
% \end{macro}
%
% \subsection{Radicals}
%
% \begin{macro}{\l_@@_radicals_tl}
% The radicals are organised in \cs{@@_set_mathsymbol:nNNn}.
% We organise radicals in the same way as nolimits-operators.
% (\cs{cuberoot} and \cs{fourthroot}, don't seem to behave as proper radicals.)
%    \begin{macrocode}
\tl_set:Nn \l_@@_radicals_tl {\sqrt \longdivision}
%    \end{macrocode}
% \end{macro}
%
%    \begin{macrocode}
%</package>
%    \end{macrocode}

\endinput

% /©
%
% This package is free software and may be redistributed and/or modified under
% the conditions of the LaTeX Project Public License, version 1.3c or higher
% (your choice): <http://www.latex-project.org/lppl/>.
%
% Copyright 2006-2017  Will Robertson, LPPL "maintainer"
% Copyright 2010-2017  Philipp Stephani
% Copyright 2011-2017  Joseph Wright
% Copyright 2012-2015  Khaled Hosny
%
% ©/
