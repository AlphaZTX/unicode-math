
% \section{Unicode sub- and super-scripts}
%
%    \begin{macrocode}
%<*package>
%    \end{macrocode}
%
% The idea here is to enter a scanning state after a superscript or subscript
% is encountered.
% If subsequent superscripts or subscripts (resp.) are found,
% they are lumped together.
% Each sub/super has a corresponding regular size
% glyph which is used by \XeTeX\ to typeset the results; this means that the
% actual subscript/superscript glyphs are never seen in the output
% document~--- they are only used as input characters.
%
% Open question: should the superscript-like `modifiers' (\unichar{1D2C}
% {modifier capital letter a} and on) be included here?
%    \begin{macrocode}
\group_begin:
%    \end{macrocode}
% \paragraph{Superscripts}
% Populate a property list with superscript characters; themselves as their
% key, and their replacement as each key's value.
% Then make the superscript active and bind it to the scanning function.
%
% \cs{scantokens} makes this process much simpler since we can activate the
% char and assign its meaning in one step.
%    \begin{macrocode}
\cs_new:Nn \@@_setup_active_superscript:nn
 {
  \prop_gput:Nnn \g_@@_supers_prop   {#1} {#2}
  \char_set_catcode_active:N #1
  \@@_char_gmake_mathactive:N #1
  \scantokens
   {
    \cs_gset:Npn #1
     {
      \tl_set:Nn \l_@@_ss_chain_tl {#2}
      \cs_set_eq:NN \@@_sub_or_super:n \sp
      \tl_set:Nn \l_@@_tmpa_tl {supers}
      \@@_scan_sscript:
     }
   }
 }
%    \end{macrocode}
% Bam:
%    \begin{macrocode}
\@@_setup_active_superscript:nn {^^^^2070} {0}
\@@_setup_active_superscript:nn {^^^^00b9} {1}
\@@_setup_active_superscript:nn {^^^^00b2} {2}
\@@_setup_active_superscript:nn {^^^^00b3} {3}
\@@_setup_active_superscript:nn {^^^^2074} {4}
\@@_setup_active_superscript:nn {^^^^2075} {5}
\@@_setup_active_superscript:nn {^^^^2076} {6}
\@@_setup_active_superscript:nn {^^^^2077} {7}
\@@_setup_active_superscript:nn {^^^^2078} {8}
\@@_setup_active_superscript:nn {^^^^2079} {9}
\@@_setup_active_superscript:nn {^^^^207a} {+}
\@@_setup_active_superscript:nn {^^^^207b} {-}
\@@_setup_active_superscript:nn {^^^^207c} {=}
\@@_setup_active_superscript:nn {^^^^207d} {(}
\@@_setup_active_superscript:nn {^^^^207e} {)}
\@@_setup_active_superscript:nn {^^^^2071} {i}
\@@_setup_active_superscript:nn {^^^^207f} {n}
\@@_setup_active_superscript:nn {^^^^02b0} {h}
\@@_setup_active_superscript:nn {^^^^02b2} {j}
\@@_setup_active_superscript:nn {^^^^02b3} {r}
\@@_setup_active_superscript:nn {^^^^02b7} {w}
\@@_setup_active_superscript:nn {^^^^02b8} {y}
%    \end{macrocode}
% \paragraph{Subscripts} Ditto above.
%    \begin{macrocode}
\cs_new:Nn \@@_setup_active_subscript:nn
 {
  \prop_gput:Nnn \g_@@_subs_prop   {#1} {#2}
  \char_set_catcode_active:N #1
  \@@_char_gmake_mathactive:N #1
  \scantokens
   {
    \cs_gset:Npn #1
     {
      \tl_set:Nn \l_@@_ss_chain_tl {#2}
      \cs_set_eq:NN \@@_sub_or_super:n \sb
      \tl_set:Nn \l_@@_tmpa_tl {subs}
      \@@_scan_sscript:
     }
   }
 }
%    \end{macrocode}
% A few more subscripts than superscripts:
%    \begin{macrocode}
\@@_setup_active_subscript:nn {^^^^2080} {0}
\@@_setup_active_subscript:nn {^^^^2081} {1}
\@@_setup_active_subscript:nn {^^^^2082} {2}
\@@_setup_active_subscript:nn {^^^^2083} {3}
\@@_setup_active_subscript:nn {^^^^2084} {4}
\@@_setup_active_subscript:nn {^^^^2085} {5}
\@@_setup_active_subscript:nn {^^^^2086} {6}
\@@_setup_active_subscript:nn {^^^^2087} {7}
\@@_setup_active_subscript:nn {^^^^2088} {8}
\@@_setup_active_subscript:nn {^^^^2089} {9}
\@@_setup_active_subscript:nn {^^^^208a} {+}
\@@_setup_active_subscript:nn {^^^^208b} {-}
\@@_setup_active_subscript:nn {^^^^208c} {=}
\@@_setup_active_subscript:nn {^^^^208d} {(}
\@@_setup_active_subscript:nn {^^^^208e} {)}
\@@_setup_active_subscript:nn {^^^^2090} {a}
\@@_setup_active_subscript:nn {^^^^2091} {e}
\@@_setup_active_subscript:nn {^^^^2095} {h}
\@@_setup_active_subscript:nn {^^^^1d62} {i}
\@@_setup_active_subscript:nn {^^^^2c7c} {j}
\@@_setup_active_subscript:nn {^^^^2096} {k}
\@@_setup_active_subscript:nn {^^^^2097} {l}
\@@_setup_active_subscript:nn {^^^^2098} {m}
\@@_setup_active_subscript:nn {^^^^2099} {n}
\@@_setup_active_subscript:nn {^^^^2092} {o}
\@@_setup_active_subscript:nn {^^^^209a} {p}
\@@_setup_active_subscript:nn {^^^^1d63} {r}
\@@_setup_active_subscript:nn {^^^^209b} {s}
\@@_setup_active_subscript:nn {^^^^209c} {t}
\@@_setup_active_subscript:nn {^^^^1d64} {u}
\@@_setup_active_subscript:nn {^^^^1d65} {v}
\@@_setup_active_subscript:nn {^^^^2093} {x}
\@@_setup_active_subscript:nn {^^^^1d66} {\beta}
\@@_setup_active_subscript:nn {^^^^1d67} {\gamma}
\@@_setup_active_subscript:nn {^^^^1d68} {\rho}
\@@_setup_active_subscript:nn {^^^^1d69} {\phi}
\@@_setup_active_subscript:nn {^^^^1d6a} {\chi}
%    \end{macrocode}
%
%    \begin{macrocode}
\group_end:
%    \end{macrocode}
% The scanning command, which collects a chain of subscripts or a chain
% of superscripts and then typesets what it has collected.
%    \begin{macrocode}
\cs_new:Npn \@@_scan_sscript:
 {
  \@@_scan_sscript:TF
   {
    \@@_scan_sscript:
   }
   {
    \@@_sub_or_super:n {\l_@@_ss_chain_tl}
   }
 }
%    \end{macrocode}
% We do not skip spaces when scanning ahead, and we explicitly wish to
% bail out on encountering a space or a brace.  These cases are filtered
% using \cs{peek_N_type:TF}.  Otherwise the token can be taken as an
% \texttt{N}-type argument.  Then we search for it in the appropriate
% property list (\cs{l_@@_tmpa_tl} is |subs| or |supers|).
% If found, add the value to the current chain of sub/superscripts.
% Remember to put the character back in the input otherwise.
% The \cs{group_align_safe_begin:} and \cs{group_align_safe_end:} are
% needed in case |#3| is |&|.
%    \begin{macrocode}
\cs_new:Npn \@@_scan_sscript:TF #1#2
 {
  \peek_N_type:TF
   {
    \group_align_safe_begin:
    \@@_scan_sscript_aux:nnN {#1} {#2}
   }
   {#2}
 }
\cs_new:Npn \@@_scan_sscript_aux:nnN #1#2#3
 {
  \prop_get:cnNTF {g_@@_\l_@@_tmpa_tl _prop} {#3} \l_@@_tmpb_tl
   {
    \tl_put_right:NV \l_@@_ss_chain_tl \l_@@_tmpb_tl
    \group_align_safe_end:
    #1
   }
   { \group_align_safe_end: #2 #3 }
 }
%    \end{macrocode}
%
%
%    \begin{macrocode}
%</package>
%    \end{macrocode}
