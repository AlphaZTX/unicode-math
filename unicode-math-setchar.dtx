
% \section{\DTXCURR --- Setting up maths chars}
%
%    \begin{macrocode}
%<*package&(XE|LU)>
%    \end{macrocode}
%
% \subsection{A token list to contain the data of the math table}
%
% Instead of \cmd\input-ing the unicode math table every time we
% want to re-read its data, we save it within a macro. This has two
% advantages: 1.~it should be slightly faster, at the expense of memory;
% 2.~we don't need to worry about catcodes later, since they're frozen
% at this point.
%
% In time, the case statement inside |set_mathsymbol| will be moved in here
% to avoid re-running it every time.
%    \begin{macrocode}
\cs_new:Npn \@@_symbol_setup:
 {
  \cs_set:Npn \UnicodeMathSymbol ##1##2##3##4
   {
    \exp_not:n { \_@@_sym:nnn {##1} {##2} {##3} }
   }
 }
%    \end{macrocode}
%
%    \begin{macrocode}
\tl_set_from_file_x:Nnn \g_@@_mathtable_tl {\@@_symbol_setup:} {unicode-math-table.tex}
%    \end{macrocode}
%
%
% \begin{macro}{\@@_input_math_symbol_table:}
% This function simply expands to the token list containing all the data.
%    \begin{macrocode}
\cs_new:Nn \@@_input_math_symbol_table: {\g_@@_mathtable_tl}
%    \end{macrocode}
% \end{macro}
%
%
% \subsection{Definitions of the active math characters}
%
% Now give \cmd\_@@_sym:nnn\ a definition in terms of \cmd\@@_cs_set_eq_active_char:Nw\
% and we're good to go.
%
% Ensure catcodes are appropriate;
% make sure |#| is an `other' so that we don't get confused with \cs{mathoctothorpe}.
%    \begin{macrocode}
\AtBeginDocument{\@@_define_math_chars:}
\cs_new:Nn \@@_define_math_chars:
 {
  \group_begin:
    \cs_set:Npn \_@@_sym:nnn ##1##2##3
     {
      \tl_if_in:nnT
       { \mathord \mathalpha \mathbin \mathrel \mathpunct \mathop \mathfence }
       {##3}
      {
        \exp_last_unbraced:NNx \cs_gset_eq:NN ##2 { \Ucharcat ##1 ~ 12 ~ }
      }
     }
    \@@_input_math_symbol_table:
  \group_end:
 }
%    \end{macrocode}
%
%
%
% \subsection{Commands for each symbol/glyph/char}
%
% \begin{macro}{\@@_set_mathsymbol:nNNn}
% \darg{A \LaTeX\ symbol font, e.g., \texttt{operators}}
% \darg{Symbol macro, \eg, \cmd\alpha}
% \darg{Type, \eg, \cmd\mathalpha}
% \darg{Slot, \eg, \texttt{"221E}}
% There are a bunch of tests to perform to process the various characters.
% The following assignments should all be fairly straightforward.
%
% The catcode setting is to work around (strange?) behaviour in LuaTeX in which catcode 11 characters don't have italic correction for maths.
% We don't adjust ascii chars, however, because certain punctuation should not have their catcodes changed.
%    \begin{macrocode}
\cs_set:Nn \@@_set_mathsymbol:nNNn
 {
  \bool_lazy_and:nnT
   {
    \int_compare_p:nNn {#4} > {127}
   }
   {
    \int_compare_p:nNn { \char_value_catcode:n {#4} } = {11}
   }
   { \char_set_catcode_other:n {#4} }

  \tl_case:Nn #3
   {
    \mathord   { \@@_set_mathcode:nnn {#4} {#3} {#1} }
    \mathalpha { \@@_set_mathcode:nnn {#4} {#3} {#1} }
    \mathbin   { \@@_set_mathcode:nnn {#4} {#3} {#1} }
    \mathrel   { \@@_set_mathcode:nnn {#4} {#3} {#1} }
    \mathpunct { \@@_set_mathcode:nnn {#4} {#3} {#1} }
    \mathop    { \@@_set_big_operator:nnn {#1} {#2} {#4} }
    \mathopen  { \@@_set_math_open:nnn    {#1} {#2} {#4} }
    \mathclose { \@@_set_math_close:nnn   {#1} {#2} {#4} }
    \mathfence { \@@_set_math_fence:nnnn  {#1} {#2} {#3} {#4} }
    \mathaccent
     { \@@_set_math_accent:Nnnn #2 {fixed} {#1} {#4} }
    \mathbotaccent
     { \@@_set_math_accent:Nnnn #2 {bottom~ fixed} {#1} {#4} }
    \mathaccentwide
     { \@@_set_math_accent:Nnnn #2 {} {#1} {#4} }
    \mathbotaccentwide
     { \@@_set_math_accent:Nnnn #2 {bottom} {#1} {#4} }
    \mathover
     { \@@_set_math_overunder:Nnnn #2 {} {#1} {#4} }
    \mathunder
     { \@@_set_math_overunder:Nnnn #2 {bottom} {#1} {#4} }
   }
 }
%    \end{macrocode}
% \end{macro}
%
%    \begin{macrocode}
\edef\mathfence{\string\mathfence}
\edef\mathover{\string\mathover}
\edef\mathunder{\string\mathunder}
\edef\mathbotaccent{\string\mathbotaccent}
\edef\mathaccentwide{\string\mathaccentwide}
\edef\mathbotaccentwide{\string\mathbotaccentwide}
%    \end{macrocode}
%
%
% \begin{macro}{\@@_set_big_operator:nnn}
% \darg{Symbol font name}
% \darg{Macro to assign}
% \darg{Glyph slot}
% In the examples following, say we're defining for the symbol \cmd\sum\ ($\sum$).
% In order for literal Unicode characters to be used in the source and still
% have the correct limits behaviour, big operators are made math-active.
% This involves three steps:
% \begin{itemize}
% \item
% The active math char is defined to expand to the macro \cs{sum_sym}.
% (Later, the control sequence \cs{sum} will be assigned the math char.)
% \item
% Declare the plain old mathchardef for the control sequence \cmd\sumop.
% (This follows the convention of \LaTeX/\pkg{amsmath}.)
% \item
% Define \cs{sum_sym} as \cmd\sumop, followed by \cmd\nolimits\ if necessary.
% \end{itemize}
% Whether the \cmd\nolimits\ suffix is inserted is controlled by the
% token list \cs{l_@@_nolimits_tl}, which contains a list of such characters.
% This list is checked dynamically to allow it to be updated mid-document.
%
% Examples of expansion, by default, for two big operators:
% \begin{quote}
% (~\cs{sum} $\to$~) $\sum$ $\to$ \cs{sum_sym} $\to$ \cs{sumop}\cs{nolimits}\par
% (~\cs{int} $\to$~) $\int$ $\to$ \cs{int_sym} $\to$ \cs{intop}
% \end{quote}
%    \begin{macrocode}
\cs_new:Nn \@@_set_big_operator:nnn
 {
  \@@_char_gmake_mathactive:n {#3}
  \cs_set_protected_nopar:Npx \@@_tmpa: { \exp_not:c { \cs_to_str:N #2 _sym } }
  \char_gset_active_eq:nN {#3} \@@_tmpa:

  \@@_set_mathchar:cNnn {\cs_to_str:N #2 op} \mathop {#1} {#3}

  \cs_gset:cpx { \cs_to_str:N #2 _sym }
   {
    \exp_not:c { \cs_to_str:N #2 op   }
    \exp_not:n { \tl_if_in:NnT \l_@@_nolimits_tl {#2} \nolimits }
   }
 }
%    \end{macrocode}
% \end{macro}
%
% \begin{macro}{\@@_set_math_open:nnn}
% \darg{Symbol font name}
% \darg{Macro to assign}
% \darg{Glyph slot}
%    \begin{macrocode}
\cs_new:Nn \@@_set_math_open:nnn
 {
  \tl_if_in:NnTF \l_@@_radicals_tl {#2}
   {
     \cs_gset_protected_nopar:cpx {\cs_to_str:N #2 sign}
       { \@@_radical:nn {#1} {#3} }
     \tl_set:cn {l_@@_radical_\cs_to_str:N #2_tl} {\use:c{sym #1}~ #3}
   }
   {
     \@@_set_delcode:nnn {#1} {#3} {#3}
     \@@_set_mathcode:nnn {#3} \mathopen {#1}
     \cs_gset_protected_nopar:Npx #2
       { \@@_delimiter:Nnn \mathopen {#1} {#3} }
   }
 }
%    \end{macrocode}
% \end{macro}
%
% \begin{macro}{\@@_set_math_close:nnn}
% \darg{Symbol font name}
% \darg{Macro to assign}
% \darg{Glyph slot}
%    \begin{macrocode}
\cs_new:Nn \@@_set_math_close:nnn
 {
  \@@_set_delcode:nnn {#1} {#3} {#3}
  \@@_set_mathcode:nnn {#3} \mathclose {#1}
  \cs_gset_protected_nopar:Npx #2
    { \@@_delimiter:Nnn \mathclose {#1} {#3} }
 }
%    \end{macrocode}
% \end{macro}
%
% \begin{macro}{\@@_set_math_fence:nnnn}
% \darg{Symbol font name}
% \darg{Macro to assign}
% \darg{Type, \eg, \cmd\mathalpha}
% \darg{Glyph slot}
%    \begin{macrocode}
\cs_new:Nn \@@_set_math_fence:nnnn
 {
  \@@_set_mathcode:nnn {#4} {#3} {#1}
  \@@_set_delcode:nnn  {#1} {#4} {#4}
  \cs_gset_protected_nopar:cpx {l \cs_to_str:N #2}
    { \@@_delimiter:Nnn \mathopen  {#1} {#4} }
  \cs_gset_protected_nopar:cpx {r \cs_to_str:N #2}
    { \@@_delimiter:Nnn \mathclose {#1} {#4} }
 }
%    \end{macrocode}
% \end{macro}
%
% \begin{macro}{\@@_set_math_accent:Nnnn}
% \darg{Accend command}
% \darg{Accent type (string)}
% \darg{Symbol font name}
% \darg{Glyph slot}
%    \begin{macrocode}
\cs_new:Nn \@@_set_math_accent:Nnnn
 {
  \cs_gset_protected_nopar:Npx #1
   { \@@_accent:nnn {#2} {#3} {#4} }
 }
%    \end{macrocode}
% \end{macro}
%
% \begin{macro}{\@@_set_math_overunder:Nnnn}
% \darg{Accend command}
% \darg{Accent type (string)}
% \darg{Symbol font name}
% \darg{Glyph slot}
%    \begin{macrocode}
\cs_new:Nn \@@_set_math_overunder:Nnnn
 {
  \cs_gset_protected_nopar:Npx #1 ##1
   {
    \mathop
     { \@@_accent:nnn {#2} {#3} {#4} {##1} }
    \limits
   }
 }
%    \end{macrocode}
% \end{macro}
%
%
%    \begin{macrocode}
%</package&(XE|LU)>
%    \end{macrocode}
%
\endinput
