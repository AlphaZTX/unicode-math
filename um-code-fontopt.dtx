%%^^A%% um-code-fontopt.dtx -- part of UNICODE-MATH <wspr.io/unicode-math>
%%^^A%% Keyval for `\setmathfont`

% \section{Font loading options}
%
%    \begin{macrocode}
%<*package>
%    \end{macrocode}
%
% \subsection{Math version}
%    \begin{macrocode}
\keys_define:nn {unicode-math}
  {
    version .code:n =
      {
        \tl_set:Nn \l_@@_mversion_tl {#1}
        \DeclareMathVersion {\l_@@_mversion_tl}
      }
  }
%    \end{macrocode}
%
% \subsection{Script and scriptscript font options}
%    \begin{macrocode}
\keys_define:nn {unicode-math}
  {
    script-features  .tl_set:N =  \l_@@_script_features_tl ,
    sscript-features .tl_set:N = \l_@@_sscript_features_tl ,
         script-font .tl_set:N =      \l_@@_script_font_tl ,
        sscript-font .tl_set:N =     \l_@@_sscript_font_tl ,
  }
%    \end{macrocode}
%
% \subsection{Range processing}
% \seclabel{rangeproc}
%
% Locally redefined all math symbol commands to their slot number prefixed by a quark.
% Similary for the math classes.
%    \begin{macrocode}
\keys_define:nn {unicode-math}
  {
    range .code:n =
      {
        \bool_if:NF \g_@@_main_font_defined_bool { \@@_error:n {no-main-font} }
        \bool_set_false:N \l_@@_init_bool
        \@@_range_init:
        \group_begin:
          \seq_map_inline:Nn \g_@@_mathclasses_seq
            {
              \cs_set:Npn ##1 { \use_none:n \q_unicode_math \exp_not:N ##1 }
            }
          \cs_set:Npn \_@@_sym:nnn ##1 ##2 ##3
            {
              \cs_set:Npn ##2 { \use_none:n \q_unicode_math ##1 }
            }
          \@@_input_math_symbol_table:
          \@@_range_process:n {#1}
        \group_end:
      }
  }
%    \end{macrocode}
%
% \begin{macro}{\@@_range_init:}
% Set processing functions if we're not defining the full Unicode math repetoire.
% Math symbols are defined with \cmd\_@@_sym:nnn; see \secref{mathsymbol}
% for the individual definitions
%    \begin{macrocode}
\cs_new:Nn \@@_range_init:
  {
    \int_incr:N \g_@@_fam_int
    \tl_set:Nx \l_@@_symfont_label_tl {@@_fam\int_use:N\g_@@_fam_int}
    \cs_set_eq:NN \_@@_sym:nnn \@@_process_symbol_parse:nnn
    \cs_set_eq:NN \@@_remap_symbol:nnn \@@_remap_symbol_parse:nnn
    \cs_set_eq:NN \@@_maybe_init_alphabet:n \use_none:n
    \cs_set_eq:NN \@@_assign_delcode:nn \@@_assign_delcode_parse:nn
    \cs_set_eq:NN \@@_make_mathactive:nNN \@@_make_mathactive_parse:nNN
%    \end{macrocode}
% Proceed by filling up the various `range' seqs according to the user options.
%    \begin{macrocode}
    \seq_gclear:N \g_@@_char_range_seq
    \seq_gclear:N \g_@@_mclass_range_seq
    \seq_gclear:N \g_@@_mathalph_seq
  }
%    \end{macrocode}
% \end{macro}
%
% \begin{macro}{\@@_range_process:}
%    \begin{macrocode}
\cs_new:Nn \@@_range_process:n
  {
    \clist_map_inline:nn {#1}
      {
        \@@_mathalph_decl:nF {##1} { \@@_range_decl:n {##1} }
      }
}
%    \end{macrocode}
% \end{macro}
%
% \begin{macro}{\@@_mathalph_decl:nF}
% Possible forms of input:\\
% |\mathscr|\\
% |\mathscr->\mathup|\\
% |\mathscr/{Latin}|\\
% |\mathscr/{Latin}->\mathup|\\
% Outputs:\\
% |tmpa|: math style (\eg, |\mathscr|)\\
% |tmpb|: alphabets (\eg, |Latin|)\\
% |tmpc|: remap style (\eg, |\mathup|). Defaults to |tmpa|.
%
% The remap style can also be |\mathcal->stixcal|, which I marginally prefer
% in the general case.
%    \begin{macrocode}
\cs_new:Nn \@@_mathalph_decl:nF
  {

    \tl_set:Nn  \l_@@_tmpa_tl {#1}
    \tl_clear:N \l_@@_tmpb_tl
    \tl_clear:N \l_@@_tmpc_tl

    \tl_if_in:NnT \l_@@_tmpa_tl {->}
      { \exp_after:wN \@@_split_arrow:w \l_@@_tmpa_tl \q_nil }

    \tl_if_in:NnT \l_@@_tmpa_tl {/}
      { \exp_after:wN \@@_split_slash:w \l_@@_tmpa_tl \q_nil }

    \tl_set:Nx \l_@@_tmpa_tl { \tl_to_str:N \l_@@_tmpa_tl }
    \exp_args:NNx \tl_remove_all:Nn \l_@@_tmpa_tl { \token_to_str:N \math }
    \exp_args:NNx \tl_remove_all:Nn \l_@@_tmpa_tl { \token_to_str:N \sym }
    \tl_trim_spaces:N \l_@@_tmpa_tl

    \tl_if_empty:NT \l_@@_tmpc_tl
      { \tl_set_eq:NN \l_@@_tmpc_tl \l_@@_tmpa_tl }

    \clist_if_in:NVT \g_@@_bad_alpha_clist \l_@@_tmpa_tl { \@@_error:n {range-not-bf-sf} }

    \prop_if_exist:cTF {g_@@_named_range_ \l_@@_tmpa_tl _prop}
      {
        \seq_gput_right:Nx \g_@@_mathalph_seq
          {
            { \exp_not:V \l_@@_tmpa_tl }
            { \exp_not:V \l_@@_tmpb_tl }
            { \exp_not:V \l_@@_tmpc_tl }
          }
      }
      {#2}
  }
%    \end{macrocode}
%
%    \begin{macrocode}
\cs_set:Npn \@@_split_arrow:w #1->#2 \q_nil
  {
    \tl_set:Nx \l_@@_tmpa_tl { \tl_trim_spaces:n {#1} }
    \tl_set:Nx \l_@@_tmpc_tl { \tl_trim_spaces:n {#2} }
  }
%    \end{macrocode}
%    \begin{macrocode}
\cs_set:Npn \@@_split_slash:w #1/#2 \q_nil
  {
    \tl_set:Nx \l_@@_tmpa_tl { \tl_trim_spaces:n {#1} }
    \tl_set:Nx \l_@@_tmpb_tl { \tl_trim_spaces:n {#2} }
  }
%    \end{macrocode}
% \end{macro}
%
% \begin{macro}{\@@_range_decl:n}
%    \begin{macrocode}
\cs_new_protected:Nn \@@_range_decl:n
  {
    \bool_lazy_and:nnTF { \tl_if_single_p:n {#1} } { \token_if_cs_p:N #1 }
      % IF A CSNAME:
      {
        \tl_if_in:VnTF #1 { \q_unicode_math }
          {
            \seq_if_in:NnTF \g__um_mathclasses_seq {#1}
              { \seq_gput_right:Nn \g_@@_mclass_range_seq {#1} }
              { \seq_gput_right:Nx \g_@@_char_range_seq   { #1 } }
          }
          { \@@_error:nx {bad-cs-in-range} { \tl_to_str:n {#1} } }
      }
      % ELSE ASSUME NUMERIC INPUT:
      {
        \seq_gput_right:Nx \g_@@_char_range_seq { #1 }
      }
  }
%    \end{macrocode}
% \end{macro}
%
%
% \begin{macro}{\@@_if_char_spec:nNT}
% \darg{Unicode character slot}
% \darg{control sequence (math class)}
% \darg{code to execute}
% This macro expands to |#3|
% if any of its arguments are contained in \cmd\g_@@_char_range_seq.
% This list can contain either character ranges (for checking with |#1|) or control sequences.
% These latter can either be the command name of a specific character, \emph{or} the math
% type of one (\eg, \cmd\mathbin).
%
% Character ranges are passed to \cs{@@_if_char_spec:nNT}, which accepts input in the form shown in \tabref{ranges}.
%
% \begin{table}[htbp]
% \centering
% \topcaption{Ranges accepted by \cs{@@_if_char_spec:nNT}.}
% \label{tab:ranges}
% \begin{tabular}{>{\ttfamily}cc}
% \textrm{Input} & Range \\
% \hline
% x & $r=x$ \\
% x- & $r\geq x$ \\
% -y & $r\leq y$ \\
% x-y & $x \leq r \leq y$ \\
% \end{tabular}
% \end{table}
%
% We have three tests, performed sequentially in order of execution time.
% Any test finding a match jumps directly to the end.
%    \begin{macrocode}
\cs_new:Nn \@@_if_char_spec:nNT
  {
    % math class:
    \seq_if_in:NnT \g_@@_mclass_range_seq {#2}
      { \use_none_delimit_by_q_nil:w }

    % character slot:
    \seq_map_inline:Nn \g_@@_char_range_seq
      {
        \@@_int_if_slot_is_last_in_range:nnT {#1} {##1}
          { \seq_gremove_all:Nn \g_@@_char_range_seq {##1} }

        \@@_int_if_slot_in_range:nnT {#1} {##1}
          { \seq_map_break:n { \use_none_delimit_by_q_nil:w } }
      }

    % the following expands to nil if no match was found:
    \use_none:nnn
    \q_nil
    \use:n
      {
        \cs_if_eq:NNT #2 \mathalpha
          {
            \clist_put_right:Nx \l_@@_mathmap_charints_clist { \int_eval:n {#1} }
          }
        #3
      }
  }
%    \end{macrocode}
% \end{macro}
%
% \begin{macro}{\@@_int_if_slot_in_range:nnT}
% Pretty basic comma separated range processing.
% Donald Arseneau's \pkg{selectp} package has a cleverer technique.
%
% A `numrange' is like |-2,5-8,12,17-| (can be unsorted).
%
% Four cases, four argument types:
% \begin{Verbatim}
% input    #2     #3      #4
% "1  "   [ 1] - [qn] - [   ] qs
% "1- "   [ 1] - [  ] - [qn-] qs
% " -3"   [  ] - [ 3] - [qn-] qs
% "1-3"   [ 1] - [ 3] - [qn-] qs
% \end{Verbatim}
%
%    \begin{macrocode}
\cs_new:Nn \@@_int_if_slot_in_range:nnT
  {
    \@@_numrange_parse:nwT {#1} #2 - \q_nil - \q_stop {#3}
  }
%    \end{macrocode}
%
%    \begin{macrocode}
\cs_set:Npn \@@_numrange_parse:nwT #1 #2 - #3 - #4 \q_stop #5
  {
    \tl_if_empty:nTF {#4} { \int_compare:nT {#1=#2} {#5} }
      {
    \tl_if_empty:nTF {#3} { \int_compare:nT {#1>=#2} {#5} }
      {
    \tl_if_empty:nTF {#2} { \int_compare:nT {#1<=#3} {#5} }
      {
    \int_compare:nT {#1>=#2} { \int_compare:nT {#1<=#3} {#5} }
      } } }
  }
%    \end{macrocode}
% \end{macro}
%
%
%    \begin{macrocode}
\cs_new:Nn \@@_int_if_slot_is_last_in_range:nnT
  {
    \@@_numrange_last_parse:nwT {#1} #2 - \q_nil - \q_stop {#3}
  }
%    \end{macrocode}
%
%    \begin{macrocode}
\cs_set:Npn \@@_numrange_last_parse:nwT #1 #2 - #3 - #4 \q_stop #5
  {
    \tl_if_empty:nTF {#4} { \int_compare:nT {#1==#2} {#5} }
      {
    \tl_if_empty:nTF {#2} { \int_compare:nT {#1==#3} {#5} }
      {
                            \int_compare:nT {#1==#3} {#5}
      } }
  }
%    \end{macrocode}
% \end{macro}
%
%    \begin{macrocode}
%</package>
%    \end{macrocode}

\endinput

% /©
%
% ------------------------------------------------
% The UNICODE-MATH package  <wspr.io/unicode-math>
% ------------------------------------------------
% This package is free software and may be redistributed and/or modified under
% the conditions of the LaTeX Project Public License, version 1.3c or higher
% (your choice): <http://www.latex-project.org/lppl/>.
% ------------------------------------------------
% Copyright 2006-2018  Will Robertson, LPPL "maintainer"
% Copyright 2010-2017  Philipp Stephani
% Copyright 2011-2017  Joseph Wright
% Copyright 2012-2015  Khaled Hosny
% ------------------------------------------------
%
% ©/
