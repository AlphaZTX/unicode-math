
% \section{\DTXCURR --- Main package code}
%
%    \begin{macrocode}
%<*package&(XE|LU)>
%    \end{macrocode}
%
%
% \subsection{\cs{setmathfontface}}
%
% \begin{macro}{\setmathfontface}
%    \begin{macrocode}
\keys_define:nn {@@_mathface}
 {
  version .code:n =
   { \tl_set:Nn \l_@@_mversion_tl {#1} }
 }
%    \end{macrocode}
% 
%    \begin{macrocode}
\DeclareDocumentCommand \setmathfontface { m O{} m O{} }
 {
  \tl_clear:N \l_@@_mversion_tl

  \keys_set_known:nnN {@@_mathface} {#2,#4} \l_@@_keyval_clist
  \exp_args:Nnx \fontspec_set_family:Nxn \l_@@_tmpa_tl
   { ItalicFont={}, BoldFont={}, \exp_not:V \l_@@_keyval_clist } {#3}

  \tl_if_empty:NT \l_@@_mversion_tl
   {
    \tl_set:Nn \l_@@_mversion_tl {normal}
    \DeclareMathAlphabet #1 {\g_fontspec_encoding_tl} {\l_@@_tmpa_tl} {\mddefault} {\updefault}
   }
  \SetMathAlphabet #1 {\l_@@_mversion_tl} {\g_fontspec_encoding_tl} {\l_@@_tmpa_tl} {\mddefault} {\updefault}

  % integrate with fontspec's \setmathrm etc:
  \tl_case:Nn #1
   {
    \mathrm { \cs_set_eq:NN \g__fontspec_mathrm_tl \l_@@_tmpa_tl }
    \mathsf { \cs_set_eq:NN \g__fontspec_mathsf_tl \l_@@_tmpa_tl }
    \mathtt { \cs_set_eq:NN \g__fontspec_mathtt_tl \l_@@_tmpa_tl }
   }
 }
%    \end{macrocode}
% 
%    \begin{macrocode}
\@onlypreamble \setmathfontface
%    \end{macrocode}
% Note that \LaTeX's SetMathAlphabet simply doesn't work to "reset" a maths alphabet font after \verb"\begin{document}", so unlike most of the other maths commands around we still restrict this one to the preamble.
% \end{macro}
%
% \begin{macro}{\setoperatorfont}
% TODO: add check?
%    \begin{macrocode}
\DeclareDocumentCommand \setoperatorfont {m}
 { \tl_set:Nn \g_@@_operator_mathfont_tl {#1} }
\setoperatorfont{\mathrm}
%    \end{macrocode}
% \end{macro}
%
% \subsection{Hooks into \pkg{fontspec}}
%
% Historically, \cs{mathrm} and so on were completely overwritten by \pkg{unicode-math}, and \pkg{fontspec}'s methods for setting these fonts in the classical manner were bypassed.
%
% While we could now re-activate the way that \pkg{fontspec} does the following, because we can now change maths fonts whenever it's better to define new commands in \pkg{unicode-math} to define the \cs{mathXYZ} fonts.
%
% \subsubsection{Text font}
%    \begin{macrocode}
\cs_generate_variant:Nn \tl_if_eq:nnT {o}
\cs_set:Nn \__fontspec_setmainfont_hook:nn
  {
    \tl_if_eq:onT {\g__fontspec_mathrm_tl} {\rmdefault}
      {
%<XE>   \fontspec_set_family:Nnn \g__fontspec_mathrm_tl {#1} {#2}
%<LU>   \fontspec_set_family:Nnn \g__fontspec_mathrm_tl {Renderer=Basic,#1} {#2}
        \SetMathAlphabet\mathrm{normal}\g_fontspec_encoding_tl\g__fontspec_mathrm_tl\mddefault\updefault
        \SetMathAlphabet\mathit{normal}\g_fontspec_encoding_tl\g__fontspec_mathrm_tl\mddefault\itdefault
        \SetMathAlphabet\mathbf{normal}\g_fontspec_encoding_tl\g__fontspec_mathrm_tl\bfdefault\updefault
      }
  }

\cs_set:Nn \__fontspec_setsansfont_hook:nn
  {
    \tl_if_eq:onT {\g__fontspec_mathsf_tl} {\sfdefault}
      {
%<XE>   \fontspec_set_family:Nnn \g__fontspec_mathsf_tl {#1} {#2}
%<LU>   \fontspec_set_family:Nnn \g__fontspec_mathsf_tl {Renderer=Basic,#1} {#2}
        \SetMathAlphabet\mathsf{normal}\g_fontspec_encoding_tl\g__fontspec_mathsf_tl\mddefault\updefault
        \SetMathAlphabet\mathsf{bold}  \g_fontspec_encoding_tl\g__fontspec_mathsf_tl\bfdefault\updefault
      }
  }

\cs_set:Nn \__fontspec_setmonofont_hook:nn
  {
    \tl_if_eq:onT {\g__fontspec_mathtt_tl} {\ttdefault}
      {
%<XE>   \fontspec_set_family:Nnn \g__fontspec_mathtt_tl {#1} {#2}
%<LU>   \fontspec_set_family:Nnn \g__fontspec_mathtt_tl {Renderer=Basic,#1} {#2}
        \SetMathAlphabet\mathtt{normal}\g_fontspec_encoding_tl\g__fontspec_mathtt_tl\mddefault\updefault
        \SetMathAlphabet\mathtt{bold}  \g_fontspec_encoding_tl\g__fontspec_mathtt_tl\bfdefault\updefault
      }
  }
%    \end{macrocode}
%
% \subsubsection{Maths font}
% If the maths fonts are set explicitly, then the text commands above will not execute their branches to set the maths font alphabets.
%    \begin{macrocode}
\cs_set:Nn \__fontspec_setmathrm_hook:nn
  {
    \SetMathAlphabet\mathrm{normal}\g_fontspec_encoding_tl\g__fontspec_mathrm_tl\mddefault\updefault
    \SetMathAlphabet\mathit{normal}\g_fontspec_encoding_tl\g__fontspec_mathrm_tl\mddefault\itdefault
    \SetMathAlphabet\mathbf{normal}\g_fontspec_encoding_tl\g__fontspec_mathrm_tl\bfdefault\updefault
  }
\cs_set:Nn \__fontspec_setboldmathrm_hook:nn
  {
    \SetMathAlphabet\mathrm{bold}\g_fontspec_encoding_tl\g__fontspec_bfmathrm_tl\mddefault\updefault
    \SetMathAlphabet\mathbf{bold}\g_fontspec_encoding_tl\g__fontspec_bfmathrm_tl\bfdefault\updefault
    \SetMathAlphabet\mathit{bold}\g_fontspec_encoding_tl\g__fontspec_bfmathrm_tl\mddefault\itdefault
  }
\cs_set:Nn \__fontspec_setmathsf_hook:nn
  {
    \SetMathAlphabet\mathsf{normal}\g_fontspec_encoding_tl\g__fontspec_mathsf_tl\mddefault\updefault
    \SetMathAlphabet\mathsf{bold}  \g_fontspec_encoding_tl\g__fontspec_mathsf_tl\bfdefault\updefault
  }
\cs_set:Nn \__fontspec_setmathtt_hook:nn
  {
    \SetMathAlphabet\mathtt{normal}\g_fontspec_encoding_tl\g__fontspec_mathtt_tl\mddefault\updefault
    \SetMathAlphabet\mathtt{bold}  \g_fontspec_encoding_tl\g__fontspec_mathtt_tl\bfdefault\updefault
  }
%    \end{macrocode}
%
%
% \subsection{The main \cs{setmathfont} macro}
%
% Using a |range| including large character sets such as \cmd\mathrel,
% \cmd\mathalpha, \etc, is \emph{very slow}!
% I hope to improve the performance somehow.
%
% \begin{macro}{\setmathfont}
% \doarg{font features (first optional argument retained for backwards compatibility)}
% \darg{font name}
% \doarg{font features}
%    \begin{macrocode}
\DeclareDocumentCommand \setmathfont { O{} m O{} }
  {
    \@@_setmathfont:nn {#1,#3} {#2}
  }
%    \end{macrocode}
%
%    \begin{macrocode}
\cs_set:Nn \@@_setmathfont:nn
 {
  \tl_set:Nn \l_@@_fontname_tl {#2}
  \@@_init:
%    \end{macrocode}
% Grab the current size information:
% (is this robust enough? Maybe it should be preceded by \cmd\normalsize).
% The macro \cmd\S@\meta{size}
% contains the definitions of the sizes used for maths letters, subscripts and subsubscripts in
% \cmd\tf@size, \cmd\sf@size, and \cmd\ssf@size, respectively.
%    \begin{macrocode}
  \cs_if_exist:cF { S@ \f@size } { \calculate@math@sizes }
  \csname S@\f@size\endcsname
%    \end{macrocode}
% Parse options and tell people what's going on:
%    \begin{macrocode}
  \keys_set_known:nnN {unicode-math} {#1} \l_@@_unknown_keys_clist
  \bool_if:NT \l_@@_init_bool { \@@_log:n {default-math-font} }
%    \end{macrocode}
% Use \pkg{fontspec} to select a font to use.
% After loading the font, we detect what sizes it recommends for scriptsize and scriptscriptsize, so after setting those values appropriately, we reload the font to take these into account.
%    \begin{macrocode}

%<debug>  \csname TIC\endcsname
  \@@_fontspec_select_font:
%<debug>  \csname TOC\endcsname
  \bool_if:nT { \l_@@_ot_math_bool && !\g_@@_mainfont_already_set_bool }
   {
    \@@_declare_math_sizes:
    \@@_fontspec_select_font:
   }
%    \end{macrocode}
% Now define |\@@_symfont_tl| as the \LaTeX\ math font to access everything:
%    \begin{macrocode}
  \cs_if_exist:cF { sym \@@_symfont_tl }
    {
      \DeclareSymbolFont{\@@_symfont_tl}
        {\encodingdefault}{\l_@@_family_tl}{\mddefault}{\updefault}
    }
  \SetSymbolFont{\@@_symfont_tl}{\l_@@_mversion_tl}
    {\encodingdefault}{\l_@@_family_tl}{\mddefault}{\updefault}
%    \end{macrocode}
% Set the bold math version.
%    \begin{macrocode}
  \tl_set:Nn \l_@@_tmpa_tl {normal}
  \tl_if_eq:NNT \l_@@_mversion_tl \l_@@_tmpa_tl
    {
     \SetSymbolFont{\@@_symfont_tl}{bold}
      {\encodingdefault}{\l_@@_family_tl}{\bfdefault}{\updefault}
    }
%    \end{macrocode}
% Declare the math sizes (i.e., scaling of superscripts) for the specific
% values for this font,
% and set defaults for math fams two and three for legacy compatibility:
%    \begin{macrocode}
  \bool_if:nT { \l_@@_ot_math_bool && !\g_@@_mainfont_already_set_bool }
   {
    \bool_set_true:N \g_@@_mainfont_already_set_bool
    \@@_setup_legacy_fam_two:
    \@@_setup_legacy_fam_three:
   }
%    \end{macrocode}
% And now we input every single maths char.
%    \begin{macrocode}
%<debug>  \csname TIC\endcsname
  \@@_input_math_symbol_table:
%<debug>  \csname TOC\endcsname
%    \end{macrocode}
% Finally,
% \begin{itemize}
% \item Remap symbols that don't take their natural mathcode
% \item Activate any symbols that need to be math-active
% \item Enable wide/narrow accents
% \item Assign delimiter codes for symbols that need to grow
% \item Setup the maths alphabets (\cs{mathbf} etc.)
% \end{itemize}
%    \begin{macrocode}
  \@@_remap_symbols:
  \@@_setup_mathactives:
  \@@_setup_delcodes:
%<debug>  \csname TIC\endcsname
  \@@_setup_alphabets:
%<debug>  \csname TOC\endcsname
  \@@_setup_negations:
%    \end{macrocode}
% Prevent spaces, and that's it:
%    \begin{macrocode}
  \ignorespaces
 }
%    \end{macrocode}
% \end{macro}
%
% Backward compatibility alias.
%    \begin{macrocode}
\cs_set_eq:NN \resetmathfont \setmathfont
%    \end{macrocode}
%
% \paragraph{Fall-back font}
%
% Want to load Latin Modern Math if nothing else.
% This needs to happen early so that all of the font-loading machinery executes before
% the other `AtBeginDocument' code.
%    \begin{macrocode}
\AtBeginDocument { \@@_load_lm_if_necessary: }
\cs_new:Nn \@@_load_lm_if_necessary:
  {
    \cs_if_exist:NF \l_@@_fontname_tl
      {
        % TODO: update this when lmmath-bold.otf is released
        \setmathfont{latinmodern-math.otf}[BoldFont={latinmodern-math.otf}]
        \bool_set_false:N \g_@@_mainfont_already_set_bool
      }
  }
%    \end{macrocode}
% Note that here we reset the `font already loaded' boolean so that a new font being set
% will do the right thing in terms of setting up defaults.
%
% TODO: need a better way to do this for the general case. (Maybe a `reset' command option?)
%
% \begin{macro}{\@@_init:}
%    \begin{macrocode}
\cs_new:Nn \@@_init:
 {
%    \end{macrocode}
% \begin{itemize}
% \item Initially assume we're using a proper OpenType font with unicode maths.
%    \begin{macrocode}
  \bool_set_true:N  \l_@@_ot_math_bool
%    \end{macrocode}
% \item Erase any conception \LaTeX\ has of previously defined math symbol fonts;
% this allows \cmd\DeclareSymbolFont\ at any point in the document.
%    \begin{macrocode}
  \cs_set_eq:NN \glb@currsize \scan_stop:
%    \end{macrocode}
% \item To start with, assume we're defining the font for every math symbol character.
%    \begin{macrocode}
  \bool_set_true:N \l_@@_init_bool
  \seq_clear:N \l_@@_char_range_seq
  \clist_clear:N \l_@@_char_nrange_clist
  \seq_clear:N \l_@@_mathalph_seq
  \seq_clear:N \l_@@_missing_alph_seq
%    \end{macrocode}
% \item By default use the `normal' math version.
%    \begin{macrocode}
  \tl_set:Nn \l_@@_mversion_tl {normal}
%    \end{macrocode}
% \item Other range initialisations.
%    \begin{macrocode}
  \tl_set:Nn \@@_symfont_tl {operators}
  \cs_set_eq:NN \_@@_sym:nnn \@@_process_symbol_noparse:nnn
  \cs_set_eq:NN \@@_set_mathalphabet_char:nnn \@@_mathmap_noparse:nnn
  \cs_set_eq:NN \@@_remap_symbol:nnn \@@_remap_symbol_noparse:nnn
  \cs_set_eq:NN \@@_maybe_init_alphabet:n \@@_init_alphabet:n
  \cs_set_eq:NN \@@_map_char_single:nn \@@_map_char_noparse:nn
  \cs_set_eq:NN \@@_assign_delcode:nn \@@_assign_delcode_noparse:nn
  \cs_set_eq:NN \@@_make_mathactive:nNN \@@_make_mathactive_noparse:nNN
%    \end{macrocode}
% \item Define default font features for the script and scriptscript font.
%    \begin{macrocode}
  \tl_set:Nn \l_@@_script_features_tl  {Style=MathScript}
  \tl_set:Nn \l_@@_sscript_features_tl {Style=MathScriptScript}
  \tl_set_eq:NN \l_@@_script_font_tl   \l_@@_fontname_tl
  \tl_set_eq:NN \l_@@_sscript_font_tl  \l_@@_fontname_tl
%    \end{macrocode}
% \end{itemize}
%    \begin{macrocode}
 }
%    \end{macrocode}
% \end{macro}
%
%
% \begin{macro}{\@@_declare_math_sizes:}
% Set the math sizes according to the recommended font parameters:
%    \begin{macrocode}
\cs_new:Nn \@@_declare_math_sizes:
  {
%<*LU>
    \fp_compare:nF { \@@_script_style_size:n {ScriptPercentScaleDown} == 0 }
      {
        \DeclareMathSizes { \f@size } { \f@size }
          { \@@_script_style_size:n {ScriptPercentScaleDown} }
          { \@@_script_style_size:n {ScriptScriptPercentScaleDown} }
      }
%</LU>
%<*XE>
    \dim_compare:nF { \fontdimen 10 \l_@@_font == 0pt }
      {
        \DeclareMathSizes { \f@size } { \f@size }
          { \@@_fontdimen_to_scale:nn {10} {\l_@@_font} }
          { \@@_fontdimen_to_scale:nn {11} {\l_@@_font} }
      }
%</XE>
  }
%    \end{macrocode}
% \end{macro}
%
%<*LU>
% \begin{macro}{\@@_script_style_size:n}
% Determine script- and scriptscriptstyle sizes using luaotfload:
%   \begin{macrocode}
\cs_new:Nn \@@_script_style_size:n
  {
    \fp_eval:n {\directlua{tex.sprint(luaotfload.aux.get_math_dimension("l_@@_font","#1"))} * \f@size / 100 }
  }
%   \end{macrocode}
% \end{macro}
%</LU>
%
%
% \begin{macro}{\@@_setup_legacy_fam_two:}
% \TeX\ won't load the same font twice at the same scale, so we need to magnify this one by an imperceptable amount.
%    \begin{macrocode}
\cs_new:Nn \@@_setup_legacy_fam_two:
  {
    \fontspec_set_family:Nxn \l_@@_family_tl
      {
      \l_@@_font_keyval_tl,
      Scale=1.00001,
      FontAdjustment =
       {
        \fontdimen8\font= \@@_get_fontparam:nn {43} {FractionNumeratorDisplayStyleShiftUp}\relax
        \fontdimen9\font= \@@_get_fontparam:nn {42} {FractionNumeratorShiftUp}\relax
        \fontdimen10\font=\@@_get_fontparam:nn {32} {StackTopShiftUp}\relax
        \fontdimen11\font=\@@_get_fontparam:nn {45} {FractionDenominatorDisplayStyleShiftDown}\relax
        \fontdimen12\font=\@@_get_fontparam:nn {44} {FractionDenominatorShiftDown}\relax
        \fontdimen13\font=\@@_get_fontparam:nn {21} {SuperscriptShiftUp}\relax
        \fontdimen14\font=\@@_get_fontparam:nn {21} {SuperscriptShiftUp}\relax
        \fontdimen15\font=\@@_get_fontparam:nn {22} {SuperscriptShiftUpCramped}\relax
        \fontdimen16\font=\@@_get_fontparam:nn {18} {SubscriptShiftDown}\relax
        \fontdimen17\font=\@@_get_fontparam:nn {18} {SubscriptShiftDownWithSuperscript}\relax
        \fontdimen18\font=\@@_get_fontparam:nn {24} {SuperscriptBaselineDropMax}\relax
        \fontdimen19\font=\@@_get_fontparam:nn {20} {SubscriptBaselineDropMin}\relax
        \fontdimen20\font=0pt\relax % delim1 = FractionDelimiterDisplaySize
        \fontdimen21\font=0pt\relax % delim2 = FractionDelimiterSize
        \fontdimen22\font=\@@_get_fontparam:nn {15} {AxisHeight}\relax
       }
      } {\l_@@_fontname_tl}
    \SetSymbolFont{symbols}{\l_@@_mversion_tl}
      {\encodingdefault}{\l_@@_family_tl}{\mddefault}{\updefault}

    \tl_set:Nn \l_@@_tmpa_tl {normal}
    \tl_if_eq:NNT \l_@@_mversion_tl \l_@@_tmpa_tl
      {
      \SetSymbolFont{symbols}{bold}
        {\encodingdefault}{\l_@@_family_tl}{\bfdefault}{\updefault}
      }
  }
%    \end{macrocode}
% \end{macro}
%
% \begin{macro}{\@@_setup_legacy_fam_three:}
% Similarly, this font is shrunk by an imperceptable amount for \TeX\ to load it again.
%    \begin{macrocode}
\cs_new:Nn \@@_setup_legacy_fam_three:
  {
    \fontspec_set_family:Nxn \l_@@_family_tl
      {
      \l_@@_font_keyval_tl,
      Scale=0.99999,
      FontAdjustment={
        \fontdimen8\font= \@@_get_fontparam:nn {48} {FractionRuleThickness}\relax
        \fontdimen9\font= \@@_get_fontparam:nn {28} {UpperLimitGapMin}\relax
        \fontdimen10\font=\@@_get_fontparam:nn {30} {LowerLimitGapMin}\relax
        \fontdimen11\font=\@@_get_fontparam:nn {29} {UpperLimitBaselineRiseMin}\relax
        \fontdimen12\font=\@@_get_fontparam:nn {31} {LowerLimitBaselineDropMin}\relax
        \fontdimen13\font=0pt\relax
      }
    } {\l_@@_fontname_tl}
    \SetSymbolFont{largesymbols}{\l_@@_mversion_tl}
      {\encodingdefault}{\l_@@_family_tl}{\mddefault}{\updefault}

    \tl_set:Nn \l_@@_tmpa_tl {normal}
    \tl_if_eq:NNT \l_@@_mversion_tl \l_@@_tmpa_tl
      {
      \SetSymbolFont{largesymbols}{bold}
        {\encodingdefault}{\l_@@_family_tl}{\bfdefault}{\updefault}
      }
  }
%    \end{macrocode}
% \end{macro}
%
%
%    \begin{macrocode}
\cs_new:Nn \@@_get_fontparam:nn
%<XE>  { \the\fontdimen#1\l_@@_font\relax }
%<LU>  { \directlua{fontspec.mathfontdimen("l_@@_font","#2")} }
%    \end{macrocode}
%
%
%
% \begin{macro}{\@@_fontspec_select_font:}
% Select the font with \cs{fontspec} and define \cs{l_@@_font} from it.
%    \begin{macrocode}
\cs_new:Nn \@@_fontspec_select_font:
 {
  \tl_set:Nx \l_@@_font_keyval_tl {
%<LU>     Renderer = Basic,
    BoldItalicFont = {}, ItalicFont = {},
    Script = Math,
    SizeFeatures =
     {
      {
       Size = \tf@size-
      } ,
      {
       Size = \sf@size-\tf@size ,
       Font = \l_@@_script_font_tl ,
       \l_@@_script_features_tl
      } ,
      {
       Size = -\sf@size ,
       Font = \l_@@_sscript_font_tl ,
       \l_@@_sscript_features_tl
      }
     } ,
    \l_@@_unknown_keys_clist
  }
  \fontspec_set_fontface:NNxn \l_@@_font \l_@@_family_tl
    {\l_@@_font_keyval_tl} {\l_@@_fontname_tl}
%    \end{macrocode}
% Check whether we're using a real maths font:
%    \begin{macrocode}
  \group_begin:
    \fontfamily{\l_@@_family_tl}\selectfont
    \fontspec_if_script:nF {math} {\bool_gset_false:N \l_@@_ot_math_bool}
  \group_end:
 }
%    \end{macrocode}
% \end{macro}
%
%
% \subsubsection{Functions for setting up symbols with mathcodes}
% \seclabel{mathsymbol}
%
% \begin{macro}{\@@_process_symbol_noparse:nnn}
% \begin{macro}{\@@_process_symbol_parse:nnn}
% If the \feat{range} font feature has been used, then only
% a subset of the Unicode glyphs are to be defined.
% See \secref{rangeproc} for the code that enables this.
%    \begin{macrocode}
\cs_set:Nn \@@_process_symbol_noparse:nnn
 {
  \@@_set_mathsymbol:nNNn {\@@_symfont_tl} #2 #3 {#1}
 }
%    \end{macrocode}
%    \begin{macrocode}
\cs_set:Nn \@@_process_symbol_parse:nnn
 {
  \@@_if_char_spec:nNNT {#1} {#2} {#3}
   {
    \@@_process_symbol_noparse:nnn {#1} {#2} {#3}
   }
 }
%    \end{macrocode}
% \end{macro}
% \end{macro}
%
%
% \begin{macro}{\@@_remap_symbols:}
% \begin{macro}{\@@_remap_symbol_noparse:nnn}
% \begin{macro}{\@@_remap_symbol_parse:nnn}
% This function is used to define the mathcodes for those chars which should
% be mapped to a different glyph than themselves.
%    \begin{macrocode}
\cs_new:Npn \@@_remap_symbols:
 {
  \@@_remap_symbol:nnn{`\-}{\mathbin}{"02212}% hyphen to minus
  \@@_remap_symbol:nnn{`\*}{\mathbin}{"02217}% text asterisk to "centred asterisk"
  \bool_if:NF \g_@@_literal_colon_bool
   {
    \@@_remap_symbol:nnn{`\:}{\mathrel}{"02236}% colon to ratio (i.e., punct to rel)
   }
 }
%    \end{macrocode}
% \end{macro}
% Where |\@@_remap_symbol:nnn| is defined to be one of these two, depending
% on the range setup:
%    \begin{macrocode}
\cs_new:Nn \@@_remap_symbol_parse:nnn
 {
  \@@_if_char_spec:nNNT {#3} {\@nil} {#2}
   { \@@_remap_symbol_noparse:nnn {#1} {#2} {#3} }
 }
\cs_new:Nn \@@_remap_symbol_noparse:nnn
 {
  \clist_map_inline:nn {#1}
   { \@@_set_mathcode:nnnn {##1} {#2} {\@@_symfont_tl} {#3} }
 }
%    \end{macrocode}
% \end{macro}
% \end{macro}
%
%
% \subsubsection{Active math characters}
%
% There are more math active chars later in the subscript/superscript section.
% But they don't need to be able to be typeset directly.
%
% \begin{macro}{\@@_setup_mathactives:}
%    \begin{macrocode}
\cs_new:Npn \@@_setup_mathactives:
 {
  \@@_make_mathactive:nNN {"2032} \@@_prime_single_mchar \mathord
  \@@_make_mathactive:nNN {"2033} \@@_prime_double_mchar \mathord
  \@@_make_mathactive:nNN {"2034} \@@_prime_triple_mchar \mathord
  \@@_make_mathactive:nNN {"2057} \@@_prime_quad_mchar   \mathord
  \@@_make_mathactive:nNN {"2035} \@@_backprime_single_mchar \mathord
  \@@_make_mathactive:nNN {"2036} \@@_backprime_double_mchar \mathord
  \@@_make_mathactive:nNN {"2037} \@@_backprime_triple_mchar \mathord
  \@@_make_mathactive:nNN {`\'} \mathstraightquote \mathord
  \@@_make_mathactive:nNN {`\`} \mathbacktick      \mathord
 }
%    \end{macrocode}
% \end{macro}
%
% \begin{macro}{\@@_make_mathactive:nNN}
% Makes |#1| a mathactive char, and gives cs |#2| the meaning of mathchar |#1|
% with class |#3|.
% You are responsible for giving active |#1| a particular meaning!
%    \begin{macrocode}
\cs_new:Nn \@@_make_mathactive_parse:nNN
  {
    \@@_if_char_spec:nNNT {#1} #2 #3
      { \@@_make_mathactive_noparse:nNN {#1} #2 #3 }
  }
\cs_new:Nn \@@_make_mathactive_noparse:nNN
  {
    \@@_set_mathchar:NNnn #2 #3 {\@@_symfont_tl} {#1}
    \@@_char_gmake_mathactive:n {#1}
  }
%    \end{macrocode}
% \end{macro}
%
% \subsubsection{Delimiter codes}
%
%
% \begin{macro}{\@@_assign_delcode:nn}
%    \begin{macrocode}
\cs_new:Nn \@@_assign_delcode_noparse:nn
 {
  \@@_set_delcode:nnn \@@_symfont_tl {#1} {#2}
 }
\cs_new:Nn \@@_assign_delcode_parse:nn
 {
  \@@_if_char_spec:nNNT {#2} {\@nil} {\@nil}
   {
    \@@_assign_delcode_noparse:nn {#1} {#2}
   }
 }
%    \end{macrocode}
% \end{macro}
%
%
% \begin{macro}{\@@_assign_delcode:n}
% Shorthand.
%    \begin{macrocode}
\cs_new:Nn \@@_assign_delcode:n { \@@_assign_delcode:nn {#1} {#1} }
%    \end{macrocode}
% \end{macro}
%
%
%
% \begin{macro}{\@@_setup_delcodes:}
% Some symbols that aren't mathopen/mathclose still need to have delimiter codes assigned.
% The list of vertical arrows may be incomplete.
% On the other hand, many fonts won't support them all being stretchy.
% And some of them are probably not meant to stretch, either. But adding them here doesn't hurt.
%    \begin{macrocode}
\cs_new:Npn \@@_setup_delcodes:
 {
  % ensure \left. and \right. work:
  \@@_set_delcode:nnn \@@_symfont_tl {`\.} {\c_zero}
  % this is forcefully done to fix a bug -- indicates a larger problem!

  \@@_assign_delcode:nn {`\/}   {\g_@@_slash_delimiter_usv}
  \@@_assign_delcode:nn {"2044} {\g_@@_slash_delimiter_usv} % fracslash
  \@@_assign_delcode:nn {"2215} {\g_@@_slash_delimiter_usv} % divslash
  \@@_assign_delcode:n {"005C} % backslash
  \@@_assign_delcode:nn {`\<} {"27E8} % angle brackets with ascii notation
  \@@_assign_delcode:nn {`\>} {"27E9} % angle brackets with ascii notation
  \@@_assign_delcode:n {"2191} % up arrow
  \@@_assign_delcode:n {"2193} % down arrow
  \@@_assign_delcode:n {"2195} % updown arrow
  \@@_assign_delcode:n {"219F} % up arrow twohead
  \@@_assign_delcode:n {"21A1} % down arrow twohead
  \@@_assign_delcode:n {"21A5} % up arrow from bar
  \@@_assign_delcode:n {"21A7} % down arrow from bar
  \@@_assign_delcode:n {"21A8} % updown arrow from bar
  \@@_assign_delcode:n {"21BE} % up harpoon right
  \@@_assign_delcode:n {"21BF} % up harpoon left
  \@@_assign_delcode:n {"21C2} % down harpoon right
  \@@_assign_delcode:n {"21C3} % down harpoon left
  \@@_assign_delcode:n {"21C5} % arrows up down
  \@@_assign_delcode:n {"21F5} % arrows down up
  \@@_assign_delcode:n {"21C8} % arrows up up
  \@@_assign_delcode:n {"21CA} % arrows down down
  \@@_assign_delcode:n {"21D1} % double up arrow
  \@@_assign_delcode:n {"21D3} % double down arrow
  \@@_assign_delcode:n {"21D5} % double updown arrow
  \@@_assign_delcode:n {"21DE} % up arrow double stroke
  \@@_assign_delcode:n {"21DF} % down arrow double stroke
  \@@_assign_delcode:n {"21E1} % up arrow dashed
  \@@_assign_delcode:n {"21E3} % down arrow dashed
  \@@_assign_delcode:n {"21E7} % up white arrow
  \@@_assign_delcode:n {"21E9} % down white arrow
  \@@_assign_delcode:n {"21EA} % up white arrow from bar
  \@@_assign_delcode:n {"21F3} % updown white arrow
 }
%    \end{macrocode}
% \end{macro}
%
%
%
%
% \subsection{(Big) operators}
%
% Turns out that \XeTeX\ is clever enough to deal with big operators for us
% automatically with \cmd\Umathchardef. Amazing!
%
% However, the limits aren't set automatically; that is, we want to define,
% a la Plain \TeX\ \etc, |\def\int{\intop\nolimits}|, so there needs to be a
% transformation from \cmd\int\ to \cmd\intop\ during the expansion of
% \cmd\_@@_sym:nnn\ in the appropriate contexts.
%
% \begin{macro}{\l_@@_nolimits_tl}
% This macro is a sequence containing those maths operators that require a
% \cmd\nolimits\ suffix.
% This list is used when processing |unicode-math-table.tex| to define such
% commands automatically (see the macro \cs{@@_set_mathsymbol:nNNn}).
% I've chosen essentially just the operators that look like integrals;
% hopefully a better mathematician can help me out here.
% I've a feeling that it's more useful \emph{not} to include the multiple
% integrals such as $\iiiint$, but that might be a matter of preference.
%    \begin{macrocode}
\tl_new:N \l_@@_nolimits_tl
\tl_set:Nn \l_@@_nolimits_tl
 {
  \int\iint\iiint\iiiint\oint\oiint\oiiint
  \intclockwise\varointclockwise\ointctrclockwise\sumint
  \intbar\intBar\fint\cirfnint\awint\rppolint
  \scpolint\npolint\pointint\sqint\intlarhk\intx
  \intcap\intcup\upint\lowint
 }
%    \end{macrocode}
% \end{macro}
%
% \begin{macro}{\addnolimits}
% This macro appends material to the macro containing the list of operators
% that don't take limits.
%    \begin{macrocode}
\DeclareDocumentCommand \addnolimits {m}
 {
  \tl_put_right:Nn \l_@@_nolimits_tl {#1}
 }
%    \end{macrocode}
% \end{macro}
%
% \begin{macro}{\removenolimits}
% Can this macro be given a better name?
% It removes an item from the nolimits list.
%    \begin{macrocode}
\DeclareDocumentCommand \removenolimits {m}
 {
  \tl_remove_all:Nn \l_@@_nolimits_tl {#1}
 }
%    \end{macrocode}
% \end{macro}
%
% \subsection{Radicals}
%
% The radical for square root is organised in \cs{@@_set_mathsymbol:nNNn}.
% I think it's the only radical ever.
% (Actually, there is also \cs{cuberoot} and \cs{fourthroot}, but they don't
%  seem to behave as proper radicals.)
%
% Also, what about right-to-left square roots?
%
% \begin{macro}{\l_@@_radicals_tl}
% We organise radicals in the same way as nolimits-operators.
%    \begin{macrocode}
\tl_new:N \l_@@_radicals_tl
\tl_set:Nn \l_@@_radicals_tl {\sqrt \longdivision}
%    \end{macrocode}
% \end{macro}
%
% \subsection{Maths accents}
%
% Maths accents should just work \emph{if they are available in the font}.
%
% \subsection{Common interface for font parameters}
%
% \XeTeX\ and \LuaTeX\ have different interfaces for math font parameters.
% We use \LuaTeX’s interface because it’s much better, but rename the primitives to be more \LaTeX3-like.
% There are getter and setter commands for each font parameter.
% The names of the parameters is derived from the \LuaTeX\ names, with underscores inserted between words.
% For every parameter \cs{Umath\meta{\LuaTeX\ name}}, we define an expandable getter command \cs{@@_\meta{\LaTeX3 name}:N} and a protected setter command \cs{@@_set_\meta{\LaTeX3 name}:Nn}.
% The getter command takes one of the style primitives (\cs{displaystyle} etc.)\ and expands to the font parameter, which is a \meta{dimension}.
% The setter command takes a style primitive and a dimension expression, which is parsed with \cs{dim_eval:n}.
%
% Often, the mapping between font dimensions and font parameters is bijective, but there are cases which require special attention:
% \begin{itemize}
% \item Some parameters map to different dimensions in display and non-display styles.
% \item Likewise, one parameter maps to different dimensions in non-cramped and cramped styles.
% \item There are a few parameters for which \XeTeX\ doesn’t seem to provide \cs{fontdimen}s; in this case the getter and setter commands are left undefined.
% \end{itemize}
%
% \paragraph{Cramped style tokens}
% \LuaTeX\ has \cs{crampeddisplaystyle} etc.,\ but they are loaded as \cs{luatexcrampeddisplaystyle} etc.\ by the \pkg{luatextra} package.
% \XeTeX, however, doesn’t have these primitives, and their syntax cannot really be emulated.
% Nevertheless, we define these commands as quarks, so they can be used as arguments to the font parameter commands (but nowhere else).
% Making these commands available is necessary because we need to make a distinction between cramped and non-cramped styles for one font parameter.
%
% \begin{macro}{\@@_new_cramped_style:N}
% \darg{command}
% Define \meta{command} as a new cramped style switch.
% For \LuaTeX, simply rename the correspronding primitive if it is not
% already defined.
% For \XeTeX, define \meta{command} as a new quark.
%    \begin{macrocode}
\cs_new_protected_nopar:Nn \@@_new_cramped_style:N
%<XE>  { \quark_new:N #1 }
%<LU>  {
%<LU>    \cs_if_exist:NF #1
%<LU>      { \cs_new_eq:Nc #1 { luatex \cs_to_str:N #1 } }
%<LU>  }
%    \end{macrocode}
% \end{macro}
%
% \begin{macro}{\crampeddisplaystyle}
% \begin{macro}{\crampedtextstyle}
% \begin{macro}{\crampedscriptstyle}
% \begin{macro}{\crampedscriptscriptstyle}
% The cramped style commands.
%    \begin{macrocode}
\@@_new_cramped_style:N \crampeddisplaystyle
\@@_new_cramped_style:N \crampedtextstyle
\@@_new_cramped_style:N \crampedscriptstyle
\@@_new_cramped_style:N \crampedscriptscriptstyle
%    \end{macrocode}
% \end{macro}
% \end{macro}
% \end{macro}
% \end{macro}
%
% \paragraph{Font dimension mapping}
% Font parameters may differ between the styles.
% \LuaTeX\ accounts for this by having the parameter primitives take a style token argument.
% To replicate this behavior in \XeTeX, we have to map style tokens to specific combinations of font dimension numbers and math fonts (\cs{textfont} etc.).
%
% \begin{macro}{\@@_font_dimen:Nnnnn}
% \darg{style token}
% \darg{font dimen for display style}
% \darg{font dimen for cramped display style}
% \darg{font dimen for non-display styles}
% \darg{font dimen for cramped non-display styles}
% Map math style to \XeTeX\ math font dimension.
% \meta{style token} must be one of the style switches (\cs{displaystyle}, \cs{crampeddisplaystyle}, \dots).
% The other parameters are integer constants referring to font dimension numbers.
% The macro expands to a dimension which contains the appropriate font dimension.
%    \begin{macrocode}
%<*XE>
  \cs_new_nopar:Npn \@@_font_dimen:Nnnnn #1 #2 #3 #4 #5 {
    \fontdimen
    \cs_if_eq:NNTF #1 \displaystyle {
      #2 \textfont
    } {
      \cs_if_eq:NNTF #1 \crampeddisplaystyle {
        #3 \textfont
      } {
        \cs_if_eq:NNTF #1 \textstyle {
          #4 \textfont
        } {
          \cs_if_eq:NNTF #1 \crampedtextstyle {
            #5 \textfont
          } {
            \cs_if_eq:NNTF #1 \scriptstyle {
              #4 \scriptfont
            } {
              \cs_if_eq:NNTF #1 \crampedscriptstyle {
                #5 \scriptfont
              } {
                \cs_if_eq:NNTF #1 \scriptscriptstyle {
                  #4 \scriptscriptfont
                } {
%    \end{macrocode}
% Should we check here if the style is invalid?
%    \begin{macrocode}
                  #5 \scriptscriptfont
                }
              }
            }
          }
        }
      }
    }
%    \end{macrocode}
% Which family to use?
%    \begin{macrocode}
    \c_two
  }
%</XE>
%    \end{macrocode}
% \end{macro}
%
% \paragraph{Font parameters}
% This paragraph contains macros for defining the font parameter interface, as well as the definition for all font parameters known to \LuaTeX.
%
% \begin{macro}{\@@_font_param:nnnnn}
% \darg{name}
% \darg{font dimension for non-cramped display style}
% \darg{font dimension for cramped display style}
% \darg{font dimension for non-cramped non-display styles}
% \darg{font dimension for cramped non-display styles}
% This macro defines getter and setter functions for the font parameter \meta{name}.
% The \LuaTeX\ font parameter name is produced by removing all underscores and prefixing the result with |Umath|.
% The \XeTeX\ font dimension numbers must be integer constants.
%    \begin{macrocode}
\cs_new_protected_nopar:Nn \@@_font_param:nnnnn
%<*XE>
{
  \@@_font_param_aux:ccnnnn { @@_ #1 :N } { @@_set_ #1 :Nn }
    { #2 } { #3 } { #4 } { #5 }
}
%</XE>
%<*LU>
{
  \tl_set:Nn \l_@@_tmpa_tl { #1 }
  \tl_remove_all:Nn \l_@@_tmpa_tl { _ }
  \@@_font_param_aux:ccc { @@_ #1 :N } { @@_set_ #1 :Nn }
    { Umath \l_@@_tmpa_tl }
}
%</LU>
%    \end{macrocode}
% \end{macro}
%
% \begin{macro}{\@@_font_param:nnn}
% \darg{name}
% \darg{font dimension for display style}
% \darg{font dimension for non-display styles}
% This macro defines getter and setter functions for the font parameter \meta{name}.
% The \LuaTeX\ font parameter name is produced by removing all underscores and prefixing the result with |Umath|.
% The \XeTeX\ font dimension numbers must be integer constants.
%    \begin{macrocode}
\cs_new_protected_nopar:Nn \@@_font_param:nnn
 {
  \@@_font_param:nnnnn { #1 } { #2 } { #2 } { #3 } { #3 }
 }
%    \end{macrocode}
% \end{macro}
%
% \begin{macro}{\@@_font_param:nn}
% \darg{name}
% \darg{font dimension}
% This macro defines getter and setter functions for the font parameter \meta{name}.
% The \LuaTeX\ font parameter name is produced by removing all underscores and prefixing the result with |Umath|.
% The \XeTeX\ font dimension number must be an integer constant.
%    \begin{macrocode}
\cs_new_protected_nopar:Nn \@@_font_param:nn
 {
  \@@_font_param:nnnnn { #1 } { #2 } { #2 } { #2 } { #2 }
 }
%    \end{macrocode}
% \end{macro}
%
% \begin{macro}{\@@_font_param:n}
% \darg{name}
% This macro defines getter and setter functions for the font parameter \meta{name}, which is considered unavailable in \XeTeX\@.
% The \LuaTeX\ font parameter name is produced by removing all underscores and prefixing the result with |Umath|.
%    \begin{macrocode}
\cs_new_protected_nopar:Nn \@@_font_param:n
%<XE>  { }
%<LU>  { \@@_font_param:nnnnn { #1 } { 0 } { 0 } { 0 } { 0 } }
%    \end{macrocode}
% \end{macro}
%
% \begin{macro}{\@@_font_param_aux:NNnnnn}
% \begin{macro}{\@@_font_param_aux:NNN}
% Auxiliary macros for generating font parameter accessor macros.
%    \begin{macrocode}
%<*XE>
\cs_new_protected_nopar:Nn \@@_font_param_aux:NNnnnn
  {
    \cs_new_nopar:Npn #1 ##1
     {
      \@@_font_dimen:Nnnnn ##1 { #3 } { #4 } { #5 } { #6 }
     }
    \cs_new_protected_nopar:Npn #2 ##1 ##2
     {
      #1 ##1 \dim_eval:n { ##2 }
     }
  }
\cs_generate_variant:Nn \@@_font_param_aux:NNnnnn { cc }
%</XE>
%<*LU>
\cs_new_protected_nopar:Nn \@@_font_param_aux:NNN
  {
    \cs_new_nopar:Npn #1 ##1
     {
      #3 ##1
     }
    \cs_new_protected_nopar:Npn #2 ##1 ##2
     {
      #3 ##1 \dim_eval:n { ##2 }
     }
  }
\cs_generate_variant:Nn \@@_font_param_aux:NNN { ccc }
%</LU>
%    \end{macrocode}
% \end{macro}
% \end{macro}
%
% Now all font parameters that are listed in the \LuaTeX\ reference follow.
%    \begin{macrocode}
\@@_font_param:nn { axis } { 15 }
\@@_font_param:nn { operator_size } { 13 }
\@@_font_param:n { fraction_del_size }
\@@_font_param:nnn { fraction_denom_down } { 45 } { 44 }
\@@_font_param:nnn { fraction_denom_vgap } { 50 } { 49 }
\@@_font_param:nnn { fraction_num_up } { 43 } { 42 }
\@@_font_param:nnn { fraction_num_vgap } { 47 } { 46 }
\@@_font_param:nn { fraction_rule } { 48 }
\@@_font_param:nn { limit_above_bgap } { 29 }
\@@_font_param:n { limit_above_kern }
\@@_font_param:nn { limit_above_vgap } { 28 }
\@@_font_param:nn { limit_below_bgap } { 31 }
\@@_font_param:n { limit_below_kern }
\@@_font_param:nn { limit_below_vgap } { 30 }
\@@_font_param:nn { over_delimiter_vgap } { 41 }
\@@_font_param:nn { over_delimiter_bgap } { 38 }
\@@_font_param:nn { under_delimiter_vgap } { 40 }
\@@_font_param:nn { under_delimiter_bgap } { 39 }
\@@_font_param:nn { overbar_kern } { 55 }
\@@_font_param:nn { overbar_rule } { 54 }
\@@_font_param:nn { overbar_vgap } { 53 }
\@@_font_param:n { quad }
\@@_font_param:nn { radical_kern } { 62 }
\@@_font_param:nn { radical_rule } { 61 }
\@@_font_param:nnn { radical_vgap } { 60 } { 59 }
\@@_font_param:nn { radical_degree_before } { 63 }
\@@_font_param:nn { radical_degree_after } { 64 }
\@@_font_param:nn { radical_degree_raise } { 65 }
\@@_font_param:nn { space_after_script } { 27 }
\@@_font_param:nnn { stack_denom_down } { 35 } { 34 }
\@@_font_param:nnn { stack_num_up } { 33 } { 32 }
\@@_font_param:nnn { stack_vgap } { 37 } { 36 }
\@@_font_param:nn { sub_shift_down } { 18 }
\@@_font_param:nn { sub_shift_drop } { 20 }
\@@_font_param:n { subsup_shift_down }
\@@_font_param:nn { sub_top_max } { 19 }
\@@_font_param:nn { subsup_vgap } { 25 }
\@@_font_param:nn { sup_bottom_min } { 23 }
\@@_font_param:nn { sup_shift_drop } { 24 }
\@@_font_param:nnnnn { sup_shift_up } { 21 } { 22 } { 21 } { 22 }
\@@_font_param:nn { supsub_bottom_max } { 26 }
\@@_font_param:nn { underbar_kern } { 58 }
\@@_font_param:nn { underbar_rule } { 57 }
\@@_font_param:nn { underbar_vgap } { 56 }
\@@_font_param:n { connector_overlap_min }
%    \end{macrocode}
%
% \section{Font features}
%
% \subsection{Math version}
%    \begin{macrocode}
\keys_define:nn {unicode-math}
  {
    version .code:n =
      {
        \tl_set:Nn \l_@@_mversion_tl {#1}
        \DeclareMathVersion {\l_@@_mversion_tl}
      }
  }
%    \end{macrocode}
%
% \subsection{Script and scriptscript font options}
%    \begin{macrocode}
\keys_define:nn {unicode-math}
 {
  script-features  .tl_set:N =  \l_@@_script_features_tl ,
  sscript-features .tl_set:N = \l_@@_sscript_features_tl ,
       script-font .tl_set:N =      \l_@@_script_font_tl ,
      sscript-font .tl_set:N =     \l_@@_sscript_font_tl ,
 }
%    \end{macrocode}
%
% \subsection{Range processing}
% \seclabel{rangeproc}
%
%    \begin{macrocode}
\keys_define:nn {unicode-math}
 {
  range .code:n =
   {
    \bool_set_false:N \l_@@_init_bool
%    \end{macrocode}
% Set processing functions if we're not defining the full Unicode math repetoire.
% Math symbols are defined with \cmd\_@@_sym:nnn; see \secref{mathsymbol}
% for the individual definitions
%    \begin{macrocode}
    \int_incr:N \g_@@_fam_int
    \tl_set:Nx \@@_symfont_tl {@@_fam\int_use:N\g_@@_fam_int}
    \cs_set_eq:NN \_@@_sym:nnn \@@_process_symbol_parse:nnn
    \cs_set_eq:NN \@@_set_mathalphabet_char:Nnn \@@_mathmap_parse:Nnn
    \cs_set_eq:NN \@@_remap_symbol:nnn \@@_remap_symbol_parse:nnn
    \cs_set_eq:NN \@@_maybe_init_alphabet:n \use_none:n
    \cs_set_eq:NN \@@_map_char_single:nn \@@_map_char_parse:nn
    \cs_set_eq:NN \@@_assign_delcode:nn \@@_assign_delcode_parse:nn
    \cs_set_eq:NN \@@_make_mathactive:nNN \@@_make_mathactive_parse:nNN
%    \end{macrocode}
% Proceed by filling up the various `range' seqs according to the user options.
%    \begin{macrocode}
    \seq_clear:N \l_@@_char_range_seq
    \seq_clear:N \l_@@_mclass_range_seq
    \seq_clear:N \l_@@_cmd_range_seq
    \seq_clear:N \l_@@_mathalph_seq

    \clist_map_inline:nn {#1}
     {
      \@@_if_mathalph_decl:nTF {##1}
       {
        \seq_put_right:Nx \l_@@_mathalph_seq
         {
          { \exp_not:V \l_@@_tmpa_tl }
          { \exp_not:V \l_@@_tmpb_tl }
          { \exp_not:V \l_@@_tmpc_tl }
         }
       }
       {
%    \end{macrocode}
% Four cases:
% math class matching the known list;
% single item that is a control sequence---command name;
% single item that isn't---edge case, must be 0--9;
% none of the above---char range.
%    \begin{macrocode}
        \seq_if_in:NnTF \g_@@_mathclasses_seq {##1}
          { \seq_put_right:Nn \l_@@_mclass_range_seq {##1} }
          {
            \bool_lazy_and:nnTF { \tl_if_single_p:n {##1} } { \token_if_cs_p:N ##1 }
              { \seq_put_right:Nn \l_@@_cmd_range_seq {##1} }
              { \seq_put_right:Nn \l_@@_char_range_seq {##1} }
          }
       }
     }
   }
 }
%    \end{macrocode}
%
%
% \begin{macro}{\@@_if_mathalph_decl:nTF}
% Possible forms of input:\\
% |\mathscr|\\
% |\mathscr->\mathup|\\
% |\mathscr/{Latin}|\\
% |\mathscr/{Latin}->\mathup|\\
% Outputs:\\
% |tmpa|: math style (\eg, |\mathscr|)\\
% |tmpb|: alphabets (\eg, |Latin|)\\
% |tmpc|: remap style (\eg, |\mathup|). Defaults to |tmpa|.
%
% The remap style can also be |\mathcal->stixcal|, which I marginally prefer
% in the general case.
%    \begin{macrocode}
\prg_new_conditional:Nnn \@@_if_mathalph_decl:n {TF}
 {
  \tl_set:Nn  \l_@@_tmpa_tl {#1}
  \tl_clear:N \l_@@_tmpb_tl
  \tl_clear:N \l_@@_tmpc_tl

  \tl_if_in:NnT \l_@@_tmpa_tl {->}
   { \exp_after:wN \@@_split_arrow:w \l_@@_tmpa_tl \q_nil }

  \tl_if_in:NnT \l_@@_tmpa_tl {/}
   { \exp_after:wN \@@_split_slash:w \l_@@_tmpa_tl \q_nil }

  \tl_set:Nx \l_@@_tmpa_tl { \tl_to_str:N \l_@@_tmpa_tl }
  \exp_args:NNx \tl_remove_all:Nn \l_@@_tmpa_tl { \token_to_str:N \math }
  \exp_args:NNx \tl_remove_all:Nn \l_@@_tmpa_tl { \token_to_str:N \sym }
  \tl_trim_spaces:N \l_@@_tmpa_tl

  \tl_if_empty:NT \l_@@_tmpc_tl
   { \tl_set_eq:NN \l_@@_tmpc_tl \l_@@_tmpa_tl }

  \seq_if_in:NVTF \g_@@_named_ranges_seq \l_@@_tmpa_tl
   { \prg_return_true: } { \prg_return_false: }
 }
%    \end{macrocode}
%    \begin{macrocode}
\cs_set:Npn \@@_split_arrow:w #1->#2 \q_nil
 {
  \tl_set:Nx \l_@@_tmpa_tl { \tl_trim_spaces:n {#1} }
  \tl_set:Nx \l_@@_tmpc_tl { \tl_trim_spaces:n {#2} }
 }
%    \end{macrocode}
%    \begin{macrocode}
\cs_set:Npn \@@_split_slash:w #1/#2 \q_nil
 {
  \tl_set:Nx \l_@@_tmpa_tl { \tl_trim_spaces:n {#1} }
  \tl_set:Nx \l_@@_tmpb_tl { \tl_trim_spaces:n {#2} }
 }
%    \end{macrocode}
% \end{macro}
%
% Pretty basic comma separated range processing.
% Donald Arseneau's \pkg{selectp} package has a cleverer technique.
%
% \begin{macro}{\@@_if_char_spec:nNNT}
% \darg{Unicode character slot}
% \darg{control sequence (character macro)}
% \darg{control sequence (math class)}
% \darg{code to execute}
% This macro expands to |#4|
% if any of its arguments are contained in \cmd\l_@@_char_range_seq.
% This list can contain either character ranges (for checking with |#1|) or control sequences.
% These latter can either be the command name of a specific character, \emph{or} the math
% type of one (\eg, \cmd\mathbin).
%
% Character ranges are passed to \cs{@@_if_char_spec:nNNT}, which accepts input in the form shown in \tabref{ranges}.
%
% \begin{table}[htbp]
% \centering
% \topcaption{Ranges accepted by \cs{@@_if_char_spec:nNNT}.}
% \label{tab:ranges}
% \begin{tabular}{>{\ttfamily}cc}
% \textrm{Input} & Range \\
% \hline
% x & $r=x$ \\
% x- & $r\geq x$ \\
% -y & $r\leq y$ \\
% x-y & $x \leq r \leq y$ \\
% \end{tabular}
% \end{table}
%
% We have three tests, performed sequentially in order of execution time.
% Any test finding a match jumps directly to the end.
%    \begin{macrocode}
\cs_new:Nn \@@_if_char_spec:nNNT
  {
    % math class:
    \seq_if_in:NnT \l_@@_mclass_range_seq {#3}
      { \use_none_delimit_by_q_nil:w }

    % command name:
    \seq_if_in:NnT \l_@@_cmd_range_seq {#2}
      { \use_none_delimit_by_q_nil:w }

    % character slot:
    \seq_map_inline:Nn \l_@@_char_range_seq
      {
        \@@_int_if_slot_in_range:nnT {#1} {##1}
          { \seq_map_break:n { \use_none_delimit_by_q_nil:w } }
      }

    % the following expands to nil if no match was found:
    \use_none:nnn
    \q_nil
    \use:n
      {
        \clist_put_right:Nx \l_@@_char_nrange_clist { \int_eval:n {#1} }
        #4
      }
  }
%    \end{macrocode}
% \end{macro}
%
% \begin{macro}{\@@_int_if_slot_in_range:nnT}
% A `numrange' is like |-2,5-8,12,17-| (can be unsorted).
%
% Four cases, four argument types:
% \begin{Verbatim}
% input    #2     #3      #4
% "1  "   [ 1] - [qn] - [   ] qs
% "1- "   [ 1] - [  ] - [qn-] qs
% " -3"   [  ] - [ 3] - [qn-] qs
% "1-3"   [ 1] - [ 3] - [qn-] qs
% \end{Verbatim}
%
%    \begin{macrocode}
\cs_new:Nn \@@_int_if_slot_in_range:nnT
  { \@@_numrange_parse:nwT {#1} #2 - \q_nil - \q_stop {#3} }
%    \end{macrocode}
%
%    \begin{macrocode}
\cs_set:Npn \@@_numrange_parse:nwT #1 #2 - #3 - #4 \q_stop #5
  {
    \tl_if_empty:nTF {#4} { \int_compare:nT {#1=#2} {#5} }
      {
    \tl_if_empty:nTF {#3} { \int_compare:nT {#1>=#2} {#5} }
      {
    \tl_if_empty:nTF {#2} { \int_compare:nT {#1<=#3} {#5} }
      {
    \int_compare:nT {#1>=#2} { \int_compare:nT {#1<=#3} {#5} }
      } } }
  }
%    \end{macrocode}
% \end{macro}
%
%
% \subsection{Resolving Greek symbol name control sequences}
%
% \begin{macro}{\@@_resolve_greek:}
% This macro defines \cmd\Alpha\dots\cmd\omega\ as their corresponding
% Unicode (mathematical italic) character. Remember that the mapping
% to upright or italic happens with the mathcode definitions, whereas these macros
% just stand for the literal Unicode characters.
%    \begin{macrocode}
\AtBeginDocument{\@@_resolve_greek:}
\cs_new:Npn \@@_resolve_greek:
 {
  \clist_map_inline:nn
   {
    Alpha,Beta,Gamma,Delta,Epsilon,Zeta,Eta,Theta,Iota,Kappa,Lambda,
    alpha,beta,gamma,delta,epsilon,zeta,eta,theta,iota,kappa,lambda,
    Mu,Nu,Xi,Omicron,Pi,Rho,Sigma,Tau,Upsilon,Phi,Chi,Psi,Omega,
    mu,nu,xi,omicron,pi,rho,sigma,tau,upsilon,phi,chi,psi,omega,
    varTheta,varsigma,vartheta,varkappa,varrho,varpi,varepsilon,varphi
   }
   {
    \tl_set:cx {##1} { \exp_not:c { mit ##1 } }
    \tl_set:cx {up ##1} { \exp_not:N \symup \exp_not:c { ##1 } }
    \tl_set:cx {it ##1} { \exp_not:N \symit \exp_not:c { ##1 } }
   }
 }
%    \end{macrocode}
% \end{macro}
%
%
%
%
%
%
%
% \section{Maths alphabets}
% \label{part:mathmap}
%
% Defining commands like \cmd\mathrm\ is not as simple with Unicode fonts.
% In traditional \TeX{} maths font setups, you simply switch between different `families' (\cmd\fam), which is analogous to changing from one font to another---a symbol such as `a' will be upright in one font, bold in another, and so on.
%
% In pkg{unicode-math}, a different mechanism is used to switch between styles. For every letter (start with ascii a-zA-Z and numbers to keep things simple for now), they are assigned a `mathcode' with \cmd\Umathcode\ that maps from input letter to output font glyph slot. This is done with the equivalent of
% \begin{Verbatim}
% \Umathcode`\a = 7 1 "1D44E\relax
% \Umathcode`\b = 7 1 "1D44F\relax
% \Umathcode`\c = 7 1 "1D450\relax
% ...
% \end{Verbatim}
% When switching from regular letters to, say, \cmd\mathrm, we now need to execute a new mapping:
% \begin{Verbatim}
% \Umathcode`\a = 7 1 `\a\relax
% \Umathcode`\b = 7 1 `\b\relax
% \Umathcode`\c = 7 1 `\c\relax
% ...
% \end{Verbatim}
% This is fairly straightforward to perform when we're defining our own commands such as \cmd\symbf\ and so on. However, this means that `classical' \TeX\ font setups will break, because with the original mapping still in place, the engine will be attempting to insert unicode maths glyphs from a standard font.
%
% \subsection{Hooks into \LaTeXe}
%
% To overcome this, we patch \cs{use@mathgroup}.
% (An alternative is to patch \cs{extract@alph@from@version}, which constructs the \cs{mathXYZ} commands, but this method fails if the command has been defined using \cs{DeclareSymbolFontAlphabet}.)
% As far as I can tell, this is only used inside of commands such as \cs{mathXYZ}, so this shouldn't have any major side-effects.
%
%    \begin{macrocode}
\cs_set:Npn \use@mathgroup #1 #2
 {
  \mode_if_math:T % <- not sure if this is really necessary since we've just checked for mmode and raised an error if not!
   {
    \math@bgroup
      \cs_if_eq:cNF {M@\f@encoding} #1 {#1}
      \@@_switchto_literal:
      \mathgroup #2 \relax
    \math@egroup
   }
 }
%    \end{macrocode}
%
%
%
% \subsection{Setting styles}
%
% Algorithm for setting alphabet fonts.
% By default, when |range| is empty, we are in \emph{implicit} mode.
% If |range| contains the name of the math alphabet, we are in \emph{explicit}
% mode and do things slightly differently.
%
% Implicit mode:
% \begin{itemize}
% \item Try and set all of the alphabet shapes.
% \item Check for the first glyph of each alphabet to detect if the font supports each
%       alphabet shape.
% \item For alphabets that do exist, overwrite whatever's already there.
% \item For alphabets that are not supported, \emph{do nothing}.
%       (This includes leaving the old alphabet definition in place.)
% \end{itemize}
%
% Explicit mode:
% \begin{itemize}
% \item Only set the alphabets specified.
% \item Check for the first glyph of the alphabet to detect if the font contains
%       the alphabet shape in the Unicode math plane.
% \item For Unicode math alphabets, overwrite whatever's already there.
% \item Otherwise, use the \ascii\ glyph slots instead.
% \end{itemize}
%
%
%
% \subsection{Defining the math style macros}
%
% We call the different shapes that a math alphabet can be a `math style'.
% Note that different alphabets can exist within the same math style. E.g.,
% we call `bold' the math style |bf| and within it there are upper and lower
% case Greek and Roman alphabets and Arabic numerals.
%
% \begin{macro}{\@@_prepare_mathstyle:n}
% \darg{math style name (e.g., \texttt{it} or \texttt{bb})}
% Define the high level math alphabet macros (\cs{mathit}, etc.) in terms of
% unicode-math definitions. Use \cs{bgroup}/\cs{egroup} so s'scripts scan the
% whole thing.
%
% The flag \cs{l_@@_mathstyle_tl} is for other applications to query the
% current math style.
%    \begin{macrocode}
\cs_new:Nn \@@_prepare_mathstyle:n
 {
  \seq_put_right:Nn \g_@@_mathstyles_seq {#1}
  \@@_init_alphabet:n {#1}
  \cs_set:cpn {_@@_sym_#1_aux:n}
   { \use:c {@@_switchto_#1:} \math@egroup }
  \cs_set_protected:cpx {sym#1}
   {
    \exp_not:n
     {
      \math@bgroup
      \mode_if_math:F
        {
          \egroup\expandafter
          \non@alpherr\expandafter{\csname sym#1\endcsname\space}
        }
      \tl_set:Nn \l_@@_mathstyle_tl {#1}
     }
    \exp_not:c {_@@_sym_#1_aux:n}
   }
 }
%    \end{macrocode}
% \end{macro}
%
%
% \begin{macro}{\@@_init_alphabet:n}
% \darg{math alphabet name (e.g., \texttt{it} or \texttt{bb})}
% This macro initialises the macros used to set up a math alphabet.
% First used when the math alphabet macro is first defined, but then used
% later when redefining a particular maths alphabet.
%    \begin{macrocode}
\cs_set:Nn \@@_init_alphabet:n
 {
  \@@_log:nx {alph-initialise} {#1}
  \cs_set_eq:cN {@@_switchto_#1:} \prg_do_nothing:
 }
%    \end{macrocode}
% \end{macro}
%
% \subsection{Definition of alphabets and styles}
%
% First of all, we break up unicode into `named ranges', such as |up|, |bb|, |sfup|, and so on, which refer to specific blocks of unicode that contain various symbols (usually alphabetical symbols).
%
%    \begin{macrocode}
\cs_new:Nn \@@_new_named_range:n
 {
  \prop_new:c {g_@@_named_range_#1_prop}
 }
\clist_set:Nn \g_@@_named_ranges_clist
 {
  up, it, tt, bfup, bfit, bb , bbit, scr, bfscr, cal, bfcal,
  frak, bffrak, sfup, sfit, bfsfup, bfsfit, bfsf
 }
\clist_map_inline:Nn \g_@@_named_ranges_clist
 { \@@_new_named_range:n {#1} }
%    \end{macrocode}
%
% Each of these styles usually contains one or more `alphabets', which are currently |latin|, |Latin|, |greek|, |Greek|, |num|, and |misc|, although there's an implicit potential for more.
% |misc| is not included in the official list to avoid checking code.
%    \begin{macrocode}
\clist_new:N  \g_@@_alphabets_seq
\clist_set:Nn \g_@@_alphabets_seq { latin, Latin, greek, Greek, num }
%    \end{macrocode}
%
% Each alphabet style needs to be configured.
% This happens in the |unicode-math-alphabets.dtx| file.
%    \begin{macrocode}
\cs_new:Nn \@@_new_alphabet_config:nnn
 {
  \prop_if_exist:cF {g_@@_named_range_#1_prop}
   { \@@_warning:nnn {no-named-range} {#1} {#2} }

  \prop_gput:cnn {g_@@_named_range_#1_prop} { alpha_tl }
    {
     \prop_item:cn {g_@@_named_range_#1_prop} { alpha_tl }
     {#2}
    }
  % Q: do I need to bother removing duplicates?

  \cs_new:cn { @@_config_#1_#2:n } {#3}
 }
%    \end{macrocode}
%    \begin{macrocode}
\cs_new:Nn \@@_alphabet_config:nnn
 {
  \use:c {@@_config_#1_#2:n} {#3}
 }
%    \end{macrocode}
%    \begin{macrocode}
\prg_new_conditional:Nnn \@@_if_alphabet_exists:nn {T,TF}
 {
  \cs_if_exist:cTF {@@_config_#1_#2:n}
   \prg_return_true: \prg_return_false:
 }
%    \end{macrocode}
%
% The linking between named ranges and symbol style commands happens here.
% It's currently not using all of the machinery we're in the process of setting up above.
% Baby steps.
%    \begin{macrocode}
\cs_new:Nn \@@_default_mathalph:nnn
 {
  \seq_put_right:Nx \g_@@_named_ranges_seq { \tl_to_str:n {#1} }
  \seq_put_right:Nn \g_@@_default_mathalph_seq {{#1}{#2}{#3}}
  \prop_gput:cnn { g_@@_named_range_#1_prop } { default-alpha } {#2}
 }
\@@_default_mathalph:nnn {up    } {latin,Latin,greek,Greek,num,misc} {up    }
\@@_default_mathalph:nnn {it    } {latin,Latin,greek,Greek,misc}     {it    }
\@@_default_mathalph:nnn {bb    } {latin,Latin,num,misc}             {bb    }
\@@_default_mathalph:nnn {bbit  } {misc}                             {bbit  }
\@@_default_mathalph:nnn {scr   } {latin,Latin}                      {scr   }
\@@_default_mathalph:nnn {cal   } {Latin}                            {scr   }
\@@_default_mathalph:nnn {bfcal } {Latin}                            {bfscr }
\@@_default_mathalph:nnn {frak  } {latin,Latin}                      {frak  }
\@@_default_mathalph:nnn {tt    } {latin,Latin,num}                  {tt    }
\@@_default_mathalph:nnn {sfup  } {latin,Latin,num}                  {sfup  }
\@@_default_mathalph:nnn {sfit  } {latin,Latin}                      {sfit  }
\@@_default_mathalph:nnn {bfup  } {latin,Latin,greek,Greek,num,misc} {bfup  }
\@@_default_mathalph:nnn {bfit  } {latin,Latin,greek,Greek,misc}     {bfit  }
\@@_default_mathalph:nnn {bfscr } {latin,Latin}                      {bfscr }
\@@_default_mathalph:nnn {bffrak} {latin,Latin}                      {bffrak}
\@@_default_mathalph:nnn {bfsfup} {latin,Latin,greek,Greek,num,misc} {bfsfup}
\@@_default_mathalph:nnn {bfsfit} {latin,Latin,greek,Greek,misc}     {bfsfit}
%    \end{macrocode}
%
% \subsubsection{Define symbol style commands}
% Finally, all of the `symbol styles' commands are set up, which are the commands to access each of the named alphabet styles. There is not a one-to-one mapping between symbol style commands and named style ranges!
%    \begin{macrocode}
\clist_map_inline:nn
 {
  up, it, bfup, bfit, sfup, sfit, bfsfup, bfsfit, bfsf,
  tt, bb, bbit, scr, bfscr, cal, bfcal, frak, bffrak,
  normal, literal, sf, bf,
 }
 { \@@_prepare_mathstyle:n {#1} }
%    \end{macrocode}
%
%
% \subsubsection{New names for legacy textmath alphabet selection}
% In case a package option overwrites, say, \cs{mathbf} with \cs{symbf}.
%    \begin{macrocode}
\clist_map_inline:nn
 { rm, it, bf, sf, tt }
 { \cs_set_eq:cc { mathtext #1 } { math #1 } }
%    \end{macrocode}
% Perhaps these should actually be defined using a hypothetical unicode-math interface to creating new such styles. To come.
%
%
% \subsubsection{Replacing legacy pure-maths alphabets}
% The following are alphabets which do not have a math/text ambiguity.
%    \begin{macrocode}
\clist_map_inline:nn
 {
   normal, bb , bbit, scr, bfscr, cal, bfcal, frak, bffrak, tt,
   bfup, bfit, sfup, sfit, bfsfup, bfsfit, bfsf
 }
 {
  \cs_set:cpx { math #1 } { \exp_not:c { sym #1 } }
 }
%    \end{macrocode}
%
%
% \subsubsection{New commands for ambiguous alphabets}
%    \begin{macrocode}
\AtBeginDocument{
\clist_map_inline:nn
 { rm, it, bf, sf, tt }
 {
  \cs_set_protected:cpx { math #1 }
   {
    \exp_not:n { \bool_if:NTF  } \exp_not:c { g_@@_ math #1 _text_bool}
     { \exp_not:c { mathtext #1 } }
     { \exp_not:c { sym #1 } }
   }
 }}
%    \end{macrocode}
%
% \paragraph{Alias \cs{mathrm} as legacy name for \cs{mathup}}
%    \begin{macrocode}
\cs_set_protected:Npn \mathup { \mathrm }
\cs_set_protected:Npn \symrm  { \symup  }
%    \end{macrocode}
%
%
% \subsubsection{Fixing up \cs{operator@font}}
%
%In LaTeX maths, the command |\operator@font| is defined that switches to the |operator| mathgroup. The classic example is the |\sin| in |$\sin{x}$|; essentially we're using |\mathrm| to typeset the upright symbols, but the syntax is |{\operator@font sin}|.
%
%It turns out that hooking into |\operator@font| is hard because all other maths font selection in 2e uses |\mathrm{...}| style.
%
%Then reading source2e a little more I stumbled upon: (in the definition of |\select@group|)
%\begin{quote}
% We surround |\select@group| with braces so that functions using it can be used directly after |_| or |^|. However, if we use oldstyle syntax where the math alphabet doesn’t have arguments (ie if |\math@bgroup| is not |\bgroup|) we need to get rid of the extra group.
%\end{quote}
%So there's a trick we can use.
%Because it's late and I'm tired, I went for the first thing that jumped out at me:
%\begin{Verbatim}
%    \documentclass{article}
%    \DeclareMathAlphabet\mathfoo{OT1}{lmdh}{m}{n}
%    \begin{document}
%    \makeatletter
%    ${\operator@font Mod}\, x$
%
%    \def\operator@font{%
%      \let \math@bgroup \relax
%      \def \math@egroup {\let \math@bgroup \@@math@bgroup
%                         \let \math@egroup \@@math@egroup}%
%      \mathfoo}
%    ${\operator@font Mod}\, x$
%    \end{document}
%\end{Verbatim}
% We define a new math alphabet |\mathfoo| to select the Latin Modern Dunhill font, and then locally redefine |\math@bgroup| to allow |\mathfoo| to be used without an argument temporarily.
%
% Now that I've written this whole thing out, another solution pops to mind:
%\begin{Verbatim}
%    \documentclass{article}
%    \DeclareSymbolFont{foo}{OT1}{lmdh}{m}{n}
%    \DeclareSymbolFontAlphabet\mathfoo{foo}
%    \begin{document}
%    \makeatletter
%    ${\operator@font Mod}\, x$
%
%    \def\operator@font{\mathgroup\symfoo}
%    ${\operator@font Mod}\, x$
%    \end{document}
%\end{Verbatim}
%I guess that's the better approach!!
%
% Or perhaps I should just use |\@fontswitch| to do the first solution with a nicer wrapper. I really should read things more carefully:
% \begin{macro}{\operator@font}
%    \begin{macrocode}
\cs_set:Npn \operator@font
 {
  \@@_switchto_literal:
  \@fontswitch {} { \g_@@_operator_mathfont_tl }
 }
%    \end{macrocode}
% \end{macro}
%
%
% \subsection{Defining the math alphabets per style}
%
% \begin{macro}{\@@_setup_alphabets:}
% This function is called within \cs{setmathfont} to configure the
% mapping between characters inside math styles.
%    \begin{macrocode}
\cs_new:Npn \@@_setup_alphabets:
 {
%    \end{macrocode}
% If |range=| has been used to configure styles, those choices will be in
% |\l_@@_mathalph_seq|. If not, set up the styles implicitly:
%    \begin{macrocode}
  \seq_if_empty:NTF \l_@@_mathalph_seq
   {
    \@@_log:n {setup-implicit}
    \seq_set_eq:NN \l_@@_mathalph_seq \g_@@_default_mathalph_seq
    \bool_set_true:N \l_@@_implicit_alph_bool
    \@@_maybe_init_alphabet:n  {sf}
    \@@_maybe_init_alphabet:n  {bf}
    \@@_maybe_init_alphabet:n  {bfsf}
   }
%    \end{macrocode}
% If |range=| has been used then we're in explicit mode:
%    \begin{macrocode}
   {
    \@@_log:n {setup-explicit}
    \bool_set_false:N \l_@@_implicit_alph_bool
    \cs_set_eq:NN \@@_set_mathalphabet_char:nnn \@@_mathmap_noparse:nnn
    \cs_set_eq:NN \@@_map_char_single:nn \@@_map_char_noparse:nn
   }

  % Now perform the mapping:
  \seq_map_inline:Nn \l_@@_mathalph_seq
   {
    \tl_set:No    \l_@@_style_tl       { \use_i:nnn   ##1 }
    \clist_set:No \l_@@_alphabet_clist { \use_ii:nnn  ##1 }
    \tl_set:No    \l_@@_remap_style_tl { \use_iii:nnn ##1 }

    % If no set of alphabets is defined:
    \clist_if_empty:NT \l_@@_alphabet_clist
     {
      \cs_set_eq:NN \@@_maybe_init_alphabet:n \@@_init_alphabet:n
      \prop_get:cnN { g_@@_named_range_ \l_@@_style_tl _prop }
       { default-alpha } \l_@@_alphabet_clist
     }

    \@@_setup_math_alphabet:
   }
  \seq_if_empty:NF \l_@@_missing_alph_seq { \@@_log:n { missing-alphabets } }
 }
%    \end{macrocode}
% \end{macro}
%
% \begin{macro}{\@@_setup_math_alphabet:}
%    \begin{macrocode}
\cs_new:Nn \@@_setup_math_alphabet:
 {
%    \end{macrocode}
% First check that at least one of the alphabets for the font shape is defined
% (this process is fast) \dots
%    \begin{macrocode}
  \clist_map_inline:Nn \l_@@_alphabet_clist
   {
    \tl_set:Nn \l_@@_alphabet_tl {##1}
    \@@_if_alphabet_exists:nnTF \l_@@_style_tl \l_@@_alphabet_tl
     {
      \str_if_eq_x:nnTF {\l_@@_alphabet_tl} {misc}
       {
        \@@_maybe_init_alphabet:n \l_@@_style_tl
        \clist_map_break:
       }
       {
        \@@_glyph_if_exist:nT { \@@_to_usv:nn {\l_@@_style_tl} {\l_@@_alphabet_tl} }
         {
          \@@_maybe_init_alphabet:n \l_@@_style_tl
          \clist_map_break:
         }
       }
     }
     { \msg_warning:nnx {unicode-math} {no-alphabet} { \l_@@_style_tl / \l_@@_alphabet_tl } }
   }
%    \end{macrocode}
% \dots and then loop through them defining the individual ranges:
% (currently this process is slow)
%    \begin{macrocode}
%<debug>  \csname TIC\endcsname
  \clist_map_inline:Nn \l_@@_alphabet_clist
   {
    \tl_set:Nx \l_@@_alphabet_tl { \tl_trim_spaces:n {##1} }
    \cs_if_exist:cT {@@_config_ \l_@@_style_tl _ \l_@@_alphabet_tl :n}
     {
      \exp_args:No \tl_if_eq:nnTF \l_@@_alphabet_tl {misc}
       {
        \@@_log:nx {setup-alph} {sym \l_@@_style_tl~(\l_@@_alphabet_tl)}
        \@@_alphabet_config:nnn {\l_@@_style_tl} {\l_@@_alphabet_tl} {\l_@@_remap_style_tl}
       }
       {
        \@@_glyph_if_exist:nTF { \@@_to_usv:nn {\l_@@_remap_style_tl} {\l_@@_alphabet_tl} }
         {
          \@@_log:nx {setup-alph} {sym \l_@@_style_tl~(\l_@@_alphabet_tl)}
          \@@_alphabet_config:nnn {\l_@@_style_tl} {\l_@@_alphabet_tl} {\l_@@_remap_style_tl}
         }
         {
          \bool_if:NTF \l_@@_implicit_alph_bool
           {
            \seq_put_right:Nx \l_@@_missing_alph_seq
             {
              \@backslashchar sym \l_@@_style_tl \space
              (\tl_use:c{c_@@_math_alphabet_name_ \l_@@_alphabet_tl _tl})
             }
           }
           {
            \@@_alphabet_config:nnn {\l_@@_style_tl} {\l_@@_alphabet_tl} {up}
           }
         }
       }
     }
   }
%<debug>  \csname TOC\endcsname
 }
%    \end{macrocode}
% \end{macro}
%
%
% \subsection{Mapping `naked' math characters}
%
% Before we show the definitions of the alphabet mappings using the functions
% |\@@_alphabet_config:nnn \l_@@_style_tl {##1} {...}|, we first want to define some functions
% to be used inside them to actually perform the character mapping.
%
% \subsubsection{Functions}
%
% \begin{macro}{\@@_map_char_single:nn}
% Wrapper for |\@@_map_char_noparse:nn| or |\@@_map_char_parse:nn|
% depending on the context.
%
% \begin{macro}{\@@_map_char_noparse:nn}
% \begin{macro}{\@@_map_char_parse:nn}
%    \begin{macrocode}
\cs_new:Nn \@@_map_char_noparse:nn
 { \@@_set_mathcode:nnnn {#1}{\mathalpha}{\@@_symfont_tl}{#2} }
%    \end{macrocode}
%
%    \begin{macrocode}
\cs_new:Nn \@@_map_char_parse:nn
 {
  \@@_if_char_spec:nNNT {#1} {\@nil} {\mathalpha}
   { \@@_map_char_noparse:nn {#1}{#2} }
 }
%    \end{macrocode}
% \end{macro}
% \end{macro}
% \end{macro}
%
% \begin{macro}{\@@_map_char_single:nnn}
% \darg{char name (`dotlessi')}
% \darg{from alphabet(s)}
% \darg{to alphabet}
% Logical interface to \cs{@@_map_char_single:nn}.
%    \begin{macrocode}
\cs_new:Nn \@@_map_char_single:nnn
 {
  \@@_map_char_single:nn { \@@_to_usv:nn {#1}{#3} }
                         { \@@_to_usv:nn {#2}{#3} }
 }
%    \end{macrocode}
% \end{macro}
%
%
% \begin{macro}{\@@_map_chars_range:nnnn}
% \darg{Number of chars (26)}
% \darg{From style, one or more (it)}
% \darg{To style (up)}
% \darg{Alphabet name (Latin)}
% First the function with numbers:
%    \begin{macrocode}
\cs_set:Nn \@@_map_chars_range:nnn
 {
  \int_step_inline:nnnn {0}{1}{#1-1}
   { \@@_map_char_single:nn {#2+##1}{#3+##1} }
 }
%    \end{macrocode}
% And the wrapper with names:
%    \begin{macrocode}
\cs_new:Nn \@@_map_chars_range:nnnn
 {
  \@@_map_chars_range:nnn {#1} { \@@_to_usv:nn {#2}{#4} }
                               { \@@_to_usv:nn {#3}{#4} }
 }
%    \end{macrocode}
% \end{macro}
%
% \subsubsection{Functions for `normal' alphabet symbols}
%
% \begin{macro}{\@@_set_normal_char:nnn}
%    \begin{macrocode}
\cs_set:Nn \@@_set_normal_char:nnn
 {
  \@@_usv_if_exist:nnT {#3} {#1}
  {
    \clist_map_inline:nn {#2}
     {
      \@@_set_mathalphabet_pos:nnnn {normal} {#1} {##1} {#3}
      \@@_map_char_single:nnn {##1} {#3} {#1}
     }
  }
 }
%    \end{macrocode}
% \end{macro}
%
%    \begin{macrocode}
\cs_new:Nn \@@_set_normal_Latin:nn
 {
  \clist_map_inline:nn {#1}
   {
    \@@_set_mathalphabet_Latin:nnn {normal} {##1} {#2}
    \@@_map_chars_range:nnnn {26} {##1} {#2} {Latin}
   }
 }
%    \end{macrocode}
%
%    \begin{macrocode}
\cs_new:Nn \@@_set_normal_latin:nn
 {
  \clist_map_inline:nn {#1}
   {
    \@@_set_mathalphabet_latin:nnn {normal} {##1} {#2}
    \@@_map_chars_range:nnnn {26} {##1} {#2} {latin}
   }
 }
%    \end{macrocode}
%
%    \begin{macrocode}
\cs_new:Nn \@@_set_normal_greek:nn
 {
  \clist_map_inline:nn {#1}
   {
    \@@_set_mathalphabet_greek:nnn {normal} {##1} {#2}
    \@@_map_chars_range:nnnn {25} {##1} {#2} {greek}
    \@@_map_char_single:nnn {##1} {#2} {epsilon}
    \@@_map_char_single:nnn {##1} {#2} {vartheta}
    \@@_map_char_single:nnn {##1} {#2} {varkappa}
    \@@_map_char_single:nnn {##1} {#2} {phi}
    \@@_map_char_single:nnn {##1} {#2} {varrho}
    \@@_map_char_single:nnn {##1} {#2} {varpi}
    \@@_set_mathalphabet_pos:nnnn {normal} {epsilon} {##1} {#2}
    \@@_set_mathalphabet_pos:nnnn {normal} {vartheta} {##1} {#2}
    \@@_set_mathalphabet_pos:nnnn {normal} {varkappa} {##1} {#2}
    \@@_set_mathalphabet_pos:nnnn {normal} {phi} {##1} {#2}
    \@@_set_mathalphabet_pos:nnnn {normal} {varrho} {##1} {#2}
    \@@_set_mathalphabet_pos:nnnn {normal} {varpi} {##1} {#2}
   }
 }
%    \end{macrocode}
%
%    \begin{macrocode}
\cs_new:Nn \@@_set_normal_Greek:nn
 {
  \clist_map_inline:nn {#1}
   {
    \@@_set_mathalphabet_Greek:nnn {normal} {##1} {#2}
    \@@_map_chars_range:nnnn {25} {##1} {#2} {Greek}
    \@@_map_char_single:nnn {##1} {#2} {varTheta}
    \@@_set_mathalphabet_pos:nnnn {normal} {varTheta} {##1} {#2}
   }
 }
%    \end{macrocode}
%
%    \begin{macrocode}
\cs_new:Nn \@@_set_normal_numbers:nn
 {
  \@@_set_mathalphabet_numbers:nnn {normal} {#1} {#2}
  \@@_map_chars_range:nnnn {10} {#1} {#2} {num}
 }
%    \end{macrocode}
%
%
% \subsection{Mapping chars inside a math style}
%
% \subsubsection{Functions for setting up the maths alphabets}
%
% \begin{macro}{\@@_set_mathalphabet_char:Nnn}
% This is a wrapper for either |\@@_mathmap_noparse:nnn| or
% |\@@_mathmap_parse:Nnn|, depending on the context.
% \end{macro}
%
% \begin{macro}{\@@_mathmap_noparse:nnn}
% \darg{Maths alphabet, \eg, `bb'}
% \darg{Input slot(s), \eg, the slot for `A' (comma separated)}
% \darg{Output slot, \eg, the slot for `$\mathbb{A}$'}
% Adds \cs{@@_set_mathcode:nnnn} declarations to the specified maths alphabet's definition.
%    \begin{macrocode}
\cs_new:Nn \@@_mathmap_noparse:nnn
 {
  \clist_map_inline:nn {#2}
   {
    \tl_put_right:cx {@@_switchto_#1:}
     {
      \@@_set_mathcode:nnnn {##1} {\mathalpha} {\@@_symfont_tl} {#3}
     }
   }
 }
%    \end{macrocode}
% \end{macro}
%
% \begin{macro}{\@@_mathmap_parse:nnn}
% \darg{Maths alphabet, \eg, `bb'}
% \darg{Input slot(s), \eg, the slot for `A' (comma separated)}
% \darg{Output slot, \eg, the slot for `$\mathbb{A}$'}
% When \cmd\@@_if_char_spec:nNNT\ is executed, it populates the \cmd\l_@@_char_nrange_clist\
% macro with slot numbers corresponding to the specified range. This range is used to
% conditionally add \cs{@@_set_mathcode:nnnn} declaractions to the maths alphabet definition.
%    \begin{macrocode}
\cs_new:Nn \@@_mathmap_parse:nnn
 {
  \clist_if_in:NnT \l_@@_char_nrange_clist {#3}
   {
    \@@_mathmap_noparse:nnn {#1}{#2}{#3}
   }
 }
%    \end{macrocode}
% \end{macro}
%
% \begin{macro}{\@@_set_mathalphabet_char:nnnn}
% \darg{math style command}
% \darg{input math alphabet name}
% \darg{output math alphabet name}
% \darg{char name to map}
%    \begin{macrocode}
\cs_new:Nn \@@_set_mathalphabet_char:nnnn
 {
  \@@_set_mathalphabet_char:nnn {#1} { \@@_to_usv:nn {#2} {#4} }
                                     { \@@_to_usv:nn {#3} {#4} }
 }
%    \end{macrocode}
% \end{macro}
%
% \begin{macro}{\@@_set_mathalph_range:nnnn}
% \darg{Number of iterations}
% \darg{Maths alphabet}
% \darg{Starting input char (single)}
% \darg{Starting output char}
% Loops through character ranges setting \cmd\mathcode.
% First the version that uses numbers:
%    \begin{macrocode}
\cs_new:Nn \@@_set_mathalph_range:nnnn
 {
  \int_step_inline:nnnn {0} {1} {#1-1}
    { \@@_set_mathalphabet_char:nnn {#2} { ##1 + #3 } { ##1 + #4 } }
 }
%    \end{macrocode}
% Then the wrapper version that uses names:
%    \begin{macrocode}
\cs_new:Nn \@@_set_mathalph_range:nnnnn
 {
  \@@_set_mathalph_range:nnnn {#1} {#2} { \@@_to_usv:nn {#3} {#5} }
                                        { \@@_to_usv:nn {#4} {#5} }
 }
%    \end{macrocode}
% \end{macro}
%
% \subsubsection{Individual mapping functions for different alphabets}
%
%    \begin{macrocode}
\cs_new:Nn \@@_set_mathalphabet_pos:nnnn
 {
  \@@_usv_if_exist:nnT {#4} {#2}
   {
    \clist_map_inline:nn {#3}
      { \@@_set_mathalphabet_char:nnnn {#1} {##1} {#4} {#2} }
   }
 }
%    \end{macrocode}
%
%    \begin{macrocode}
\cs_new:Nn \@@_set_mathalphabet_numbers:nnn
 {
  \clist_map_inline:nn {#2}
    { \@@_set_mathalph_range:nnnnn {10} {#1} {##1} {#3} {num} }
 }
%    \end{macrocode}
%
%    \begin{macrocode}
\cs_new:Nn \@@_set_mathalphabet_Latin:nnn
 {
  \clist_map_inline:nn {#2}
    { \@@_set_mathalph_range:nnnnn {26} {#1} {##1} {#3} {Latin} }
 }
%    \end{macrocode}
%
%    \begin{macrocode}
\cs_new:Nn \@@_set_mathalphabet_latin:nnn
 {
  \clist_map_inline:nn {#2}
   {
    \@@_set_mathalph_range:nnnnn {26} {#1} {##1} {#3} {latin}
    \@@_set_mathalphabet_char:nnnn    {#1} {##1} {#3} {h}
   }
 }
%    \end{macrocode}
%
%    \begin{macrocode}
\cs_new:Nn \@@_set_mathalphabet_Greek:nnn
 {
  \clist_map_inline:nn {#2}
   {
    \@@_set_mathalph_range:nnnnn {25} {#1} {##1} {#3} {Greek}
    \@@_set_mathalphabet_char:nnnn    {#1} {##1} {#3} {varTheta}
   }
 }
%    \end{macrocode}
%
%    \begin{macrocode}
\cs_new:Nn \@@_set_mathalphabet_greek:nnn
 {
  \clist_map_inline:nn {#2}
   {
    \@@_set_mathalph_range:nnnnn {25} {#1} {##1} {#3} {greek}
    \@@_set_mathalphabet_char:nnnn    {#1} {##1} {#3} {epsilon}
    \@@_set_mathalphabet_char:nnnn    {#1} {##1} {#3} {vartheta}
    \@@_set_mathalphabet_char:nnnn    {#1} {##1} {#3} {varkappa}
    \@@_set_mathalphabet_char:nnnn    {#1} {##1} {#3} {phi}
    \@@_set_mathalphabet_char:nnnn    {#1} {##1} {#3} {varrho}
    \@@_set_mathalphabet_char:nnnn    {#1} {##1} {#3} {varpi}
   }
 }
%    \end{macrocode}
%
%
%
% \section{A token list to contain the data of the math table}
%
% Instead of \cmd\input-ing the unicode math table every time we
% want to re-read its data, we save it within a macro. This has two
% advantages: 1.~it should be slightly faster, at the expense of memory;
% 2.~we don't need to worry about catcodes later, since they're frozen
% at this point.
%
% In time, the case statement inside |set_mathsymbol| will be moved in here
% to avoid re-running it every time.
%    \begin{macrocode}
\cs_new:Npn \@@_symbol_setup:
 {
  \cs_set:Npn \UnicodeMathSymbol ##1##2##3##4
   {
    \exp_not:n { \_@@_sym:nnn {##1} {##2} {##3} }
   }
 }
%    \end{macrocode}
%
%    \begin{macrocode}
\tl_set_from_file_x:Nnn \g_@@_mathtable_tl {\@@_symbol_setup:} {unicode-math-table.tex}
%    \end{macrocode}
%
%
% \begin{macro}{\@@_input_math_symbol_table:}
% This function simply expands to the token list containing all the data.
%    \begin{macrocode}
\cs_new:Nn \@@_input_math_symbol_table: {\g_@@_mathtable_tl}
%    \end{macrocode}
% \end{macro}
%
%
% \section{Definitions of the active math characters}
%
% Now give \cmd\_@@_sym:nnn\ a definition in terms of \cmd\@@_cs_set_eq_active_char:Nw\
% and we're good to go.
%
% Ensure catcodes are appropriate;
% make sure |#| is an `other' so that we don't get confused with \cs{mathoctothorpe}.
%    \begin{macrocode}
\AtBeginDocument{\@@_define_math_chars:}
\cs_new:Nn \@@_define_math_chars:
 {
  \group_begin:
    \cs_set:Npn \_@@_sym:nnn ##1##2##3
     {
      \tl_if_in:nnT
       { \mathord \mathalpha \mathbin \mathrel \mathpunct \mathop \mathfence }
       {##3}
      {
        \exp_last_unbraced:NNx \cs_gset_eq:NN ##2 { \Ucharcat ##1 ~ 12 ~ }
      }
     }
    \@@_input_math_symbol_table:
  \group_end:
 }
%    \end{macrocode}
%
%
%    \begin{macrocode}
%</package&(XE|LU)>
%    \end{macrocode}
%
\endinput
