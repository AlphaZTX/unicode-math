%%^^A%% um-code-mathmap.dtx -- part of UNICODE-MATH <wspr.io/unicode-math>
%%^^A%% Setup of symbol alphabets.

% \section{Defining the math alphabets per style}
% \label{sec:mathmap}
%
%    \begin{macrocode}
%<*package>
%    \end{macrocode}
%
% \begin{macro}{\@@_setup_alphabets:}
% This function is called within \cs{setmathfont} to configure the
% mapping between characters inside math styles. Three modes:
% \begin{decription}
% \item[IMPLICIT] No ranges specified, set up everything
% \item[EXPLICIT] Some ranges specified, set up what is requested only
% \item[INHERIT]  Of the slots in the ranges specified, compare against
%                 slots in each styled alphabet and only set up those needed
% \end{description}
% The INHERIT mode saves less time than I was hoping for but is still beneficial
% in simple cases.
%    \begin{macrocode}
\cs_new:Npn \@@_setup_alphabets:
  {

    \bool_if:NTF \l_@@_init_bool
      {
        % IMPLICIT
        \@@_log:n {setup-implicit}
        \seq_gset_eq:NN \g_@@_mathalph_seq \g_@@_default_mathalph_seq
        \bool_set_true:N \l_@@_implicit_alph_bool
        \@@_maybe_init_alphabet:n  {sf}
        \@@_maybe_init_alphabet:n  {bf}
        \@@_maybe_init_alphabet:n  {bfsf}
        \cs_set_eq:NN \@@_set_mathalphabet_char:nnn \@@_mathmap_noparse:nnn
        \cs_set_eq:NN \@@_map_char_single:nn \@@_map_char_noparse:nn
        \@@_mathalph_map:
        \seq_if_empty:NF \l_@@_missing_alph_seq { \@@_log:n { missing-alphabets } }
      }
      {
        \seq_if_empty:NF \g_@@_mathalph_seq
          {
            % EXPLICIT
            \@@_log:n {setup-explicit}
            \bool_set_false:N \l_@@_implicit_alph_bool
            \cs_set_eq:NN \@@_set_mathalphabet_char:nnn \@@_mathmap_noparse:nnn
            \cs_set_eq:NN \@@_map_char_single:nn \@@_map_char_noparse:nn
            \@@_mathalph_map:
            \seq_if_empty:NF \l_@@_missing_alph_seq { \@@_log:n { missing-alphabets } }
          }

        % INHERIT
        \clist_if_empty:NF \l_@@_mathmap_charints_clist
          {
            \seq_gclear:N \g_@@_mathalph_seq
            \seq_map_inline:Nn \g_@@_default_mathalph_seq
              {
                \tl_set:No    \l_@@_style_tl       { \use_i:nnn   ##1 }
                \clist_set:No \l_@@_alphabet_clist { \use_ii:nnn  ##1 }

                \clist_map_inline:Nn \l_@@_alphabet_clist
                  {
                    \clist_map_inline:cn {g_@@_named_slots_ \l_@@_style_tl _ ####1 _clist}
                      {
                        \clist_map_inline:Nn \l_@@_mathmap_charints_clist
                          {
                            \@@_int_if_slot_in_range:nnT {################1} {########1}
                              {
                                \seq_gput_right:Nn \g_@@_mathalph_seq {##1}
                                \clist_map_break:n { \clist_map_break:n { \clist_map_break: } }
                              }
                          }
                      }
                  }
              }
          }

        \cs_set_eq:NN \@@_set_mathalphabet_char:nnn \@@_mathmap_parse:nnn
        \cs_set_eq:NN \@@_map_char_single:nn \@@_map_char_parse:nn
        \@@_mathalph_map:
      }
  }
%    \end{macrocode}
% \end{macro}
%
% \begin{macro}{\@@_mathalph_map:}
%    \begin{macrocode}
\cs_set:Nn \@@_mathalph_map:
  {
   \seq_map_inline:Nn \g_@@_mathalph_seq
      {
        \tl_set:No    \l_@@_style_tl       { \use_i:nnn   ##1 }
        \clist_set:No \l_@@_alphabet_clist { \use_ii:nnn  ##1 }
        \tl_set:No    \l_@@_remap_style_tl { \use_iii:nnn ##1 }

        % If no set of alphabets is defined:
        \clist_if_empty:NT \l_@@_alphabet_clist
          {
            \cs_set_eq:NN \@@_maybe_init_alphabet:n \@@_init_alphabet:n
            \prop_get:cnN { g_@@_named_range_ \l_@@_style_tl _prop }
              { default-alpha } \l_@@_alphabet_clist
          }

        \@@_check_math_alphabet:
        \@@_setup_math_alphabet:
      }
  }
%    \end{macrocode}
% \end{macro}
%
% \begin{macro}{\@@_check_math_alphabet:}
% First check that at least one of the alphabets for the font shape is defined
% (this process is fast) \dots
%    \begin{macrocode}
\cs_new:Nn \@@_check_math_alphabet:
  {
    \clist_map_inline:Nn \l_@@_alphabet_clist
      {
        \tl_set:Nn \l_@@_alphabet_tl {##1}
        \@@_if_alphabet_exists:nnTF \l_@@_style_tl \l_@@_alphabet_tl
          {
            \str_if_eq_x:nnTF {\l_@@_alphabet_tl} {misc}
              {
                \@@_maybe_init_alphabet:n \l_@@_style_tl
                \clist_map_break:
              }
              {
                \@@_glyph_if_exist:NnT \g_@@_curr_font_cmd_tl
                  { \@@_to_usv:nn {\l_@@_style_tl} {\l_@@_alphabet_tl} }
                  {
                    \@@_maybe_init_alphabet:n \l_@@_style_tl
                    \clist_map_break:
                  }
              }
          }
          {
            \msg_warning:nnx {unicode-math} {no-alphabet}
              { \l_@@_style_tl / \l_@@_alphabet_tl }
          }
      }
  }
%    \end{macrocode}
% \end{macro}
%
% \begin{macro}{\@@_setup_math_alphabet:}
% \dots and then loop through them defining the individual ranges:
% (currently this process is slow)
%    \begin{macrocode}
\cs_new:Nn \@@_setup_math_alphabet:
  {
    \clist_map_inline:Nn \l_@@_alphabet_clist
      {
        \tl_set:Nx \l_@@_alphabet_tl { \tl_trim_spaces:n {##1} }

%<debug>\typeout{_setup_math_alphabet:~\l_@@_style_tl/\l_@@_alphabet_tl}

        \@@_if_alphabet_exists:nnT {\l_@@_style_tl} {\l_@@_alphabet_tl}
          {
            \exp_args:No \tl_if_eq:nnTF \l_@@_alphabet_tl {misc}
              {
                \@@_log:nx {setup-alph} {sym \l_@@_style_tl~(\l_@@_alphabet_tl)}
                \@@_alphabet_config:nnn {\l_@@_style_tl} {\l_@@_alphabet_tl} {\l_@@_remap_style_tl}
              }
              {
                \@@_glyph_if_exist:NnTF \g_@@_curr_font_cmd_tl { \@@_to_usv:nn {\l_@@_remap_style_tl} {\l_@@_alphabet_tl} }
                  {
                    \@@_log:nx {setup-alph} {sym \l_@@_style_tl~(\l_@@_alphabet_tl)}
                    \@@_alphabet_config:nnn {\l_@@_style_tl} {\l_@@_alphabet_tl} {\l_@@_remap_style_tl}
                  }
                  {
                    \bool_if:NTF \l_@@_implicit_alph_bool
                      {
                        \seq_put_right:Nx \l_@@_missing_alph_seq
                          {
                            \@backslashchar sym \l_@@_style_tl \space
                            (\tl_use:c{c_@@_math_alphabet_name_ \l_@@_alphabet_tl _tl})
                          }
                      }
                      {
                        \@@_alphabet_config:nnn {\l_@@_style_tl} {\l_@@_alphabet_tl} {up}
                      }
                  }
              }
          }
      }
  }
%    \end{macrocode}
% \end{macro}
%
% Each alphabet style needs to be configured.
% This happens in Section~\ref{sec:setupalphabets}.
%    \begin{macrocode}
\cs_new:Nn \@@_new_alphabet_config:nnn
  {
    \prop_if_exist:cF {g_@@_named_range_#1_prop}
      { \@@_warning:nnn {no-named-range} {#1} {#2} }


    \prop_gput:cnn {g_@@_named_range_#1_prop} { alpha_tl }
      {
        \prop_item:cn {g_@@_named_range_#1_prop} { alpha_tl } {#2}
      }
    % Q: do I need to bother removing duplicates?

%    \end{macrocode}
% Create list of all chars defined in this named range:
%    \begin{macrocode}
    \cs_new:cn { @@_config_#1_#2:n }
      {
        \clist_gclear_new:c {g_@@_named_slots_#1_#2_clist}
        \tl_set:Nn \l_@@_curr_named_slot { g_@@_named_slots_#1_#2_clist }
        #3
        \clist_gremove_duplicates:c {g_@@_named_slots_#1_#2_clist}
      }

  }
%    \end{macrocode}
%    \begin{macrocode}
\cs_new:Nn \@@_alphabet_config:nnn
  {
    \use:c {@@_config_#1_#2:n} {#3}
  }
%    \end{macrocode}
%    \begin{macrocode}
\prg_new_conditional:Nnn \@@_if_alphabet_exists:nn {T,TF}
  {
    \cs_if_exist:cTF {@@_config_#1_#2:n}
      \prg_return_true: \prg_return_false:
  }
%    \end{macrocode}
%
% \subsection{Mapping `naked’ math characters}
%
% Before we show the definitions of the alphabet mappings using the functions
% |\@@_alphabet_config:nnn \l_@@_style_tl {##1} {...}|, we first want to define some functions
% to be used inside them to actually perform the character mapping.
%
% \subsubsection{Functions}
%
% \begin{macro}{\@@_map_char_single:nn}
% Wrapper for |\@@_map_char_noparse:nn| or |\@@_map_char_parse:nn|
% depending on the context.
%
% \begin{macro}{\@@_map_char_noparse:nn}
% \begin{macro}{\@@_map_char_parse:nn}
%    \begin{macrocode}
\cs_new:Nn \@@_map_char_noparse:nn
  {
    \@@_set_mathcode:nnnn {#1} {\mathalpha} {\l_@@_symfont_label_tl} {#2}
  }
%    \end{macrocode}
%
%    \begin{macrocode}
\cs_new:Nn \@@_map_char_parse:nn
  {
    \@@_if_char_spec:nNT {#1} {\mathalpha}
      { \@@_map_char_noparse:nn {#1}{#2} }
  }
%    \end{macrocode}
% \end{macro}
% \end{macro}
% \end{macro}
%
% \begin{macro}{\@@_map_char_single:nnn}
% \darg{char name (`dotlessi’)}
% \darg{from alphabet(s)}
% \darg{to alphabet}
% Logical interface to \cs{@@_map_char_single:nn}.
%    \begin{macrocode}
\cs_new:Nn \@@_map_char_single:nnn
  {
    \@@_map_char_single:nn { \@@_to_usv:nn {#1} {#3} }
                           { \@@_to_usv:nn {#2} {#3} }
  }
%    \end{macrocode}
% \end{macro}
%
%
% \begin{macro}{\@@_map_chars_range:nnnn}
% \darg{Number of chars (26)}
% \darg{From style, one or more (it)}
% \darg{To style (up)}
% \darg{Alphabet name (Latin)}
% First the function with numbers:
%    \begin{macrocode}
\cs_set:Nn \@@_map_chars_range:nnn
  {
    \int_step_inline:nnnn {0} {1} {#1-1}
      { \@@_map_char_single:nn {#2+##1} {#3+##1} }

    \clist_gput_right:cx { \l_@@_curr_named_slot }
      { \int_eval:n { #3 } - \int_eval:n { #3 + #1-1 } }
  }
%    \end{macrocode}
% And the wrapper with names:
%    \begin{macrocode}
\cs_new:Nn \@@_map_chars_range:nnnn
  {
    \@@_map_chars_range:nnn {#1} { \@@_to_usv:nn {#2} {#4} }
                                 { \@@_to_usv:nn {#3} {#4} }
  }
%    \end{macrocode}
% \end{macro}
%
% \subsubsection{Functions for `normal’ alphabet symbols}
%
% \begin{macro}{\@@_set_normal_char:nnn}
%    \begin{macrocode}
\cs_set:Nn \@@_set_normal_char:nnn
  {
    \@@_usv_if_exist:nnT {#3} {#1}
      {
        \clist_map_inline:nn {#2}
          {
            \@@_set_mathalphabet_pos:nnnn {normal} {#1} {##1} {#3}
            \@@_map_char_single:nnn {##1} {#3} {#1}

            \clist_gput_right:cx {\l_@@_curr_named_slot}
              { \int_eval:n { \@@_to_usv:nn {#3} {#1} } }
          }
      }
  }
%    \end{macrocode}
% \end{macro}
%
%    \begin{macrocode}
\cs_new:Nn \@@_set_normal_Latin:nn
  {
    \clist_map_inline:nn {#1}
      {
        \@@_set_mathalphabet_Latin:nnn {normal} {##1} {#2}
        \@@_map_chars_range:nnnn {26} {##1} {#2} {Latin}
      }
  }
%    \end{macrocode}
%
%    \begin{macrocode}
\cs_new:Nn \@@_set_normal_latin:nn
  {
    \clist_map_inline:nn {#1}
      {
        \@@_set_mathalphabet_latin:nnn {normal} {##1} {#2}
        \@@_map_chars_range:nnnn {26} {##1} {#2} {latin}
      }
  }
%    \end{macrocode}
%
%    \begin{macrocode}
\cs_new:Nn \@@_set_normal_greek:nn
  {
    \clist_map_inline:nn {#1}
      {
        \@@_set_mathalphabet_greek:nnn {normal} {##1} {#2}
        \@@_map_chars_range:nnnn {25} {##1} {#2} {greek}
        \@@_map_char_single:nnn {##1} {#2} {epsilon}
        \@@_map_char_single:nnn {##1} {#2} {vartheta}
        \@@_map_char_single:nnn {##1} {#2} {varkappa}
        \@@_map_char_single:nnn {##1} {#2} {phi}
        \@@_map_char_single:nnn {##1} {#2} {varrho}
        \@@_map_char_single:nnn {##1} {#2} {varpi}
        \@@_set_mathalphabet_pos:nnnn {normal} {epsilon} {##1} {#2}
        \@@_set_mathalphabet_pos:nnnn {normal} {vartheta} {##1} {#2}
        \@@_set_mathalphabet_pos:nnnn {normal} {varkappa} {##1} {#2}
        \@@_set_mathalphabet_pos:nnnn {normal} {phi} {##1} {#2}
        \@@_set_mathalphabet_pos:nnnn {normal} {varrho} {##1} {#2}
        \@@_set_mathalphabet_pos:nnnn {normal} {varpi} {##1} {#2}
      }
  }
%    \end{macrocode}
%
%    \begin{macrocode}
\cs_new:Nn \@@_set_normal_Greek:nn
  {
    \clist_map_inline:nn {#1}
      {
        \@@_set_mathalphabet_Greek:nnn {normal} {##1} {#2}
        \@@_map_chars_range:nnnn {25} {##1} {#2} {Greek}
        \@@_map_char_single:nnn {##1} {#2} {varTheta}
        \@@_set_mathalphabet_pos:nnnn {normal} {varTheta} {##1} {#2}
      }
  }
%    \end{macrocode}
%
%    \begin{macrocode}
\cs_new:Nn \@@_set_normal_numbers:nn
  {
    \@@_set_mathalphabet_numbers:nnn {normal} {#1} {#2}
    \@@_map_chars_range:nnnn {10} {#1} {#2} {num}
  }
%    \end{macrocode}
%
%
% \subsection{Mapping chars inside a math style}
%
% \subsubsection{Functions for setting up the maths alphabets}
%
% \begin{macro}{\@@_set_mathalphabet_char:nnn}
% \darg{Maths alphabet, \eg, `bb’}
% \darg{Input slot, \eg, the slot for `A’ (comma separated)}
% \darg{Output slot, \eg, the slot for `$\mathbb{A}$’}
% This is a wrapper for either |\@@_mathmap_noparse:nnn| or
% |\@@_mathmap_parse:nnn|, depending on the context.
% \end{macro}
%
% \begin{macro}{\@@_mathmap_noparse:nnn}
% \darg{Maths alphabet, \eg, `bb’}
% \darg{Input slot, \eg, the slot for `A’ (comma separated)}
% \darg{Output slot, \eg, the slot for `$\mathbb{A}$’}
% Adds \cs{@@_set_mathcode:nnnn} declarations to the specified maths alphabet’s definition.
%    \begin{macrocode}
\cs_new:Nn \@@_mathmap_noparse:nnn
  {
    \tl_put_right:cx { @@_switchto_#1: }
      {
        \@@_set_mathcode:nnnn {#2} {\mathalpha} {\l_@@_symfont_label_tl} {#3}
      }
  }
%    \end{macrocode}
% \end{macro}
%
% \begin{macro}{\@@_mathmap_parse:nnn}
% \darg{Maths alphabet, \eg, `bb’}
% \darg{Input slot, \eg, the slot for `A’ (comma separated)}
% \darg{Output slot, \eg, the slot for `$\mathbb{A}$’}
% When \cmd\@@_if_char_spec:nNT\ is executed, it populates the \cmd\l_@@_mathmap_charints_clist\
% macro with slot numbers corresponding to the specified range. This range is used to
% conditionally add \cs{@@_set_mathcode:nnnn} declaractions to the maths alphabet definition.
%    \begin{macrocode}
\cs_new:Nn \@@_mathmap_parse:nnn
  {
    \exp_args:NNx \clist_if_in:NnT \l_@@_mathmap_charints_clist { \int_eval:n {#3} }
      {
        \@@_mathmap_noparse:nnn {#1} {#2} {#3}
      }
  }
%    \end{macrocode}
% \end{macro}
%
% \begin{macro}{\@@_set_mathalphabet_char:nnnn}
% \darg{math style command}
% \darg{input math alphabet name}
% \darg{output math alphabet name}
% \darg{char name to map}
%    \begin{macrocode}
\cs_new:Nn \@@_set_mathalphabet_char:nnnn
  {
    \@@_set_mathalphabet_char:nnn {#1} { \@@_to_usv:nn {#2} {#4} }
                                       { \@@_to_usv:nn {#3} {#4} }
  }
%    \end{macrocode}
% \end{macro}
%
% \begin{macro}{\@@_set_mathalph_range:nnnn}
% \darg{Number of iterations}
% \darg{Sym command suffix}
% \darg{Starting input char}
% \darg{Starting output char}
% Loops through character ranges setting \cmd\mathcode.
% First the version that uses numbers:
%    \begin{macrocode}
\cs_new:Nn \@@_set_mathalph_range:nnnn
  {
    \int_step_inline:nnnn {0} {1} {#1-1}
      { \@@_set_mathalphabet_char:nnn {#2} { ##1 + #3 } { ##1 + #4 } }
  }
%    \end{macrocode}
% \end{macro}
%
% \begin{macro}{\@@_set_mathalph_range:nnnn}
% \darg{Number of iterations}
% \darg{Sym command suffix}
% \darg{input style}
% \darg{output style}
% \darg{alphabet}
% Then the wrapper version that uses names:
%    \begin{macrocode}
\cs_new:Nn \@@_set_mathalph_range:nnnnn
  {
    \clist_gput_right:cx { \l_@@_curr_named_slot }
      { \int_eval:n { \@@_to_usv:nn {#4} {#5} } - \int_eval:n { (#1-1)+\@@_to_usv:nn {#4} {#5} } }

    \@@_set_mathalph_range:nnnn {#1} {#2} { \@@_to_usv:nn {#3} {#5} }
                                          { \@@_to_usv:nn {#4} {#5} }
  }
%    \end{macrocode}
% \end{macro}
%
% \subsubsection{Individual mapping functions for different alphabets}
%
%    \begin{macrocode}
\cs_new:Nn \@@_set_mathalphabet_pos:nnnn
  {
    \@@_usv_if_exist:nnT {#4} {#2}
      {
        \clist_map_inline:nn {#3}
          { \@@_set_mathalphabet_char:nnnn {#1} {##1} {#4} {#2} }

        \clist_gput_right:cx {\l_@@_curr_named_slot}
          { \int_eval:n { \@@_to_usv:nn {#4} {#2} } }
      }
  }
%    \end{macrocode}
%
%    \begin{macrocode}
\cs_new:Nn \@@_set_mathalphabet_numbers:nnn
  {
    \clist_map_inline:nn {#2}
      { \@@_set_mathalph_range:nnnnn {10} {#1} {##1} {#3} {num} }
  }
%    \end{macrocode}
%
%    \begin{macrocode}
\cs_new:Nn \@@_set_mathalphabet_Latin:nnn
  {
    \clist_map_inline:nn {#2}
      { \@@_set_mathalph_range:nnnnn {26} {#1} {##1} {#3} {Latin} }
  }
%    \end{macrocode}
%
%    \begin{macrocode}
\cs_new:Nn \@@_set_mathalphabet_latin:nnn
  {
    \clist_map_inline:nn {#2}
      {
        \@@_set_mathalph_range:nnnnn {26} {#1} {##1} {#3} {latin}
        \@@_set_mathalphabet_char:nnnn    {#1} {##1} {#3} {h}
      }
  }
%    \end{macrocode}
%
%    \begin{macrocode}
\cs_new:Nn \@@_set_mathalphabet_Greek:nnn
  {
    \clist_map_inline:nn {#2}
      {
        \@@_set_mathalph_range:nnnnn {25} {#1} {##1} {#3} {Greek}
        \@@_set_mathalphabet_char:nnnn    {#1} {##1} {#3} {varTheta}
      }
  }
%    \end{macrocode}
%
%    \begin{macrocode}
\cs_new:Nn \@@_set_mathalphabet_greek:nnn
  {
    \clist_map_inline:nn {#2}
      {
        \@@_set_mathalph_range:nnnnn {25} {#1} {##1} {#3} {greek}
        \@@_set_mathalphabet_char:nnnn    {#1} {##1} {#3} {epsilon}
        \@@_set_mathalphabet_char:nnnn    {#1} {##1} {#3} {vartheta}
        \@@_set_mathalphabet_char:nnnn    {#1} {##1} {#3} {varkappa}
        \@@_set_mathalphabet_char:nnnn    {#1} {##1} {#3} {phi}
        \@@_set_mathalphabet_char:nnnn    {#1} {##1} {#3} {varrho}
        \@@_set_mathalphabet_char:nnnn    {#1} {##1} {#3} {varpi}
      }
  }
%    \end{macrocode}
%
%    \begin{macrocode}
%</package>
%    \end{macrocode}

\endinput

% /©
%
% ————————————————
% The UNICODE-MATH package  <wspr.io/unicode-math>
% ————————————————
% This package is free software and may be redistributed and/or modified under
% the conditions of the LaTeX Project Public License, version 1.3c or higher
% (your choice): <http://www.latex-project.org/lppl/>.
% ————————————————
% Copyright 2006-2017  Will Robertson, LPPL “maintainer”
% Copyright 2010-2017  Philipp Stephani
% Copyright 2011-2017  Joseph Wright
% Copyright 2012-2015  Khaled Hosny
% ————————————————
%
% ©/
