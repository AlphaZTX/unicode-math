%%^^A%% um-code-mathmap.dtx -- part of UNICODE-MATH <wspr.io/unicode-math>
%%^^A%% The "math symbol alphabets" such as `\symbf` and co.

% \section{Mapping in maths alphabets}
% \label{sec:mathmap}
%
%    \begin{macrocode}
%<*package>
%    \end{macrocode}
%
% Switching to a different style of alphabetic symbols was traditionally performed with
% commands like \cmd\mathbf, which literally changes fonts to access alternate symbols.
% This is not as simple with Unicode fonts.
%
% In traditional \TeX{} maths font setups, you simply switch between different `families' (\cmd\fam), which is analogous to changing from one font to another---a symbol such as `a' will be upright in one font, bold in another, and so on.
% In pkg{unicode-math}, a different mechanism is used to switch between styles. For every letter (start with ascii a-zA-Z and numbers to keep things simple for now), they are assigned a `mathcode' with \cmd\Umathcode\ that maps from input letter to output font glyph slot. This is done with the equivalent of
% \begin{Verbatim}
% \Umathcode`\a = 7 1 "1D44E\relax
% \Umathcode`\b = 7 1 "1D44F\relax
% \Umathcode`\c = 7 1 "1D450\relax
% ...
% \end{Verbatim}
% When switching from regular letters to, say, \cmd\mathrm, we now need to execute a new mapping:
% \begin{Verbatim}
% \Umathcode`\a = 7 1 `\a\relax
% \Umathcode`\b = 7 1 `\b\relax
% \Umathcode`\c = 7 1 `\c\relax
% ...
% \end{Verbatim}
% This is fairly straightforward to perform when we're defining our own commands such as \cmd\symbf\ and so on. However, this means that `classical' \TeX\ font setups will break, because with the original mapping still in place, the engine will be attempting to insert unicode maths glyphs from a standard font.
%
% \subsection{Hooks into \LaTeXe}
%
% To overcome this, we patch \cs{use@mathgroup}.
% (An alternative is to patch \cs{extract@alph@from@version}, which constructs the \cs{mathXYZ} commands, but this method fails if the command has been defined using \cs{DeclareSymbolFontAlphabet}.)
% As far as I can tell, this is only used inside of commands such as \cs{mathXYZ}, so this shouldn't have any major side-effects.
%
%    \begin{macrocode}
\cs_set:Npn \use@mathgroup #1 #2
  {
    \mode_if_math:T % <- not sure if this is really necessary since we've just checked for mmode and raised an error if not!
      {
        \math@bgroup
          \cs_if_eq:cNF {M@\f@encoding} #1 {#1}
          \@@_switchto_literal:
          \mathgroup #2 \relax
        \math@egroup
      }
  }
%    \end{macrocode}
%
% In LaTeX maths, the command |\operator@font| is defined that switches to the |operator| mathgroup. The classic example is the |\sin| in |$\sin{x}$|; essentially we’re using |\mathrm| to typeset the upright symbols, but the syntax is |{\operator@font sin}|.
% I thought that hooking into |\operator@font| would be hard because all other maths font selection in 2e uses |\mathrm{...}| style.
% Then reading source2e a little more I stumbled upon:
% \begin{macro}{\operator@font}
%    \begin{macrocode}
\cs_set:Npn \operator@font
  {
    \@@_switchto_literal:
    \@fontswitch {} { \g_@@_operator_mathfont_tl }
  }
%    \end{macrocode}
% \end{macro}
%
%
% \subsection{Setting styles}
%
% Algorithm for setting alphabet fonts.
% By default, when |range| is empty, we are in \emph{implicit} mode.
% If |range| contains the name of the math alphabet, we are in \emph{explicit}
% mode and do things slightly differently.
%
% Implicit mode:
% \begin{itemize}
% \item Try and set all of the alphabet shapes.
% \item Check for the first glyph of each alphabet to detect if the font supports each
%       alphabet shape.
% \item For alphabets that do exist, overwrite whatever’s already there.
% \item For alphabets that are not supported, \emph{do nothing}.
%       (This includes leaving the old alphabet definition in place.)
% \end{itemize}
%
% Explicit mode:
% \begin{itemize}
% \item Only set the alphabets specified.
% \item Check for the first glyph of the alphabet to detect if the font contains
%       the alphabet shape in the Unicode math plane.
% \item For Unicode math alphabets, overwrite whatever’s already there.
% \item Otherwise, use the \ascii\ glyph slots instead.
% \end{itemize}
%
%
%
% \subsection{Defining the math style macros}
%
% We call the different shapes that a math alphabet can be a `math style’.
% Note that different alphabets can exist within the same math style. E.g.,
% we call `bold’ the math style |bf| and within it there are upper and lower
% case Greek and Roman alphabets and Arabic numerals.
%
% \begin{macro}{\@@_prepare_mathstyle:n}
% \darg{math style name (e.g., \texttt{it} or \texttt{bb})}
% Define the high level math alphabet macros (\cs{mathit}, etc.) in terms of
% unicode-math definitions. Use \cs{bgroup}/\cs{egroup} so s’scripts scan the
% whole thing.
%
% The flag \cs{l_@@_mathstyle_tl} is for other applications to query the
% current math style.
%    \begin{macrocode}
\cs_new:Nn \@@_prepare_mathstyle:n
  {
    \seq_put_right:Nn \g_@@_mathstyles_seq {#1}
    \@@_init_alphabet:n {#1}
    \cs_set:cpn {_@@_sym_#1_aux:n} 
      {
        \use:c {@@_switchto_#1:} \math@egroup
      }
    \cs_set_protected:cpx {sym#1}
      {
        \exp_not:n
          {
            \math@bgroup
            \mode_if_math:F
              {
                \egroup\expandafter
                \non@alpherr\expandafter{\csname sym#1\endcsname\space}
              }
            \tl_set:Nn \l_@@_mathstyle_tl {#1}
          }
        \exp_not:c {_@@_sym_#1_aux:n}
      }
  }
%    \end{macrocode}
% \end{macro}
%
%
% \begin{macro}{\@@_init_alphabet:n}
% \darg{math alphabet name (e.g., \texttt{it} or \texttt{bb})}
% This macro initialises the macros used to set up a math alphabet.
% First used when the math alphabet macro is first defined, but then used
% later when redefining a particular maths alphabet.
%    \begin{macrocode}
\cs_set:Nn \@@_init_alphabet:n
  {
    \@@_log:nx {alph-initialise} {#1}
    \cs_set_eq:cN {@@_switchto_#1:} \prg_do_nothing:
  }
%    \end{macrocode}
% \end{macro}
%
% \subsection{Definition of alphabets and styles}
%
% First of all, we break up unicode into `named ranges’, such as |up|, |bb|, |sfup|, and so on, which refer to specific blocks of unicode that contain various symbols (usually alphabetical symbols).
%
%    \begin{macrocode}
\cs_new:Nn \@@_new_named_range:n
  {
    \prop_new:c {g_@@_named_range_#1_prop}
  }
%    \end{macrocode}
%
%    \begin{macrocode}
\clist_set:Nn \g_@@_named_ranges_clist
  {
    up, it, tt, bfup, bfit, bb , bbit, scr, bfscr, cal, bfcal,
    frak, bffrak, sfup, sfit, bfsfup, bfsfit, bfsf
  }
%    \end{macrocode}
%
%    \begin{macrocode}
\clist_map_inline:Nn \g_@@_named_ranges_clist
  {
    \@@_new_named_range:n {#1}
  }
%    \end{macrocode}
%
%
% Each alphabet style needs to be configured.
% This happens in Section~\ref{sec:setupalphabets}.
%    \begin{macrocode}
\cs_new:Nn \@@_new_alphabet_config:nnn
  {
    \prop_if_exist:cF {g_@@_named_range_#1_prop}
    { \@@_warning:nnn {no-named-range} {#1} {#2} }

    \prop_gput:cnn {g_@@_named_range_#1_prop} { alpha_tl }
      {
      \prop_item:cn {g_@@_named_range_#1_prop} { alpha_tl }
      {#2}
      }
    % Q: do I need to bother removing duplicates?

    \cs_new:cn { @@_config_#1_#2:n } {#3}
  }
%    \end{macrocode}
%    \begin{macrocode}
\cs_new:Nn \@@_alphabet_config:nnn
  {
    \use:c {@@_config_#1_#2:n} {#3}
  }
%    \end{macrocode}
%    \begin{macrocode}
\prg_new_conditional:Nnn \@@_if_alphabet_exists:nn {T,TF}
  {
    \cs_if_exist:cTF {@@_config_#1_#2:n}
      \prg_return_true: \prg_return_false:
  }
%    \end{macrocode}
%
% The linking between named ranges and symbol style commands happens here.
% It’s currently not using all of the machinery we’re in the process of setting up above.
% Baby steps.
%    \begin{macrocode}
\cs_new:Nn \@@_default_mathalph:nnn
  {
    \seq_put_right:Nx \g_@@_named_ranges_seq { \tl_to_str:n {#1} }
    \seq_put_right:Nn \g_@@_default_mathalph_seq {{#1}{#2}{#3}}
    \prop_gput:cnn { g_@@_named_range_#1_prop } { default-alpha } {#2}
  }
%    \end{macrocode}
%    \begin{macrocode}
\@@_default_mathalph:nnn {up    } {latin,Latin,greek,Greek,num,misc} {up    }
\@@_default_mathalph:nnn {it    } {latin,Latin,greek,Greek,misc}     {it    }
\@@_default_mathalph:nnn {bb    } {latin,Latin,num,misc}             {bb    }
\@@_default_mathalph:nnn {bbit  } {misc}                             {bbit  }
\@@_default_mathalph:nnn {scr   } {latin,Latin}                      {scr   }
\@@_default_mathalph:nnn {cal   } {Latin}                            {scr   }
\@@_default_mathalph:nnn {bfcal } {Latin}                            {bfscr }
\@@_default_mathalph:nnn {frak  } {latin,Latin}                      {frak  }
\@@_default_mathalph:nnn {tt    } {latin,Latin,num}                  {tt    }
\@@_default_mathalph:nnn {sfup  } {latin,Latin,num}                  {sfup  }
\@@_default_mathalph:nnn {sfit  } {latin,Latin}                      {sfit  }
\@@_default_mathalph:nnn {bfup  } {latin,Latin,greek,Greek,num,misc} {bfup  }
\@@_default_mathalph:nnn {bfit  } {latin,Latin,greek,Greek,misc}     {bfit  }
\@@_default_mathalph:nnn {bfscr } {latin,Latin}                      {bfscr }
\@@_default_mathalph:nnn {bffrak} {latin,Latin}                      {bffrak}
\@@_default_mathalph:nnn {bfsfup} {latin,Latin,greek,Greek,num,misc} {bfsfup}
\@@_default_mathalph:nnn {bfsfit} {latin,Latin,greek,Greek,misc}     {bfsfit}
%    \end{macrocode}
%
% \subsubsection{Define symbol style commands}
% Finally, all of the `symbol styles’ commands are set up, which are the commands to access each of the named alphabet styles. There is not a one-to-one mapping between symbol style commands and named style ranges!
%    \begin{macrocode}
\clist_map_inline:nn
  {
    up, it, bfup, bfit, sfup, sfit, bfsfup, bfsfit, bfsf,
    tt, bb, bbit, scr, bfscr, cal, bfcal, frak, bffrak,
    normal, literal, sf, bf,
  }
  { 
    \@@_prepare_mathstyle:n {#1}
  }
%    \end{macrocode}
%
%
% \subsubsection{New names for legacy textmath alphabet selection}
% In case a package option overwrites, say, \cs{mathbf} with \cs{symbf}.
%    \begin{macrocode}
\clist_map_inline:nn
  { rm, it, bf, sf, tt }
  { \cs_set_eq:cc { mathtext #1 } { math #1 } }
%    \end{macrocode}
% Perhaps these should actually be defined using a hypothetical unicode-math interface to creating new such styles. To come.
%
%
% \subsubsection{Replacing legacy pure-maths alphabets}
% The following are alphabets which do not have a math/text ambiguity.
%    \begin{macrocode}
\clist_map_inline:nn
  {
    normal, bb , bbit, scr, bfscr, cal, bfcal, frak, bffrak, tt,
    bfup, bfit, sfup, sfit, bfsfup, bfsfit, bfsf
  }
  {
    \cs_set:cpx { math #1 } { \exp_not:c { sym #1 } }
  }
%    \end{macrocode}
%
%
% \subsubsection{New commands for ambiguous alphabets}
%
%    \begin{macrocode}
\AtBeginDocument
  {
    \clist_map_inline:nn
      { rm, it, bf, sf, tt }
      {
        \cs_set_protected:cpx { math #1 }
        {
          \exp_not:n { \bool_if:NTF  } \exp_not:c { g_@@_ math #1 _text_bool}
            { \exp_not:c { mathtext #1 } }
            { \exp_not:c { sym #1 } }
        }
      }
  }
%    \end{macrocode}
%
% \paragraph{Alias \cs{mathrm} as legacy name for \cs{mathup}}
%    \begin{macrocode}
\cs_set_protected:Npn \mathup { \mathrm }
\cs_set_protected:Npn \symrm  { \symup  }
%    \end{macrocode}
%
%
%
%
% \subsection{Defining the math alphabets per style}
%
% \begin{macro}{\@@_setup_alphabets:}
% This function is called within \cs{setmathfont} to configure the
% mapping between characters inside math styles.
%    \begin{macrocode}
\cs_new:Npn \@@_setup_alphabets:
  {
%    \end{macrocode}
% If |range=| has been used to configure styles, those choices will be in
% |\l_@@_mathalph_seq|. If not, set up the styles implicitly:
%    \begin{macrocode}
    \seq_if_empty:NTF \l_@@_mathalph_seq
      {
        \@@_log:n {setup-implicit}
        \seq_set_eq:NN \l_@@_mathalph_seq \g_@@_default_mathalph_seq
        \bool_set_true:N \l_@@_implicit_alph_bool
        \@@_maybe_init_alphabet:n  {sf}
        \@@_maybe_init_alphabet:n  {bf}
        \@@_maybe_init_alphabet:n  {bfsf}
      }
%    \end{macrocode}
% If |range=| has been used then we’re in explicit mode:
%    \begin{macrocode}
      {
        \@@_log:n {setup-explicit}
        \bool_set_false:N \l_@@_implicit_alph_bool
        \cs_set_eq:NN \@@_set_mathalphabet_char:nnn \@@_mathmap_noparse:nnn
        \cs_set_eq:NN \@@_map_char_single:nn \@@_map_char_noparse:nn
      }
%    \end{macrocode}
% Now perform the mapping:
%    \begin{macrocode}
   \seq_map_inline:Nn \l_@@_mathalph_seq
      {
        \tl_set:No    \l_@@_style_tl       { \use_i:nnn   ##1 }
        \clist_set:No \l_@@_alphabet_clist { \use_ii:nnn  ##1 }
        \tl_set:No    \l_@@_remap_style_tl { \use_iii:nnn ##1 }

        % If no set of alphabets is defined:
        \clist_if_empty:NT \l_@@_alphabet_clist
        {
          \cs_set_eq:NN \@@_maybe_init_alphabet:n \@@_init_alphabet:n
          \prop_get:cnN { g_@@_named_range_ \l_@@_style_tl _prop }
          { default-alpha } \l_@@_alphabet_clist
        }

        \@@_setup_math_alphabet:
      }
    \seq_if_empty:NF \l_@@_missing_alph_seq { \@@_log:n { missing-alphabets } }
 }
%    \end{macrocode}
% \end{macro}
%
% \begin{macro}{\@@_setup_math_alphabet:}
%    \begin{macrocode}
\cs_new:Nn \@@_setup_math_alphabet:
  {
%    \end{macrocode}
% First check that at least one of the alphabets for the font shape is defined
% (this process is fast) \dots
%    \begin{macrocode}
    \clist_map_inline:Nn \l_@@_alphabet_clist
      {
        \tl_set:Nn \l_@@_alphabet_tl {##1}
        \@@_if_alphabet_exists:nnTF \l_@@_style_tl \l_@@_alphabet_tl
          {
            \str_if_eq_x:nnTF {\l_@@_alphabet_tl} {misc}
              {
                \@@_maybe_init_alphabet:n \l_@@_style_tl
                \clist_map_break:
              }
              {
                \@@_glyph_if_exist:NnT \l_@@_font { \@@_to_usv:nn {\l_@@_style_tl} {\l_@@_alphabet_tl} }
                {
                  \@@_maybe_init_alphabet:n \l_@@_style_tl
                  \clist_map_break:
                }
              }
          }
          {
            \msg_warning:nnx {unicode-math} {no-alphabet}
              { \l_@@_style_tl / \l_@@_alphabet_tl }
          }
      }
%    \end{macrocode}
% \dots and then loop through them defining the individual ranges:
% (currently this process is slow)
%    \begin{macrocode}
%<debug>  \csname TIC\endcsname
    \clist_map_inline:Nn \l_@@_alphabet_clist
      {
        \tl_set:Nx \l_@@_alphabet_tl { \tl_trim_spaces:n {##1} }
        \cs_if_exist:cT {@@_config_ \l_@@_style_tl _ \l_@@_alphabet_tl :n}
          {
            \exp_args:No \tl_if_eq:nnTF \l_@@_alphabet_tl {misc}
              {
                \@@_log:nx {setup-alph} {sym \l_@@_style_tl~(\l_@@_alphabet_tl)}
                \@@_alphabet_config:nnn {\l_@@_style_tl} {\l_@@_alphabet_tl} {\l_@@_remap_style_tl}
              }
              {
                \@@_glyph_if_exist:NnTF \l_@@_font { \@@_to_usv:nn {\l_@@_remap_style_tl} {\l_@@_alphabet_tl} }
                  {
                    \@@_log:nx {setup-alph} {sym \l_@@_style_tl~(\l_@@_alphabet_tl)}
                    \@@_alphabet_config:nnn {\l_@@_style_tl} {\l_@@_alphabet_tl} {\l_@@_remap_style_tl}
                  }
                  {
                    \bool_if:NTF \l_@@_implicit_alph_bool
                      {
                        \seq_put_right:Nx \l_@@_missing_alph_seq
                          {
                            \@backslashchar sym \l_@@_style_tl \space
                            (\tl_use:c{c_@@_math_alphabet_name_ \l_@@_alphabet_tl _tl})
                          }
                      }
                      {
                        \@@_alphabet_config:nnn {\l_@@_style_tl} {\l_@@_alphabet_tl} {up}
                      }
                  }
              }
          }
      }
%<debug>  \csname TOC\endcsname
  }
%    \end{macrocode}
% \end{macro}
%
%
% \subsection{Mapping `naked’ math characters}
%
% Before we show the definitions of the alphabet mappings using the functions
% |\@@_alphabet_config:nnn \l_@@_style_tl {##1} {...}|, we first want to define some functions
% to be used inside them to actually perform the character mapping.
%
% \subsubsection{Functions}
%
% \begin{macro}{\@@_map_char_single:nn}
% Wrapper for |\@@_map_char_noparse:nn| or |\@@_map_char_parse:nn|
% depending on the context.
%
% \begin{macro}{\@@_map_char_noparse:nn}
% \begin{macro}{\@@_map_char_parse:nn}
%    \begin{macrocode}
\cs_new:Nn \@@_map_char_noparse:nn
  {
    \@@_set_mathcode:nnnn {#1}{\mathalpha}{\l_@@_symfont_label_tl}{#2}
  }
%    \end{macrocode}
%
%    \begin{macrocode}
\cs_new:Nn \@@_map_char_parse:nn
  {
    \@@_if_char_spec:nNNT {#1} {\@nil} {\mathalpha}
      { \@@_map_char_noparse:nn {#1}{#2} }
  }
%    \end{macrocode}
% \end{macro}
% \end{macro}
% \end{macro}
%
% \begin{macro}{\@@_map_char_single:nnn}
% \darg{char name (`dotlessi’)}
% \darg{from alphabet(s)}
% \darg{to alphabet}
% Logical interface to \cs{@@_map_char_single:nn}.
%    \begin{macrocode}
\cs_new:Nn \@@_map_char_single:nnn
  {
    \@@_map_char_single:nn { \@@_to_usv:nn {#1}{#3} }
                           { \@@_to_usv:nn {#2}{#3} }
  }
%    \end{macrocode}
% \end{macro}
%
%
% \begin{macro}{\@@_map_chars_range:nnnn}
% \darg{Number of chars (26)}
% \darg{From style, one or more (it)}
% \darg{To style (up)}
% \darg{Alphabet name (Latin)}
% First the function with numbers:
%    \begin{macrocode}
\cs_set:Nn \@@_map_chars_range:nnn
  {
    \int_step_inline:nnnn {0}{1}{#1-1}
      { \@@_map_char_single:nn {#2+##1}{#3+##1} }
  }
%    \end{macrocode}
% And the wrapper with names:
%    \begin{macrocode}
\cs_new:Nn \@@_map_chars_range:nnnn
  {
    \@@_map_chars_range:nnn {#1} { \@@_to_usv:nn {#2}{#4} }
                                 { \@@_to_usv:nn {#3}{#4} }
  }
%    \end{macrocode}
% \end{macro}
%
% \subsubsection{Functions for `normal’ alphabet symbols}
%
% \begin{macro}{\@@_set_normal_char:nnn}
%    \begin{macrocode}
\cs_set:Nn \@@_set_normal_char:nnn
  {
    \@@_usv_if_exist:nnT {#3} {#1}
      {
        \clist_map_inline:nn {#2}
          {
            \@@_set_mathalphabet_pos:nnnn {normal} {#1} {##1} {#3}
            \@@_map_char_single:nnn {##1} {#3} {#1}
          }
      }
  }
%    \end{macrocode}
% \end{macro}
%
%    \begin{macrocode}
\cs_new:Nn \@@_set_normal_Latin:nn
  {
    \clist_map_inline:nn {#1}
      {
        \@@_set_mathalphabet_Latin:nnn {normal} {##1} {#2}
        \@@_map_chars_range:nnnn {26} {##1} {#2} {Latin}
      }
  }
%    \end{macrocode}
%
%    \begin{macrocode}
\cs_new:Nn \@@_set_normal_latin:nn
  {
    \clist_map_inline:nn {#1}
      {
        \@@_set_mathalphabet_latin:nnn {normal} {##1} {#2}
        \@@_map_chars_range:nnnn {26} {##1} {#2} {latin}
      }
  }
%    \end{macrocode}
%
%    \begin{macrocode}
\cs_new:Nn \@@_set_normal_greek:nn
  {
    \clist_map_inline:nn {#1}
      {
        \@@_set_mathalphabet_greek:nnn {normal} {##1} {#2}
        \@@_map_chars_range:nnnn {25} {##1} {#2} {greek}
        \@@_map_char_single:nnn {##1} {#2} {epsilon}
        \@@_map_char_single:nnn {##1} {#2} {vartheta}
        \@@_map_char_single:nnn {##1} {#2} {varkappa}
        \@@_map_char_single:nnn {##1} {#2} {phi}
        \@@_map_char_single:nnn {##1} {#2} {varrho}
        \@@_map_char_single:nnn {##1} {#2} {varpi}
        \@@_set_mathalphabet_pos:nnnn {normal} {epsilon} {##1} {#2}
        \@@_set_mathalphabet_pos:nnnn {normal} {vartheta} {##1} {#2}
        \@@_set_mathalphabet_pos:nnnn {normal} {varkappa} {##1} {#2}
        \@@_set_mathalphabet_pos:nnnn {normal} {phi} {##1} {#2}
        \@@_set_mathalphabet_pos:nnnn {normal} {varrho} {##1} {#2}
        \@@_set_mathalphabet_pos:nnnn {normal} {varpi} {##1} {#2}
      }
  }
%    \end{macrocode}
%
%    \begin{macrocode}
\cs_new:Nn \@@_set_normal_Greek:nn
  {
    \clist_map_inline:nn {#1}
      {
        \@@_set_mathalphabet_Greek:nnn {normal} {##1} {#2}
        \@@_map_chars_range:nnnn {25} {##1} {#2} {Greek}
        \@@_map_char_single:nnn {##1} {#2} {varTheta}
        \@@_set_mathalphabet_pos:nnnn {normal} {varTheta} {##1} {#2}
      }
  }
%    \end{macrocode}
%
%    \begin{macrocode}
\cs_new:Nn \@@_set_normal_numbers:nn
  {
    \@@_set_mathalphabet_numbers:nnn {normal} {#1} {#2}
    \@@_map_chars_range:nnnn {10} {#1} {#2} {num}
  }
%    \end{macrocode}
%
%
% \subsection{Mapping chars inside a math style}
%
% \subsubsection{Functions for setting up the maths alphabets}
%
% \begin{macro}{\@@_set_mathalphabet_char:Nnn}
% This is a wrapper for either |\@@_mathmap_noparse:nnn| or
% |\@@_mathmap_parse:Nnn|, depending on the context.
% \end{macro}
%
% \begin{macro}{\@@_mathmap_noparse:nnn}
% \darg{Maths alphabet, \eg, `bb’}
% \darg{Input slot(s), \eg, the slot for `A’ (comma separated)}
% \darg{Output slot, \eg, the slot for `$\mathbb{A}$’}
% Adds \cs{@@_set_mathcode:nnnn} declarations to the specified maths alphabet’s definition.
%    \begin{macrocode}
\cs_new:Nn \@@_mathmap_noparse:nnn
  {
    \clist_map_inline:nn {#2}
      {
        \tl_put_right:cx {@@_switchto_#1:}
          {
            \@@_set_mathcode:nnnn {##1} {\mathalpha} {\l_@@_symfont_label_tl} {#3}
          }
      }
  }
%    \end{macrocode}
% \end{macro}
%
% \begin{macro}{\@@_mathmap_parse:nnn}
% \darg{Maths alphabet, \eg, `bb’}
% \darg{Input slot(s), \eg, the slot for `A’ (comma separated)}
% \darg{Output slot, \eg, the slot for `$\mathbb{A}$’}
% When \cmd\@@_if_char_spec:nNNT\ is executed, it populates the \cmd\l_@@_char_nrange_clist\
% macro with slot numbers corresponding to the specified range. This range is used to
% conditionally add \cs{@@_set_mathcode:nnnn} declaractions to the maths alphabet definition.
%    \begin{macrocode}
\cs_new:Nn \@@_mathmap_parse:nnn
  {
    \clist_if_in:NnT \l_@@_char_nrange_clist {#3}
      {
        \@@_mathmap_noparse:nnn {#1}{#2}{#3}
      }
  }
%    \end{macrocode}
% \end{macro}
%
% \begin{macro}{\@@_set_mathalphabet_char:nnnn}
% \darg{math style command}
% \darg{input math alphabet name}
% \darg{output math alphabet name}
% \darg{char name to map}
%    \begin{macrocode}
\cs_new:Nn \@@_set_mathalphabet_char:nnnn
  {
    \@@_set_mathalphabet_char:nnn {#1} { \@@_to_usv:nn {#2} {#4} }
                                       { \@@_to_usv:nn {#3} {#4} }
  }
%    \end{macrocode}
% \end{macro}
%
% \begin{macro}{\@@_set_mathalph_range:nnnn}
% \darg{Number of iterations}
% \darg{Maths alphabet}
% \darg{Starting input char (single)}
% \darg{Starting output char}
% Loops through character ranges setting \cmd\mathcode.
% First the version that uses numbers:
%    \begin{macrocode}
\cs_new:Nn \@@_set_mathalph_range:nnnn
  {
    \int_step_inline:nnnn {0} {1} {#1-1}
      { \@@_set_mathalphabet_char:nnn {#2} { ##1 + #3 } { ##1 + #4 } }
  }
%    \end{macrocode}
% Then the wrapper version that uses names:
%    \begin{macrocode}
\cs_new:Nn \@@_set_mathalph_range:nnnnn
  {
    \@@_set_mathalph_range:nnnn {#1} {#2} { \@@_to_usv:nn {#3} {#5} }
                                          { \@@_to_usv:nn {#4} {#5} }
  }
%    \end{macrocode}
% \end{macro}
%
% \subsubsection{Individual mapping functions for different alphabets}
%
%    \begin{macrocode}
\cs_new:Nn \@@_set_mathalphabet_pos:nnnn
  {
    \@@_usv_if_exist:nnT {#4} {#2}
      {
        \clist_map_inline:nn {#3}
          { \@@_set_mathalphabet_char:nnnn {#1} {##1} {#4} {#2} }
      }
  }
%    \end{macrocode}
%
%    \begin{macrocode}
\cs_new:Nn \@@_set_mathalphabet_numbers:nnn
  {
    \clist_map_inline:nn {#2}
      { \@@_set_mathalph_range:nnnnn {10} {#1} {##1} {#3} {num} }
  }
%    \end{macrocode}
%
%    \begin{macrocode}
\cs_new:Nn \@@_set_mathalphabet_Latin:nnn
  {
    \clist_map_inline:nn {#2}
      { \@@_set_mathalph_range:nnnnn {26} {#1} {##1} {#3} {Latin} }
  }
%    \end{macrocode}
%
%    \begin{macrocode}
\cs_new:Nn \@@_set_mathalphabet_latin:nnn
  {
    \clist_map_inline:nn {#2}
      {
        \@@_set_mathalph_range:nnnnn {26} {#1} {##1} {#3} {latin}
        \@@_set_mathalphabet_char:nnnn    {#1} {##1} {#3} {h}
      }
  }
%    \end{macrocode}
%
%    \begin{macrocode}
\cs_new:Nn \@@_set_mathalphabet_Greek:nnn
  {
    \clist_map_inline:nn {#2}
      {
        \@@_set_mathalph_range:nnnnn {25} {#1} {##1} {#3} {Greek}
        \@@_set_mathalphabet_char:nnnn    {#1} {##1} {#3} {varTheta}
      }
  }
%    \end{macrocode}
%
%    \begin{macrocode}
\cs_new:Nn \@@_set_mathalphabet_greek:nnn
  {
    \clist_map_inline:nn {#2}
      {
        \@@_set_mathalph_range:nnnnn {25} {#1} {##1} {#3} {greek}
        \@@_set_mathalphabet_char:nnnn    {#1} {##1} {#3} {epsilon}
        \@@_set_mathalphabet_char:nnnn    {#1} {##1} {#3} {vartheta}
        \@@_set_mathalphabet_char:nnnn    {#1} {##1} {#3} {varkappa}
        \@@_set_mathalphabet_char:nnnn    {#1} {##1} {#3} {phi}
        \@@_set_mathalphabet_char:nnnn    {#1} {##1} {#3} {varrho}
        \@@_set_mathalphabet_char:nnnn    {#1} {##1} {#3} {varpi}
      }
  }
%    \end{macrocode}
%
%    \begin{macrocode}
%</package>
%    \end{macrocode}

\endinput

% /©
%
% ————————————————
% The UNICODE-MATH package  <wspr.io/unicode-math>
% ————————————————
% This package is free software and may be redistributed and/or modified under
% the conditions of the LaTeX Project Public License, version 1.3c or higher
% (your choice): <http://www.latex-project.org/lppl/>.
% ————————————————
% Copyright 2006-2017  Will Robertson, LPPL “maintainer”
% Copyright 2010-2017  Philipp Stephani
% Copyright 2011-2017  Joseph Wright
% Copyright 2012-2015  Khaled Hosny
% ————————————————
%
% ©/
